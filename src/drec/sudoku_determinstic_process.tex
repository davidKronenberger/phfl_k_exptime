In 2006 B. Borchert and K. Reinhardt introduced a stronger version of determinism which they called
Sudoku-determinism~\cite{borchert2006deterministically}. They also considered a more general
deterministic process than the one presented in Section~\ref{deterministicprocess}. 

The \emph{Sudoku-determinism} is defined in a process that reduces a picture step by step. The
difference with the determinism described in the section above is that instead of determining the
pre-image symbol on a position in one shot, they keep a set of possible pre-image symbols for each
position and reduce one set per step of the process by excluding impossible pre-images. 

Accordingly, let $\tau = (\Sigma, \Gamma, \Delta, \pi)$ be a DS. For a picture $p \in \Sigma$, $s_p$
is a local picture of the same size in which every position $(i, j)$ is initialized by the set
$\pi^{-1}(p(i, j)) \in 2^\Gamma$ of possible pre-image symbols. With this information, one step in a
Sudoku-deterministic process is defined as follows:
\begin{definition} For $s, s' \in (2^\Gamma)^{*,*}$ with $l_1(s) = l_1(s')$ and $l_2(s) = l_2(s')$,
we allow a \emph{Sudoku-deterministic step} $\hat{s} \underset{sd(\tau)}{\Rightarrow} \hat{s}'$ if
the following condition holds for all positions $(i, j)$ of $s'$:
\begin{align*}
\left.\begin{tabular}{C{0.8cm}} 
 \tabularnewline 
 \tabularnewline
\end{tabular}s'(i,j) = \right\lbrace &x \in s(i,j) \mid \exists \gamma_1, \gamma_2, \gamma_3,
\gamma_4 \in \Gamma \cup \{\#\} \text{ such that } \\
&\gamma_1 \in \hat{s}(i, j + 1), \gamma_2 \in \hat{s}(i, j - 1),
\gamma_3 \in \hat{s}(i + 1, j), \gamma_4 \in \hat{s}(i - 1, j)\\
&\text{ and }
\begin{tabular}{|C{0.8cm}|C{0.8cm}|}
\hline
 \hspace{0.1cm}x\hspace{0.1cm} & $\gamma_1$ \tabularnewline
\hline 
\end{tabular}, 
\begin{tabular}{|C{0.8cm}|C{0.8cm}|} 
\hline 
$\gamma_2$ & \hspace{0.1cm}x\hspace{0.1cm} \tabularnewline
\hline 
\end{tabular}, 
\begin{tabular}{|C{0.8cm}|} 
\hline 
x \tabularnewline 
\hline 
$\gamma_3$ \tabularnewline
\hline 
\end{tabular},
\left. 
\begin{tabular}{|C{0.8cm}|} 
\hline 
$\gamma_4$ \tabularnewline 
\hline 
x \tabularnewline
\hline 
\end{tabular}
\in \Delta\right\rbrace.
\end{align*}
\end{definition}
One step in a deterministic process which is described in Section~\ref{deterministicprocess}
can be formulated as a special case (see~\cite{borchert2006deterministically}).
\begin{definition}
For $s, s', s'' \in (2^\Gamma)^{*,*}$ with $l_1(s) = l_1(s')$ and $l_2(s) = l_2(s')$ we allow a
\emph{deterministic step} $\hat{s} \underset{d(\tau)}{\Rightarrow} \hat{s}'$ if $\hat{s}
\underset{sd(\tau)}{\Rightarrow} \hat{s}''$ and, for all positions $(i, j)$ of $s'$, we define
$s'(i, j) := s''(i, j)$ if $\left| s''(i, j) \right| = 1$ or $\hat{s}'(i, j) = \hat{s}(i, j)$.
\label{deterministic_step_definition}
\end{definition}
Remark that by reducing all positions to singletons we got the forced choice of
Definition~\ref{deterministic_process_definition}. The replacement of each symbol by one of its
local symbols dependend on the dominos of $\Delta'$ of
Definition~\ref{deterministic_process_definition}, is similar to starting with all possible local
symbols of one symbol and minimize them to a one element dependent on the dominos of $\Delta$ of 
Definition~\ref{deterministic_step_definition}.

In the following we use the notion from~\cite{borchert2006deterministically}, where $\{p\}$
describes a picture which has the same size as $p$ and has a singleton $\{p(i, j)\}$ instead of
the letter $p(i, j)$ on every position.

Let $\tau = (\Sigma, \Gamma, \Delta, \pi)$ be a DS. Then $L_{sd}(\tau) := \{p \in \Sigma^{*,*}
\mid \exists p' \in L(\tau)$ s.t.
$\hat{s}_p \overset{*}{\underset{sd(\tau)}{\Rightarrow}} \{\hat{p}'\}\}$ is a
\emph{Sudoku-deterministically recognizable picture language}.
That implies that the local picture $\hat{s}_p$ can be transformed with a finite number of steps of
a Sudoku-deterministic process into the picture $\{\hat{p}'\}$. Remark that the language
$L_{d}(\tau)$ is defined in the same way using $\overset{*}{\underset{d(\tau)}{\Rightarrow}}$ instead of
$\overset{*}{\underset{sd(\tau)}{\Rightarrow}}$. The \emph{family of all Sudoku-deterministically
recognizable picture languages} is denoted by \emph{SDREC}. The \emph{class of all
deterministically recognizable picture languages} $L_d(\tau)$ is denoted by DREC.

We now consider an example of a DS $\tau$, some subpictures $t_p, t'_p, t''_p \in B_{2, 2}(s_p)$,
where $p \in \Sigma^{*,*}$, and how a Sudoku-deterministic and a deterministic process work with
these subpictures depending on the given $\tau$. For convenience, we only test two of the four possible
neighbours of every position in the subpictures.
\begin{example}
For a DS $\tau = (\Sigma, \Gamma, \Delta, \pi)$ with $\Gamma = \{a, b, c, d\}$ and\newline
$\Delta = \left\lbrace
\begin{tabular}{|C{0.7cm}|} 
\hline 
\hspace{0.05cm}$a$\hspace{0.05cm} \tabularnewline 
\hline 
$b$ \tabularnewline
\hline 
\end{tabular}, 
\begin{tabular}{|C{0.7cm}|} 
\hline 
\hspace{0.05cm}$b$\hspace{0.05cm} \tabularnewline 
\hline 
$a$ \tabularnewline
\hline 
\end{tabular},\begin{tabular}{|C{0.7cm}|} 
\hline 
\hspace{0.05cm}$c$\hspace{0.05cm} \tabularnewline 
\hline 
$d$ \tabularnewline
\hline 
\end{tabular},\begin{tabular}{|C{0.7cm}|} 
\hline 
\hspace{0.05cm}$d$\hspace{0.05cm} \tabularnewline 
\hline 
$c$ \tabularnewline
\hline 
\end{tabular}, 
\begin{tabular}{|C{0.7cm}|} 
\hline 
\hspace{0.05cm}$x$\hspace{0.05cm} \tabularnewline 
\hline 
$x$ \tabularnewline
\hline 
\end{tabular},
\begin{tabular}{|C{0.7cm}|} 
\hline 
\hspace{0.05cm}$x$\hspace{0.05cm} \tabularnewline 
\hline 
\# \tabularnewline
\hline 
\end{tabular},
\begin{tabular}{|C{0.7cm}|} 
\hline 
\# \tabularnewline 
\hline 
$x$ \tabularnewline
\hline 
\end{tabular},
\begin{tabular}{|C{0.7cm}|C{0.7cm}|} 
\hline 
\hspace{0.05cm}$a$\hspace{0.05cm} & \hspace{0.05cm}$c$\hspace{0.05cm} \tabularnewline
\hline 
\end{tabular},
\begin{tabular}{|C{0.7cm}|C{0.7cm}|} 
\hline 
\hspace{0.05cm}$c$\hspace{0.05cm} & \hspace{0.05cm}$a$\hspace{0.05cm} \tabularnewline
\hline 
\end{tabular},
\begin{tabular}{|C{0.7cm}|C{0.7cm}|} 
\hline 
\hspace{0.05cm}$b$\hspace{0.05cm} & \hspace{0.05cm}$d$\hspace{0.05cm} \tabularnewline
\hline 
\end{tabular},
\begin{tabular}{|C{0.7cm}|C{0.7cm}|} 
\hline 
\hspace{0.05cm}$d$\hspace{0.05cm} & \hspace{0.05cm}$b$\hspace{0.05cm} \tabularnewline
\hline 
\end{tabular}\right.$,\newline
\begin{tabular}{|C{0.7cm}|C{0.7cm}|} 
\hline 
\hspace{0.05cm}$x$\hspace{0.05cm} & \hspace{0.05cm}$x$\hspace{0.05cm} \tabularnewline
\hline 
\end{tabular}, 
\begin{tabular}{|C{0.7cm}|C{0.7cm}|} 
\hline 
\hspace{0.05cm}$x$\hspace{0.05cm} & \# \tabularnewline
\hline 
\end{tabular},
\begin{tabular}{|C{0.7cm}|C{0.7cm}|} 
\hline 
\# & \hspace{0.05cm}$x$\hspace{0.05cm} \tabularnewline
\hline 
\end{tabular}
$\mid x \in \Gamma\left\rbrace\begin{tabular}{C{0.7cm}} 
 \tabularnewline 
 \tabularnewline
\end{tabular}\right.$ \newline
holds.
\begin{center}
$t_p = $ 
\begin{tabular}{|C{1.75cm}|C{1.75cm}|} 
\hline 
$a, b, c, d$ & $a, b, c, d$ \tabularnewline
\hline 
$b, d$ & $a, b, c, d$ \tabularnewline
\hline 
\end{tabular} $\underset{sd(\tau)}{\Rightarrow}$
\begin{tabular}{|C{1.75cm}|C{1.75cm}|} 
\hline 
$a, b, c, d$ & $a, b, c, d$ \tabularnewline
\hline 
$b, d$ & $b, d$ \tabularnewline
\hline 
\end{tabular}
\end{center}
\begin{center}
$t_p' = $ 
\begin{tabular}{|C{1.75cm}|C{1.75cm}|} 
\hline 
$a, b, c, d$ & $c, d$ \tabularnewline
\hline 
$a, b, c, d$ & $a, b, c, d$ \tabularnewline
\hline 
\end{tabular} $\underset{sd(\tau)}{\Rightarrow}$
\begin{tabular}{|C{1.75cm}|C{1.75cm}|} 
\hline 
$a, b, c, d$ & \hspace{0.4cm}$c, d$\hspace{0.4cm} \tabularnewline
\hline 
$a, b, c, d$ & $c, d$ \tabularnewline
\hline 
\end{tabular}
\end{center}
\begin{center}
$t_p'' = $ 
\begin{tabular}{|C{1.75cm}|C{1.75cm}|} 
\hline 
$a, b, c, d$ & \hspace{0.4cm}$c, d$\hspace{0.4cm} \tabularnewline
\hline 
$b, d$ & $a, b, c, d$ \tabularnewline
\hline 
\end{tabular} $\underset{d(\tau)}{\Rightarrow}$
\begin{tabular}{|C{1.75cm}|C{1.75cm}|} 
\hline 
$a, b, c, d$ & \hspace{0.4cm}$c, d$\hspace{0.4cm} \tabularnewline
\hline 
$b, d$ & $d$ \tabularnewline
\hline 
\end{tabular}
\end{center}
\end{example}
Furthermore, the language family dependencies will be talked about. The
following class relationships are proved
in~\cite{borchert2006deterministically}:
\begin{theorem} The family DREC is a subset of
\begin{compactitem}
\item SDREC, 
\item REC and 
\item co-REC $ = \lbrace L \mid \bar{L} \in REC \rbrace$.
\end{compactitem}
\end{theorem}