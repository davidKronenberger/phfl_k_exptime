In~\cite{reinhardt1998recognizable} K. Reinhard introduced another way to describe deterministically
recognizable picture languages. For this he defined a \emph{deterministic process} that
starts with a given picture $p$ over $\Sigma$ and ends with the local picture $p'$ over $\Gamma$ using a DS. One step in
this process is the replacement of one symbol $s \in \Sigma$ by its local symbol $g \in \Gamma$,
which can only be performed if it is locally the only possible choice. Formally this is the
following:
\begin{definition}
Let $\tau = (\Sigma, \Gamma, \Delta, \pi)$ be a DS. Extend $\Delta$ to $\Delta' = \Delta \cup
\left\lbrace\begin{tabular}{|C{0.65cm}|}
\hline
 s \tabularnewline
\hline
 r \tabularnewline
\hline
\end{tabular}\right.$, \begin{tabular}{|C{0.65cm}|} 
\hline
 s \tabularnewline
\hline
 f \tabularnewline
\hline
\end{tabular}, \begin{tabular}{|C{0.65cm}|} 
\hline
 g \tabularnewline
\hline
 r \tabularnewline
\hline
\end{tabular}, \begin{tabular}{|C{0.65cm}|C{0.65cm}|} 
\hline 
q & o \tabularnewline
\hline 
\end{tabular}, \begin{tabular}{|C{0.65cm}|C{0.65cm}|} 
\hline 
q & d \tabularnewline
\hline 
\end{tabular}, \begin{tabular}{|C{0.65cm}|C{0.65cm}|} 
\hline 
e & o \tabularnewline
\hline 
\end{tabular} $\mid \left. \begin{tabular}{|C{0.65cm}|} 
\hline
 g \tabularnewline
\hline
 f \tabularnewline
\hline
\end{tabular}, \begin{tabular}{|C{0.65cm}|C{0.65cm}|} 
\hline 
e & d \tabularnewline
\hline 
\end{tabular} \in \Delta, s = \pi(g), r = \pi(f), q = \pi(e), o = \pi(d)\right\rbrace$
by also allowing the image symbols in the dominos.

For two intermediate configuations $p, p' \in (\Sigma \cup \Gamma)^{m, n},$ $n, m \geq 1$, we allow
a \emph{replacement step} $\hat{p} \underset{\tau}{\Rightarrow} \hat{p}'$ if for all positions
$(i,j)$ of $p'$, we have either $p'(i, j) = p(i, j)$ or $\pi(p'(i, j)) = p(i, j)$
and
\begin{center}
\begin{tabular}{|C{2.2cm}|} 
\hline
 $p(i - 1, j)$ \tabularnewline
\hline
 $p'(i, j)$  \tabularnewline
\hline
\end{tabular}, \begin{tabular}{|C{2.2cm}|} 
\hline
 $p'(i, j)$  \tabularnewline
\hline
 $p(i + 1, j)$ \tabularnewline
\hline
\end{tabular}, \begin{tabular}{|C{2.2cm}|C{2.2cm}|} 
\hline 
 $p(i, j - 1)$ & \hspace{0.3cm}$p'(i, j)$\hspace{0.3cm} \tabularnewline
\hline 
\end{tabular}, \begin{tabular}{|C{2.2cm}|C{2.2cm}|} 
\hline  
 \hspace{0.3cm}$p'(i, j)$\hspace{0.3cm} & $p(i, j + 1)$  \tabularnewline
\hline 
\end{tabular} 
$\in \Delta'$
\end{center} 
and the choice $p'(i,j)$ was 'forced'. 

Forced means that, for each $g\in\Gamma$ with $\pi(g) =
p(i,j)$ and $g \neq p'(i,j)$, the replacement of $p(i, j)$ in $p$ by $g$ would result in at least
one of these tiles to not be in $\Theta'$.
\label{deterministic_process_definition}
\end{definition}
The accepted language for a given DS $\tau$ is $L_d(\tau) := \{p \in \Sigma^{*,*} \mid
\hat{p} \overset{*}{\underset{\tau}{\Rightarrow}} \hat{p}' \in (\Gamma \cup
\{\#\})^{*,*}\}$. A picture language $L \subseteq \Sigma^{*,*}$ is called
\emph{deterministically recognizable} if there is a DS $\tau$ with $L = L_d(\tau)$.
Remark that $L_d(\tau) \subseteq L(\tau)$.

The language of pictures over $\{a, b\}$, where all occurring $b$'s
are connected, is one example picture language that is deterministically
recognizable. For the proof of this statement see~\cite{reinhardt1998recognizable}.
\label{deterministicprocess}