%%
%% Author: DKron
%% 24.07.2018
%%

\chapter{Upper Bounds}\label{ch:upperBounds}

In this chapter we regard the upper bounds of the expressive power of PHFL$^k$ and PHFL$^k_{tail}$. First, we have to show that the upper
bound of the expressive power of PHFL$^k$ is \exptime{$k$}.

To show that the upper bound of the expressive power of PHFL$^k$ is \exptime{$k$} we reduce the model checking
problem of PHFL$^k$ that to HFL$^k$. Remember that HFL$^k$ is the set of $1$-adic PHFL$^k$
formulas. In combination with Theorem~\ref{theorem:hfl_k_in_k_exptime} we get the upper bound of the expressive power of PHFL$^k$.

\begin{theorem}{\cite{axelsson2007complexity}}
    \label{theorem:hfl_k_in_k_exptime}
	The model checking problem of HFL$^k$ is solvable in $k$-EXPTIME, i.e. on input $\mathcal{T}, s$ and $\Phi$ it is decidable in $k$-EXPTIME whether $\mathcal{T}, s$ models $\Phi$, where $\mathcal{T}$ is an LTS, $s$ is a state and $\Phi$ in HFL$^k$.
\end{theorem}

In the following, we have to reduce the semantics of PHFL$^k$ to that of HFL$^k$.
Thus, we have to convert the input LTS $\mathcal{T}$ and the input PHFL$^k$ formula $\Phi$ of the problem of deciding whether
$\mathcal{T}$ with a state tuple $\emph{s}$ is a model of $\Phi$. We first define a mapping that transforms an LTS into the $d$-fold product of itself, where $d \in \mathbb{N}$
 and give an example for this transformation process. In the next step we define a function that maps a
PHFL$^k$ type to a HFL$^k$ type. We then continue to define one additional function that maps a PHFL$^k$ formulas to
HFL$^k$ formulas and give an example for such a mapping. Finally, we show that the semantics of the original
formula with the original types and the original LTS in the PHFL$^k$ context coincide with the semantics of the
converted formula with the converted types and the converted LTS in the HFL$^k$ context\footnote{Because the new LTS will have as underlying set $d$-tuples of states of the original LTS, the semantics of both logics represents the same mathematically object.}. 

As mentioned above we do start by defining a mapping that transforms a given LTS.

\begin{definition}
    \label{definition:lts_transformation}
    Let $d \in \mathbb{N}$ and $\mathcal{T} = (Q, \Sigma, P, \Delta, v)$ an LTS, then
    $\mathcal{T}_d = (Q^d, \Sigma_d \cup S_d, P_d, \Delta_d, v_d)$, where
    \begin{compactitem}
        \item $\Sigma_d = \underset{a \in \Sigma}{\bigcup}(\underset{i = 1}{\overset{d}{\bigcup}} \{a_i\})$,
        \item $S_d = \{s_{(e(1), \dots, e(d))} \mid e: \{1, \dots d\} \rightarrow \{1, \dots, d\}\}$,
        \item $P_d = \underset{p \in P}{\bigcup}(\underset{i = 1}{\overset{d}{\bigcup}} \{p_i\})$,
        \item $\Delta_d = \{((q_1, \dots ,q_{i - 1}, q_i, q_{i + 1}, \dots, q_d), a_i, (q_1, \dots ,q_{i - 1},
        {q_i}', q_{i + 1}, \dots, q_d)) \mid (q_i, a, {q_i}') \in \Delta\}$

        $\cup\,\{((q_{e(1)}, \dots, q_{e(d)}), s_{(e(1),
        \dots, e(d))}, (q_1, \dots, q_d)) \mid e: \{1, \dots, d\} \rightarrow \{1, \dots, d\}\}$ and
        \item $v_d \colon Q^d \rightarrow 2^{P_d}, $
        $v_d((q_1, \dots, q_d)) = \underset{i = 1}{\overset{d}{\bigcup}} \{p_i \mid p \in v(q_i)\}$.
    \end{compactitem}
\end{definition}

The following example shows the construction of an LTS as described in Definition~\ref{definition:lts_transformation}.

\begin{example}
    Let $\mathcal{T}$ be an LTS given by
    \begin{center}
        \begin{tikzpicture}[]
            \node [place] (q1) {$1$};
            \node  (temp1) [left=of q1] {};
            \node  (label) [left=of q1] {$\mathcal{T}:$};
            \node  (temp2) [right=of q1] {};
            \node [place] (q2) [below=of temp2,label=left:$p$] {$2$}
            edge [pre] node[left] {a} (q1);
            \node [place] (q3) [right=of temp2,label=right:$q$] {$3$}
            edge [pre] node[auto, swap] {b} (q1)
            edge [pre] node[right] {c} (q2);
        \end{tikzpicture}
    \end{center}
    Let $d = 2$ be a natural number then $\mathcal{T}_d$ is the
    following LTS. Note, that for readability reasons not all edges are drawn in this representation of $\mathcal{T}_d$.
    The edges that are missing are those that uses action $s_{(1, 2)}$ (except $(1, 1)$ to $(1,
    1)$) and the following edges are also not drawn
    $\{(1, 2) \overset{s_{(1, 1)}}{\rightarrow} (1, 1),
    (1, 3) \overset{s_{(1, 1)}}{\rightarrow} (1, 1),
    (1, 3) \overset{s_{(2, 1)}}{\rightarrow} (3, 1),
    (2, 1) \overset{s_{(2, 2)}}{\rightarrow} (1, 1),
    (2, 2) \overset{S_d}{\rightarrow} (2, 2),
    (2, 3) \overset{s_{(2, 1)}}{\rightarrow} (3, 2),
    (3, 1) $ $\overset{s_{(2, 1)}}{\rightarrow} (1, 3),
    (3, 1) \overset{s_{(2, 2)}}{\rightarrow} (1, 1),
    (3, 2) \overset{s_{(2, 1)}}{\rightarrow} (2, 3),
    (3, 3) \overset{S_d}{\rightarrow} (3, 3),
    (1, 2) \overset{b_1}{\rightarrow} (3, 2),
    $ $(1, 3) \overset{b_1}{\rightarrow} (3, 3),
    (2, 1) \overset{b_2}{\rightarrow} (2, 3),
    (3, 1) \overset{b_2}{\rightarrow} (3, 3)
    \}$, where $q \overset{S_d}{\rightarrow} q' =
    \{q \overset{a}{\rightarrow} q' \mid a \in S_d\}$.
    \begin{center}
        \begin{tikzpicture}[]
            \node [place] (q11) {$(1, 1)$}
            edge [loop] node[above] {$S_d$} ();
            \node (label) [left=of q11] {$\mathcal{T}_d:$};
            \node (temp1) [right=of q11] {};
            \node [place] (q12) [right=of temp1,label=above:{$p_2$}] {$(1, 2)$}
            edge [pre] node[above] {$a_2$} (q11);
            \node  (temp2) [right=of q12] {};
            \node [place] (q13) [right=of temp2, label=above:{$q_2$}] {$(1, 3)$}
            edge [pre] node[above] {$c_2$} (q12)
            edge [pre, bend right=40] node[above] {$b_2$} (q11);
            \node [place] (q21) [below=of q11,label=left:{$p_1$}] {$(2, 1)$}
            edge [post, bend right=10] node[auto, swap] {$s_{(2, 1)}$} (q12)
            edge [pre, bend left=10] node[auto] {$s_{(2, 1)}$} (q12)
            edge [pre] node[left] {$a_1$} (q11);
            \node [place] (q22) [below=of q12,label=right:{$p_1, p_2$}] {$(2, 2)$}
            edge [pre] node[below] {$a_2, s_{(1, 1)}$} (q21)
            edge [pre, bend left=25] node[right] {$a_1, s_{(2, 2)}$} (q12);
            \node [place] (q23) [below=of q13,label=right:{$p_1, q_2$}] {$(2, 3)$}
            edge [pre, bend left=30] node[below] {$c_2$} (q22)
            edge [post, bend right=30] node[above] {$s_{(1, 1)}$} (q22)
            edge [pre] node[right] {$a_1$} (q13);
            \node [place] (q31) [below=of q21,label=below:{$q_1$}] {$(3, 1)$}
            edge [pre, bend left=80] node[left] {$b_1$} (q11)
            edge [pre] node[left] {$c_1$} (q21);
            \node [place] (q32) [below=of q22,label=below:{$q_1, p_2$}] {$(3, 2)$}
            edge [pre] node[above] {$a_2$} (q31)
            edge [pre, bend left=25] node[left] {$c_1$} (q22)
            edge [post, bend right=25] node[right] {$s_{(2, 2)}$} (q22);
            \node [place] (q33) [below=of q23,label=below:{$q_1, q_2$}] {$(3, 3)$}
            edge [pre] node[below] {$c_2, s_{(1, 1)}$} (q32)
            edge [pre] node[right] {$c_1, s_{(2, 2)}$} (q23)
            edge [pre, bend right=100] node[right] {$s_{(2, 2)}$} (q13)
            edge [pre, bend left=40] node[right, below] {$s_{(1, 1)}$} (q31);
        \end{tikzpicture}
    \end{center}
\end{example}

The next step is to define a function that maps a PHFL type to a HFL type.

\begin{definition}
    \label{definition:type_function}
    Let $\tau_{PHFL}$, $\sigma_{PHFL}$ be arbitrary PHFL types, $\bullet_{PHFL}$ the base type of PHFL, $\bullet_{HFL}$
    the base type of HFL and $v$ an arbitrary variance, then $T$ is a function that maps a PHFL type to a HFL type
    defined inductive over the type of PHFL as follows:
    \begin{align*}
        T(\bullet_{PHFL}) &= \bullet_{HFL},\\
        T(\sigma_{PHFL}^v \rightarrow \tau_{PHFL}) &= T(\sigma_{PHFL})^v \rightarrow T(\tau_{PHFL})
    \end{align*}
\end{definition}

The type function $T$ of Definition~\ref{definition:type_function} can be adapted on type environments. If
$\Gamma = X_1^{v_1} \colon \tau_1, \dots, X_n^{v_n} \colon \tau_n$ is a type environment, then $T(\Gamma) =
X_1^{v_1} \colon T(\tau_1), \dots, X_n^{v_n} \colon T(\tau_n)$.

\begin{definition}
    \label{definition:variable_mapping_function}
    Let $\mathcal{T}$ be an LTS, $d \in \mathbb{N}$ with $d >0$ and $\mathcal{T}_d$ the transformed LTS by
    Definition~\ref{definition:lts_transformation} and $T$ the type mapping of
    Definition~\ref{definition:type_function}. Furthermore, let $\eta$ be a variable mapping over $\mathcal{T}$ for PHFL
    formulas, then $V(\eta)$ is a variable mapping over $\mathcal{T}_d$ for HFL formulas, where $V(\eta)(X) = \eta(X) = \mathcal{X}$ with $\mathcal{X} \in \llbracket \tau \rrbracket_\mathcal{T}$. Note, that for $\mathcal{X}$ it holds that $\mathcal{X} \in \llbracket \tau
    \rrbracket_\mathcal{T}$ iff $\mathcal{X} \in \llbracket T(\tau)\rrbracket_{\mathcal{T}_d}$ due to the definition
    of $\mathcal{T}_d$ and $T(\tau)$.
\end{definition}

We continue with the definition of the function that maps a PHFL$^k$ formula to a HFL$^k$ formula.

\begin{definition}
    \label{definition:formula_function}
    Let $T$ be the type function from Definition~\ref{definition:type_function} and let $P$ be a set of propositions,
    $\Sigma$ a set of actions and $\mathcal{V} = \{X_1, \dots, X_n\}$ a finite set of variables for a $d$-adic
    PHFL$^k$ formula $\Phi$, then $F$ is a function that maps a $d$-adic PHFL$^k$ formula over $P$, $\Sigma$ and
    $\mathcal{V}$ to a HFL$^k$ formula over proposition set $P_d = \underset{p \in P}{\bigcup}(\underset{i =
    1}{\overset{d}{\bigcup}} \{p_i\})$, action set $\Sigma_d \cup S_d = \underset{a \in \Sigma}{\bigcup}(\underset{i =
    1}{\overset{d}{\bigcup}} \{a_i\}) \cup \{s_{(e(1), \dots, e(d))} \mid e: \{1, \dots d\} \rightarrow \{1, \dots,
    d\}\}$ and variable set $\mathcal{V}$ which is defined inductive over the $d$-adic PHFL$^k$ formula as follows:
    \begin{align*}
        F(\top) &= \top\\
        F(X) &= X\\
        F(p_i) &= p_i\\
        F(\langle a \rangle_i \psi) &= \langle a_i \rangle F(\psi) \\
        F(\psi \vee \psi') &= F(\psi) \vee F(\psi') \\
        F(\neg \psi) &= \neg F(\psi) \\
        F(\{\emph{j}\,\} \psi) &= \langle s_{\emph{j}} \rangle F(\psi)  \\
        F(\mu (X \colon \tau).\,\psi) &= \mu (X \colon T(\tau)).\,F(\psi) \\
        F(\lambda (X^v \colon \tau).\, \psi) &= \lambda (X^v \colon T(\tau)).\, F(\psi) \\
        F(\psi\,{\psi'}) &= F(\psi)\,F({\psi'})
    \end{align*}
\end{definition}

\begin{example}
    We transform the $2$-adic PHFL$^2$ formula of Example~\ref{example:phfl_order_2} with the function of Definition~\ref{definition:formula_function} to a HFL$^2$ formula.
    \begin{align*}
        \Phi = &(\mu (F \colon \bullet_{PHFL}^0 \rightarrow (\bullet_{PHFL}^0 \rightarrow \bullet_{PHFL})).\,
        \lambda (X \colon \bullet_{PHFL}).\, \lambda (Y \colon \bullet_{PHFL}).\, \\&X \Leftrightarrow Y \wedge
        \underset{a \in \Sigma}{\bigwedge} F \langle a \rangle_1 X \langle a \rangle_2 Y)\top \top
    \end{align*}
    will be transformed to
\begin{align*}
    F(\Phi) = &(\mu (F \colon \bullet_{HFL}^0 \rightarrow (\bullet_{HFL}^0 \rightarrow \bullet_{HFL})).\,
    \lambda (X \colon \bullet_{HFL}).\, \lambda (Y \colon \bullet_{HFL}).\, \\&X \Leftrightarrow Y \wedge \underset{a
    \in \Sigma}{\bigwedge} F \langle a_1 \rangle X \langle a_2 \rangle Y)\top \top.
\end{align*}
\end{example}

\begin{remark}
    It holds for type environment $\Gamma$ and PHFL$^k$ formula $\Phi$ of PHFL$^k$ type $\tau$ that if $\Gamma \vdash
    \Phi \colon \tau$ is derivable, then $T(\Gamma) \vdash F(\Phi) \colon T(\tau)$ is also derivable. This statement
    can easily be proven by induction over the structure of formula $\Phi$.
\end{remark}

Because the type judgement is always derivable we ignore the type environment in the following proof and write just $\llbracket \Phi \rrbracket^\eta_\mathcal{T}$ instead of $\llbracket \Gamma \vdash \Phi \colon \tau \rrbracket^\eta_\mathcal{T}$, where $\Phi$ is a PHFL formula, $\eta$ is a variable mapping, $\mathcal{T}$ is an LTS, $\Gamma$ is a type environment and $\tau$ is a PHFL type. 

In the final step we show that the semantics of a given PHFL$^k$ formula coincide with the transformed HFL$^k$ formula.

\begin{lemma}
    \label{lemma:model_check_phfl_k}
    Let $\mathcal{T}$ be an LTS, $\eta$ a variable mapping, $\Phi$ a well-typed $d$-adic
    PHFL$^k$ formula, $\mathcal{T}_d$ the LTS transformed by the process of
    Defintion~\ref{definition:lts_transformation}, $T$ the type function of Defintion~\ref{definition:type_function},
    $V$ the variable mapping function of Definition~\ref{definition:variable_mapping_function}
    and $F$ the formula function of Definition~\ref{definition:formula_function} then $\llbracket \Phi \rrbracket^\eta_\mathcal{T} = \llbracket F(\Phi)
    \rrbracket^{V(\eta)}_{\mathcal{T}_d}$
\end{lemma}
Although it looks unexpected this is exactly the statement that we have to prove. The semantics of a PHFL formula is a set of tuples but in the HFL formula this is just a set. Because this set is the set of states of $\mathcal{T}_d$ and this is a set of tuples, the both semantics operates on the same kind of sets.
\begin{proof}
    We show that $\llbracket \Phi \rrbracket^\eta_\mathcal{T} = \llbracket F(\Phi) \rrbracket^{V(\eta)}_{\mathcal{T}_d}$ by induction on formula $\Phi$.
    \begin{compactitem}
        \item In case of $\Phi = \top$, $\llbracket \Phi
        \rrbracket^\eta_\mathcal{T}$ is the set of $d$-tuples of state set $Q$ of $\mathcal{T}$ what means that
        \[\llbracket \Phi \rrbracket^\eta_\mathcal{T} = Q^d.\]
        By construction of $\mathcal{T}_d$ the set of states is also the set of $d$-tuples of state set $Q$.
        Moreover, due to formula function $F$ it is true that $F(\top) = \top$. It holds that
        \[\llbracket F(\Phi) \rrbracket^{V(\eta)}_{\mathcal{T}_d} = Q^d\]
        because the set of states of $\mathcal{T}_d$ is $Q^d$.

        \item In case of $\Phi = p_i$,
        \[\llbracket \Phi
        \rrbracket^\eta_\mathcal{T} = \{(q_1, \dots, q_d)\in Q^d \mid p \in v(q_i)\}.\]
        By construction of $\mathcal{T}_d$ it holds that $v_d((q_1, \dots, q_d)) = \overset{d}{\underset{i = 1}{\bigcup}}\{p_i
        \mid p \in v(q_i)\}$ and by formula function $F$ it is true that $F(p_i) = p_i$. Because $p_i \in v_d((q_1,
        \dots, q_d))$ iff $p \in v(q_i)$ it holds that
        \[\llbracket F(\Phi) \rrbracket^{V(\eta)}_{\mathcal{T}_d} = \{
        (q_1, \dots, q_d) \in Q^d \mid p_i \in v_d((q_1, \dots, q_d))\}\]
        and furthermore that
        \[\{(q_1, \dots, q_d)\in Q^d \mid p \in v(q_i)\} =\{(q_1, \dots, q_d) \in Q^d \mid p_i \in
        v_d((q_1, \dots, q_d))\}.\]

        \item The last base case is $\Phi = X$. It holds that
        \[\llbracket \Phi \rrbracket^\eta_\mathcal{T} = \eta(X).\]
        Moreover, it is true that
        \[\llbracket F(\Phi) \rrbracket^{V(\eta)}_{\mathcal{T}_d} = V(\eta)(X).\]
        By definition of the function $F$ it is also true that $F(X) = X$. The combination of construction $\mathcal{T}_d$, type
        function $T$ and by construction of variable mapping $V(\eta)$ in
        Definition~\ref{definition:variable_mapping_function} it follows that $V(\eta)(X) = \eta(X)$. 
    \end{compactitem}
    The induction hypothesis is that it holds for all subformulas $\psi$ and $\psi'$ of $\Phi$ that $\llbracket
    \psi \rrbracket^\eta_\mathcal{T} = \llbracket F(\psi)
    \rrbracket^{V(\eta)}_{\mathcal{T}_d}$ and $\llbracket
    \psi' \rrbracket^\eta_\mathcal{T} = \llbracket F(\psi') 
    \rrbracket^{V(\eta)}_{\mathcal{T}_d}$
    \begin{compactitem}
        \item In case of $\Phi = \langle a \rangle_i \psi$,
        \begin{align*}
            \llbracket \Phi \rrbracket^\eta_\mathcal{T} = &\,
            \{(q_1, \dots, q_d) \in Q^d \mid \\&\text{ it exists } ({q_1}', \dots, {q_d}') \in \llbracket \Gamma
                \vdash \psi 
                \rrbracket^\eta_\mathcal{T} \text{ such that }\\&\,q_i \overset{a}{\rightarrow} {q_i}' \text{ and for
            all
            } i \neq j
            \text{ it holds } q_j =
                {q_j}'\}.
        \end{align*}
        By the induction hypothesis $({q_1}', \dots, {q_d}') \in \llbracket F(\psi) \rrbracket^{V(\eta)}_{\mathcal{T}_d}$ because $({q_1}', \dots,$ $ {q_d}') \in \llbracket \psi \rrbracket^\eta_\mathcal{T}$. Based on construction of $\mathcal{T}_d$
        and $q_i \overset{a}{\rightarrow} {q_i}' \in \Delta$ it follows that $({q_1}', \dots, {q_{i - 1}'}, q_i,
        {q_{i + 1}'}, \dots, {q_d}') \overset{a_i}{\rightarrow} ({q_1}', \dots, {q_{i - 1}'}, {q_i}',
        {q_{i + 1}'}, \dots, {q_d}') \in \Delta_d$. That means that
        \begin{align*}
            \{(q_1, \dots, q_d) \in Q^d \mid &\text{ it exists } ({q_1}', \dots, {q_d}') \in \llbracket \psi 
            \rrbracket^\eta_\mathcal{T} \text{ such that }\\&\,q_i \overset{a}{\rightarrow} {q_i}' \text{ and for
            all } i \neq j \text{ it holds } q_j = {q_j}'\}
        \end{align*}
        is equal to
        \begin{align*}
            \{(q_1, \dots, q_d) \in Q^d \mid &\text{ it exists } ({q_1}', \dots, {q_d}') \in \llbracket F(\psi) \rrbracket^{V(\eta)}_{\mathcal{T}_d} \text{ such that }\\&\,q_i
            \overset{a_i}{\rightarrow} {q_i}' \text{ and for all } i \neq j \text{ it holds } q_j = {q_j}'\}.
        \end{align*}
        As a result of of $T$, $V$ and that $F(\langle a \rangle_i \psi) = \langle a_i \rangle \psi$ holds the second set is
        exactly the definition of the semantics of
        \[\llbracket F(\Phi) \rrbracket^{V(\eta)}_{\mathcal{T}_d}.\]

        \item In case of $\Phi = \psi \vee {\psi'}$, 
        \[\llbracket \Phi \rrbracket^\eta_\mathcal{T} = \llbracket \psi
        \rrbracket^\eta_\mathcal{T} \sqcup_\bullet \llbracket \psi'
        \rrbracket^\eta_\mathcal{T}.\]
        By the induction hypothesis it holds that \[\llbracket F(\psi) \rrbracket^{V(\eta)}_{\mathcal{T}_d} = \llbracket \psi \rrbracket^\eta_\mathcal{T}\]
        and
        \[\llbracket F(\psi') \rrbracket^{V(\eta)}_{\mathcal{T}_d} = \llbracket \psi' \rrbracket^\eta_\mathcal{T}.\]
        By construction of $T$ it is true that
        \[\llbracket F(\psi) \rrbracket^{V(\eta)}_{\mathcal{T}_d} \sqcup_{T(\bullet)} \llbracket F(\psi') \rrbracket^{V(\eta)}_{\mathcal{T}_d}\]
        is equal to
        \[\llbracket \psi
        \colon \tau \rrbracket^\eta_\mathcal{T} \sqcup_\bullet \llbracket \psi' 
        \rrbracket^\eta_\mathcal{T}.\]
        Because $F(\psi \vee \psi') = F(\psi) \vee F(\psi')$ one can see that
        \[\llbracket F(\Phi) \rrbracket^{V(\eta)}_{\mathcal{T}_d}\]
        is equal to
        \[\llbracket F(\psi) \rrbracket^{V(\eta)}_{\mathcal{T}_d} \sqcup_{T(\bullet)}
        \llbracket F(\psi') \rrbracket^{V(\eta)}_{\mathcal{T}_d}.\]

        \item In case of $\Phi = \neg \psi$,
        \[\llbracket \Phi \rrbracket^\eta_\mathcal{T} = Q^d \setminus
        \llbracket \psi \rrbracket^\eta_\mathcal{T}.\]
        By the induction hypothesis it holds that \[\llbracket F(\psi) \rrbracket^{V(\eta)}_{\mathcal{T}_d} = \llbracket \psi \rrbracket^\eta_\mathcal{T}.\]
        Due to the equality of the two sets it follows that
        \[Q^d \setminus \llbracket F(\psi) \rrbracket^{V(\eta)
        }_{\mathcal{T}_d} =
        Q^d \setminus \llbracket \psi \rrbracket^\eta_\mathcal{T}.\]
        Since $F(\neg \psi) = \neg F(\psi)$ the first set is exactly the semantics of
        \[\llbracket F(\Phi) \rrbracket^{V(\eta)}_{\mathcal{T}_d}.\]

        \item In case of $\Phi = \{(e(1), \dots, e(d))\}\,\psi$, 
        \begin{align*}
            \llbracket \Phi \rrbracket^\eta_\mathcal{T} = \{&(q_1, \dots,
            q_d) \in Q^d \mid \\&(q_{e(1)}, \dots, q_{e(d)}) \in \llbracket \psi
            \rrbracket ^\eta_\mathcal{T}\}.
        \end{align*}
        By the induction hypothesis $({q_{e(1)}}, \dots, {q_{e(d)}}) \in \llbracket F(\psi) \rrbracket^{V(\eta)}_{\mathcal{T}_d}$ because $({q_{e(1)}},$ $ \dots, {q_{e(d)}}) \in
        \llbracket \psi \rrbracket^\eta_\mathcal{T}$.
        Moreover, the state tuple $({q_{e(1)}}, \dots, {q_{e(d)}})$ that fulfills $\psi$ is reached by 'moving' from
        state tuple $({q_{1}}, \dots, {q_{d}})$ to $({q_{e(1)}}, \dots, {q_{e(d)}})$. This movement is
        integrated in the construction of $\mathcal{T}_d$. There exists for each endomorphism $e$ a
        substitution action $s_{(e(1), \dots, e(d))}$ for each tuple state. With $F(\{(e(1), \dots, e(d))\}
        \psi) = \langle s_{(e(1), \dots, e(d))} \rangle F(\psi)$ we receive the state tuples that have the action $(q_1,
        \dots, q_d)$ $ \overset{s_{(e(1), \dots, e(d))}}{\rightarrow} (q_{e
        (1)}, \dots, q_{e(d)})$ where $(q_1, \dots, q_d) \in \llbracket F(\psi) \rrbracket^{V(\eta)}_{\mathcal{T}_d}$.
        It follows that
        \[\llbracket \Phi \rrbracket^\eta_\mathcal{T} = \llbracket F(\Phi) \rrbracket^{V(\eta)}_{\mathcal{T}_d}.\]

        \item In case of $\Phi = \mu(X \colon \tau).\,\psi$,
        \begin{align*}
            \llbracket \Phi \rrbracket^\eta_\mathcal{T} =\bigsqcap\,
            _{\llbracket\tau\rrbracket_\mathcal{T}} \{\mathcal{X} \in \llbracket \tau \rrbracket_\mathcal{T}
            \mid 
            \llbracket \psi \rrbracket^{\eta[X \mapsto
            \mathcal{X}]}_\mathcal{T}
            \leq_{\llbracket \tau \rrbracket_\mathcal{T}} \mathcal{X}\}.
        \end{align*}
        By the induction hypothesis it applies that 
        \[\llbracket F(\psi) \rrbracket^{V(\eta)[X \mapsto \mathcal{X}]}_{\mathcal{T}_d} = \llbracket 
        \psi \rrbracket^{\eta[X \mapsto \mathcal{X}]}_\mathcal{T}.\]
        Moreover, $F(\mu(X \colon \tau).\psi) = \mu(X \colon T(\tau)).F(\psi)$. In particular, it holds that
        \begin{align*}
            \llbracket \Phi \rrbracket^\eta_\mathcal{T} =&\,\bigsqcap\,
            _{\llbracket\tau\rrbracket_{\mathcal{T}_d}} \{\mathcal{X} \in \llbracket \tau \rrbracket_{\mathcal{T}_d}
            \mid \\
            &\llbracket F(\psi) \rrbracket^{V(\eta)[X \mapsto
            \mathcal{X}]}_{\mathcal{T}_d}
            \leq_{\llbracket \tau \rrbracket_{\mathcal{T}_d}} \mathcal{X}\}.
        \end{align*}
        This is exactly the semantics of
        \[\llbracket F(\Phi) \rrbracket^{V(\eta)}_{\mathcal{T}_d}.\]

        \item In case of $\Phi = \lambda(X^v \colon \sigma).\,\psi$,
        \begin{align*}
            \llbracket \Phi \colon \sigma^v \rightarrow \tau \rrbracket^\eta_\mathcal{T} =&\,F \in
            \llbracket \sigma^v \rightarrow \tau \rrbracket_\mathcal{T}
        \end{align*}
        such that for all $\mathcal{X} \in \llbracket \sigma \rrbracket_\mathcal{T}$ it holds that $F(\mathcal{X}) =
        \llbracket \psi \colon \tau \rrbracket^{\eta[X \mapsto
        \mathcal{X}]}_\mathcal{T}$.
        By the induction hypothesis it is true that 
        \[\llbracket F(\psi) \colon T
        (\tau) \rrbracket^{V(\eta)[X \mapsto \mathcal{X}]}_{\mathcal{T}_d} = \llbracket \psi \colon \tau \rrbracket^{\eta[X \mapsto \mathcal{X}]}_\mathcal{T}.\]
        Moreover, $F(\lambda(X^v \colon \sigma).\psi) = \lambda(X^v \colon T(\sigma)).F(\psi)$. Consequently it holds that
        \begin{align*}
            \llbracket \Phi \colon \sigma^v \rightarrow \tau \rrbracket^\eta_\mathcal{T} =&\,F \in
            \llbracket T(\sigma^v \rightarrow \tau) \rrbracket_{\mathcal{T}_d}
        \end{align*}
        such that for all $\mathcal{X} \in \llbracket T(\sigma) \rrbracket_{\mathcal{T}_d}$ the following is true \[F(\mathcal{X}) =
        \llbracket F(\psi) \colon T(\tau) \rrbracket^{V(\eta)[X \mapsto
        \mathcal{X}]}_{\mathcal{T}_d}.\]
        This is exactly the semantics of
        \[\llbracket F(\Phi) \colon T(\sigma^v \rightarrow \tau) \rrbracket^{V(\eta)
        }_{\mathcal{T}_d}.\]

        \item In case of $\Phi = \psi\,\psi'$,
        \[\llbracket \Phi \colon \tau \rrbracket^\eta_\mathcal{T} = \llbracket \psi
        \colon \sigma^v \rightarrow \tau \rrbracket^\eta_\mathcal{T}(\llbracket \psi' \colon
        \sigma \rrbracket^\eta_\mathcal{T}).\]
        By the induction hypothesis it applies that \[\llbracket F(\psi) \colon T
        (\sigma^v \rightarrow \tau) \rrbracket^{V(\eta)}_{\mathcal{T}_d} = \llbracket \Gamma
        \vdash \psi \colon \sigma^v \rightarrow \tau \rrbracket^\eta_\mathcal{T}\]
        and
        \[\llbracket F(\psi') \colon T
        (\sigma) \rrbracket^{V(\eta)}_{\mathcal{T}_d} = \llbracket \Gamma
        \vdash \psi' \colon \sigma \rrbracket^\eta_\mathcal{T}.\]
        It follows that
        \[\llbracket F(\psi) \colon T
        (\sigma^v \rightarrow \tau) \rrbracket^{V(\eta)}_{\mathcal{T}_d}(\llbracket F(\psi') \colon T
        (\sigma) \rrbracket^{V(\eta)}_{\mathcal{T}_d}\]
        is equal to
        \[\llbracket \psi
        \colon \sigma^v \rightarrow \tau \rrbracket^\eta_\mathcal{T}(\llbracket \psi' \colon \sigma
        \rrbracket^\eta_\mathcal{T}).\]
        Because $F(\psi\,\psi') = F(\psi)F(\psi')$ the first set is equal to the semantics of
        \[\llbracket F(\Phi) \colon T(\tau) \rrbracket^{V(\eta)}_{\mathcal{T}_d}.\]
    \end{compactitem}
    This shows the correctness of the construction. 
\end{proof}

To make sure that this construction does not exceed the bounds of HFL$^k$ we have to verify, that the growth of the defined construction is at most polynomial. Thereafter it is proven that the semantics of PHFL$^k$ is
reducible to the semantics of HFL$^k$.

\begin{lemma}
\label{lemma:phfl_to_hfl_polynomial}
Let $\mathcal{T} = (Q, \Sigma, P, \Delta, v)$ be an LTS, $\eta$ a variable mapping, $\Gamma$ a type environment, $\Phi$ a well-typed $d$-adic
    PHFL$^k$ formula of type $\tau$, $\mathcal{T}_d$ the LTS transformed by the process of
    Definition~\ref{definition:lts_transformation}, $T$ the type function of Definition~\ref{definition:type_function},
    $V$ the variable mapping function of Definition~\ref{definition:variable_mapping_function}
    and $F$ the formula function of Definition~\ref{definition:formula_function} then 
    \begin{compactitem}
    \item $\mathcal{T}_d$ grows polynomial related to $\mathcal{T}$, 
    \item $T(\tau)$ grows polynomial related to $\tau$,
    \item $T(\Gamma)$ grows polynomial related to $\Gamma$,
    \item $V(\eta)$ grows polynomial related to $\eta$ and
    \item $F(\Phi)$ grows polynomial related to $\Phi$.
    \end{compactitem}
\end{lemma}

\begin{proof}
	$F(\Phi)$ and $T(\tau)$ grow obviously linear related to $\Phi$ and $\tau$ 
	respectively. It follows that also $T(\Gamma)$ grows linear related to $\Gamma$. To show 
	that $\mathcal{T}_d$ grows polynomial related to $\mathcal{T}$, we take a look at the particular 
	components of the LTS. For the set of states it holds that $|Q^d| = |Q|^d$. That means $Q^d$ 
	grows polynomial related to $Q$. The cardinality of the set of actions of $\mathcal{T}_d$ is 
	$|\Sigma_d| + |S_d|$. Since $|S_d|$ is a constant it holds also that $|\Sigma_d| = |\Sigma|^d$. 
	By combining these two statements, one can see that the set of actions also grows polynomial. The set of propositions $P^d$ also grows 
	polynomial, because it is constructed like the subset of actions $\Sigma_d$. The labelled 
	transition relation $\Delta_d$ also grows polynomial because it is constructed like the set of 
	actions of $\mathcal{T}_d$. The first set has cardinality $|\Delta|^d$ and the second set is a 
	constant. Again, by combining the previous statements it is clear that $\Delta_d$ grows polynomial related to $\Delta$. At least, the 
	mapping $v_d$ grows polynomial related to $v$ because the set of states and the set of 
	propositions of $\mathcal{T}$ grow polynomial. As a result it follows, that $
	\mathcal{T}_d$ grows polynomial related to $\mathcal{T}$. Because $\mathcal{T}_d$ grows 
	polynomial related to $\mathcal{T}$ and $T(\tau)$ grows polynomial related to $\tau$  
	$V(\eta)$ also grows polynomial related to $\eta$. 
\end{proof}

The following theorem is given by Lemma~\ref{lemma:model_check_phfl_k}, Lemma~\ref{lemma:phfl_to_hfl_polynomial} and Theorem~\ref{theorem:hfl_k_in_k_exptime}.

\begin{theorem}
    \label{theorem:phfl_k_in_k_exptime}
	The model checking problem of PHFL$^k$ for $k > 0$ is solvable in $k$-EXPTIME, i.e. on input $\mathcal{T}, s$ and $\Phi$ it is decidable in $k$-EXPTIME whether $\mathcal{T}, s$ models $\Phi$, where $\mathcal{T}$ is an LTS, $s$ is a state and $\Phi$ in PHFL$^k$.    
\end{theorem}

To show that the upper bound of the expressive power of PHFL$^{k + 1}_{tail}$ is \expspace{$k$} we can reduce the semantics of
PHFL$^{k}_{tail}$ to the semantics of HFL$^k_{tail}$. Keep in mind, that HFL$^k_{tail}$ is the set of tail-recursive
$1$-adic PHFL$^k$ formulas. In combination with the following theorem we get the upper bound
of the expressive power of PHFL$^k_{tail}$.

\begin{theorem}{\cite{bruse2017space}}
    \label{theorem:hfl_k_plus_1_in_k_expspace}
    Given an LTS $\mathcal{T}$, a state $s$ and an HFL$^{k + 1}_{tail}$ formula $\Phi$, the model checking problem i.e., deciding whether
    $\mathcal{T}, s \models \Phi$ is in $k$-EXPSPACE, where $k > 0$.
\end{theorem}

\begin{lemma}
    \label{lemma:model_check_phfl_k_tail}
    Let $\mathcal{T}$ be an LTS, $\eta$ a variable mapping, $\Phi$ a well-typed $d$-adic
    PHFL$^k_{tail}$ formula, $\mathcal{T}_d$ the LTS transformed by process of
    Defintion~\ref{definition:lts_transformation},
    $V$ the variable mapping function of Definition~\ref{definition:variable_mapping_function}
    and $F$ the formula function of Definition~\ref{definition:formula_function} then $\llbracket \Phi \rrbracket^\eta_\mathcal{T} = \llbracket F(\Phi)
    \rrbracket^{V(\eta)}_{\mathcal{T}_d}$.
\end{lemma}

\begin{proof}
    This proof is similar to the proof of Lemma~\ref{lemma:model_check_phfl_k}. Because the definition of $F$ obviously retains tail-recursiveness this lemma holds.
\end{proof}

The following theorem is given by Lemma~\ref{lemma:model_check_phfl_k_tail} and
Theorem~\ref{theorem:hfl_k_plus_1_in_k_expspace}.

\begin{theorem}
    \label{theorem:phfl_k_plus_1_tail_in_k_expspace}
The model checking problem of PHFL$^{k + 1}_{tail}$ for $k > 0$ is solvable in $k$-EXPSPACE, i.e. on input $\mathcal{T}, s$ and $\Phi$ it is decidable in $k$-EXPSPACE whether $\mathcal{T}, s$ models $\Phi$, where $\mathcal{T}$ is an LTS, $s$ is a state and $\Phi$ in PHFL$^{k + 1}_{tail}$.        
\end{theorem}