\section{Upper Bound of PHFL$^k$}
\label{sec:phfl_k_equals_k_exptime_upper_bounds}
To show that the upper bound of PHFL$^k$ is \exptime{$k$} we want to make a reduction from the model checking
problem of PHFL$^k$ to the model checking problem of HFL$^k$. Remember, that HFL$^k$ is set of $1$-adic PHFL$^k$
formulas. In combination to the following theorem we get the upper bound of PHFL$^k$.

\begin{theorem}{\cite{axelsson2007complexity}}
    \label{theorem:hfl_k_in_k_exptime}
    The model checking problem of HFL$^k$ formulas lies in $k$-EXPTIME, where $k > 0$.
\end{theorem}

\begin{lemma}
    \label{lemma:model_check_phfl_k}
    The model checking problem of PHFL$^k$ formulas is reducible to the model checking problem of HFL$^k$ formulas.
\end{lemma}

\begin{proof}
    To reduce the model checking problem of PHFL$^k$ to the model checking problem of HFL$^k$, we have to transform
    the input LTS $\mathcal{T}$ and the input formula $\varphi$ of the problem $\mathcal{T}, \emph{s}
    \overset{?}{\models} \varphi$.
    Let $\mathcal{T}_{PHFL} = (Q, \Sigma, P, \Delta_{PHFL}, \nu_{PHFL})$ a LTS and $\emph{s} = (s_1, \dots, s_d)$ a
    state tuple to check if it is a model of a $d$-adic PHFL$^k$ formula $\varphi_{PHFL}$, i.e. $\mathcal{T}_{PHFL},
    \emph{s} \overset{?}{\models} \varphi_{PHFL}$. Then $\mathcal{T}_{HFL} = (Q^d, \Sigma_1 \cup
    \dots \cup \Sigma_d \cup S_1 \cup \dots \cup S_d, P_1 \times \dots \times P_d, \Delta_{HFL}, \nu_{HFL})$, where
    \begin{compactitem}
        \item $\Sigma_i = \{a_i \mid a \in \Sigma\}$,
        \item $S_k = \{s_{(i_1, \dots, i_k) \leftarrow (j_1, \dots, j_k)} \mid i_1, \dots, i_k, j_1, \dots, j_k \in
        \{1, \dots, d\} \wedge i_1 \neq \dots \neq i_k \wedge j_1 \neq \dots \neq j_k\}$,
        \item $P_i = \{p_i \mid p \in P\}$,
        \item $i \in \{1, \dots, d\}$,
        \item $\Delta_{HFL} =$
        \begin{align*}
                  &\,\{((q_1, \dots, q_d), a_i, ({q_1}', \dots, {q_d}')) \mid (q, a, q') \in \Delta_{PHFL}
                  \wedge i \in \{1, \dots, d\} \\&\wedge a_i \in \Sigma_i \wedge q_i = q \wedge {q_i}' = q' \wedge
                  \forall j \neq i.\, q_j = {q_j}'\}  \\
                  \cup & \,\{((q_1, \dots, q_{i_1 - 1}, q_{i_1}, q_{i_1 + 1}, \dots, q_d), s_{(i_1) \leftarrow (j_1)
                  }, \\&
                  (q_1, \dots, q_{i_1 - 1}, q_{j_1}, q_{i_1 + 1}, \dots, q_d)) \mid \\&
                  i_1, j_1 \in \{1, \dots, d\} \wedge (q_{i_1}, a_1, q_{j_1}) \in \Delta_{PHFL} \wedge a_1 \in
                  \Sigma\} \\
                  \cup & \,\{((q_1, \dots, q_{i_1 - 1}, q_{i_1}, q_{i_1 + 1}, \dots, q_{i_2 - 1}, q_{i_2}, q_{i_2 +
                  1}, \dots, q_d), s_{(i_1, i_2) \leftarrow (j_1, j_2)}, \\&
                  (q_1, \dots, q_{i_1 - 1}, q_{j_1}, q_{i_1 + 1}, \dots, q_{i_2 - 1}, q_{j_2}, q_{i_2 + 1}, \dots, q_d))
                  \mid \\& i_1, i_2, j_1, j_2 \in \{1, \dots, d\} \wedge i_1 \neq i_2 \wedge j_1 \neq j_2 \,\wedge\\&
                   (q_{i_1}, a_1, q_{j_1}) \in \Delta_{PHFL} \wedge (q_{i_2}, a_2, q_{j_2}) \in \Delta_{PHFL}
                  \wedge a_1, a_2 \in \Sigma\} \\
            \cup &\\
            &\dots \\
                  \cup & \,\{((q_{i_1}, \dots, q_{i_d}), s_{(i_1, \dots, i_d) \leftarrow
                  (j_1, \dots, j_d)},
                  (q_{j_1}, \dots, q_{j_d})) \mid \\&
                  i_1, \dots, i_d, j_1, \dots, j_d \in \{1, \dots, d\} \wedge i_1 \neq \dots \neq i_d \wedge j_1
                  \neq \dots \neq j_d \,\wedge\\& (q_{i_1}, a_1,  q_{j_1}) \in
                  \Delta_{PHFL} \wedge \dots \wedge (q_{i_d}, a_d, q_{j_d}) \in \Delta_{PHFL}
                  \wedge a_1, \dots a_d \in \Sigma\}
        \end{align*} and
        \item $\nu_{HFL} \colon Q^d \rightarrow 2^{P_1, \dots, P_d}, $

        $\nu_{HFL}((q_1, \dots, q_d)) = \{p_1 \mid p \in \nu_{PHFL}(q_1)\} \cup \dots \cup \{p_d \mid p \in
        \nu_{PHFL}(q_d)\}$,
    \end{compactitem}
    is the transformed LTS to check if it is a model of the transformed HFL formula $\varphi_{HFL}$, i.e.
    $\mathcal{T}_{HFL}, \emph{s} \overset{?}{\models} \varphi_{HFL}$.

    The following example shows the transforming of a LTS for the model check of a $2$-adic PHFL$^k$ formula to the
    LTS for the model check of the transformed HFL$^k$ formula. Let $\mathcal{T}_{PHFL}$ the LTS for the model check
    $\mathcal{T}_{PHFL}, \emph{s} \overset{?}{\models} \varphi_{PHFL}$ and $\mathcal{T}_{HFL}$ the transformed LTS
    for the model check $\mathcal{T}_{HFL}, \emph{s}
    \overset{?}{\models} \varphi_{PHFL}$.
        \begin{center}
            \begin{tikzpicture}[]
                \node [place] (q1) {$1$};
                \node  (temp1) [left=of q1] {};
                \node  (label) [left=of q1] {$\mathcal{T}_{PHFL}:$};
                \node  (temp2) [right=of q1] {};
                \node [place] (q2) [below=of temp2,label=left:$p$] {$2$}
                edge [pre] node[left] {a} (q1);
                \node [place] (q3) [right=of temp2,label=right:$q$] {$3$}
                edge [pre] node[auto, swap] {b} (q1)
                edge [pre] node[right] {c} (q2);
            \end{tikzpicture}
        \end{center}
        \begin{center}
            \begin{tikzpicture}[]
                \node [place] (q11) {$(1, 1)$};
                \node (label) [left=of q11] {$\mathcal{T}_{HFL}:$};
                \node  (temp1) [right=of q11] {};
                \node [place] (q12) [right=of temp1,label=above:{$p_2$}] {$(1, 2)$}
                edge [pre] node[above] {$a_2$} (q11);
                \node  (temp2) [right=of q12] {};
                \node [place] (q13) [right=of temp2, label=above:{$q_2$}] {$(1, 3)$}
                edge [pre] node[above] {$c_2$} (q12)
                edge [pre, bend right=40] node[above] {$b_2$} (q11);
                \node [place] (q21) [below=of q11,label=left:{$p_1$}] {$(2, 1)$}
                edge [pre] node[left] {$a_1$} (q11);
                \node [place] (q22) [below=of q12,label=right:{$p_1, p_2$}] {$(2, 2)$}
                edge [pre, bend right] node[above] {$a_2, s_{(2)\leftarrow(1)}$} (q21)
                edge [pre] node[right] {$a_1, s_{(1)\leftarrow(2)}$} (q12);
                \node [place] (q23) [below=of q13,label=right:{$p_1, q_2$}] {$(2, 3)$}
                edge [pre, bend right=30] node[below] {$c_2$} (q22)
                edge [pre] node[right] {$a_1$} (q13)
                edge [pre, bend left=18] node[right, below] {$b_2$} (q21);
                \node [place] (q31) [below=of q21,label=below:{$q_1$}] {$(3, 1)$}
                edge [pre, bend left=80] node[left] {$b_1$} (q11)
                edge [pre] node[left] {$c_1$} (q21);
                \node [place] (q32) [below=of q22,label=below:{$q_1, p_2$}] {$(3, 2)$}
                edge [pre] node[above] {$a_2, s_{(2)\leftarrow(1)}$} (q31)
                edge [pre, bend left=28] node[left] {$b_1$} (q12)
                edge [pre, bend right=40] node[right] {$c_1$} (q22);
                \node [place] (q33) [below=of q23,label=below:{$q_1, q_2$}] {$(3, 3)$}
                edge [pre] node[above] {$c_2, s_{(2)\leftarrow(1)}$} (q32)
                edge [pre, bend left=40] node[right] {$c_1, s_{(1)\leftarrow(2)}$} (q23)
                edge [pre, bend right=80] node[right] {$b_1, s_{(1)\leftarrow(2)}$} (q13)
                edge [pre, bend left=40] node[right, below] {$b_2, s_{(2)\leftarrow(1)}$} (q31);
            \end{tikzpicture}
        \end{center}

    As mentioned above, the PHFL$^k$ formula $\varphi_{PHFL}$ has also be transformed to $\varphi_{HFL}$. We
    transform $\varphi_{PHFL}$ by induction. For induction base we have three cases.
    \begin{compactitem}
        \item $\varphi_{PHFL} = \top$ have to be transformed to $\varphi_{HFL} = \top$,

        \item $\varphi_{PHFL} = X$ to $\varphi_{HFL} = X$ and

        \item $\varphi_{PHFL} = p_i$ to $\varphi_{HFL} = p_i$. Note, that $p_i$ of formula $\varphi_{PHFL}$ is
        property $p \in P$ of the element with index $i$ in a $d$-tuple and $p_i$ of formula $\varphi_{HFL}$ is
        property $p_i \in P_i$.
    \end{compactitem}
    By induction hypothesis we have for subformulas $\psi_{PHFL}$ and ${\psi'}_{PHFL}$ of $\varphi_{PHFL}$ subformulas
    $\psi_{HFL}$ and ${\psi'}_{HFL}$ of $\varphi_{HFL}$.
    \begin{compactitem}
        \item Then $\varphi_{PHFL} = \langle a \rangle_i \psi_{PHFL}$ will be transformed to $\varphi_{HFL} =
        \langle a_i \rangle \psi_{HFL}$,

        \item $\varphi_{PHFL} = \psi_{PHFL} \vee {\psi'}_{PHFL}$ to $\varphi_{HFL} =
        \psi_{HFL} \vee {\psi'}_{HFL}$ and

        \item $\varphi_{PHFL} = \neg \psi_{PHFL}$ to $\varphi_{HFL} = \neg \psi_{HFL}$.

        \item Next, $\varphi_{PHFL} = \{\emph{i} \leftarrow \emph{j}\} \psi_{PHFL}$, where $\emph{i} = (i_1, \dots,
        i_n)$, $\emph{j} = (j_i, \dots, j_n)$ and $n, i_1, \dots, i_n, j_1, \dots, j_n \in \{1, \dots, d\}$ will be
        transformed to $\varphi_{HFL} = \langle s_{\emph{i} \leftarrow \emph{j}} \rangle \psi_{HFL}$.

        \item Furthermore, $\varphi_{PHFL} = \mu (X \colon \tau).\,\psi_{PHFL}$ will be transformed to $\varphi_{HFL}
        = (X \colon \tau).\,\psi_{HFL}$,

        \item $\varphi_{PHFL} = \lambda (X^v \colon \tau).\, \psi_{PHFL}$ to $\varphi_{HFL} = \lambda (X^v \colon \tau).\,
        \psi_{HFL}$ and

        \item $\varphi_{PHFL} = \psi_{PHFL}{\psi'}_{PHFL}$ to $\varphi_{HFL} = \psi_{HFL}{\psi'}_{HFL}$.
    \end{compactitem}

    As an example we want to transform the PHFL$^1$ formula of Example~\ref{example:phfl} to a HFL$^1$ formula.
    \[\varphi_{PHFL} = (\mu F. \lambda X, Y. X \Leftrightarrow Y \wedge \underset{a \in \Sigma}{\bigwedge} F
    \langle a \rangle_1 X \langle a \rangle_2 Y)\top \top\]
    will be transformed to
    \[\varphi_{HFL} = (\mu F. \lambda X, Y. X \Leftrightarrow Y \wedge \underset{a \in \Sigma}{\bigwedge} F \langle a_1
    \rangle X \langle a_2 \rangle Y)\top \top.\]

    Next, we want to show the correctness of this construction. For this, we prove by induction on formula
    $\varphi_{PHFL}$: $\mathcal{T}_{PHFL}, \emph{s} \models \varphi_{PHFL}$ iff $\mathcal{T}_{HFL}, \emph{s} \models
    \varphi_{HFL}$, where $\emph{s} = (s_1, \dots, s_d)$.
    \begin{compactitem}
        \item In case of $\varphi_{PHFL} = \top$, $\mathcal{T}_{PHFL}, \emph{s} \models \varphi_{PHFL}$
        is always true. The same holds for $\mathcal{T}_{HFL}, \emph{s} \models \varphi_{HFL}$, where $\varphi_{HFL} =
        \top$.

        \item If $\varphi_{PHFL} = p_i$ then $\mathcal{T}_{PHFL},
        \emph{s} \models \varphi_{PHFL}$ iff $p \in \nu_{PHFL}(s_i)$. By our construction that is exactly the case if
        $p_i \in \nu_{HFL}(\emph{s})$ which means that $\mathcal{T}_{HFL}, \emph{s} \models \varphi_{HFL}$ where
        $\varphi_{HFL} = p_i$.

        \item The last induction base case is $\varphi_{PHFL} = X$. $\mathcal{T}_{PHFL}, \emph{s} \models
        \varphi_{PHFL}$ exactly then if $\emph{s} \in \eta{X}$, where $\eta$ is a variable mapping. The same variable
        mapping is used for the HFL$^k$ formula which means that $\mathcal{T}_{HFL}, \emph{s} \models \varphi_{HFL}$.
    \end{compactitem}
    By induction hypothesis we have for subformulas $\psi_{PHFL}$ and ${\psi'}_{PHFL}$ of $\varphi_{PHFL}$
    subformulas $\psi_{HFL}$ and ${\psi'}_{HFL}$ of $\varphi_{HFL}$ such that $\mathcal{T}_{PHFL}, \emph{s} \models
    \psi_{PHFL}$ iff $\mathcal{T}_{HFL}, \emph{s} \models \psi_{HFL}$ and $\mathcal{T}_{PHFL}, \emph{s} \models
    {\psi'}_{PHFL}$ iff $\mathcal{T}_{HFL}, \emph{s} \models {\psi'}_{HFL}$.
    \begin{compactitem}
        \item Let $\varphi_{PHFL} = \langle a \rangle_i \psi_{PHFL}$, then $\mathcal{T}_{PHFL}, \emph{s} \models
        \varphi_{PHFL}$ iff there exists $\emph{s'}$ with $\emph{s} \overset{a, i}{\rightarrow} \emph{s'}$ and
        $\mathcal{T}_{PHFL}, \emph{s'} \models \psi_{PHFL}$. This is exactly the case if there exists $\emph{s'}$
        with $(s_i, a, {s_i}') \in \Delta_{PHFL}$ and $\mathcal{T}_{PHFL}, \emph{s'} \models \psi_{PHFL}$ if there
        exists $\emph{s'}$ with $(\emph{s}, a_i, \emph{s'}) \in \Delta_{HFL}$ and by induction hypothesis it exists
        $\psi_{HFL}$ such that $\mathcal{T}_{HFL}, \emph{s'} \models \psi_{HFL}$. This is the definition of the
        semantics of HFL formula $\varphi_{HFL} = \langle a_i \rangle \psi_{HFL}$ which means that
        $\mathcal{T}_{HFL}, \emph{s} \models \varphi_{HFL}$.

        \item If $\varphi_{PHFL} = \psi_{PHFL} \vee {\psi'}_{PHFL}$ then $\mathcal{T}_{PHFL}, \emph{s} \models
        \varphi_{PHFL}$ iff $\mathcal{T}_{PHFL}, \emph{s} \models \psi_{PHFL}$ or $\mathcal{T}_{PHFL}, \emph{s}
        \models {\psi'}_{PHFL}$. By induction hypothesis it holds that $\mathcal{T}_{HFL}, \emph{s} \models
        \psi_{HFL}$ or $\mathcal{T}_{HFL}, \emph{s} \models {\psi'}_{HFL}$ what is exactly the same as
        $\mathcal{T}_{HFL}, \emph{s} \models \varphi_{HFL}$, where $\varphi_{HFL} = \psi_{HFL} \vee {\psi'}_{HFL}$.

        \item If $\varphi_{PHFL} = \neg \psi_{PHFL}$ then $\mathcal{T}_{PHFL}, \emph{s} \models \varphi_{PHFL}$ iff
        $\mathcal{T}_{PHFL}, \emph{s} \not\models \psi_{PHFL}$. By induction hypothesis it holds that
        $\mathcal{T}_{HFL}, \emph{s} \not\models \psi_{HFL}$ what is exactly the same as $\mathcal{T}_{HFL}, \emph{s}
        \models \varphi_{HFL}$, where $\varphi_{HFL} = \neg\psi_{HFL}$.

        \item If $\varphi_{PHFL} = \{(i_1, \dots, i_n) \leftarrow (j_1, \dots, j_n)\} \psi_{PHFL}$ then
        $\mathcal{T}_{PHFL}, \emph{s} \models \varphi_{PHFL}$ iff there exists $\emph{s'}$ and $a_1,
        \dots, a_n \in \Sigma$ with $(s_{i_1}, a_1, {s'}_{j_1}), \dots, (s_{i_n}, a_n, {s'}_{j_n})$ $ \in \Delta_{PHFL}$
        and $\mathcal{T}_{PHFL}, \emph{s'} \models \psi_{PHFL}$ by construction it exists $\emph{s'}$ with $
        (\emph{s}, s_{(i_1, \dots, i_n) \leftarrow (j_1, \dots, j_n)}, \emph{s'}) \in \Delta_{HFL}$ and by induction
        hypothesis it exists $\psi_{HFL}$ such that $\mathcal{T}_{HFL}, \emph{s'} \models \psi_{HFL}$. That is
        exactly the case if $\mathcal{T}_{HFL}, \emph{s} \models \varphi_{HFL}$ where $\varphi_{HFL} = \langle s_{
        (i_1, \dots, i_n) \leftarrow (j_1, \dots, j_n)} \rangle \psi_{HFL}$.

        \item If $\varphi_{PHFL} = \mu(X \colon \tau).\,\psi_{PHFL}$ then $\mathcal{T}_{PHFL}, \emph{s}
        \models \varphi_{PHFL}$ iff $\mathcal{T}_{PHFL}, \emph{s} \models \psi_{PHFL}$ where for variable mapping
        $\eta$ holds $\eta[X \mapsto \mathcal{X}]$ and $\mathcal{X} \in \mathcal{T}_{PHFL} \llbracket \tau
        \rrbracket$ is the least fixpoint. By induction hypothesis it exists $\psi_{HFL}$ such that
        $\mathcal{T}_{HFL}, \emph{s} \models \psi_{HFL}$ where for variable mapping $\eta$ holds $\eta[X \mapsto
        \mathcal{X}]$ and $\mathcal{X} \in \mathcal{T}_{HFL} \llbracket \tau \rrbracket$. That is exactly the case if
        $\mathcal{T}_{HFL}, \emph{s} \models \varphi_{HFL}$ where $\varphi_{HFL} = \mu(X \colon \tau).\,\psi_{HFL}$.

        \item If $\varphi_{PHFL} = \lambda (X^v \colon \sigma).\,\psi_{PHFL}$ then $\mathcal{T}_{PHFL}, \emph{s}
        \models \varphi_{PHFL}$ iff it exists a function $F \in \mathcal{T}_{PHFL}\llbracket \sigma^v \rightarrow \tau
        \rrbracket$ such that $\mathcal{T}_{PHFL}, \emph{s} \models \psi_{PHFL}$ where for variable mapping $\eta$
        holds $\eta[X \mapsto F]$. This is exactly the case if there exists a function $F \in
        \mathcal{T}_{HFL}\llbracket \sigma^v \rightarrow \tau \rrbracket$ where for variable mapping $\eta$ holds
        $\eta[X \mapsto F]$ and by induction hypothesis it exists $\psi_{HFL}$ with $\mathcal{T}_{HFL}, \emph{s}
        \models \varphi_{HFL}$ where $\varphi_{HFL} = \lambda (X^v \colon \sigma).\,\psi_{HFL}$.

        \item If $\varphi_{PHFL} = \psi_{PHFL}{\psi'}_{PHFL}$ then $\mathcal{T}_{PHFL}, \emph{s}
        \models \varphi_{PHFL}$ iff $\mathcal{T}_{PHFL}, \emph{s} \models \psi_{PHFL}({\psi'}_{PHFL})$. By induction
        hypothesis it exists $\psi_{HFL}$ and ${\psi'}_{HFL}$ with $\mathcal{T}_{HFL}, \emph{s} \models \psi_{HFL}
        ({\psi'}_{HFL})$. This is exactly the same as $\mathcal{T}_{HFL}, \emph{s} \models \varphi_{HFL})$
        where $\varphi_{HFL} = \lambda (X^v \colon \sigma).\,\psi_{HFL}{\psi'}_{HFL}$.
    \end{compactitem}
    This shows the correctness of the construction and so that the model checking problem of PHFL$^k$ is reducable to
    the model checking problem of HFL$^k$.
\end{proof}

The following theorem is given by Lemma~\ref{lemma:model_check_phfl_k} and Theorem~\ref{theorem:hfl_k_in_k_exptime}.

\begin{theorem}
    \label{theorem:phfl_k_in_k_exptime}
    The model checking problem of PHFL$^k$ formulas lies in $k$-EXPTIME, where $k > 0$.
\end{theorem}