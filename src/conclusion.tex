\chapter{Conclusion}

In this thesis, we contributed to descriptive complexity theory by relating any order of PHFL to the according complexity class. In detail, we showed that the logic PHFL$^k$ captures the complexity class \exptime{$k$} for any $k > 1$ on finite labelled transition systems. Due to the fact, that the above statement is also true for $0 \leq k \leq 1$~\cite{lange2014capturing} we were able to verify, that PHFL$^k$ captures  for any $k \geq 0$ on finite labelled transition systems. Furthermore, it was proven that the logic PHFL$^{k+1}_{tail}$ captures the complexity class \expspace{$k$} for any $k > 1$. In analogy to the exponential time classes, it was also proven that PHFL$^{k+1}_{tail}$ captures \expspace{$k$} for any $k \geq 0$ on finite labelled transition systems.
Since quantification is not included in PHFL a lot of effort had to be spent into the development of the encoding of the existential quantifiers of any order.

To obtain higher-order quantification in PHFL we used the existential quantifiers of type $\tau = (\odot, \dots, \odot)$ to define the order of domains of kind
$D_{(\tau, \dots, \tau)}(\mathcal{U})$. This order was then be used to define a formula that returns the successor
of a given element of $D_{(\tau, \dots, \tau)}(\mathcal{U})$ in respect to this order. Finally, we used this
formula to define the existential quantifier of type $(\tau, \dots, \tau)$. This procedure was applied to all
possible types of HO. In this way we got higher-order quantification of any type in PHFL.

The presented results enlarge the angle of vision on these complexity classes, which opens the possibilities for additional research, especially for further characterization of $k$-EXPTIME and $k$-EXPSPACE. That could lead to a better understanding of these classes and might even facilitate a differentiation approach.
