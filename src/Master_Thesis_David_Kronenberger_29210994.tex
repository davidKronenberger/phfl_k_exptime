%%
%% Author: Davidov
%% 23.04.2018
%%

\documentclass[12pt,a4paper]{book}

%underline emph
\renewcommand{\emph}[1]{\textbf{#1}}

%packages for math symbols
\usepackage{amsmath}
\usepackage{amssymb}
\usepackage{textcomp}
\usepackage{mathpartir}
\usepackage{stmaryrd}
\usepackage{mathtools}

% package for graphs
\usepackage{tikz}
\usetikzlibrary{graphs}
\usetikzlibrary{positioning}
\usetikzlibrary{automata}
\usetikzlibrary{arrows,decorations.pathmorphing,backgrounds,positioning,fit,petri}

% package for inverse and reverse search
\usepackage{pdfsync}

% package for urls
\usepackage{hyperref}

%package for using
\usepackage{graphicx}

%package for proof and other theorem environments
\usepackage{amsthm}

%this package provides a pendant to "itemize" with better spacing - use compactitem
\usepackage{paralist}

%supports some nice characters with mathscr
\usepackage{mathrsfs}

%enables better support for tables
\usepackage{array}
\usepackage{multirow}

% various theorems, numbered by section
\newtheorem{theorem}{Theorem}[chapter]
\newtheorem{lemma}[theorem]{Lemma}
\newtheorem{proposition}[theorem]{Proposition}
\newtheorem{corollary}[theorem]{Corollary}
\newtheorem{example}[theorem]{Example}
\theoremstyle{definition}
\newtheorem{definition}[theorem]{Definition}
\theoremstyle{remark}
\newtheorem{remark}[theorem]{Remark}
\newtheorem{observation}[theorem]{Observation}

\newcolumntype{C}[1]{>{\parbox[c][#1][c]{0cm}{}}c<{}}

%some own commands
\newcommand{\expspace}[1]{#1-EXPSPACE/$_{\raise.17ex\hbox{$\scriptstyle\sim$}}$}
\newcommand{\exptime}[1]{#1-EXPTIME/$_{\raise.17ex\hbox{$\scriptstyle\sim$}}$}
\newcommand{\nexptime}[1]{#1-NEXPTIME/$_{\raise.17ex\hbox{$\scriptstyle\sim$}}$}
\newcommand{\nexpspace}[1]{#1-NEXPSPACE/$_{\raise.17ex\hbox{$\scriptstyle\sim$}}$}

%set of own commands
%\newcommand{\hcat}{\rotatebox{90}{$\ominus$}}
\newcommand{\hcat}{
\mathchoice{\rotatebox{90}{$\displaystyle\ominus$}}
{\rotatebox{90}{$\ominus$}}
{\rotatebox{90}{$\scriptstyle\ominus$}}
{\rotatebox{90}{$\scriptscriptstyle\ominus$}}}
\newcommand{\vcat}{
\mathchoice{\raisebox{1pt}{$\displaystyle\ominus$}}
{\raisebox{1pt}{$\ominus$}}
{\raisebox{0.5pt}{$\scriptstyle\ominus$}}
{\raisebox{0.2pt}{$\scriptscriptstyle\ominus$}}}
\newcommand{\plusinbox}{
\setlength\fboxsep{0pt}
\setlength{\fboxrule}{0.00001pt}
\text{ \framebox[8pt]{+} }
\setlength\fboxsep{3pt}
\setlength{\fboxrule}{0.4pt}
}
\newcommand{\mirroredL}{
\resizebox{0.31cm}{!}{\tiny\begin{tabular}[b]{C{0.4cm}|C{0.4cm}|}
\cline{2-2}
&\tabularnewline
\hline
\multicolumn{1}{|c|}{}&\tabularnewline
\hline
\end{tabular}}\normalsize}
%document information
\title{Capturing Bisimulation-Invariant Complexity Classes by Polyadic Higher-Order Fixpoint Logic}

\author{David Kronenberger}

\makeatletter
\def\maketitle{%

\begin{center}
\textbf{\textsf{\Huge \@title}}\\
\vspace{2cm}
{\Large by\\
\@author\\
\vspace{2.5cm}
Dipl.-Math. Florian Bruse, Advisor\\
Prof. Dr. Martin Lange, Reviewer\\
\vspace{0.2cm}
Prof. Dr. Stefan Göller, Reviewer
}\\
\vspace{2.0cm}
A thesis submitted in partial fulfillment\\
of the requirements for the\\
Degree of Master of Science\\
in Computer Science\\
\vspace{2cm}
UNIVERSITY OF KASSEL\\
Hesse, Germany\\
\vspace{1cm}
\@date
\end{center}
}

\addtocontents{toc}{\protect\thispagestyle{empty}}
\begin{document}

\maketitle
\thispagestyle{empty}

\pagebreak
\thispagestyle{empty}

\cleardoublepage
\vspace*{30\baselineskip}
\hbox to \textwidth{\hrulefill}
\par
\hbox to \textwidth{I declare that I have developed and written the enclosed thesis entirely}
\hbox to \textwidth{by myself, and have not used sources or means without declaration in the text.}
\vspace{0.9cm}
\hbox{\textbf{Kassel, \today}}
\vspace{0.6cm}

\hbox{\ldots\ldots\ldots\ldots\ldots\ldots\ldots\ldots\ldots\ldots\ldots\ldots}~\\
\hbox{\hspace*{1cm}(\textbf{David Kronenberger})} %center name with hspace

\thispagestyle{empty}

\pagebreak
\thispagestyle{empty}

\begingroup
	\flushbottom
	\setlength{\parskip}{12pt}%
	\tableofcontents
	\endgroup
\thispagestyle{empty}
\pagebreak

\setcounter{page}{1}
\chapter{Introduction}\label{ch:introduction}

The higher order cases of PHFL and a restriction called
tail-recursive for this higher order cases we are interested to compare with the in
Section~\ref{sec:descriptiveComplexity} introduced complexity classes \exptime{$k$} and \expspace{$k$}.

\chapter{Preliminaries}
\label{ch:preliminaries}
This chapter introduces all necessary definitions to prove that PHFL$^k =$~\exptime{$k$} and
PHFL$^{k+1}_{tail} =$~\expspace{$k$}. The notions are mainly from~\cite{immerman1999descriptive},
~\cite{papadimitriou1994complexity},~\cite{otto1999bisimulation},~\cite{freireMartins2011descriptive}
and~\cite{lange2014capturing}.

We assume that the reader is already familiar with basic notions of first order logic and 
computational complexity. In the first section we define a special kind of graphs called LTS and 
define some properties on it. Also in this section we define queries with special forms of queries. 
In the next section we give some information on fixpoints that are used the section that follows. 
There is defined the logic PHFL. In Section~\ref{sec:descriptiveComplexity} we present the 
descriptive complexity and the complexity clases \exptime{$k$} and \expspace{$k$}. In the last section we define the higher-order logic and combinations with LFP and PFP.

%%
%% Author: Davidov
%% 16.05.2018
%%

\subsection{Bisimulation Invariance}\label{subsec:bisimulationInvariance}

First of all, we need the definition of \textit{labeled transition systems}. A labeled transition system is a graph
with labeled vertices and edges. Formally, it is the following.

\begin{definition}
    \label{definition:lts}
    A quintuple $\mathcal{T} = (Q, \Sigma, P, \Delta, \nu)$ is called a \emph{labeled transition system} (\emph{LTS}),
    where
    \begin{compactitem}
        \item $Q$ is a set of states,
        \item $\Sigma$ is a finite set of actions,
        \item $P$ is a finite set of propositions,
        \item $\Delta \subseteq Q \times \Sigma \times Q$ is the labeled transition relation and
        \item $\nu: Q \rightarrow 2^P$ is a function that maps each state to a set of propositions.
    \end{compactitem}
\end{definition}

For all $q_1, q_2 \in Q$ and all $a \in \Sigma$ we write $q_1 \overset{a}{\rightarrow} q_2$ for $(q_1, a, q_2) \in
\Delta$.

\begin{example}
    As mentioned above, LTS can be seen as a graph with labeled vertices and edges. One example for a LTS is
    $\mathcal{T} = (\{0, 1, 2, 3, 4\}, \{a, b\}, \{p, q\}, \Delta, \nu)$
\begin{center}\begin{tikzpicture}[]
                  \node [place] (w1) [label=above:$q$] {0};
                  \node [place] (c1) [below=of w1,label=above:$p~q$] {2};
                  \node [place] (s) [below=of c1,label=above:$q$] {4};
                  \node [place] (e1) [left=of c1] {1}
                  edge [pre,bend left] (w1)
                  edge [post,bend right] (s)
                  edge [post] (c1);
                  \node [place] (l1) [right=of c1] {3}
                  edge [pre] (c1)
                  edge [pre,bend left] (s)
                  edge [post,bend right] node[auto, swap] {2} (w1);
\end{tikzpicture}
\end{center}
\end{example}

On this systems or rather on the states of the systems it is possible to define relations. The
following relation describes those states that have the same behaviour. For this, let be $\mathcal{T}_1 = (Q_1,
\Sigma_1, P_1, \Delta_1, \nu_1)$ and $\mathcal{T}_2 = (Q_2, \Sigma_2, P_2, \Delta_2, \nu_2)$ two LTSs.

\begin{definition}
    A \emph{bisimulation} is a binary relation $R \subseteq Q_1 \times Q_2$ that fulfills for all $(q_1, q_2) \in R$
    \begin{compactitem}
        \item $\nu_1 (q_1) = \nu_2 (q_2)$,
        \item for all $a_1 \in \Sigma_1$ and all $q_1' \in Q_1$, if $q_1 \overset{a_1}{\rightarrow} q_1' \in
        \Delta_1$, then there is a state $q_2' \in Q_2$ with $a_1 \in \Sigma_2$, $q_2
        \overset{a_1}{\rightarrow} q_2' \in \Delta_2$ and $(q_1', q_2') \in R$ and
        \item for all $a_2 \in \Sigma_2$ and all $q_2' \in Q_2$, if $q_2 \overset{a_2}{\rightarrow} q_2' \in
        \Delta_2$, then there is a state $q_1' \in Q_1$ with $a_2 \in \Sigma_1$, $q_1
        \overset{a_2}{\rightarrow} q_1' \in \Delta_1$ and $(q_1', q_2') \in R$.
    \end{compactitem}
    We call two states $q_1 \in Q_1$, $q_2 \in Q_2$ \emph{bisimilar}, noted as $(\mathcal{T}_1, q_1) \sim
    (\mathcal{T}_2, q_2)$, if there
    is a bisimulation $R$ such that $(q_1, q_2) \in R$.
\end{definition}

Furthermore, we can describe properties of LTS. \textit{Queries} are one way to describe these properties. A query
is a mapping that associates a LTS $\mathcal{T} = (Q, \Sigma, P, \Delta, \nu)$ to a subset
$M^{\mathcal{T}}$ of $Q \times \dots \times Q$. Remark, that any isomorphism $f: \mathcal{T} \simeq
\mathcal{T}'$ is also an isomorphism of $M^{\mathcal{T}}$ and $M^{{\mathcal{T}}'}$.

\begin{definition}
    \label{definition:query}
    A function $\mathcal{Q} : \mathscr{T} \rightarrow \mathscr{Q} \times \dots \times \mathscr{Q}, [\mathcal{T}]_\simeq
    \mapsto M^{\mathcal{T}}$ is called a
    \emph{$r$-adic query},
    where
    \begin{compactitem}
        \item $\mathscr{T}$ is the set of all equivalence classes of LTS relative to isomorphism,
        \item $\mathscr{Q}$ is the set of all equivalence classes of sets of states relative to isomorphism,
        \item $[\mathcal{T}]_\simeq = ([Q^\mathcal{T}]_\simeq, [\Sigma^\mathcal{T}]_\simeq, [P^\mathcal{T}]_\simeq,
        [\Delta^\mathcal{T}]_\simeq, [\nu^\mathcal{T}]_\simeq) \in \mathscr{T}$ is the equivalence class of LTS
        $\mathcal{T}$ relative to isomorphism and
        \item $M^{\mathcal{T}} \in \mathcal{P}([Q^\mathcal{T}]_\simeq \times \dots \times [Q^\mathcal{T}]_\simeq)$ is a
        set of tuples of states from $[\mathcal{T}]_\simeq$, i.e. $(q_1, \dots, q_r) \in M^{\mathcal{T}}$ with $q_1,
        \dots, q_r \in [Q^\mathcal{T}]_\simeq$.
    \end{compactitem}
\end{definition}

The queries defined in Definition~\ref{definition:query} can be categorized. Here we are interested in two of these
categories. The first category is called \textit{bisimulation invariant}. This category describes those queries that
can't distinguish bisimilar states. In~\cite{otto1999bisimulation} this property is defined over so called
\textit{Kripke structures}. A Kripke structure
is a transition system. Remark, that transition systems  have only one type of actions. This means the edges of the
graph haven't labels.

\begin{definition}
    \label{definition:bisimulationInvariant}
    Let $\mathcal{T}$, $\mathcal{T}'$ be two LTSs with $\mathcal{T} = (Q, \Sigma, P, \Delta, \nu)$
    and $\mathcal{T}' = (Q', \Sigma', P', \Delta', \nu')$. Furthermore, let be $(q_1, \dots, q_r) \in Q \times \dots
    \times Q$ and $({q_1}', \dots, {q_n}') \in Q' \times \dots \times Q'$.

    A query $\mathcal{Q}$ is called \emph{bisimulation invariant} if $(\mathcal{T}, q_i) \sim (\mathcal{T}', q_i')$
    for all $1 \leq i \leq r$ implies that $(q_1, \dots, q_r) \in \mathcal{Q}([\mathcal{T}]_\simeq)$ iff $({q_1}',
    \dots, {q_n}') \in \mathcal{Q}([\mathcal{T}']_\simeq)$.
\end{definition}

The second category of a query tells us which complexity class a query belongs to.

\begin{definition}
    \label{definition:queryBelongsToComplexityClass}
    Let $\mathcal{T}$ be a LTS with $\mathcal{T} = (Q, \Sigma, P, \Delta, \nu)$ and $(q_1, \dots, q_{r}) \in Q \times
    \dots \times Q$.

    A query $\mathcal{Q}$ belongs to complexity class $\mathcal{C}$ if there is an algorithm in $\mathcal{C}$ for
    deciding on input $(\mathcal{T}, (q_1, \dots, q_{r}))$ whether $(q_1, \dots, q_{r}) \in \mathcal{Q}
    ([\mathcal{T}]_\simeq)$.
\end{definition}

This definition leads us to the next chapter and the definitions for descriptive complexity.

%%
%% Author: Davidov
%% 16.05.2018
%%

\section{Fixpoints}\label{sec:fixpoints}

To define the polyadic higher-order fixpoint logic and the higher-order logic with least and partial fixpoints, we examine fixpoints in general in this section. The first fixpoint we consider is the least fixpoint.

\begin{definition}
   Let $F\colon A \rightarrow A$ be an operator on a finite set $A$, then $x \in A$
   is called a \emph{fixpoint} of $F$ if $F(x) = x$. Let $x$ be a fixpoint of $F$ and $\sqsubseteq$ an partial order on $A$, then $x$ is called the \emph{least
   fixpoint} of $F$, abbreviated as $\mathit{LFP}$($F$), if for all other fixpoints $y$ of $F$ the condition $x
   \sqsubseteq y$ holds. A fixpoint $x$ is called the \emph{greatest fixpoint} if $y \sqsubseteq x$ for all fixpoints $y$ of $F$.
\end{definition}

From the Knaster-Tarski Theorem~\cite{tarski1955lattice} we know that if an operator $F\colon A \rightarrow 
A$ is monotone and $A$ is a complete lattice regarding to $\sqsubseteq$ then the least and greatest fixpoints of $F$ exists. $F$ is monotone if for all $x, y
 \in A$ if $x \sqsubseteq y$ then $F(x) \sqsubseteq F(y)$ holds.

\begin{example}
    \label{example:lfp} Let $\mathcal{T} = (Q, \Sigma, P, \Delta, v)$ be an LTS and $F: Q^2 \rightarrow Q^2$ an operator on $Q^2$ defined as 
\begin{align*}
    F(X) =\, &\{(q, p) \in Q^2 \mid v(q) \neq v(p)\}\, \cup \\&
    \{(q,p) \in Q^2 \mid \text{it exists } a\in\Sigma \text{ with } q\overset{a}{\rightarrow} q' \text{ such that for all } p' \in Q \\&\text{ it holds } p\overset{a}{\rightarrow} p' \text{ implies } (q', p') \in X\}\,\cup \\&\{(q,p) \in Q^2 \mid  \text{it exists } a\in\Sigma \text{ with } p\overset{a}{\rightarrow} p' \text{ such that for all } q' \in Q \\&\text{ it holds } q\overset{a}{\rightarrow} q' \text{ implies } (q', p') \in X\}
\end{align*}   
Then $LFP(F)$ represents all those pairs of states $(q, p)$ such that $q\not\sim p$.
\end{example}

The following theorem shows a possibility to calculate the least fixpoint of a monotone operator if the operating set is a complete lattice with respect to its order.

\begin{theorem}[Kleene Fixed-Point Theorem~\cite{stoltenberghHansen1994mathematical}]
\label{theorem:kleene}
Let $F: A \rightarrow A$ be a monotone operator on $A$ and $A$ regarding to $\sqsubseteq$ a complete lattice, then it exists a finite sequence $X_0, \dots, X_m$ such that the first part $X_0$ is the smallest element in $A$ with respect to $\sqsubseteq$, the $(i+1)$-th part $X_{i+1}$ is $F(X_i)$ and $X_m = X_{m+1}$.
\end{theorem}

Next, we define the partial fixpoint. Since the
$\mathit{LFP}$ restricts the operator to be monotone, the partial fixpoint need no restriction on the operator.

\begin{definition}
\label{definition:pfp}
    Let $F\colon A \rightarrow A$ be an operator on a finite set $A$, then the \emph{partial
    fixpoint} of $F$, abbreviated as $\mathit{PFP}$($F$), is defined as follows:
    \[\mathit{PFP}(F)\coloneqq\begin{cases}
               F^{i+1}(\varnothing)=F^i(\varnothing),  & \text{if such } i \in \{0,\dots,|A|\} \text{ exists}\\
               \varnothing, & \text{otherwise,}
    \end{cases}\]
    where $F^0(\varnothing) = \varnothing$, $F^1(\varnothing) = F(\varnothing)$, $F^2(\varnothing) = F(F(\varnothing))$, and so on.
\end{definition}

Note, that for monotone $F$ holds $\mathit{PFP}(F)$ equals $\mathit{LFP}(F)$. 


\begin{example}
\label{example:pfp}
Let $F: \mathcal{P}(\{1, \dots, n\}) \rightarrow  \mathcal{P}(\{1, \dots, n\})$ be an operator on $ \mathcal{P}(\{1, \dots, n\})$ defined as
\begin{align*}
	F(X) = \{x \in \{1, \dots, n\}\mid &\; x\in X \text{ and it exists } y \in \{1, \dots, n\} \text{ such that }\\ &\;y < x \text{ and } y\not\in X \text{ or } \\&\;x\not\in X \text{ and for all } y\in \{1,\dots, n\} \text{ holds if } \\&\;y<x \text{ then } y\in X\}. 
\end{align*}
If we see $X \in \mathcal{P}(\{1,\dots, n\})$ as a binary string $b$, where a $1$ at the $i$-th position means that $i$ is in $X$, then $F(X)$ returns the set $Y$ such that the binary string $b'$ of $Y$ is $b' = b+1\mod n$. Then $PFP(F)$ returns $\varnothing$ because for every $i$ it holds $F^i(\varnothing) \neq F^{i+1}(\varnothing)$.
\end{example}

%%
%% Author: Davidov
%% 16.05.2018
%%

\section{Polyadic Higher Order Fixpoint Logic}\label{sec:polyadichigherorderfixpointlogic}

In this section, we present a logic with name Polyadic Higher Order Fixpoint Logic, abbreviated with PHFL, that was
introduced by M. Lange and E. Lozes in~\cite{lange2014capturing}. It is defined over LTS (see
Definition~\ref{definition:lts}) and extends the polyadic modal $\mu$-calculus~\cite{otto1999bisimulation} with
higher order fixpoints like M. Viswanathan and R. Viswanathan it did with monadic case of modal
$\mu$-calculus~\cite{kozen1983results} in~\cite{viswanathan2004higher}. The logic of M. Viswanathan and R
.Viswanathan with name higher order fixed point logic is a combination of propositional logic, modal operators and
a simply typed $\lambda$-calculus with fixed point operators. The higher order cases of PHFL and a restriction called
tail-recursive for this higher order cases we are interested to compare with the in
Chapter~\ref{sec:descriptiveComplexity} introduced complexity classes \exptime{$k$} and \expspace{$k$}.

\subsection{PHFL Types}\label{subsec:phflTypes}

Before defining formulas of PHFL we need to introduce the PHFL types. These definitions are guided
by~\cite{viswanathan2004higher} and~\cite{lange2014capturing}.

\begin{definition}
    \emph{PHFL types} are given by the grammar
    \[\sigma, \tau \Coloneqq \bullet \mid \sigma^v \rightarrow \tau,\]
    where $v$ is called \textit{variance}. The \emph{variances} of PHFL are defined by the grammar
    \[v \Coloneqq + \mid - \mid 0.\]
\end{definition}

All types will be interpreted as a partially ordered sets. Partial orders are relations that are reflexive, transitive
and antisymmetric. Let $\mathcal{A} = (A, \leq_A)$ and $\mathcal{B} = (B, \leq_B)$ be two partial orders. Then
$\mathcal{A} \rightarrow \mathcal{B}$ is the partial order of monotone functions ordered pointwise, i.e.
\[\mathcal{A} \rightarrow \mathcal{B} = \{f\colon A\rightarrow B \mid \text{ for all } x,y \in A.\,x\leq_A y \text{
implies }
f(x)
\leq_B f(y)\}\]
and the ordering relation is given by
\[f \leq_{\mathcal{A}\rightarrow\mathcal{B}} g\text{ iff } \text{ for all } x\in \mathcal{A}.\,f(x) \leq_{\mathcal{B}} g
(x)\].

\begin{definition}
    Let $\mathcal{T} = (Q, \Sigma, P, \Delta, v)$ be a LTS and $d \in \mathbb{N}$ the dimension of PHFL,
    then $\llbracket\tau\rrbracket_\mathcal{T}$ the
    semantics
    of type $\tau$ is defined by $\tau$ as follows:
        \[\llbracket\tau\rrbracket_\mathcal{T}=
        \begin{cases}
            (\mathcal{P}(Q^d), \subseteq),  & \text{if }\tau = \bullet\\
            ((\llbracket\sigma_1\rrbracket_\mathcal{T})^v \rightarrow \llbracket\sigma_2\rrbracket_\mathcal{T}, \leq_{
            (\llbracket\sigma_1\rrbracket_\mathcal{T})^v \rightarrow \llbracket\sigma_2\rrbracket_\mathcal{T}}), &
            \text{if }\tau = \sigma_1^v\rightarrow \sigma_2,
        \end{cases}\]
    where for any partial order $\mathcal{A} = (A, \leq_A)$, $\mathcal{A}^v = (A, \leq_A^v)$ is a partial order
    with $\leq_A^+ = \leq_A$, $\leq_A^- = \{(a, b) \mid (b, a) \in \leq_A\}$ and $\leq_A^0 = \leq_A^+ \cap \leq_A^-$.
\end{definition}

The partial orders $\llbracket\tau\rrbracket_\mathcal{T}$ for any PHFL type $\tau$ are complete lattices. That means we
have meets and joins, denoted by $\sqcap_{\llbracket\tau\rrbracket_\mathcal{T}}$ and
$\sqcup_{\llbracket\tau\rrbracket_\mathcal{T}}$, and least and greatest elements, denoted by
$\bot_{\llbracket\tau\rrbracket_\mathcal{T}}$ and $\top_{\llbracket\tau\rrbracket_\mathcal{T}}$ for any subset of
$\llbracket\tau\rrbracket_\mathcal{T}$. This ensures that the least and greatest fixpoint over all monotone PHFL types
exist~\cite{tarski1955lattice}. See Chapter~\ref{subsec:hoPlusLfp} for further information about fixpoints.

\begin{definition}
    The \emph{maximal arity} $ma(\tau)$ and the \emph{order} $ord(\tau)$ of a PHFL type $\tau$ are defined
    inductively on
    $\tau$ as follows:
\[ma(\tau)=
\begin{cases}
    1, & \text{if }\tau = \bullet\\
    max(\{n\} \cup \{ma(\tau_i)\mid1,\dots,n\}), &
    \text{if }\tau = \tau_1\rightarrow\dots\rightarrow\tau_n\rightarrow\bullet
\end{cases}\]
\[ord(\tau)=
\begin{cases}
    0, & \text{if }\tau = \bullet\\
    max(\{1 + ord(\sigma_1), ord(\sigma_2)\}), & \text{if }\tau = \sigma_1 \rightarrow \sigma_2
\end{cases}\]
\end{definition}

Next, we want to define the syntax of PHFL formulas.

\subsection{PHFL Syntax}\label{subsec:phflSyntax}

\begin{figure}
    \caption{Derivation Rules for PHFL formulas.}
    \label{figure:phfl-typing-rules}
    \begin{mathpar}
        \Gamma \vdash \top \colon \bullet \and
        \Gamma \vdash p_i \colon \bullet \and
        \inferrule{\Gamma \vdash \Phi \colon \bullet}{\Gamma \vdash \langle a \rangle_i \Phi \colon \bullet} \and
        \inferrule{\Gamma \vdash \Phi \colon \bullet}{\Gamma \vdash \{\emph{i} \leftarrow \emph{j}\}\Phi \colon
        \bullet} \and
        \inferrule{\Gamma^-\vdash\Phi\colon \bullet}{\Gamma \vdash \neg \Phi \colon \bullet} \and
        \inferrule{\Gamma\vdash\Phi \colon \tau \\ \Gamma\vdash\Psi\colon \tau}{\Gamma \vdash \Phi \vee
        \Psi \colon  \tau} \and
        \inferrule{v \in \{+, 0\} }{\Gamma, X^v \colon\tau \vdash X\colon\tau} \and
        \inferrule{\Gamma,X^v\colon\sigma\vdash \Phi\colon\tau}{\Gamma\vdash \lambda (X^v \colon \tau)
        .\Phi\colon\sigma^v\rightarrow\tau} \and
        \inferrule{\Gamma,X^+ \colon \tau \vdash \Phi\colon\tau}{\Gamma \vdash \mu (X \colon\tau). \Phi\colon\tau} \and
        \inferrule{\Gamma\vdash \Phi\colon\sigma^+ \rightarrow \tau \\ \Gamma\vdash\Psi\colon\sigma}{\Gamma \vdash
        \Phi\,\Psi \colon \tau} \and
        \inferrule{\Gamma\vdash \Phi\colon\sigma^- \rightarrow \tau \\ \Gamma^-\vdash\Psi\colon\sigma}{\Gamma \vdash
        \Phi\,\Psi \colon \tau} \and
        \inferrule{\Gamma\vdash \Phi\colon\sigma^0 \rightarrow \tau \\ \Gamma \vdash \Psi\colon\sigma \\ \Gamma^-
        \vdash\Psi\colon\sigma}{\Gamma \vdash \Phi\,\Psi \colon \tau} \and
    \end{mathpar}
\end{figure}

\begin{definition}
    Let $P$ a set of propositions, $\Sigma$ a set of actions and $\mathcal{V} = \{X_1, X_2, \dots\}$ a countable
    infinite
    set of variables, then
    \emph{$d$-adic PHFL formulas} $\Phi, \Psi,\dots$ are defined by the grammar
    \begin{align*}
        \Phi,\Psi\Coloneqq&\top \mid p_i \mid \Phi \vee \Psi \mid \neg \Phi \mid \langle a \rangle_i \Phi \mid
        \{\emph{j}\,\} \Phi \mid X \mid \lambda (X^v\colon\tau).\Phi \mid \Phi\,\Psi\mid  \mu (X\colon\tau).\Phi
    \end{align*}
    where
    \begin{compactitem}
        \item $\emph{j} = (e(1), \dots, e(d))$ and $e: \{1, \dots, d\} \rightarrow \{1, \dots, d\}$,
        \item $i \in \{1, \dots, d\}$
        \item $v$ is a variance,
        \item $\tau$ is a type,
        \item $p \in P$,
        \item $a \in \Sigma$ and
        \item $X \in \mathcal{V}$.
    \end{compactitem}
\end{definition}

For convenience, we use some other further standard notations like $\Phi \wedge \Psi$, $[a]_i\Phi$, $\nu
X \colon \tau.\Phi$ or $\Phi \Leftrightarrow \Psi$. Note, that this logic is defined over LTS. The formulas are
often interpreted as a game played by two players moving pebbles along the transitions of an LTS. The two players
are called Prover and Refuter. So, $p_i$ can be interpreted as, the position of the $i$-th pebble fulfills
property $p$. $\langle a \rangle_i \Phi$ means, Prover has to move the $i$-th pebble along an $a$-transition and
check if there holds $\Phi$. With the formula $\{\emph{j}\,\} \Phi$ is mentioned, that all pebbles
are moved from Prover to the positions described by tuple $\emph{j}$ and after this $\Phi$ have to
be fulfilled. The player who has to move the pebbles changes on negations. $\lambda (X^v\colon\tau).\Phi$ is
interpreted as a function that expects arguments of $(\llbracket\tau\rrbracket_\mathcal{T})^v$. We see, that the
formulas also have types. For this, we have ensure that a formula is well-typed.

\begin{definition}
    Let $X_1, \dots, X_n$ variables, $\Phi$ a PHFL formula, $v_1, \dots, v_n$ variances and $\tau, \tau_1, \dots,
    \tau_n$ types, then $\Gamma = X_1^{v_1}\colon \tau_1, \dots X_n^{v_n} \colon \tau_n$ is
    called a \emph{type environment} and $\Gamma \vdash \Phi\colon\tau$
    is called a \emph{type judgement}. Let $\Gamma^- = X_1^{v_1^-}\colon \tau_1, \dots
    X_n^{v_n^-} \colon \tau_n$ be a type environment then $\Gamma^- = X_1^{v_1^-}\colon \tau_1, \dots
    X_n^{v_n^-} \colon \tau_n$, where $-^- = +$, $+^- = -$ and $0^- = 0$.
\end{definition}

A type judgment is called \textit{derivable} if it generates a derivation tree according to the rules of
Figure~\ref{figure:phfl-typing-rules}. This type system ensures that we do not create senseless formulas like
$\langle a \rangle_i p_j p_k$. Furthermore, it can order formulas with semantics and guarantees so monotonicity. A
formula $\Phi$ is called \textit{well-typed} if the type judgement $\emptyset \vdash \Phi:\tau$ is derivable for some
type $\tau$. Note, that we are here only interested in well-typed formulas. For those formulas where also the variable
types are obvious, we omit the type on the variables.

\subsection{PHFL Semantics}\label{subsec:phflSemantics}

\begin{figure}
    \caption{Semantics of PHFL formulas.}
    \label{figure:phfl-semantics}
    \begin{align*}
        \llbracket \Gamma \vdash \top \colon \bullet \rrbracket^\eta_\mathcal{T} =\,& Q^d\\
        \llbracket \Gamma \vdash p_i \colon \bullet \rrbracket^\eta_\mathcal{T} =\,& \{(q_1, \dots, q_d) \in Q^d \mid p \in v
        (q_i)\}\\
        \llbracket \Gamma \vdash \langle a \rangle_i \Phi \colon \bullet \rrbracket^\eta_\mathcal{T} =\,& \{(q_1,
        \dots, q_d) \in Q^d \mid \text{ it exists } \\& ({q'}_1, \dots, {q'}_d) \in \llbracket \Gamma \vdash \Phi \colon
\bullet \rrbracket^\eta_\mathcal{T} \text{ such that } \\&q_i \overset{a}{\rightarrow} {q'}_i \text{ and for all } j
        \neq
        i \text{ holds } q_j = {q'}_j\}\\
        \llbracket \Gamma \vdash \Phi \vee \Psi \colon \tau \rrbracket^\eta_\mathcal{T} =\,& \llbracket \Gamma \vdash \Phi
        \colon \tau \rrbracket ^\eta_\mathcal{T} \sqcup_\tau \llbracket \Gamma \vdash \Psi \colon \tau \rrbracket ^\eta_\mathcal{T}\\
        \llbracket \Gamma \vdash \neg \Phi \colon \bullet \rrbracket^\eta_\mathcal{T} =\,& Q^d \setminus \llbracket
        \Gamma^- \vdash \Phi
        \colon \bullet \rrbracket ^\eta_\mathcal{T}\\
        \llbracket \Gamma \vdash \{\emph{j}\} \Phi \colon \bullet \rrbracket^\eta_\mathcal{T} =\,&
        \{\{(q_1, \dots, q_d) \in Q^d \mid \\ &(q_{j_1}, \dots, q_{j_d}) \in \llbracket \Gamma \vdash \Phi
        \colon \bullet
        \rrbracket ^\eta_\mathcal{T}\}\\
        \llbracket \Gamma, X \colon \tau \vdash X \colon \tau \rrbracket^\eta_\mathcal{T} =\,& \eta(X)\\
        \llbracket \Gamma \vdash \mu (X \colon \tau).\,\Phi \colon \tau \rrbracket^\eta_\mathcal{T} =\,&
        \bigsqcap\,_{\llbracket\tau\rrbracket_\mathcal{T}} \{\mathcal{X} \in \llbracket \tau \rrbracket_\mathcal{T}
        \mid \\
        &\llbracket \Gamma, X^+ \colon \tau \vdash \Phi \colon \tau \rrbracket^{\eta[X \mapsto \mathcal{X}]}_\mathcal{T}
        \leq_{\llbracket \tau \rrbracket_\mathcal{T}} \mathcal{X}\}\\
        %\llbracket         \Gamma
 %       \vdash
  %      \lambda X^+ \colon \tau.\Phi \colon \tau \rrbracket ^\eta_\mathcal{T}\\
        \llbracket \Gamma \vdash \lambda (X^v \colon \sigma).\,\Phi \colon \sigma^v \rightarrow \tau \rrbracket
        ^\eta_\mathcal{T} =\,& F \in \llbracket \sigma^v \rightarrow \tau \rrbracket_\mathcal{T} \text{ such that for
        all }
        \mathcal{X} \in \llbracket \sigma \rrbracket_\mathcal{T}.\, \\
        &F(\mathcal{X}) = \llbracket \Gamma, X^v \colon \sigma \vdash \Phi \colon \tau \rrbracket^{\eta[X \mapsto
        \mathcal{X}]}_\mathcal{T}\\
        \llbracket \Gamma \vdash \Phi\,\Psi \colon \tau \rrbracket^\eta_\mathcal{T} =\,& \llbracket \Gamma \vdash \Phi
        \colon \sigma
        ^v \rightarrow \tau \rrbracket ^\eta_\mathcal{T}(\llbracket \Gamma \vdash \Psi \colon \sigma \rrbracket ^\eta_\mathcal{T})
    \end{align*}
\end{figure}

To define the semantics of PHFL formulas we need a mapping $\eta$ that associates to each variable an element of its
type semantics, i.e. $\eta(X) \in \llbracket\tau\rrbracket_\mathcal{T}$ for $X$ of type $\tau$. Let $\Phi$ be a
well-typed formula of type $\tau$, $\mathcal{T}$ a LTS and $\eta$ a variable mapping, then the semantics
$\llbracket\Gamma \vdash \Phi\colon \tau \rrbracket^\eta_\mathcal{T}$ are defined inductively on $\Phi$ which maps to
an element of $\llbracket\tau\rrbracket_\mathcal{T}$ as explained in Figure~\ref{figure:phfl-semantics}.
Note, that $\eta[X \mapsto \mathcal{X}]$ is a mapping $\eta'$ that is equal to $\eta$ but $\eta'(X) = \mathcal{X}$.

In this thesis we are interested in PHFL formulas that have a specific order. For this, a formula $\Phi$ has order $k$
if $k = max(\{ord(\tau)\mid \Psi \colon \tau$ \textit{is a subformula of} $\Phi\})$. The set of formulas that have
order at most $k$ is denoted by PHFL$^k$.

\begin{example}{\cite{lange2014capturing}}
    \label{example:phfl_order_2}
    The following $2$-adic PHFL$^1$ formula $\Phi$ describes trace equivalence of two states in a LTS i.e. it denotes
    those pairs $(q_1, q_2)$ for which $q_1$ has the same traces as $q_2$ and vice versa.
    \begin{align*}
        \Phi = &(\mu (F \colon \bullet^0 \rightarrow (\bullet^0 \rightarrow \bullet)).\,
        \lambda (X \colon \bullet).\, \lambda (Y \colon \bullet).\, \\&X \Leftrightarrow Y \wedge
        \underset{a \in \Sigma}{\bigwedge} F \langle a \rangle_1 X \langle a \rangle_2 Y)\top \top
    \end{align*}
\end{example}

\begin{example}{\cite{lange2014capturing}}
    \label{example:phfl_order_0}
    The following $2$-adic PHFL$^0$ formula $\Phi_\sim$ describes bisimilarity i.e. it denotes
    those pairs $(q_1, q_2)$ such that $q_1 \sim q_2$ and vice versa.
    \begin{align*}
        \Phi_\sim = \nu (X \colon \bullet).\,
        \underset{a \in \Sigma}{\bigwedge} [a]_1 \langle a \rangle_2 X \wedge [a]_2 \langle a \rangle_1 X \wedge
        \underset{p \in P}{\bigwedge} p_1 \Leftrightarrow p_2
    \end{align*}
\end{example}

Before looking at the tail-recursive fragment see the following definition of $r$-adic queries that are associated to a
closed $d$-adic formula $\Phi$.

\begin{definition}
    Given a dimension $d$ of PHFL, a type judgment $\Gamma$, a variable assignment $\eta$ and a closed $d$-adic PHFL
    formula $\Phi$ we call a $r$-adic query $\mathcal{Q}^r_\Phi$ associated to $\Phi$ if there is for all LTS
    $\mathcal{T}$ and all $(q_1, \dots, q_r) \in {\mathcal{Q}^r_\Phi}^\mathcal{T}$ a $(s_1, \dots, s_d) \in
    \llbracket \Gamma \vdash \Phi \colon \bullet \rrbracket^\eta_\mathcal{T}$ such that $q_i = s_i$ for all $i \in
    \{1, \dots, min(\{r, d\})\}$.
\end{definition}

For example $\mathcal{Q}^2_{\Phi_\sim}$ is the same query as in Example~\ref{example:query_bisimulation}, where
$\Phi_\sim$ is the formula from Example~\ref{example:phfl_order_0}.

\subsection{Tail-Recursive PHFL}\label{subsec:tail-recursivePhfl}


\begin{figure}
    \caption{Derivation Rules for PHFL formulas that shall be tail-recursive.}
    \label{figure:phfl-tail-recursive}
    \begin{mathpar}
        \bar{Y} \vdash tail(p_i, \bar{X}) \and
        \inferrule{X \in \bar{X} \cup \bar{Y}}{\bar{Y} \vdash tail(X, \bar{X})} \and
        \inferrule{\bar{Y} \vdash tail(\Phi, \emptyset)}{\bar{Y} \vdash tail(\neg \Phi, \bar{X})} \and
        \inferrule{\bar{Y} \vdash tail(\Phi, \bar{X})}{\bar{Y} \vdash tail(\{\emph{j}\} \Phi,
        \bar{X})} \and
        \inferrule{\bar{Y} \vdash tail(\Phi, \bar{X}) \\ \bar{Y} \vdash tail(\Psi, \bar{X})}{\bar{Y} \vdash tail
        (\Phi \vee \Psi, \bar{X})} \and
        \inferrule{\bar{Y} \vdash tail(\Phi, \bar{X})}{\bar{Y} \vdash tail(\langle a \rangle_i \Phi, \bar{X})} \and
        \inferrule{\bar{Y} \vdash tail(\Phi, \emptyset)}{\bar{Y} \vdash tail([a]_i \Phi, \bar{X})} \and
        \inferrule{\bar{Y} \cup \{Z\} \vdash tail(\Phi, \bar{X})}{\bar{Y} \vdash tail(\lambda Z^v \colon \tau . \Phi,
        \bar{X})} \and
        \inferrule{\bar{Y} \vdash tail(\Phi, \emptyset) \\ \bar{Y} \vdash tail(\Psi, \bar{X})}{\bar{Y} \vdash tail
        (\Phi \wedge \Psi, \bar{X})} \and
        \inferrule{\bar{Y} \vdash tail(\Phi, \bar{X}) \\ \bar{Y} \vdash tail(\Psi, \emptyset)}{\bar{Y} \vdash tail
        (\Phi\,\Psi, \bar{X})} \and
        \inferrule{\bar{Y} \vdash tail(\Phi, \bar{X} \cup \{Z\})}{\bar{Y} \vdash tail(\mu Z \colon \tau . \Phi,
        \bar{X})} \and
        \inferrule{\bar{Y} \vdash tail(\Phi, \bar{X} \cup \{Z\})}{\bar{Y} \vdash tail(\nu Z \colon \tau . \Phi,
        \bar{X})}
    \end{mathpar}
\end{figure}

Next, we want to define a fragment of PHFL formulas. This fragment is called tail-recursive and ensures that
some combinations of subformulas do not appear in an PHFL formula. For this, let the logical connective
$\wedge$, the modality operators $[a]_i$ and the greatest fixpoint operator $\nu$ be further primitives of PHFL formula
syntax. Intuitively, tail-recursive PHFL formulas are PHFL formulas where all fixpoint variables do not occur freely
under the operators $\neg$ and $[a]_i$ nor in $\Psi$ of formulas of the type $\Phi\,\Psi$ or $\Psi \wedge \Phi$.

\begin{definition}
    A closed PHFL formula $\Phi$ is called \emph{tail-recursive} if $\emptyset \vdash tail(\Phi, \emptyset)$ is
    derivable via the rules in Figure~\ref{figure:phfl-tail-recursive}.
\end{definition}

The set of all tail-recursive PHFL formulas that have order at most $k$ is denoted by PHFL$^k_{tail}$.

\begin{example}{\cite{lange2014capturing}}
    Looking at Figure~\ref{figure:phfl-tail-recursive}, we can see, that $\Phi_1 = \mu X.[a]_1 X$ is not tail
    recursive, because $X$ occurs under $[a]_1$. Also $\Phi_2 = \mu F .\lambda X. (F X) \wedge (F(F X))$
    is not tail-recursive because $F$ occurs on the left side of the logical operator $\wedge$. The second reason why
    $\Phi_2$ is not tail-recursive is because $F$ also occurs in $F X$ of subformula $F (F X)$. An example of a
    formula that is tail-recursive is given by:
    \[\nu F. \lambda X. (F \langle a \rangle_1 X) \vee (X \wedge \langle a \rangle_2 (F X))\bot\]
\end{example}

%%
%% Author: Davidov
%% 27.04.2018
%%

\subsection{Descriptive Complexity}\label{subsec:descriptiveComplexity}

One way to describe complexity classes is with the help of \textit{Turing Machines}~\cite{hopcroft1994einfuehrung}.

\begin{definition}
    The seven-tuple $M = (Q, \Sigma, \Gamma, \delta, q_0, \Box, F, R)$ is called a \emph{Deterministic Turing Machine}
    ($\mathit{DTM}$),
    where
    \begin{compactitem}
        \item $Q$ is the finite set of states,
        \item $\Sigma$ is the input alphabet,
        \item $\Gamma$ is the working alphabet with $\Sigma \subset \Gamma$,
        \item $\delta : (Q \setminus (F \cup R)) \times \Gamma \rightarrow Q \times \Gamma \times \{L, R, N\}$ is the
        transition function,
        \item $q_0 \in Q$ is the initial state,
        \item $\Box \in \Gamma \setminus \Sigma$ is the blank symbol,
        \item $F \subseteq Q$ is the set of accepting states and
        \item $R \subseteq Q$ is the set of rejecting states.
    \end{compactitem}
\end{definition}

\begin{example}
    \label{example:dtm}
    As an example for a $\mathit{DTM}$ let
    \[M = (\{q_0, q_f, q_r\}, \{a, b\}, \{a, b, \Box\}, \delta, q_0, \Box, \{q_f\}, \{q_r\})\]
    where $\delta(q_0, a)= (q_0, \Box, R)$, $\delta(q_0, b) = (q_r, b, L)$, $\delta(q_0, \Box) = (q_f, \Box, N)$.
    $M$ is a $\mathit{DTM}$ that accepts all input words that contain no symbol $b$, i.e. $L(M) = \{a\}^*$.
\end{example}

Configurations are snapshots of $\mathit{DTM}$s working on an input word. This includes the working tape, the current
state and the current position of the reading head. Formally, $C_i^M(w) = \Gamma^m \cdot Q \cdot (\Gamma^n | \Box)$
is called the $i$-th configuration of a $\mathit{DTM}$ $M = (Q, \Sigma, \Gamma, \delta, q_0, \Box, F)$ for input word
$w \in \Sigma^*$, where $m \geq 0$ and $n \geq 1$. In addition, $\Gamma^m$ represents the word of all symbols left
from the reading head position to the last symbol that is unequal to the blank symbol. $\Gamma^n$ represents the word
of all symbols right of the reading head position to the last symbol that is unequal to the blank symbol. If there
are no symbols unequal to the blank symbol right of the reading head position, the blank symbol $\Box$ of $M$ is used
instead of $\Gamma^n$.

\begin{definition}
    Let $C_i^M(w) = \gamma_1^i\dots\gamma_{m_i}^i{q_i}{\gamma_1^i}'\dots{\gamma_{n_i}^i}'$, $C_j^M(w) =
    \gamma_1^j\dots\gamma_{m_j}^j{q_j}{\gamma_1^j}'\dots{\gamma_{n_j}^j}'$ two configurations of a $\mathit{DTM}$ $M = (Q, \Sigma,
    \Gamma,
    \delta, q_0, \Box, F)$ for input word $w \in \Sigma^*$ with $i \neq j$. $C_j^M(w)$ is the next configuration
    of $C_i^M(w)$ written as $C_i^M(w) \rightarrow_M C_j^M(w)$ iff $j = i + 1$ and
    \begin{compactitem}
        \item $m_j = m_i - 1$, $\gamma_1^j = \gamma_1^i, \dots \gamma_{m_j}^j = \gamma_{m_j}^i$, $n_j = n_i + 1$,
        ${\gamma_1^j}' = \gamma_{m_i}^i, {\gamma_2^j}' = a, {\gamma_3^j}' = {\gamma_2^i}' \dots {\gamma_{n_j}^j}' =
        {\gamma_{{n_j}- 1}^i}'$ and $\delta(q_i, {\gamma_1^i}') = (q_j, a, L)$ or
        \item $m_j = m_i + 1$, $\gamma_1^j = \gamma_1^i, \dots \gamma_{m_j-1}^j = \gamma_{m_j-1}^i, \gamma_{m_j}^j
        = a$, $n_j = n_i - 1$, ${\gamma_1^j}' = {\gamma_2^i}', \dots {\gamma_{n_j}^j}' = {\gamma_{{n_j}+1}^i}'$ and
$\delta (q_i, {\gamma_1^i}') = (q_j, a, R)$ or
        \item $m_j = m_i$, $\gamma_1^j = \gamma_1^i, \dots \gamma_{m_j}^j = \gamma_{m_j}^i$, $n_j = n_i$, ${\gamma_1^j}'
= a, {\gamma_2^j}' = {\gamma_2^i}' \dots {\gamma_{n_j}^j}' = {\gamma_{n_j}^i}'$ and $\delta
        (q_i, {\gamma_1^i}') = (q_j, a, N)$.
    \end{compactitem}
    $C_i^M(w) \rightarrow_M C_j^M(w)$ is called a \emph{transition} of $M$ on $w$.
\end{definition}

The start configuration for an input word $w$ of a $\mathit{DTM}$ $M$ is $C_0^M(w) = q_0w$, where $q_0$ is the
initial state of $M$. A run of $\mathit{DTM}$ $M$ on input word $w$ is a sequence of transitions, $C_0^M(w)
\rightarrow_M C_1^M(w) \rightarrow_M C_2^M(w) \rightarrow_M \dots \rightarrow_M C_n^M(w)$, where $C_n^M(w)$ includes
either an accepting or rejecting state. A run is accepting if there is a $n \in \mathcal{N}$ such that $C_0^M(w)
\rightarrow_M C_1^M(w) \rightarrow_M \dots \rightarrow_M C_n^M(w)$ and $C_n^M(w)$ contains an accepting state of $M$.

\begin{example}
    \label{example:run_of_dtm}
    Let $M$ be the $\mathit{DTM}$ from Example~\ref{example:dtm} and $w_1 = aaba$, $w_2 = aaaa$ two input words. The
    run of $M$ on $w_1$ is
    \[C_0^M(w_1) = q_0aaba \rightarrow_M q_0aba \rightarrow_M q_0ba \rightarrow_M q_r\Box ba = C_3^M(w_1)\]
    and the run of $M$ on $w_2$ is
    \[C_0^M(w_2) = q_0aaaa \rightarrow_M q_0aaa \rightarrow_M q_0aa \rightarrow_M q_0a \rightarrow_M q_0\Box
    \rightarrow_M q_f\Box = C_5^M(w_2)\]
\end{example}

Remark, that it is possible to define $\mathit{DTM}$s that does not accept or reject any input word. For example, let
$M = (\{q_0\}, \{a\}, \{a, \Box\}, \delta, \emptyset, \emptyset)$, where $\delta(q_0, x) = (q_0, x, N)$ with $x \in
\{a, \Box\}$. $M$ is a DTM where any calculation of an input word $w$ looks as follows $q_0w \rightarrow_M q_0w
\rightarrow_M \dots$. It never reaches an accepting or rejecting state. In this thesis, we are only interested in
$\mathit{DTM}$s that reaches an accepting or rejecting state in finite time on any input word.

Known from computational complexity theory~\cite{papadimitriou1994complexity}, the time and space classes
can be defined by functions. These functions have as input a natural number that represents the length of an input
word of a $\mathit{DTM}$. In case of time classes the output of the functions depends on the number of configuration
steps. In case of space classes the output is based on the longest transition.

\begin{definition}
    Let $M$ be a $\mathit{DTM}$. $\mathit{TIME}(n):= max(\mathit{STEPS}(w)\mid |w| = n)$, where $\mathit{STEPS}(w)$
    is the number of transitions while $M$ computes $w$. $\mathit{SPACE}(n) := max(\mathit{STORAGE}(w)\mid |w| = n)$,
    where $\mathit{STORAGE}(w) := max(length(C_i^M)\mid i\in\{1, \dots m\})$ is the longest configuration while $M$
    computes $w$ and $C_0^M(w) \rightarrow_M C_1^M(w) \rightarrow_M \dots \rightarrow_M C_m^M(w)$ is a run of $M$ on
    $w$.
\end{definition}

\begin{example}
    
\end{example}

Now, it is possible to group the $\mathit{DTM}$s by functions. Those groups are our computational complexity classes.
In this thesis we are interested in exponential time classes and exponential space classes.

\begin{definition}
    Let $f: \mathbb{N} \rightarrow \mathbb{N}$ be a polynomial function, then $\emph{exp}: \mathbb{N} \times \mathbb{N}
    \rightarrow \mathbb{N}$ is a function defined inductive as follows:
    \begin{compactitem}
        \item $exp(0, f(n)) = f(n)$,
        \item $exp(i, f(n)) = 2^{exp(i - 1, f(n))}$ for $i \geq 1$.
    \end{compactitem}
\end{definition}

With the help of function $exp$, we are now able to define the complexity classes $k$-EXPTIME and $k$-EXPSPACE for
all $k \geq 1$.

\begin{definition}
    Let $k \in \mathbb{N} \setminus \{0\}$ and $f: \mathbb{N} \rightarrow \mathbb{N}$ be a polynomial function then
    $k$-EXPTIME $=$ $\mathit{TIME}$($exp(k, f(n))$) and $k$-EXPSPACE $= \mathit{SPACE}(exp(k, f(n)))$.
\end{definition}

Remark, that $\mathit{TIME}$ is the maximal number of transitions and $\mathit{SPACE}$ the biggest configuration of a
$\mathit{DTM}$ while computing a word.

From Definition~\ref{definition:bisimulationInvariant}, Definition~\ref{definition:queryBelongsToComplexityClass}
and the complexity classes $k$-EXPTIME and $k$-EXPSPACE follow the queries that we want to investigate.

\begin{definition}
    \label{definition:kExptimekExpspace}
    \exptime{$k$} are the bisimulation invariant queries that belong to complexity class $k$-EXPTIME and
    \expspace{$k$} are the bisimulation invariant queries that belong to complexity class $k$-EXSPACE, where $k \geq 1$.
\end{definition}

The main aim of \emph{descriptive complexity} is to describe the complexity classes known from
computational complexity theory with logics. While computational complexity theory distinguishes time and space
classes, descriptive complexity theory characterizes classes with logical resources instead of a reference to
automaton models or space and time bounds.

The first known result in the area of descriptive complexity comes from Fagin. In 1974 he showed that the complexity
class NP coincides with $\exists SO$~\cite{fagin1974generalized}, the existential fragment of second-order logic.

In the next chapter we introduce the polyadic higher order fixed point logic and a restriction called tail-recursive
devised by Lange and Lozes in~\cite{lange2014capturing}. We want to compare the in
Definition~\ref{definition:kExptimekExpspace} presented queries with these logics to make a further contribution in
the theory of descriptive complexity.

%%
%% Author: Davidov
%% 16.05.2018
%%


\subsection{Higher Order Logic}\label{subsec:higherOrderLogic}

For comparing the complexity classes with PHFL, we have to detour over combinations of extensions of FO. The first well
known extension is called Higher Order Logic~\cite{vanBenthem2001higher}, abbreviated with HO. In HO we
increase the expressive power of FO by allowing relation variables over any order. For this, we have to define the
types of higher order variables.

\begin{definition}
    \emph{HO types} are defined inductive as follows:
    \begin{compactitem}
        \item $\tau = \odot$ is a HO type,
        \item $\tau = (\tau_1, \dots, \tau_n)$ is a HO type, if $\tau_1, \dots, \tau_n$ are
        HO types.
    \end{compactitem}
\end{definition}

The HO type of individuals is $\tau = \odot$. These objects have the \textit{HO order} $0$. The HO type $\tau = (\tau_1,
\dots, \tau_n)$ is that of relations between objects of HO types $\tau_1, \dots, \tau_n$ and have the HO order $1 + max
(order(\tau_1), \dots, order(\tau_n))$.

For each HO type we have a countable infinite set of variables. The \textit{terms} of HO type $\odot$ are
generated as in FO. Terms of a higher HO type are just a variable of that HO type. Furthermore, let $\sigma$ be
a signature holding the constants, functions and relations of arbitrary HO types.

\begin{definition}
    The set of \emph{HO formulas} over $\sigma$ is defined inductively as follows:
    \begin{compactitem}
        \item $X = Y$ is a HO formula over $\sigma$ if $X$ and $Y$ are variables of the same type,
        \item $R(t_1, \dots, t_n)$ is a HO formula over $\sigma$ if $R \in \sigma$ is a relation with arity $n$ and
        $t_1, \dots, t_n$ are terms of type $\odot$,
        \item $X(t_1, \dots, t_n)$ is a HO formula over $\sigma$ if $X$ is a variable of type $(\tau_1, \dots, \tau_n)$
        and $t_i$ is a term over $\tau_i$, with $i \in \{1, \dots, n\}$,
        \item if $\varphi$ and $\psi$ are two HO formulas over $\sigma$, then $\neg\varphi$, $\varphi\wedge\psi$ and $\varphi
        \vee \psi$ are also HO formulas over $\sigma$,
        \item if $\varphi$ is a HO formula over $\sigma$ and $X$ a variable of arbitrary type, then $\exists X\varphi$ and
        $\forall X\varphi$ are also HO formulas over $\sigma$.
    \end{compactitem}
\end{definition}

The first step in direction of the semantics of HO formulas is to interpret the universes of the different HO types.

\begin{definition}
    Let $\mathcal{U}$ be the global universe. The universes of the HO types are defined inductively as follows:
    \begin{compactitem}
        \item $D_\odot(\mathcal{U}) = \mathcal{U}$,
        \item $D_{(\tau_1, \dots, \tau_n)}(\mathcal{U}) = \mathcal{P}(D_{\tau_1} \times \dots \times D_{\tau_n})$
    \end{compactitem}
\end{definition}

Moreover, $\alpha$ is a function that assigns all variables to an element of the appropriated universe, i.e. if
variable $X$ is of type $\tau$, then $\alpha(X) \in D_{\tau}(\mathcal{U})$.

\begin{definition}
    Let $\mathcal{A}$ be a $\sigma$-structure and $\alpha$ a variable allocation over universe $\mathcal{U}$. The
    satisfaction of a HO formula is defined inductively as follows:
    \begin{compactitem}
        \item $\mathcal{A}, \alpha \models (X = Y)$ iff $\alpha(X) = \alpha(Y)$
        \item $\mathcal{A}, \alpha \models R(t_1, \dots t_n)$ iff $(t_1^{\mathcal{A}}[\alpha], \dots
        t_n^{\mathcal{A}}[\alpha]) \in R^{\mathcal{A}}$, where $t_i^{\mathcal{A}}[\alpha]$ is the value of term
        $t_i$ under $\alpha$ in $\mathcal{A}$ defined as in FO,
        \item $\mathcal{A}, \alpha \models X(t_1, \dots t_n)$ iff $(t_1^{\mathcal{A}}[\alpha], \dots
        t_n^{\mathcal{A}}[\alpha]) \in \alpha(X)$, where $t_i^{\mathcal{A}}[\alpha]$ is the value of term
        $t_i$ under $\alpha$ in $\mathcal{A}$ defined as in FO,
        \item $\mathcal{A}, \alpha \models \neg\varphi$ iff $\mathcal{A}, \alpha\not\models\varphi$,
        \item $\mathcal{A}, \alpha \models \varphi \wedge \psi$ iff $\mathcal{A}, \alpha\models\varphi$ and $\mathcal{A},
        \alpha\models\psi$,
        \item $\mathcal{A}, \alpha \models \varphi \vee \psi$ iff $\mathcal{A}, \alpha\models\varphi$ or $\mathcal{A},
        \alpha\models\psi$,
        \item $\mathcal{A}, \alpha \models \exists X\varphi$ iff there is an allocation of $X$, $a = \alpha(X) \in D_{\tau}
        (\mathcal{U})$, that $\mathcal{A}, \alpha[X/a] \models \varphi$, where $\tau$ is the type of $X$ and
        \item $\mathcal{A}, \alpha \models \forall X\varphi$ iff all possible allocations of $X$, $a = \alpha(X) \in
        D_{\tau}(\mathcal{U})$ holds that $\mathcal{A}, \alpha[X/a] \models \varphi$, where $\tau$ is the type of $X$.
        \end{compactitem}
\end{definition}

We can categorize the formulas by the order of all occurring variables. With HO$^k$ we mean the set of all those
formulas whose variables have order less or equal $k$.

Another possibility to extend FO is to add operators that are not expressible in FO. Here, we are interested in two
of them, the least fixpoint and the partial fixpoint operator. Instead of defining the operators for FO we are
here interested to define this operators for HO. First, we regard the least fixpoint operator.

\begin{definition}
   Let $F: \mathscr{P}(A) \rightarrow \mathscr{P}(A)$ be an operator on a finite set $A$, then $X \in \mathscr{P}(A)$
   is called a \emph{fixpoint} of $F$ if $F(X) = X$. Let $X$ be a fixpoint of $F$, then $X$ is called the \emph{least
   fixpoint} of $F$, abbreviated as LFP($F$), if all other fixpoints $Y$ of $F$ including $X$ i.e. $X \subseteq Y$.
\end{definition}

Like in~\cite{freireMartins2011descriptive} we want to define special operators that are working on HO type
universes.

\begin{definition}
    Let $\sigma$ an arbitrary signature, $X$ a relation variable of HO type $\tau = (\tau_1, \dots, \tau_k)$,
    $\tau_1, \dots \tau_k$ arbitrary HO types, $x_1$ a variable of HO type $\tau_1$, \dots, $x_k$ a
    variable of HO type $\tau_k$ and $\varphi(X, x_1, \dots, x_k)$ a formula over $\sigma$ with free variables $X, x_1,
    \dots, x_k$. Each $\sigma$-structure $\mathcal{A}$ with universe $\mathcal{U}$ induces the operator
    \begin{align*}
        F_\varphi^\mathcal{A} : \mathscr{P}(D_\tau(\mathcal{U})) &\longrightarrow \mathscr{P}(D_\tau(\mathcal{U}))\\
        A &\longmapsto F_\varphi^\mathcal{A}(A) := \{(a_1, \dots, a_k) \mid \mathcal{A} \models \varphi(A, a_1, \dots, a_k)\}
    \end{align*}
\end{definition}

From Knaster-Tarski~\cite{tarski1955lattice} we know that for an operator $F: \mathscr{P}(A) \rightarrow \mathscr{P}
(A)$ the LFP($F$) exists if $F$ is monoton, that means for all $X, Y \subseteq A$ if $X \subseteq Y$ then $F(X)
\subseteq F(Y)$ holds. To make $F_\varphi^\mathcal{A}$ monoton we have to restrict $\varphi(X, x_1, \dots, x_k)$ in this way
that variable $X$ occurs in an even number of negations within of $\varphi$~\cite{freireMartins2011descriptive}. Those
functions are called \textit{positive in} $X$. With these information we are able to define the least fixpoint
operator for HO formulas, denoted by HO(LFP).

\begin{definition}
    Let $\sigma$ be a signature. The set of \emph{HO(LFP) formulas} enhances the set of HO formulas with the
    following formation rule:
    \begin{compactitem}
        \item $[LFP_{X, x_1, \dots, x_k}\varphi(X, x_1, \dots, x_k)](v_1, \dots, v_k)$ is a HO(LFP) formula over
        $\sigma$ with free variables $v_1, \dots, v_k$ iff $\varphi(X, x_1, \dots, x_k)$ is a HO(LFP) formula with
        free variables $X, x_1, \dots, x_k$, where $\varphi$ is positive in $X$, $X$ has HO type $\tau = (\tau_1,
        \dots, \tau_k)$, $x_1$ and $v_1$ have HO type $\tau_1$, \dots, $x_k$ and $v_k$ have HO type $\tau_k$ and
        $\tau_1, \dots \tau_k$ are arbitrary.
    \end{compactitem}
\end{definition}

As in HO, with HO(LFP)$^k$ we mean the set of all those HO(LFP) formulas whose variables have order less or equal $k$.

\begin{definition}
    Let $\mathcal{A}$ be a $\sigma$-structure and $\alpha$ a variable allocation over universe $\mathcal{U}$. The
    satisfaction of a HO(LFP) formula extends that of HO formulas with the following definition:
    \begin{compactitem}
        \item $\mathcal{A}, \alpha \models [LFP_{X, x_1, \dots, x_k}\varphi(X, x_1, \dots, x_k)](v_1, \dots, v_k)$
        iff $(\alpha(v_1), \dots, \alpha(v_k)) \in LFP(F_\varphi^\mathcal{A})$.
    \end{compactitem}
\end{definition}

Next, we define the partial fixpoint operator for HO formulas like in~\cite{schewe2006fixpoint}. For this, we define
first the partial fixpoint operator generally.

\begin{definition}
    Let $F: \mathscr{P}(A) \rightarrow \mathscr{P}(A)$ be an operator on a finite set $A$, then the \emph{partial
    fixpoint} of $F$, abbreviated as PFP($F$), is defined as follows:
    \[PFP(F):=\begin{cases}
               F^{i+1}(\emptyset)=F^i(\emptyset),  & \text{if such } i >= 0 \text{ exists}\\
               \emptyset, & \text{otherwise,}
    \end{cases}\]
    where $F^0(\emptyset) = \emptyset$, $F^1(\emptyset) = F(\emptyset)$, $F^2(\emptyset) = F(F(\emptyset))$, and so on.
\end{definition}

With this knowledge we can define and add the partial fixpoint operator to HO formulas, denoted as HO(PFP).

\begin{definition}
    Let $\sigma$ be a signature. The set of \emph{HO(PFP) formulas} enhances the set of HO formulas with the
    following formation rule:
    \begin{compactitem}
        \item $[PFP_{X, x_1, \dots, x_k}\varphi(X, x_1, \dots, x_k)](v_1, \dots, v_k)$ is a HO(PFP) formula over
        $\sigma$ with free variables $v_1, \dots, v_k$ iff $\varphi(X, x_1, \dots, x_k)$ is a HO(PFP) formula with
        free variables $X, x_1, \dots, x_k$, where $X$ has HO type $\tau = (\tau_1, \dots, \tau_k)$, $x_1$ and $v_1$
        have HO type $\tau_1$, \dots, $x_k$ and $v_k$ have HO type $\tau_k$ and $\tau_1, \dots \tau_k$ are arbitrary.
    \end{compactitem}
\end{definition}

HO(PFP)$^k$ is the set of all those HO(PFP) formulas whose variables have order less or equal $k$.

\begin{definition}
    Let $\mathcal{A}$ be a $\sigma$-structure and $\alpha$ a variable allocation over universe $\mathcal{U}$. The
    satisfaction of a HO(PFP) formula extends that of HO formulas with the following definition:
    \begin{compactitem}
        \item $\mathcal{A}, \alpha \models [PFP_{X, x_1, \dots, x_k}\varphi(X, x_1, \dots, x_k)](v_1, \dots, v_k)$
        iff $(\alpha(v_1), \dots, \alpha(v_k)) \in PFP(F_\varphi^\mathcal{A})$.
    \end{compactitem}
\end{definition}

In the next chapter we want to compare the here defined HO(PFP) and HO(LFP) with PHFL from
Chapter~\ref{subsec:polyadicHigherOrderFixpointLogic}.

%%
%% Author: DKron
%% 24.07.2018
%%

\chapter{Upper Bounds}\label{ch:upperBounds}

In this chapter we want to regard the upper bounds of PHFL$^k$ and PHFL$^k_{tail}$. First, we show that the upper
bound of PHFL$^k$ is \exptime{$k$}.

\begin{definition}
    Let $\mathcal{T}$ a LTS, $\emph{s}$ a state tuple, $\eta$ a variable mapping, $\Gamma$ a type environment and
    $\varphi$ a PHFL$^k$ formula of type $\bullet$ then we call $\mathcal{T}$ with $\emph{s}$ a model of $\varphi$,
    written as $\mathcal{T}, \emph{s} \models \varphi$ iff $\emph{s} \in \llbracket \Gamma
    \vdash \varphi \colon \bullet \rrbracket^\eta_\mathcal{T}$.
\end{definition}

To show that the upper bound of PHFL$^k$ is \exptime{$k$} we want to make a reduction from the model checking
problem of PHFL$^k$ to the model checking problem of HFL$^k$. Remember, that HFL$^k$ is the set of $1$-adic PHFL$^k$
formulas. In combination to the following theorem we get the upper bound of PHFL$^k$.

\begin{theorem}{\cite{axelsson2007complexity}}
    \label{theorem:hfl_k_in_k_exptime}
    Given a LTS $\mathcal{T}$, a state $s$ and a HFL$^k$ formula $\varphi$, the model checking problem $\mathcal{T}, s
    \models \varphi$ is in $k$-EXPTIME, where $k > 0$.
\end{theorem}

In the following, we want to reduce the semantics of PHFL$^k$ to the semantics of HFL$^k$.
For this we have to transform the input LTS $\mathcal{T}$ and the input PHFL$^k$ formula $\varphi$ of the problem
$\mathcal{T}, \emph{s} \overset{?}{\models} \varphi$. We first define a mapping that transforms an LTS to another
LTS and give an example for this transforming process. In the next step we define a function that transforms a
PHFL$^k$ type to a HFL$^k$ type. After this, we define a further function that transforms a PHFL$^k$ formula to a
HFL$^k$ formula and give an example for this transformation. At last, we show that the semantics of the original
formula with the original types and the original LTS in PHFL$^k$ context coincides with the semantics of the
transformed formula with the transformed types and the transformed LTS in HFL$^k$ context.

As mentioned above we first define a mapping that transforms a given LTS.

\begin{definition}
    \label{definition:lts_transformation}
    Let $d \in \mathbb{N}$ the dimension of PHFL and $\mathcal{T} = (Q, \Sigma, P, \Delta, v)$ a LTS, then
    $\mathcal{T}_d = (Q^d, \Sigma_d \cup S_d, P_d, \Delta_d, v_d)$, where
    \begin{compactitem}
        \item $\Sigma_d = \underset{a \in \Sigma}{\bigcup}(\underset{i = 1}{\overset{d}{\bigcup}} \{a_i\})$,
        \item $S_d = \{s_{(e(1), \dots, e(d))} \mid e: \{1, \dots d\} \rightarrow \{1, \dots, d\}\}$,
        \item $P_d = \underset{p \in P}{\bigcup}(\underset{i = 1}{\overset{d}{\bigcup}} \{p_i\})$,
        \item $\Delta_d = \{((q_1, \dots ,q_{i - 1}, q_i, q_{i + 1}, \dots, q_d), a_i, (q_1, \dots ,q_{i - 1},
        {q_i}', q_{i + 1}, \dots, q_d)) \mid (q_i, a, {q_i}') \in \Delta\}$

        $\cup\,\{((q_{e(1)}, \dots, q_{e(d)}), s_{(e(1),
        \dots, e(d))}, (q_1, \dots, q_d)) \mid e: \{1, \dots, d\} \rightarrow \{1, \dots, d\}\}$ and
        \item $v_d \colon Q^d \rightarrow 2^{P_d}, $
        $v_d((q_1, \dots, q_d)) = \underset{i = 1}{\overset{d}{\bigcup}} \{p_i \mid p \in v(q_i)\}$,
    \end{compactitem}
\end{definition}

The following example shows the transforming of a LTS by Definition~\ref{definition:lts_transformation}.

\begin{example}
    Let $\mathcal{T}$ a LTS given by
    \begin{center}
        \begin{tikzpicture}[]
            \node [place] (q1) {$1$};
            \node  (temp1) [left=of q1] {};
            \node  (label) [left=of q1] {$\mathcal{T}:$};
            \node  (temp2) [right=of q1] {};
            \node [place] (q2) [below=of temp2,label=left:$p$] {$2$}
            edge [pre] node[left] {a} (q1);
            \node [place] (q3) [right=of temp2,label=right:$q$] {$3$}
            edge [pre] node[auto, swap] {b} (q1)
            edge [pre] node[right] {c} (q2);
        \end{tikzpicture}
    \end{center}
    Let $d = 2$ then $\mathcal{T}_d$ is by construction of Definition~\ref{definition:lts_transformation} the
    following LTS. Note, that for readability not all edges are drawn in this representation of $\mathcal{T}_d$.
    The edges that are missing are those that uses action $s_{(1, 2)}$ (except $(1, 1)$ to $(1,
    1)$) and the following edges are also not drawn
    $\{(1, 2) \overset{s_{(1, 1)}}{\rightarrow} (1, 1),
    (1, 3) \overset{s_{(1, 1)}}{\rightarrow} (1, 1),
    (1, 3) \overset{s_{(2, 1)}}{\rightarrow} (3, 1),
    (2, 1) \overset{s_{(2, 2)}}{\rightarrow} (1, 1),
    (2, 2) \overset{S_d}{\rightarrow} (2, 2),
    (2, 3) \overset{s_{(2, 1)}}{\rightarrow} (3, 2),
    (3, 1) $ $\overset{s_{(2, 1)}}{\rightarrow} (1, 3),
    (3, 1) \overset{s_{(2, 2)}}{\rightarrow} (1, 1),
    (3, 2) \overset{s_{(2, 1)}}{\rightarrow} (2, 3),
    (3, 3) \overset{S_d}{\rightarrow} (3, 3),
    (1, 2) \overset{b_1}{\rightarrow} (3, 2),
    $ $(1, 3) \overset{b_1}{\rightarrow} (3, 3),
    (2, 1) \overset{b_2}{\rightarrow} (2, 3),
    (3, 1) \overset{b_2}{\rightarrow} (3, 3)
    \}$, where $q \overset{S_d}{\rightarrow} q' =
    \{q \overset{a}{\rightarrow} q' \mid a \in S_d\}$.
    \begin{center}
        \begin{tikzpicture}[]
            \node [place] (q11) {$(1, 1)$}
            edge [loop] node[above] {$S_d$} ();
            \node (label) [left=of q11] {$\mathcal{T}_d:$};
            \node (temp1) [right=of q11] {};
            \node [place] (q12) [right=of temp1,label=above:{$p_2$}] {$(1, 2)$}
            edge [pre] node[above] {$a_2$} (q11);
            \node  (temp2) [right=of q12] {};
            \node [place] (q13) [right=of temp2, label=above:{$q_2$}] {$(1, 3)$}
            edge [pre] node[above] {$c_2$} (q12)
            edge [pre, bend right=40] node[above] {$b_2$} (q11);
            \node [place] (q21) [below=of q11,label=left:{$p_1$}] {$(2, 1)$}
            edge [post, bend right=10] node[auto, swap] {$s_{(2, 1)}$} (q12)
            edge [pre, bend left=10] node[auto] {$s_{(2, 1)}$} (q12)
            edge [pre] node[left] {$a_1$} (q11);
            \node [place] (q22) [below=of q12,label=right:{$p_1, p_2$}] {$(2, 2)$}
            edge [pre] node[below] {$a_2, s_{(1, 1)}$} (q21)
            edge [pre, bend left=25] node[right] {$a_1, s_{(2, 2)}$} (q12);
            \node [place] (q23) [below=of q13,label=right:{$p_1, q_2$}] {$(2, 3)$}
            edge [pre, bend left=30] node[below] {$c_2$} (q22)
            edge [post, bend right=30] node[above] {$s_{(1, 1)}$} (q22)
            edge [pre] node[right] {$a_1$} (q13);
            \node [place] (q31) [below=of q21,label=below:{$q_1$}] {$(3, 1)$}
            edge [pre, bend left=80] node[left] {$b_1$} (q11)
            edge [pre] node[left] {$c_1$} (q21);
            \node [place] (q32) [below=of q22,label=below:{$q_1, p_2$}] {$(3, 2)$}
            edge [pre] node[above] {$a_2$} (q31)
            edge [pre, bend left=25] node[left] {$c_1$} (q22)
            edge [post, bend right=25] node[right] {$s_{(2, 2)}$} (q22);
            \node [place] (q33) [below=of q23,label=below:{$q_1, q_2$}] {$(3, 3)$}
            edge [pre] node[below] {$c_2, s_{(1, 1)}$} (q32)
            edge [pre] node[right] {$c_1, s_{(2, 2)}$} (q23)
            edge [pre, bend right=100] node[right] {$s_{(2, 2)}$} (q13)
            edge [pre, bend left=40] node[right, below] {$s_{(1, 1)}$} (q31);
        \end{tikzpicture}
    \end{center}
\end{example}

The next step is to define a function that maps a PHFL type to a HFL type.

\begin{definition}
    \label{definition:type_function}
    Let $\tau_{PHFL}$, $\sigma_{PHFL}$ arbitrary PHFL types, $\bullet_{PHFL}$ the base type of PHFL, $\bullet_{HFL}$
    the base type of HFL and $v$ an arbitrary variance, then $T$ is a function that maps a PHFL type to a HFL type
    defined inductive over the type of PHFL as follows:
    \begin{align*}
        T(\bullet_{PHFL}) &= \bullet_{HFL},\\
        T(\sigma_{PHFL}^v \rightarrow \tau_{PHFL}) &= T(\sigma_{PHFL})^v \rightarrow T(\tau_{PHFL})
    \end{align*}
\end{definition}

The type function $T$ of Definition~\ref{definition:type_function} can be adapted on type environments. If
$\Gamma = X_1^{v_1} \colon \tau_1, \dots, X_n^{v_n} \colon \tau_n$ is a type environment, then $T(\Gamma) =
X_1^{v_1} \colon T(\tau_1), \dots, X_n^{v_n} \colon T(\tau_n)$.

\begin{definition}
    \label{definition:variable_mapping_function}
    Let $\mathcal{T}$ a LTS, $d \in \mathbb{N}$ the dimension of PHFL and $\mathcal{T}_d$ the transformed LTS by
    Definition~\ref{definition:lts_transformation} and $T$ the type mapping of
    Definition~\ref{definition:type_function}, then if $\eta$ is a variable mapping over $\mathcal{T}$ for PHFL
    formulas, then $V(\eta)$ is a variable mapping over $\mathcal{T}_d$ for HFL formulas, where $V(\eta)$ is defined
    as if $\eta(X) = \mathcal{X}$ where $\mathcal{X} \in \llbracket \tau \rrbracket_\mathcal{T}$ then $V(\eta)(X)$ is
    exactly the same set $\mathcal{X}$. Note, that for $\mathcal{X}$ it holds that $\mathcal{X} \in \llbracket \tau
    \rrbracket_\mathcal{T}$ iff $\mathcal{X} \in \llbracket T(\tau)\rrbracket_{\mathcal{T}_d}$ because of definition
    of $\mathcal{T}_d$ and $T(\tau)$.
\end{definition}

Now, we are ready to define the function that maps a PHFL$^k$ formula to a HFL$^k$ formula.

\begin{definition}
    \label{definition:formula_function}
    Let $T$ be the type function from Definition~\ref{definition:type_function} and let $P$ a set of propositions,
    $\Sigma$ a set of actions and $\mathcal{V} = \{X_1, \dots, X_n\}$ a finite set of variables for a $d$-adic
    PHFL$^k$ formula $\varphi$, then $F$ is a function that maps a $d$-adic PHFL$^k$ formula over $P$, $\Sigma$ and
    $\mathcal{V}$ to a HFL$^k$ formula over proposition set $P_d = \underset{p \in P}{\bigcup}(\underset{i =
    1}{\overset{d}{\bigcup}} \{p_i\})$, action set $\Sigma_d \cup S_d = \underset{a \in \Sigma}{\bigcup}(\underset{i =
    1}{\overset{d}{\bigcup}} \{a_i\}) \cup \{s_{(e(1), \dots, e(d))} \mid e: \{1, \dots d\} \rightarrow \{1, \dots,
    d\}\}$ and variable set $\mathcal{V}$ which is defined inductive over the $d$-adic PHFL$^k$ formula as follows:
    \begin{align*}
        F(\top) &= \top\\
        F(X) &= X\\
        F(p_i) &= p_i\\
        F(\langle a \rangle_i \psi) &= \langle a_i \rangle F(\psi) \\
        F(\psi \vee \psi') &= F(\psi) \vee F(\psi') \\
        F(\neg \psi) &= \neg F(\psi) \\
        F(\{\emph{j}\,\} \psi) &= \langle s_{\emph{j}} \rangle F(\psi)  \\
        F(\mu (X \colon \tau).\,\psi) &= \mu (X \colon T(\tau)).\,F(\psi) \\
        F(\lambda (X^v \colon \tau).\, \psi) &= \lambda (X^v \colon T(\tau)).\, F(\psi) \\
        F(\psi{\psi'}) &= F(\psi)F({\psi'})
    \end{align*}
\end{definition}

\begin{example}
    As an example we want to transform the $2$-adic PHFL$^2$ formula of Example~\ref{example:phfl_order_2} to a HFL$^2$ formula.
    \begin{align*}
        \varphi = &(\mu (F \colon \bullet_{PHFL}^0 \rightarrow (\bullet_{PHFL}^0 \rightarrow \bullet_{PHFL})).\,
        \lambda (X \colon \bullet_{PHFL}).\, \lambda (Y \colon \bullet_{PHFL}).\, \\&X \Leftrightarrow Y \wedge
        \underset{a \in \Sigma}{\bigwedge} F \langle a \rangle_1 X \langle a \rangle_2 Y)\top \top
    \end{align*}
    will be transformed to
\begin{align*}
    F(\varphi) = &(\mu (F \colon \bullet_{HFL}^0 \rightarrow (\bullet_{HFL}^0 \rightarrow \bullet_{HFL})).\,
    \lambda (X \colon \bullet_{HFL}).\, \lambda (Y \colon \bullet_{HFL}).\, \\&X \Leftrightarrow Y \wedge \underset{a
    \in \Sigma}{\bigwedge} F \langle a_1 \rangle X \langle a_2 \rangle Y)\top \top.
\end{align*}
\end{example}

\begin{remark}
    It holds for type environment $\Gamma$ and PHFL$^k$ formula $\Phi$ of PHFL$^k$ type $\tau$ that if $\Gamma \vdash
    \Phi \colon \tau$ is derivable, then $T(\Gamma) \vdash F(\Phi) \colon T(\tau)$ is also derivable. This statement
    is easy proved by induction over the structure of formula $\Phi$.
\end{remark}

We are now interested to prove the following lemma that shows that the semantics of PHFL$^k$ is
reducible to the semantics of HFL$^k$.

\begin{lemma}
    \label{lemma:model_check_phfl_k}
    Let $\mathcal{T}$ a LTS, $\eta$ a variable mapping, $\Gamma$ a type environment, $\varphi$ a well-typed $d$-adic
    PHFL$^k$ formula of type $\tau$, $\mathcal{T}_d$ the LTS transformed by process of
    Defintion~\ref{definition:lts_transformation}, $T$ the type function of Defintion~\ref{definition:type_function},
    $V$ the variable mapping function of Definition~\ref{definition:variable_mapping_function}
    and $F$ the formula function of Definition~\ref{definition:formula_function} then $\llbracket \Gamma \vdash
    \varphi \colon \tau \rrbracket^\eta_\mathcal{T} = \llbracket T(\Gamma) \vdash F(\varphi) \colon T(\tau)
    \rrbracket^{V(\eta)}_{\mathcal{T}_d}$
\end{lemma}

\begin{proof}
    We show that $\llbracket \Gamma \vdash \varphi \colon \tau \rrbracket^\eta_\mathcal{T} = \llbracket T(\Gamma)
    \vdash F(\varphi) \colon T(\tau) \rrbracket^{V(\eta)}_{\mathcal{T}_d}$ by induction on formula $\varphi$.
    \begin{compactitem}
        \item In case of $\varphi = \top$, $\llbracket \Gamma \vdash \varphi \colon \bullet_{PHFL}
        \rrbracket^\eta_\mathcal{T}$ is the set of $d$-tuples of state set $Q$ of $\mathcal{T}$ what means that
        \[\llbracket \Gamma \vdash \varphi \colon \bullet_{PHFL} \rrbracket^\eta_\mathcal{T} = Q^d.\]
        By construction of $\mathcal{T}_d$ the set of states is also the set of $d$-tuples of state set $Q$.
        Moreover, by formula function $F$ it is true that $F(\top) = \top$. Because type environment $\Gamma$, the
        type $\bullet_{PHFL}$ and variable mapping $\eta$ do not matter in the semantics of $\llbracket \Gamma \vdash
        \varphi \colon \bullet_{PHFL} \rrbracket^\eta_\mathcal{T}$, $T(\Gamma)$, $T(\bullet_{PHFL})$ and $V(\eta)$
        are irrelevant and can be arbitrary. It holds that
        \[\llbracket T(\Gamma) \vdash F(\varphi) \colon T(\bullet_{PHFL}) \rrbracket^{V(\eta)}_{\mathcal{T}_d} = Q^d\]
        because the set of states of $\mathcal{T}_d$ is $Q^d$.

        \item In case of $\varphi = p_i$,
        \[\llbracket \Gamma \vdash \varphi \colon \bullet_{PHFL}
        \rrbracket^\eta_\mathcal{T} = \{(q_1, \dots, q_d)\in Q^d \mid p \in v(q_i)\}.\]
        By construction of $\mathcal{T}_d$ $v_d((q_1, \dots, q_d)) = \overset{d}{\underset{i = 1}{\bigcup}}\{p_i
        \mid p \in v(q_i)\}$ and by formula function $F$ it is true that $F(p_i) = p_i$. As in case of $\varphi =
        \top$ the type environment $\Gamma$, the type $\bullet_{PHFL}$ and variable mapping $\eta$ do not matter in
        the semantics of $\llbracket \Gamma \vdash \varphi \colon \bullet_{PHFL} \rrbracket^\eta_\mathcal{T}$ and so $T
        (\Gamma)$, $T(\bullet_{PHFL})$ and $V(\eta)$ are irrelevant and can be arbitrary. Because $p_i \in v_d((q_1,
        \dots, q_d))$ iff $p \in v(q_i)$ it holds that
        \[\llbracket T (\Gamma) \vdash F(\varphi) \colon T(\bullet_{PHFL}) \rrbracket^{V(\eta)}_{\mathcal{T}_d} = \{
        (q_1, \dots, q_d) \in Q^d \mid p_i \in v_d((q_1, \dots, q_d))\}\]
        and moreover that
        \[\{(q_1, \dots, q_d)\in Q^d \mid p \in v(q_i)\} =\{(q_1, \dots, q_d) \in Q^d \mid p_i \in
        v_d((q_1, \dots, q_d))\}.\]

        \item The last induction base case is $\varphi = X$. It holds that
        \[\llbracket \Gamma \vdash \varphi \colon \tau \rrbracket^\eta_\mathcal{T} = \eta(X).\]
        Moreover, it is true that
        \[\llbracket T(\Gamma) \vdash F(\varphi) \colon T(\tau) \rrbracket^{V(\eta)}_{\mathcal{T}_d} = V(\eta)(X).\]
        By formula function $F$ it is true that $F(X) = X$. The combination of construction $\mathcal{T}_d$, type
        function $T$ and by construction of variable mapping $V(\eta)$ in
        Definition~\ref{definition:variable_mapping_function} it follows that $V(\eta)(X) = \eta(X)$. As in the both
        cases above the type environment $\Gamma$ do not matter in the semantics of $\llbracket \Gamma \vdash \varphi
        \colon \tau \rrbracket^\eta_\mathcal{T}$ and so $T(\Gamma)$ is irrelevant and can be arbitrary.
    \end{compactitem}
    By induction hypothesis it holds for subformulas $\psi$ and $\psi'$ of $\varphi$ that $\llbracket \Gamma \vdash
    \psi \colon \tau \rrbracket^\eta_\mathcal{T} = \llbracket T(\Gamma) \vdash F(\psi) \colon T(\tau)
    \rrbracket^{V(\eta)}_{\mathcal{T}_d}$ and $\llbracket \Gamma \vdash
    \psi' \colon \tau \rrbracket^\eta_\mathcal{T} = \llbracket T(\Gamma) \vdash F(\psi') \colon T(\tau)
    \rrbracket^{V(\eta)}_{\mathcal{T}_d}$
    \begin{compactitem}
        \item In case of $\varphi = \langle a \rangle_i \psi$,
        \begin{align*}
            \llbracket \Gamma \vdash \varphi \colon \bullet_{PHFL} \rrbracket^\eta_\mathcal{T} = &\,
            \{(q_1, \dots, q_d) \in Q^d \mid \\&\text{ it exists } ({q_1}', \dots, {q_d}') \in \llbracket \Gamma
                \vdash \psi \colon \bullet_{PHFL}
                \rrbracket^\eta_\mathcal{T} \text{ such that }\\&\,q_i \overset{a}{\rightarrow} {q_i}' \text{ and for
            all
            } i \neq j
            \text{ it holds } q_j =
                {q_j}'\}.
        \end{align*}
        By induction hypothesis $({q_1}', \dots, {q_d}') \in \llbracket T(\Gamma) \vdash F(\psi) \colon T
        (\bullet_{PHFL}) \rrbracket^{V(\eta)}_{\mathcal{T}_d}$ because $({q_1}', \dots, {q_d}') \in \llbracket \Gamma
        \vdash \psi \colon \bullet_{PHFL} \rrbracket^\eta_\mathcal{T}$. Because of construction of $\mathcal{T}_d$
        and $q_i \overset{a}{\rightarrow} {q_i}' \in \Delta$ it follows that $({q_1}', \dots, {q_{i - 1}'}, q_i,
        {q_{i + 1}'}, \dots, {q_d}') \overset{a_i}{\rightarrow} ({q_1}', \dots, {q_{i - 1}'}, {q_i}',
        {q_{i + 1}'}, \dots, {q_d}') \in \Delta_d$. That means that
        \begin{align*}
            \{(q_1, \dots, q_d) \in Q^d \mid &\text{ it exists } ({q_1}', \dots, {q_d}') \in \llbracket \Gamma
            \vdash \psi \colon \bullet_{PHFL}
            \rrbracket^\eta_\mathcal{T} \text{ such that }\\&\,q_i \overset{a}{\rightarrow} {q_i}' \text{ and for
            all } i \neq j \text{ it holds } q_j = {q_j}'\}
        \end{align*}
        is equal to
        \begin{align*}
            \{(q_1, \dots, q_d) \in Q^d \mid &\text{ it exists } ({q_1}', \dots, {q_d}') \in \llbracket T(\Gamma)
            \vdash F(\psi) \colon T(\bullet_{PHFL}) \rrbracket^{V(\eta)}_{\mathcal{T}_d}\\& \text{ such that }\,q_i
            \overset{a_i}{\rightarrow} {q_i}' \text{ and for all } i \neq j \text{ it holds } q_j = {q_j}'\}.
        \end{align*}
        Because of $T$, $V$ and that $F(\langle a \rangle_i \psi) = \langle a_i \rangle \psi$ the second set is
        exactly the definition of the semantics of
        \[\llbracket T(\Gamma) \vdash F(\varphi) \colon T(\bullet_{PHFL}) \rrbracket^{V(\eta)}_{\mathcal{T}_d}.\]

        \item In case of $\varphi = \psi \vee {\psi'}$ then
        \[\llbracket \Gamma \vdash \varphi \colon \tau \rrbracket^\eta_\mathcal{T} = \llbracket \Gamma \vdash \psi
        \colon \tau \rrbracket^\eta_\mathcal{T} \sqcup_\tau \llbracket \Gamma \vdash \psi' \colon \tau
        \rrbracket^\eta_\mathcal{T}.\]
        By induction hypothesis it holds that \[\llbracket T(\Gamma) \vdash F(\psi) \colon T
        (\tau) \rrbracket^{V(\eta)}_{\mathcal{T}_d} = \llbracket \Gamma
        \vdash \psi \colon \tau \rrbracket^\eta_\mathcal{T}\]
        and
        \[\llbracket T(\Gamma) \vdash F(\psi') \colon T
        (\tau) \rrbracket^{V(\eta)}_{\mathcal{T}_d} = \llbracket \Gamma
        \vdash \psi' \colon \tau \rrbracket^\eta_\mathcal{T}.\]
        By construction of $T$ it holds that
        \[\llbracket T(\Gamma) \vdash F(\psi) \colon T
        (\tau) \rrbracket^{V(\eta)}_{\mathcal{T}_d} \sqcup_{T(\tau)} \llbracket T(\Gamma) \vdash F(\psi') \colon T
        (\tau) \rrbracket^{V(\eta)}_{\mathcal{T}_d}\]
        is equal to
        \[\llbracket \Gamma \vdash \psi
        \colon \tau \rrbracket^\eta_\mathcal{T} \sqcup_\tau \llbracket \Gamma \vdash \psi' \colon \tau
        \rrbracket^\eta_\mathcal{T}.\]
        Because $F(\psi \vee \psi') = F(\psi) \vee F(\psi')$ it holds that
        \[\llbracket T(\Gamma) \vdash F(\varphi) \colon T(\tau) \rrbracket^{V(\eta)}_{\mathcal{T}_d}\]
        is equal to
        \[\llbracket T(\Gamma) \vdash F(\psi) \colon T(\tau) \rrbracket^{V(\eta)}_{\mathcal{T}_d} \sqcup_{T(\tau)}
        \llbracket T(\Gamma) \vdash F(\psi') \colon T(\tau) \rrbracket^{V(\eta)}_{\mathcal{T}_d}.\]

        \item In case of $\varphi = \neg \psi$ then
        \[\llbracket \Gamma \vdash \varphi \colon \bullet_{PHFL} \rrbracket^\eta_\mathcal{T} = Q^d \setminus
        \llbracket
        \Gamma^-
        \vdash \psi \colon \bullet_{PHFL} \rrbracket^\eta_\mathcal{T}.\]
        By induction hypothesis it holds that \[\llbracket T(\Gamma) \vdash F(\psi) \colon T
        (\bullet_{PHFL}) \rrbracket^{V(\eta)}_{\mathcal{T}_d} = \llbracket \Gamma
        \vdash \psi \colon \bullet_{PHFL} \rrbracket^\eta_\mathcal{T}.\]
        Because of construction of $T(\Gamma)$ it also holds that
        \[\llbracket T(\Gamma^-) \vdash F(\psi) \colon T
        (\bullet_{PHFL}) \rrbracket^{V(\eta)}_{\mathcal{T}_d} = \llbracket \Gamma^-
        \vdash \psi \colon \bullet_{PHFL} \rrbracket^\eta_\mathcal{T}.\]
        Because of the equality of the two sets it follows that
        \[Q^d \setminus \llbracket T(\Gamma^-) \vdash F(\psi) \colon T(\bullet_{PHFL}) \rrbracket^{V(\eta)
        }_{\mathcal{T}_d} =
        Q^d \setminus \llbracket \Gamma^- \vdash \psi \colon \bullet_{PHFL} \rrbracket^\eta_\mathcal{T}.\]
        Because $F(\neg \psi) = \neg F(\psi)$ the first set is exactly the semantics of
        \[\llbracket T(\Gamma) \vdash F(\varphi) \colon T(\bullet_{PHFL}) \rrbracket^{V(\eta)}_{\mathcal{T}_d}.\]

        \item In case of $\varphi = \{(e(1), \dots, e(d))\}\,\psi$ then
        \begin{align*}
            \llbracket \Gamma \vdash \varphi \colon \bullet_{PHFL} \rrbracket^\eta_\mathcal{T} = \{&(q_1, \dots,
            q_d) \in Q^d \mid \\&(q_{e(1)}, \dots, q_{e(d)}) \in \llbracket \Gamma \vdash \psi
            \colon \bullet_{PHFL}
            \rrbracket ^\eta_\mathcal{T}\}.
        \end{align*}
        By induction hypothesis $({q_{e(1)}}, \dots, {q_{e(d)}}) \in \llbracket T(\Gamma) \vdash F(\psi) \colon T
        (\bullet_{PHFL}) \rrbracket^{V(\eta)}_{\mathcal{T}_d}$ because $({q_{e(1)}}, \dots, {q_{e(d)}}) \in
        \llbracket \Gamma \vdash \psi \colon \bullet_{PHFL} \rrbracket^\eta_\mathcal{T}$.
        Moreover, the state tuple $({q_{e(1)}}, \dots, {q_{e(d)}})$ that fulfills $\psi$ is reached by 'moving' from
        state tuple $({q_{1}}, \dots, {q_{d}})$ to $({q_{e(1)}}, \dots, {q_{e(d)}})$. This movement is
        integrated in the construction of $\mathcal{T}_d$. There is for each endomorphism $e$ a
        substitution action $s_{(e(1), \dots, e(d))}$ for each tuple state. With $F(\{(e(1), \dots, e(d))\}
        \psi) = \langle s_{(e(1), \dots, e(d))} \rangle F(\psi)$ we get the state tuples that have the action $(q_1,
        \dots, q_d)$ $ \overset{s_{(e(1), \dots, e(d))}}{\rightarrow} (q_{e
        (1)}, \dots, q_{e(d)})$ where $(q_1, \dots, q_d) \in \llbracket T(\Gamma) \vdash F(\psi) \colon T
        (\bullet_{PHFL}) \rrbracket^{V(\eta)}_{\mathcal{T}_d}$.
        It follows that
        \[\llbracket \Gamma \vdash \varphi \colon \bullet_{PHFL} \rrbracket^\eta_\mathcal{T} = \llbracket T(\Gamma)
        \vdash F(\varphi) \colon T(\bullet_{PHFL}) \rrbracket^{V(\eta)}_{\mathcal{T}_d}.\]

        \item In case of $\varphi = \mu(X \colon \tau).\,\psi$ then
        \begin{align*}
            \llbracket \Gamma \vdash \varphi \colon \tau \rrbracket^\eta_\mathcal{T} =&\,\bigsqcap\,
            _{\llbracket\tau\rrbracket_\mathcal{T}} \{\mathcal{X} \in \llbracket \tau \rrbracket_\mathcal{T}
            \mid \\
            &\llbracket \Gamma, X^+ \colon \tau \vdash \psi \colon \tau \rrbracket^{\eta[X \mapsto
            \mathcal{X}]}_\mathcal{T}
            \leq_{\llbracket \tau \rrbracket_\mathcal{T}} \mathcal{X}\}.
        \end{align*}
        By induction hypothesis it holds that \[\llbracket T(\Gamma) \vdash F(\psi) \colon T
        (\tau) \rrbracket^{V(\eta)}_{\mathcal{T}_d} = \llbracket \Gamma
        \vdash \psi \colon \tau \rrbracket^\eta_\mathcal{T}.\]
        Because of construction of $T(\Gamma)$, $\mathcal{T}_d$ and $V(\eta)$ it also holds that
        \[\llbracket T(\Gamma), X^+ \colon T(\tau) \vdash F(\psi) \colon T
        (\tau) \rrbracket^{V(\eta)[X \mapsto \mathcal{X}]}_{\mathcal{T}_d} = \llbracket \Gamma, X^+ \colon \tau
        \vdash \psi \colon \tau \rrbracket^{\eta[X \mapsto \mathcal{X}]}_\mathcal{T}.\]
        Moreover, $F(\mu(X \colon \tau).\psi) = \mu(X \colon T(\tau)).F(\psi)$. Summarized it holds
        \begin{align*}
            \llbracket \Gamma \vdash \varphi \colon \tau \rrbracket^\eta_\mathcal{T} =&\,\bigsqcap\,
            _{\llbracket\tau\rrbracket_{\mathcal{T}_d}} \{\mathcal{X} \in \llbracket \tau \rrbracket_{\mathcal{T}_d}
            \mid \\
            &\llbracket T(\Gamma), X^+ \colon T(\tau) \vdash F(\psi) \colon T(\tau) \rrbracket^{V(\eta)[X \mapsto
            \mathcal{X}]}_{\mathcal{T}_d}
            \leq_{\llbracket \tau \rrbracket_{\mathcal{T}_d}} \mathcal{X}\}.
        \end{align*}
        This is exactly the semantics of
        \[\llbracket T(\Gamma) \vdash F(\varphi) \colon T(\tau) \rrbracket^{V(\eta)}_{\mathcal{T}_d}.\]

        \item In case of $\varphi = \lambda(X^v \colon \sigma).\,\psi$ then
        \begin{align*}
            \llbracket \Gamma \vdash \varphi \colon \sigma^v \rightarrow \tau \rrbracket^\eta_\mathcal{T} =&\,F \in
            \llbracket \sigma^v \rightarrow \tau \rrbracket_\mathcal{T}
        \end{align*}
        such that for all $\mathcal{X} \in \llbracket \sigma \rrbracket_\mathcal{T}$ holds that $F(\mathcal{X}) =
        \llbracket \Gamma, X^v \colon \sigma \vdash \psi \colon \tau \rrbracket^{\eta[X \mapsto
        \mathcal{X}]}_\mathcal{T}$.
        By induction hypothesis it holds that \[\llbracket T(\Gamma) \vdash F(\psi) \colon T
        (\tau) \rrbracket^{V(\eta)}_{\mathcal{T}_d} = \llbracket \Gamma
        \vdash \psi \colon \tau \rrbracket^\eta_\mathcal{T}.\]
        Because of construction of $T(\Gamma)$, $\mathcal{T}_d$ and $V(\eta)$ it also holds that
        \[\llbracket T(\Gamma), X^v \colon T(\sigma) \vdash F(\psi) \colon T
        (\tau) \rrbracket^{V(\eta)[X \mapsto \mathcal{X}]}_{\mathcal{T}_d} = \llbracket \Gamma, X^v \colon \sigma
        \vdash \psi \colon \tau \rrbracket^{\eta[X \mapsto \mathcal{X}]}_\mathcal{T}.\]
        Moreover, $F(\lambda(X^v \colon \sigma).\psi) = \lambda(X^v \colon T(\sigma)).F(\psi)$. Summarized it holds
        \begin{align*}
            \llbracket \Gamma \vdash \varphi \colon \sigma^v \rightarrow \tau \rrbracket^\eta_\mathcal{T} =&\,F \in
            \llbracket T(\sigma^v \rightarrow \tau) \rrbracket_{\mathcal{T}_d}
        \end{align*}
        such that for all $\mathcal{X} \in \llbracket T(\sigma) \rrbracket_{\mathcal{T}_d}$ holds that $F(\mathcal{X}) =
        \llbracket T(\Gamma), X^v \colon T(\sigma) \vdash F(\psi) \colon T(\tau) \rrbracket^{V(\eta)[X \mapsto
        \mathcal{X}]}_{\mathcal{T}_d}$.
        This is exactly the semantics of
        \[\llbracket T(\Gamma) \vdash F(\varphi) \colon T(\sigma^v \rightarrow \tau) \rrbracket^{V(\eta)
        }_{\mathcal{T}_d}.\]

        \item In case of $\varphi = \psi\psi'$ then
        \[\llbracket \Gamma \vdash \varphi \colon \tau \rrbracket^\eta_\mathcal{T} = \llbracket \Gamma \vdash \psi
        \colon \sigma^v \rightarrow \tau \rrbracket^\eta_\mathcal{T}(\llbracket \Gamma \vdash \psi' \colon
        \sigma \rrbracket^\eta_\mathcal{T}).\]
        By induction hypothesis it holds that \[\llbracket T(\Gamma) \vdash F(\psi) \colon T
        (\sigma^v \rightarrow \tau) \rrbracket^{V(\eta)}_{\mathcal{T}_d} = \llbracket \Gamma
        \vdash \psi \colon \sigma^v \rightarrow \tau \rrbracket^\eta_\mathcal{T}\]
        and
        \[\llbracket T(\Gamma) \vdash F(\psi') \colon T
        (\sigma) \rrbracket^{V(\eta)}_{\mathcal{T}_d} = \llbracket \Gamma
        \vdash \psi' \colon \sigma \rrbracket^\eta_\mathcal{T}.\]
        It follows that
        \[\llbracket T(\Gamma) \vdash F(\psi) \colon T
        (\sigma^v \rightarrow \tau) \rrbracket^{V(\eta)}_{\mathcal{T}_d}(\llbracket T(\Gamma) \vdash F(\psi') \colon T
        (\sigma) \rrbracket^{V(\eta)}_{\mathcal{T}_d}\]
        is equal to
        \[\llbracket \Gamma \vdash \psi
        \colon \sigma^v \rightarrow \tau \rrbracket^\eta_\mathcal{T}(\llbracket \Gamma \vdash \psi' \colon \sigma
        \rrbracket^\eta_\mathcal{T}.\]
        Because $F(\psi\psi') = F(\psi)F(\psi')$ the first set is the semantics of
        \[\llbracket T(\Gamma) \vdash F(\varphi) \colon T(\tau) \rrbracket^{V(\eta)}_{\mathcal{T}_d}.\]
    \end{compactitem}
    This shows the correctness of the construction and so that the model checking problem of PHFL$^k$ is reducible to
    the model checking problem of HFL$^k$.
\end{proof}

The following theorem is given by Lemma~\ref{lemma:model_check_phfl_k} and Theorem~\ref{theorem:hfl_k_in_k_exptime}.

\begin{theorem}
    \label{theorem:phfl_k_in_k_exptime}
    Given a LTS $\mathcal{T}$, a state $s$ and a PHFL$^k$ formula $\varphi$, the model checking problem $\mathcal{T}, s
    \models \varphi$ is in $k$-EXPTIME, where $k > 0$.
\end{theorem}

To show that the upper bound of PHFL$^{k + 1}_{tail}$ is \expspace{$k$} we can make a reduction from the semantics of
PHFL$^{k}_{tail}$ to the semantics of HFL$^k_{tail}$. Remember, that HFL$^k_{tail}$ is the set of tail-recursive
$1$-adic PHFL$^k$ formulas. In combination to the following theorem we get the upper bound
of PHFL$^k_{tail}$.

\begin{theorem}{\cite{bruse2017space}}
    \label{theorem:hfl_k_plus_1_in_k_expspace}
    Given a LTS $\mathcal{T}$, a state $s$ and a HFL$^{k + 1}_{tail}$ formula $\varphi$, the model checking problem
    $\mathcal{T}, s \models \varphi$ is in $k$-EXPSPACE, where $k > 0$.
\end{theorem}

\begin{lemma}
    \label{lemma:model_check_phfl_k_tail}
    Let $\mathcal{T}$ a LTS, $\eta$ a variable mapping, $\Gamma$ a type environment, $\varphi$ a well-typed $d$-adic
    PHFL$^k_{tail}$ formula of type $\tau$, $\mathcal{T}_d$ the LTS transformed by process of
    Defintion~\ref{definition:lts_transformation}, $T$ the type function of Defintion~\ref{definition:type_function},
    $V$ the variable mapping function of Definition~\ref{definition:variable_mapping_function}
    and $F$ the formula function of Definition~\ref{definition:formula_function} then $\llbracket \Gamma \vdash
    \varphi \colon \tau \rrbracket^\eta_\mathcal{T} = \llbracket T(\Gamma) \vdash F(\varphi) \colon T(\tau)
    \rrbracket^{V(\eta)}_{\mathcal{T}_d}$
\end{lemma}

\begin{proof}
    This proof is similar to the proof of Lemma~\ref{lemma:model_check_phfl_k}. The tail-recursive fragment of the
    PHFL$^k$ and HFL$^k$ formulas do not influence the construction of the proof of
    Lemma~\ref{lemma:model_check_phfl_k}.
\end{proof}

The following theorem is given by Lemma~\ref{lemma:model_check_phfl_k_tail} and
Theorem~\ref{theorem:hfl_k_plus_1_in_k_expspace}.

\begin{theorem}
    \label{theorem:phfl_k_plus_1_tail_in_k_expspace}
    Given a LTS $\mathcal{T}$, a state $s$ and a PHFL$^{k + 1}_{tail}$ formula $\varphi$, the model checking problem
    $\mathcal{T}, s \models \varphi$ is in $k$-EXPSPACE, where $k > 0$.
\end{theorem}
%%
%% Author: DKron
%% 24.07.2018
%%

\chapter{Lower Bounds}\label{ch:lowerBounds}

In this chapter we want to establish that the lower bounds of the expressive power of PHFL$^k$ and PHFL$^k_{tail}$ are \exptime{$k$} and \expspace{$k$} respectively. The
lower bounds of PHFL$^k$ and PHFL$^k_{tail}$ can be proven by encoding the run of a Turing Machine as queries.
As another possibility one can use intermediate logics HO(LFP)$^{k+1}$ and HO(PFP)$^{k+1}$ and encode the bisimulation-invariant logic of those as PHFL$^k$ and
PHFL$^k_{tail}$, respectively. Note that PHFL cannot distinguish between bisimilar structures~\cite{viswanathan2004higher}. That means PHFL formulas can only define bisimulation-invariant
graph problems. Moreover, PHFL is sufficient to encode the bisimulation-invariant logic of HO(LFP)$^{k+1}$ and HO(PFP)$^{k+1}$. To encode the bisimulation-invariant fragment of HO(LF\-P)$^{k+1}$ and HO(PFP)$^{k+1}$ as PHFL$^k$ and
PHFL$^{k}_{tail}$ respectively we want to translate HO(LFP)$^{k+1}$ formulas into PHFL$^k$ formulas and 
HO(PFP)$^{k+1}$ into a PHFL$^k_{tail}$ formula. 

Encoding existential quantifiers is the most complex part from a computational point of view. In the 
first section we consider some preparations that are necessary to model them in PHFL. Thereafter,  we show that existential 
quantifiers with bound variable of order $k \geq 1$ can be expressed by a PHFL$^{k-1}$ formula. In the subsequent section we 
use this formula to show that the bisimulation invariant fragment of HO(LFP)$^{k+1}$ can be encoded into PHFL$^k$ and so 
that the lower bound of the expressive power of PHFL$^k$ is \exptime{$k$}. And finally we show that the lower bound 
of the expressive power of PHFL$^k_{tail}$ is \expspace{$k$}.

\section{Preparation}\label{sec:lower_bounds_preparation}

Before we can start with the encodings there are some important steps that we have to consider.  Let $\mathcal{T} = (Q, 
\Sigma, P, \Delta, v)$ be an LTS, $q_1, \dots, q_n, p_1, \dots, p_m \in Q$ some states of $\mathcal{T}$ and let $\mathcal{T}'$ be the 
reduced LTS of $\mathcal{T}$ with respect to $p_1, \dots, p_m$.  
As mentioned in the introduction of this chapter it is known from~\cite{lange2014capturing} that PHFL cannot distinguish 
between bisimilar structures. That means PHFL formulas can only define bisimulation-invariant graph problems. That means that, without loss of generality, we can check if $([q_1]_\sim, \dots, [q_n]_\sim)$ satisfies a formula $\Phi$ with respect to $\mathcal{T}'$ instead of checking if $(q_1, \dots, q_n)$ satisfies 
$\Phi$ with respect to $\mathcal{T}$. With $[q_i]_\sim$ we denote the equivalence class of $q_i$ with respect to $\sim$.

From this point all LTS are reduced LTS with respect to some states of their state sets. In more detail, let $\mathcal{T} = (Q, \Sigma, P, \Delta, v)$ be a reduced LTS with respect to $(q_1, \dots, q_r)$, where $q_1, \dots, q_r \in Q$. On those LTS it is possible to define a total order on their states. 

\begin{remark} \label{remark:transitive_relation}
    In~\cite{otto1999bisimulation} it was shown that it is possible to define a $2$-adic PHFL$^0$ formula $\Phi_<$ over reduced LTS that defines a
    transitive relation $<$ such that $< \cap > = \emptyset$ and $< \cup > = \not\sim$. This relation $<$ defines a total
    order on states of a reduced LTS.
\end{remark}


%%
%% Author: DKron
%% 17.08.2018
%%

\section{Existential Quantifiers in PHFL}\label{sec:existential_quantifiers_in_phfl}

In this section we define PHFL$^{k}$ formulas that describes existential quantification over HO variables of order $k
\geq 1$. But before we can define these formulas we have to translate the types.

The types of HO have to be translated because the most types in HO$^{k + 1}$ do not exist in PHFL$^k$. While HO$^{k +
1}$ includes variables for example of kind set of sets, PHFL$^k$ does not support this kind of type.
But each set $A$ in HO$^{k+1}$ can be described by the characteristic function of $A$ in PHFL$^k$.

The following definition translates all HO types with order $2$ or greater to types in PHFL. The base type of HO
has to be encode differently, and will be regarded after this definition.

\begin{definition}
    \label{definition:lower_bound_type_function}
    Let $T$ be a function that maps a type of HO of order $2$ or greater to a type of PHFL defined inductive over the
    type of HO as follows:

    \begin{align*}
        T((\odot, \dots, \odot)) &= \bullet\\
        T((\tau', \dots, \tau')) &= T(\tau')^+ \rightarrow (T(\tau')^+ \rightarrow \dots \rightarrow (T(\tau')^+
        \rightarrow \bullet) \dots )
    \end{align*}
\end{definition}

Note that $ord(T(\tau)) = order(\tau) - 2$ for all $\tau$ with order $2$ or greater.

\begin{example}
    Let $\tau$ be a type of HO
    \[\tau = (\tau', \tau')\]
    with
    \[\tau' = (((\odot)), ((\odot)))\]
    then by Definition~\ref{definition:lower_bound_type_function} of type function $T$
    \[T(\tau) = T(\tau') \rightarrow (T(\tau') \rightarrow \bullet).\]
    with
    \[T(\tau') = (\bullet \rightarrow \bullet) \rightarrow ((\bullet \rightarrow \bullet) \rightarrow \bullet)\]
\end{example}

With this type function $T$ a HO$^{k + 1}$ variable $X$ of type $\tau$ can be translated to a PHFL$^k$ variable
with type $T(\tau)$. Please note that $\tau$ has order $2$ or greater. Intuitively the variable $X$ which represents a
set in HO$^{k+1}$ is represented in PHFL$^k$ as the characteristic function of $X$. Please note that $X$ for type
$\tau = (\odot, \dots, \odot)$ is also a set in PHFL$^k$ of PHFL type $\bullet$.

As mentioned above the base type of HO have to be encoded differently. The reason is that the base type in PHFL is at
least a set of tuples of states. So single states can not be depict directly by a variable. In this thesis the
idea is to use the polyadic fragment of a PHFL$^k$ formula $\Phi$ to represent the different first-order variables of an
HO$^{k+1}$ formula $\Psi$. Each first-order variable of $\Psi$ represents one component in $\Phi$, that means each
variable
increases the dimension of $\Phi$. The assignment of a first-order variable $x_i$ in $\Psi$ is then the current state
of the $i$-th component in $\Phi$. This is mainly from~\cite{lange2014capturing}.

Without loss of generality the first-order variables are enumerated as $x_1, \dots, x_f, x_{f + 1}, \dots, x_q$,
where $x_1, \dots, x_f$ are the free and $x_{f+1}, \dots, x_q$ are the quantified variables.

\begin{remark}
    The PHFL formula $\Phi$ that we get through the encoding of a given HO formula $\Psi$ has dimension
    $d$ that is always big enough to translate all second-order variables of $\Psi$ to an order $0$ variable in
    $\Phi$. In more detail $s$ is the maximal arity of second-order variables in $\Psi$ and $d > s$. To distiguish all
    different first-order variables in $\Psi$, the dimension $d$ of $\Phi$ is also bigger then $q$.  Finally, to compare
    two
    elements of $Q^{m}$, where $Q$ is the state set of an LTS and $m = max({s, q})$, the dimension $d$ of $\Phi$ is
    twice as big as the maximum of $s$ and $q$. That means $d = 2 * m$.
\end{remark}

After we know how to interpret the different HO types and variables we are now able to consider the existential
quantification. Before we regard higher-order quantification we look at first-order quantification.

Because we encode the bisimulation-invariant fragment of HO(LFP)$^{k + 1}$, the first-order quantification can be
encoded by moving through all states reachable by one of the free variables and check if we reach a state tuple where
the bounded formula holds. If we regard $\exists (x_i \colon \odot).\,\Phi$ then it can be understand as, check if we
reach a state tuple where $\Phi$ holds once the $i$-th component is replaced by one of the free variables. This can
be formalized in PHFL as
\[\exists_i \Phi \coloneqq \bigvee_{j=1}^f \{(1, \dots, i-1, j, i + 1, \dots, d)\} \mu (X
\colon \bullet).\,\Phi \vee \bigvee_{a \in \Sigma} \langle a \rangle_i X.\]
Now we regard the higher-order quantification. Let $\tau$ a HO type, $\sigma$ a signature and $\mathcal{U}$ the
universe of a $\sigma$ structure. Because the idea we use for first-order quantification is not obvious to adapt to
the higher-order quantification, we use another encoding. The idea to obtain higher-order quantification in PHFL is that
we have to iterate through all possible elements in a domain $D_\tau(\mathcal{U})$ and check if the given formula is
fulfilled. To make it possible to iterate through $D_\tau(\mathcal{U})$ we need a formula that returns us the
successor of a given element in $D_\tau(\mathcal{U})$. Eventually, to get the successor of a given element we need some
order on $D_\tau(\mathcal{U})$. So the first thing we need is the order of any domain.

\begin{remark}
    In~\cite{otto1999bisimulation} was shown that it is possible to define a $2$-adic formula $\Phi_<$ that defines a
    transitive relation $<$ such that $< \cap > = \emptyset$ and $< \cup > = \not\sim$. In this thesis $<$ is a total
    order on states of an LTS.
\end{remark}

To get an order of any type we define formulas that tells us given two elements of same type which one is the smaller
one. We say that a tuple $x$ is smaller than a tuple $y$ if there is an index $i$ such that the element in $x$ on $i$
is smaller then the element in $y$ on $i$ and there is no position $j$ left of $i$ such that the element in $x$ on
$j$ is bigger then the element in $y$ on $j$. We say that a set $x$ is smaller than a set $y$ if there is an element
$e$ in $x$ that is not in $y$ and all smaller elements $f$ to $e$ are only in $x$ if $f$ is also in $y$. This is
formalized in PHFL in the following definition.

\begin{definition}
    \label{definition:lower_bound_less}
    The PHFL$^k$ formula $<^\tau$ where $k = ord(T(\tau))$ is defined inductive over the type $\tau$ as follows:

    \begin{align*}
        <^\odot \coloneqq &\,\Phi_< \\
        <^{\odot \times \dots \times \odot} \coloneqq &\,\underset{i = 1}{\overset{m}{\bigvee}}\{(i, i + m, 3, \dots,
        d)\} <^\odot \wedge \\
        &\,\underset{j = 1}{\overset{i - 1}{\bigwedge}}\{(j + m, j, 3, \dots, d)\} \neg
        <^\odot\\
        <^{\tau' \times \dots \times \tau'}(x_1, y_1, \dots, x_1, y_n) \coloneqq &\,\underset{i =
        1}{\overset{n}{\bigvee}}<^{\tau'}(x_i, y_i) \wedge \underset{j = 1}{\overset{i - 1}{\bigwedge}}
        \neg <^{\tau'}(y_j, x_j)\\
        <^{(\odot, \dots, \odot)}(X, Y) \coloneqq &\,\exists_{i_1}.\, \dots \exists_{i_n}. \{(i_1, \dots, i_n, n
        + 1, \dots, d)\}Y \wedge \\&\,\neg \{(i_1, \dots,
        i_n, n + 1, \dots, d)\} X\,\wedge\\&\, \forall_{j_1}. \,\dots \forall_{j_n}. \{(j_1, \dots, j_n, n+1,
        \dots, m, \\&\,i_1, \dots, i_n, m + n + 1, \dots, d)\}<^{\odot
        \times \dots \times \odot} \Rightarrow \\&\,(\{(j_1, \dots, j_n, n + 1, \dots, d)\}X
        \Rightarrow \\&\,\neg \{(j_1, \dots, j_n, n + 1, \dots, d)\} Y)
        \\
        <^{(\tau', \dots, \tau')}(X, Y) \coloneqq &\,\exists^{\tau'}x_1. \,\dots \exists^{\tau'}x_n.\,(\dots((Y
        (x_1))(x_2))
        \dots) (x_n)
         \\&\,\wedge \neg (\dots((X(x_1))(x_2)) \dots)(x_n)\,\wedge \\&\,\forall^{\tau'}y_1. \,\dots
        \forall^{\tau'}y_n.
        \,<^{\tau'
        \times \dots \times \tau'}
        (y_1, x_1, \dots, y_n, x_n) \\&\,\Rightarrow ((\dots((X(y_1))(y_2)) \dots)(y_n) \\&\,\Rightarrow (\dots((Y(y_1))
        (y_2)) \dots(y_n))
    \end{align*}
    where the formulas of kind $\exists^{\tau'} x.\,\phi$ are the in PHFL$^{k-1}$ defined higher-order quantification
    for HO
    type $\tau'$ and formulas of kind $\forall^{\tau'} x.\,\phi$ are the counterparts.
\end{definition}

After we have now orders of any HO type we can define formulas that returns the successor of the input element.

\begin{definition}
    The PHFL$^k$ formula $next^\tau$ where $k = ord(T(\tau))$ is defined by the type $\tau$ as follows:
    \begin{align*}
        next^{(\odot, \dots, \odot)} \coloneqq &\,\lambda (X \colon \bullet).\, (\neg X \wedge \forall_{m +
        1}\dots\forall_{m + m}<^{\odot \times \dots \times \odot}\, \Rightarrow \\&\,\{(m +
        1, \dots, m + m, m + 1, \dots, d)\} X) \,\vee \\&\,(X \wedge \exists_{m + 1}\dots\exists_{m + m} <^{\odot
        \times \dots \times \odot} \,\wedge \\&\,\{(m + 1, \dots, m + m, m + 1, \dots, d)\}
        \neg X)\\
        next^{(\tau', \dots, \tau')} \coloneqq &\,\lambda (X \colon T ((\tau', \dots, \tau'))).\,(\neg (\dots((X
        (x_1))(x_2))\dots) (x_n) \\&\, \wedge \forall^{\tau'}y_1.\, \dots \forall^{\tau'}y_n.\,<^{\tau' \times
        \dots \times \tau'}(y_1, x_1, \dots, y_n, x_n) \\&\,\Rightarrow  (\dots((X(y_1))(y_2))\dots)(y_n)) \,\vee
        \\&\,((\dots ((X(x_1))(x_2)) \dots)(x_n) \wedge \exists^{\tau'}y_1.\, \dots \exists^{\tau'}y_n.\, \\&\,
        <^{\tau' \times \dots \times \tau'}
        (y_1, x_1,
        \dots, y_n, x_n)\,\wedge \neg (\dots((X(y_1))(y_2))\dots)(y_n))
    \end{align*}
\end{definition}

The idea of the two formulas are based on binary incrementation. Let $\tau = (\tau', \dots, \tau')$ an HO type and
$\mathcal{U}$ the universe of a $\sigma$-structure. Remember that a set $X \in D_\tau(\mathcal{U})$ can be
represented by its characteristic function. This can be transformed to a binary string where each position of
this string represents an element of $D_{\tau'}(\mathcal{U})^n$. Because each position in the binary string represents
an element of $D_{\tau'}(\mathcal{U})^n$ and a position have always to represent the same element in $D_{\tau'}
(\mathcal{U})^n$, the elements in $D_{\tau'}(\mathcal{U})^n$ have to be ordered. The order of the elements of
$D_{\tau'}(\mathcal{U})^n$ is given by the formula $<^{\tau' \times \dots \times \tau'}$ of
Definition~\ref{definition:lower_bound_less}. If the position $i$ in the binary string is $1$ then this means that
the element with index $i$ in $D_{\tau'}(\mathcal{U})^n$ is also in $X$ and if the position $i$ in the binary string
is $0$ then the element with index $i$ in $D_{\tau'}(\mathcal{U})^n$ is not in $X$. This binary representation of $X$
in regard to $D_{\tau'}(\mathcal{U})^n$ can be extended to a function $f\colon D_\tau(\mathcal{U}) \rightarrow {0,
\dots, |D_\tau(\mathcal{U})| - 1}$ such that each element $X$ of $D_\tau(\mathcal{U})$ will be mapped to its binary
string in regard to $D_{\tau'}(\mathcal{U})^n$. With this knowledge we can say that $Y \in D_\tau(\mathcal{U})$ is the
direct successor of $X \in D_\tau(\mathcal{U})$ if $f(Y) = (f(X) + 1)$ modulo $|D_\tau(\mathcal{U})|$. In detail that
means that the $i$-th bit is $1$ in $f(Y)$ if it is either $0$ in $X$ and all lower bits are $1$ in $X$ or it is $1$
in $X$ and there is a bit lower then $i$ that is $0$ in $X$.

With the previous definitions we are now able to define the higher-order quantification in PHFL.
\begin{definition}
    \label{definition:existential_quantification}
    Let $\tau = (\tau', \dots, \tau')$ be an HO type where $\tau' \neq \odot$ then $\exists^{\tau}X .\,\Phi(X)$ is
    defined as follows:
    \begin{align*}
        \exists^{\tau}X.\, \Phi(X) \coloneqq &\,(\mu (F \colon T(\tau) \rightarrow \bullet).\, \lambda (X \colon T(\tau)
        ).\,
        \Phi(X)
        \vee F(next^\tau X))\\&\,(\lambda (x_1 \colon \tau').\, \dots \lambda (x_n \colon \tau').\,\bot)
    \end{align*}
    In case of $\tau = (\odot, \dots, \odot)$, $\exists^{\tau}X .\,\Phi(X)$ is defined as
    \[        \exists^{\tau}X.\, \Phi(X) \coloneqq (\mu (F \colon \bullet \rightarrow \bullet).\, \lambda (X
    \colon \bullet).\, \Phi(X) \vee F(next^{\tau} X)) \bot
    \]
\end{definition}

The last step is to show that the given formula of Definition~\ref{definition:existential_quantification} defines
existential quantification.

\begin{lemma}
    \label{lemma:existential_quantifier}
    For all HO types $\tau$ of order $2$ or greater, all variable mappings $\eta$ and all LTS $\mathcal{T}$ holds
    \[\llbracket \exists^\tau X.\,\Phi(X)\rrbracket^\eta_\mathcal{T} \equiv \underset{\mathcal{X} \in \llbracket \tau
    \rrbracket_\mathcal{T}}{\bigsqcup} \llbracket \Phi(X) \rrbracket^{\eta[X\rightarrow \mathcal{X}]}_\mathcal{T}.\]
\end{lemma}

\begin{proof}
    TODO
\end{proof}

%%
%% Author: DKron
%% 17.08.2018
%%

\section{Lower Bound of PHFL$^k$}\label{sec:lowerBoundOfPhfl}

As mentioned in the introduction of this chapter we can show that the lower bound of PHFL$^k$ is \exptime{$k$} by
make a detour over HO(LFP)$^{k+1}$. This and following ideas are oriented on~\cite{lange2014capturing} where it was
shown that PHFL$^1$ captures \exptime{$1$}. At first we will see that it was proven that HO(LFP)$^{k +
1}$ coincides with $k$-fold exponential time over finite and ordered structures. To use this we have to encode the
bisimulation invariant fragment of HO$^{k+1}$ into PHFL$^k$. Therefore, we define an abbreviation for the HO(LFP)
formulas that uses the LFP operator first and then a function that uses this abbreviation and those of
Section~\ref{sec:existential_quantifiers_in_phfl} to map an HO(LFP)$^{k+1}$ formula to a PHFL$^k$ formula. Finally, we
show that for any bisimulation-invariant HO(LFP)$^{k+1}$ formula $\Phi$ there is a transformed PHFL$^k$ formula $\Psi$ such that the query $\mathcal{Q}^d_\Psi$ defined by $\Psi$ can be obtained from the query $\mathcal{Q}^f_\Phi$ defined by $\Phi$ via projection to the relevant components. Hence, PHFL$^k$ can define all queries defined by the bisimulation-invariant fragment of HO(LFP)$^{k+1}$.

\begin{theorem}{\cite{freireMartins2011descriptive}}\label{theorem:hoLfpEqualsExptime}
    For all $k \geq 1$, HO(LFP)$^{k + 1}$ captures $k$-EXPTIME over finite and ordered structures.
\end{theorem}

The proof follows the idea to encode the run of a $k$-EXPTIME Turing Machine $M$ by a formula $\phi$ of HO(LFP)$^{k +
1}$ in such a way that $M$ accepts the standard coding of $\mathcal{A}$ iff $\mathcal{A} \models \phi$. On the other hand each HO(LFP)$^{k +
1}$ formula $\phi$ can be evaluated by a $k$-EXPTIME Turing Machine $M_\phi$.

Because Theorem~\ref{theorem:hoLfpEqualsExptime} holds, it is also possible to prove that the lower bound of PHFL$^k$
is in \exptime{$k$} by encoding the bisimulation invariant fragment of HO(LFP)$^{k + 1}$ into PHFL$^k$. To encode the
bisimulation invariant fragment of HO(LFP)$^{k + 1}$ into PHFL$^k$ we have to define a function that transforms a HO
(LFP)$^{k + 1}$ formula into a PHFL$^k$ formula. Note that the types and variables of an HO formula also need to be
transformed. See Section~\ref{sec:existential_quantifiers_in_phfl} for further details.

Before we consider the definition of the transforming function, we define a PHFL formula for HO formulas that uses the
LFP operator.

\begin{definition}
\label{definition:lfp_in_phfl}
   Let $d$ be the constant as described in Remark~\ref{remark:dimension} and let $X$ be an HO variable of HO type $\tau = (\tau', \dots, \tau')$ where $\tau' \neq \odot$. Furthermore, let
    $\Phi$ be a PHFL$^k$ formula, then $LFP^\tau X.\,\Phi$ is a PHFL$^k$ formula with dimension $d$, defined as:
    \[LFP^\tau X.\,\Phi \coloneqq \mu (X \colon T(\tau)).\,\Phi(X).\]
    In case of $\tau = (\odot, \dots, \odot)$ let
    \[LFP^{\tau} X.\,\Phi \coloneqq \mu (X \colon \bullet).\,\Phi(X).\]
\end{definition}

Now we are able to define the function in the following definition by using the abbreviations of Definitions~\ref{definition:existential_quantification_second},~\ref{definition:existential_quantification_higher},~\ref{definition:existential_quantification_first} and~\ref{definition:lfp_in_phfl}. The function translates a bisimulation invariant HO(LFP)$^{k+1}$ formula to a PHFL$^k$ formula.

\begin{definition}
    \label{definition:lower_bounds_phfl_formula_function}
   Define $F$ as the function that maps a bisimulation invariant HO(LFP)$^{k+1}$ formula $\varphi$ to a PHFL$^k$ formula with dimension $d$, where $d$ and $s$ are the constants as described in Remark~\ref{remark:dimension} and $\Phi_\sim$ is the formula of Example~\ref{example:phfl_order_0}, then $F$ is defined
    inductive over $\varphi$ as follows:
    \begin{align*}
        F(p(x_i)) \coloneqq &\, p_{2s+i} \\
        F(a(x_i, x_j)) \coloneqq &\, \langle a \rangle_{2s+i} \{(2s+i, 2s+j, \\
        &\,3, \dots, d)\} \Phi_\sim \\
        F(\Phi \vee \Psi) \coloneqq &\, F(\Phi) \vee F(\Psi) \\
        F(\neg \Phi) \coloneqq &\, \neg F(\Phi) \\
        F(\exists (x_i \colon \odot).\,\Phi) \coloneqq &\, \exists_{2s+i} F(\Phi) \\
        F(\exists (X \colon \tau).\,\Phi) \coloneqq &\, \exists^\tau X.\,F(\Phi(X)) \\
        F([LFP\;\Phi(X, x_{i_1}, \dots, x_{i_n})](v_{j_1}, \dots, v_{j_n})) \coloneqq &\,\{(j_1, \dots, j_n, n + 1, \dots, d)\} \\
        &\,LFP^{(\odot, \dots, \odot)} X.\, F(\Phi) \\
        F([LFP\;\Phi(X, X_1, \dots, X_n)](V_1, \dots, V_n)) \coloneqq &\,(\dotsb \big(LFP^\tau X.\, F(\Phi)\big)\,V_1)\dotsb)\,V_n \\
        F(X(x_{i_1}, \dots, x_{i_n})) \coloneqq &\, \{(2s+i_1, \dots, 2s+i_n, \\
        &\,n + 1, \dots, d)\}X\\
        F(X(X_1, \dots, X_n)) \coloneqq &\, (\dotsb (X\,X_1)\dotsb)\,X_n
    \end{align*}
\end{definition}
Keep in mind that we are working on LTS, that means that the relations in the signatures for HO(LFP) formulas have either arity one or two. Relations with arity one represent the 
propositions and those with arity two the actions of an LTS.

\subsection{Variables}\label{subsec:lower_bounds_variables}

After we encoded HO(LFP)$^{k+1}$ syntactically in PHFL$^k$ formulas, the last step is to translate the 
interpretation of variables. As described in Section~\ref{sec:existential_quantifiers_in_phfl} the variables in HO of 
types with order $3$ or higher are not supported in PHFL. Also first-order variables are not supported. Therefore, we 
define a function that maps a given variable mapping for a HO formula to the correct variable mapping for PHFL 
semantics. This function ignores the mapping of first-order variables, maps second-order variables to sets and 
higher-order variables to the corresponding characteristic function. Note, that the sets of order $2$ in HO have order 
$0$ in PHFL. The first-order variables of an HO formula that are marked as undefined in the following function, can be 
mapped to any arbitrary value because we do not use them in PHFL directly.

    Note that HO variables of order $2$ or higher are syntactically equivalent to the variables used in the translated PHFL formulas. To distinguish them in this definition an HO variable $X$ is denoted by $\hat{X}$ for the usage in PHFL context. 

\begin{definition}
    \label{definition:lower_bound_variable_function}
    Let $d$ be the constant as described in Remark~\ref{remark:dimension}, let $\eta$ be a
    variable mapping for a HO$^k$ formula and let $\mathcal{T} = (Q, \Sigma, P, \Delta, v)$ be a reduced LTS, then $\eta_V$ is a 
    variable mapping for a PHFL$^k$ formula with dimension $d$. 
    
    Variable mapping $\eta_V$ is defined as:
    \[\eta_V(\hat{X})\coloneqq
    \begin{cases}
        \text{undefined}, & \text{if } X \text{ is of type } \odot \\
        A,  & \text{if } X \text{ is of type } (\odot, \dots, \odot)\\
        G, & \text{if } X \text{ is of type } (\tau, \dots, \tau) \text{ and } \tau \neq \odot,
    \end{cases}\]
    where $A \subseteq Q^d$ such that $(q_1, \dots, q_n, q_{n + 1}, \dots, q_d) \in A$ iff $(q_1, \dots, q_n) \in
    \eta(X)$ and $G$ is a function of type $T((\tau, \dots, \tau))$ defined as follows:
    \begin{align*}
        (\dotsb\big(G\,\eta_V(\hat{X_1})\big)\dotsb)\,\eta_V(\hat{X_n}) &= Q^d &\text{ iff } (\eta(X_1), \dots, \eta(X_n)) \in \eta(X)\\
        (\dotsb\big(G\,\eta_V(\hat{X_1})\big)\dotsb)\,\eta_V(\hat{X_n}) &= \emptyset &\text{ iff } (\eta(X_1), \dots, \eta(X_n))
        \not\in \eta(X)
    \end{align*}
\end{definition}

The following example shows how a set of higher type variables will be translated into the characteristic function via the
variable mapping $\eta_V$ of Definition~\ref{definition:lower_bound_variable_function}.

\begin{example}
    Let $\mathcal{A}$ be a $\sigma$-structure over universe $\mathcal{U} = \{1, 2, 3\}$, let $X$ be a HO(LFP)$^{k + 1}$
    variable of type $((\odot, \odot), (\odot, \odot))$ mapped through variable mapping $\eta$ to
    \[\eta(X) = \{(\{(1, 1)\}, \{(2, 2)\}), (\{(1, 1), (2, 2)\}, \{(3, 3)\})\},\]
    let $\mathcal{T} = (\mathcal{U}, \Sigma, P, \Delta, v)$ be a reduced LTS and $d$ a dimension of PHFL, then $\eta_V(X)$ is a PHFL$^k$ function of type $\bullet \rightarrow (\bullet \rightarrow \bullet)$ such
    that $\eta_V(X)$ $(\{(1, 1)\}) = f$, $\eta_V(X)(\{(1, 1), (2, $ $2)\}) = g$ and $\eta_V(X)(z) = h$ for $z \in
    \mathcal{U}^d$ where $z \neq \{(1, 1)\}$ and $z \neq \{(1, 1), (2, 2)\}$. Moreover, $f$, $g$ and $h$ are functions of type $\bullet
    \rightarrow \bullet$ where $f(\{(2, 2)\}) = g(\{(3, 3)\}) = \mathcal{U}^d$ and $f(z) = g(z') = h(z'') = \emptyset$ for $z,
    z', z'' \in \mathcal{U}^d$ where $z \neq \{(2, 2)\}$ and $z' \neq \{(3, 3)\}$.
\end{example}

\subsection{Correctness Proof}\label{subsec:lower_bounds_correctness_lfp}

The last step is to show such that the query that can be defined by the PHFL$^k$ formula $\Psi$ that is obtained by transforming a HO(LFP)$^{k+1}$ formula $\Phi$ can be achieved from the query defined by $\Phi$ via projection to the relevant components. that for any query that can be associated to a HO(LFP)$^{k+1}$ formula $\Phi$ coincides with the query that can be associated to the PHFL$^k$ formula that is obtained by transforming $\Phi$. As mentioned in Section~\ref{sec:lower_bounds_preparation} without loss of generality the statement can be proven by consider only  reduced LTS. 

To make the correctness proof clearer there is one last remark.

\begin{remark}
    It holds for any HO(LFP)$^{k+1}$ formula $\Phi$ that for PHFL$^k$ formula $F(\Phi)$ and some type environment $\Gamma$ the type judgment $\Gamma \vdash
    \Phi \colon \bullet$ is derivable. This statement
    is easy proved by induction over the structure of formula $\Phi$. 
\end{remark}

Because the type judgement is always derivable we ignore the type environment in the following proof and write just $\llbracket \Phi \rrbracket^\eta_\mathcal{T}$ instead of $\llbracket \Gamma \vdash F(\Phi) \colon \tau \rrbracket^\eta_\mathcal{T}$, where $\Phi$ is a PHFL formula, $\eta$ is a variable mapping, $\mathcal{T}$ is an LTS, $\Gamma$ is a type environment and $\tau$ is a PHFL type.

\begin{theorem}
    \label{theorem:ho_lfp_equals_phfl}
    Let $d, f \geq 1$ and $k \geq 0$. For every bisimulation-invariant formula $\Phi$ of HO(LFP)$^{k + 1}$ there is a
    PHFL$^k$ formula $\Psi$ such that a projection on the $d$-adic query $\mathcal{Q}_\Phi^d$ that is defined by $\Phi$ is equal to the $f$-adic query $\mathcal{Q}_\Psi^f$ that is defined by $\Psi$.
\end{theorem}

\begin{proof}
    This lemma can be proven by showing for all HO(LFP)$^{k+1}$ formulas $\Phi$ with first-order variables $x_1,
    \dots, x_q$, all reduced LTS $\mathcal{T} = (Q, \Sigma, P,
    \Delta, v)$ with respect to $\emph{q}_r = q_1, \dots, q_r$ and all variable mappings $\eta$ that it holds that $\mathcal{T}, \eta \models \Phi$ iff $\emph{q} =
    (\emph{q}_s, \emph{q}_s, \emph{q}_q, \emph{q}_r)$ and $\emph{q} \in \llbracket
   F(\Phi)\rrbracket^{\eta_V}_{\mathcal{T}}$, where $\emph{q}_s = q_1',\dots,q_s'$ is a sequence of $s$ placeholders used for the interaction of second-order variables, $\emph{q}_q = \eta(x_1), \dots, \eta(x_q)$ is a sequence of first-order variables that are mapped by $\eta$ where $\eta(x_i) = q_0$ if $x_i$ is bound to a quantifier, $q_0, q_1', \dots, q_s' \in Q$ are arbitrary states, $F$ is the formula function of
    Definition~\ref{definition:lower_bounds_phfl_formula_function} and $\eta_V$ the variable mapping of
    Definition~\ref{definition:lower_bound_variable_function}. This statement can be proven by induction over formula
    $\Phi$.
    \begin{compactitem}
        \item In case of $\Phi = p(x_i)$ where $x_i$ is a free first-order variable then $\mathcal{T}, \eta \models \Phi$ holds exactly then if $\eta(x_i) \in
        p^\mathcal{T}$. Translated to the normal LTS definition of $\mathcal{T}$ it is the same as $p \in v(\eta(x_i))$.
        With $F(\Phi) = p_{2s+i}$ this is exactly
        \begin{align*}
            (\emph{q}_s, \emph{q}_s, \eta(x_1),& \dots, \eta(x_{i-1}), \eta(x_{i}), \eta(x_{i+1}), \dots, \eta(x_q), \emph{q}_r) \in
            \llbracket F(\Phi) \rrbracket^{\eta_V}_\mathcal{T}.
        \end{align*}
        
        \item In case of $\Phi = a(x_i, x_j)$ where $x_i$ and $x_j$ are free first-order variables then $\mathcal{T}, \eta \models \Phi$ holds exactly then if $(\eta(x_i)
        , \eta(x_j))$ $ \in a^\mathcal{T}$. Translated to the normal LTS definition of $\mathcal{T}$ it is the same as $
        \eta(x_i) \overset{a}{\rightarrow} \eta(x_j)$. Let $g: \{1, 2\} \rightarrow \{i, j\}$ be a function.
        By  definition of the semantics of $\langle a \rangle_{2s+i}$ the tuple
        \begin{align*}
            (\emph{q}_s, &\emph{q}_s, \eta(x_1), \dots, \eta(x_{g(1) - 1}), \eta(x_{g(1)}), \eta(x_{g(1)+1}), \dots, \eta(x_{g(2)-1}), \eta
            (x_{g(2)}),\\& \eta(x_{g(2)+1}), \dots, \eta(x_q), \emph{q}_r)
        \end{align*}
        is an element of the semantics of $\langle a \rangle_{2s+i} \Phi_\sim$ if
        \begin{align*}
            (\emph{q}_s, &\emph{q}_s, \eta(x_1), \dots, \eta(x_{g(1) - 1}), q_m, \eta(x_{g(1)+1}), \dots, \eta(x_{g(2)-1}), q_n,\\& \eta(x_{g(2)
            +1}), \dots, \eta(x_q), \emph{q}_r)
        \end{align*}
        is an element of the semantics of $\Phi_\sim$ where $\eta(x_i)$ can
        move by an $a$ action to $q_m$ if $g(1) = i$ or to $q_n$ if $g(2) = i$. Because $\eta(x_j)$ is the state that
        have to be reached via an $a$ action from $\eta(x_i)$ we have to check if $q_m = \eta(x_j)$ in case of $g(1)
        = j$ and in case of $g(2) = j$ we have to check if $q_n = \eta(x_i)$. If two states are equal in a reduced LTS is
        the same to check if those states are bisimilar. $\sim$ is given by formula $\Phi_\sim$ of
        Example~\ref{example:phfl_order_0}.
        Because this formula returns those $d$-tuples where the first and second component are bisimilar, we have to
        move the $2s+i$-th and $2s+j$-th component to the first and second component. This is given by $\{(2s+i, 2s+j, 3, \dots, d
        )\} \Phi_\sim$. Summarizing all these steps with $F(\Phi) = \langle a \rangle_{2s+i} \{(2s+i, 2s+j, 3, \dots, d)\}
        \Phi_\sim$ it follows
        \begin{align*}
            (\emph{q}_s, &\emph{q}_s, \eta(x_1), \dots, \eta(x_{g(1) - 1}), \eta(x_{g(1)}), \eta(x_{g(1)+1}), \dots, \eta(x_{g(2)-1}), \eta
            (x_{g(2)}),\\& \eta(x_{g(2)+1}), \dots, \eta(x_q), \emph{q}_r) \in \llbracket F(\Phi) \rrbracket^{\eta_V}_\mathcal{T}
        \end{align*}
        if $(\eta(x_i), \eta(x_j))$ $ \in a^\mathcal{T}$ and
        \begin{align*}
            (\emph{q}_s, &\emph{q}_s, \eta(x_1), \dots, \eta(x_{g(1) - 1}), \eta(x_{g(1)}), \eta(x_{g(1)+1}), \dots, \eta(x_{g(2)-1}), \eta
            (x_{g(2)}),\\& \eta(x_{g(2)+1}), \dots, \eta(x_q), \emph{q}_r) \not\in \llbracket F(\Phi) \rrbracket^{\eta_V}_\mathcal{T}
        \end{align*}
        if $(\eta(x_i), \eta(x_j))$ $ \not\in a^\mathcal{T}$.

        \item In case of $\Phi = X(x_{i_1}, \dots, x_{i_n})$ where $X$ is a free variable of HO type $(\odot, \dots,
        \odot)$ and $x_{i_1}, \dots, x_{i_n}$ are free first-order variables then $\mathcal{T}, \eta \models \Phi$
        holds exactly then if $(\eta(x_{i_1}), \dots \eta(x_{i_n})) \in \eta(X)$. Because of definition of $\eta_V$ the
        tuple $(\eta(x_{i_1}), \dots \eta(x_{i_n}), q_{n + 1}', \dots, q_{s}', \emph{q}_s, \emph{q}_q, \emph{q}_r)$ is in $
        \eta_V(X)$ if $(\eta(x_{i_1}),\dots \eta(x_{i_n})) \in \eta(X)$ and is not in $\eta_V(X)$ otherwise. Because
         components $1, \dots, n$ are not set to the mappings of first-order variables $x_{i_1}, \dots, x_{i_n}$, we first move the components $2s+i_1, \dots, 2s+i_n$ to components $1, \dots, n$ respectively and check then if
        \[(\eta(x_{i_1}), \dots \eta(x_{i_n}), q_{n + 1}', \dots, q_{s}', \emph{q}_s, \emph{q}_q, \emph{q}_r) \in \eta_V(X).\]
        So it holds with $F(\Phi) = {(2s+i_1, \dots, 2s+i_n, n+1, \dots, d)}X$ and function $g: \{1, \dots, n\}
        \rightarrow \{i_1, \dots, i_n\}$ that
        \begin{align*}
            (\emph{q}_s, &\emph{q}_s, \eta(x_1), \dots, \eta(x_{g(1)-1}), \eta(x_{g(1)}), \eta(x_{g(1)+1}), \dots \eta(x_{g(2)-1}), \eta
            (x_{g(2)}),\\& \eta(x_{g(2)+1}), \dots, \eta(x_{g(n)-1}), \eta(x_{g(n)}), \eta(x_{g(n)+1}), \dots, \eta
            (x_q),\\& \emph{q}_r) \in \llbracket F(\Phi) \rrbracket^{\eta_V
            }_\mathcal{T}
        \end{align*}
        if $(\eta(x_{i_1}), \dots \eta(x_{i_n})) \in \eta(X)$ and
        \begin{align*}
            (\emph{q}_s, &\emph{q}_s, \eta(x_1), \dots, \eta(x_{g(1)-1}), \eta(x_{g(1)}), \eta(x_{g(1)+1}), \dots \eta(x_{g(2)-1}), \eta
            (x_{g(2)}),\\& \eta(x_{g(2)+1}), \dots, \eta(x_{g(n)-1}), \eta(x_{g(n)}), \eta(x_{g(n)+1}), \dots, \eta
            (x_q),\\& \emph{q}_r) \not\in \llbracket F(\Phi) \rrbracket^{\eta_V
            }_\mathcal{T}
        \end{align*}
        if $(\eta(x_{i_1}), \dots \eta(x_{i_n})) \not\in \eta(X)$.

        \item In case of $\Phi = X(X_1, \dots, X_n)$ where $X$ is a free variable of HO type $(\tau, \dots,
        \tau)$ and $X_1, \dots, X_n$ are free variables of HO type $\tau$ then $\mathcal{T}, \eta \models \Phi$
        holds exactly then if $(\eta(X_1), \dots \eta(X_n)) \in \eta(X)$. Because of definition of $\eta_V$ it follows
        \[(\dotsb\big(\eta_V(X)\,\eta_V(X_1)\big)\dotsb)\,\eta_V(X_n) = Q^d\]
        if $(\eta(X_1), \dots \eta(X_n)) \in \eta(X)$ and
        \[(\dotsb\big(\eta_V(X)\,\eta_V(X_1)\big)\dotsb)\,\eta_V(X_n) = \emptyset\]
        if $(\eta(X_1), \dots \eta(X_n)) \not\in \eta(X)$. With $F(\Phi) = (\dotsb (X\,X_1)\dotsb)\,X_n$
        it follows
        \[ \emph{q} \in \llbracket F(\Phi) \rrbracket^{\eta_V}_\mathcal{T} = Q^d.\]
        if $(\eta(X_1), \dots \eta(X_n))\in \eta(X)$ and
        \[ \emph{q} \not\in \llbracket F(\Phi)
        \rrbracket^{\eta_V}_\mathcal{T} = \emptyset.\]
        if $(\eta(X_1), \dots \eta(X_n)) \not\in \eta(X)$.
    \end{compactitem}
    
    By induction hypothesis it holds for HO(LFP)$^{k+1}$ formulas $\Psi$ and $\Psi'$ with first-order variables $x_1,
    \dots, x_q$, all reduced LTS $\mathcal{T} = (Q, \Sigma, P,
    \Delta, v)$ with respect to $\emph{q}_r$ and all variable mappings $\eta$  that $\mathcal{T}, \eta \models \Psi$ iff $\emph{q} \in \llbracket
   F(\Psi)\rrbracket^{\eta_V}_{\mathcal{T}}$ and $\mathcal{T}, \eta \models \Psi'$ iff $\emph{q} \in \llbracket
   F(\Psi')\rrbracket^{\eta_V}_{\mathcal{T}}$, where $\emph{q} =
    (\emph{q}_s, \emph{q}_s, \emph{q}_q, \emph{q}_r)$.
    
    \begin{compactitem}
        \item In case of $\Phi = \neg \Psi$ it follows that $\mathcal{T}, \eta \models \Phi$ exactly then if
        $\mathcal{T}, \eta \not\models \Psi$. By induction hypothesis that is exactly then the case when
        \[ \emph{q} \not\in \llbracket F(\Psi) \rrbracket^{\eta_V}_\mathcal{T}.\]
        This is exactly the case if
        \[ \emph{q} \in Q^d \setminus \llbracket F(\Psi) \rrbracket^{\eta_V}_\mathcal{T}.\]
        And this is exactly the semantics of $F(\Phi) = \neg F(\Psi)$.

        \item In case of $\Phi = \Psi \vee \Psi'$ it follows that $\mathcal{T}, \eta \models \Phi$ exactly then if
        $\mathcal{T}, \eta \models \Psi$ or $\mathcal{T}, \eta \models \Psi'$. By induction hypothesis that is
        exactly then the case when
        \[\emph{q} \in \llbracket F
        (\Psi) \rrbracket^{\eta_V}_\mathcal{T}\]
        or
        \[\emph{q} \in \llbracket F
        (\Psi') \rrbracket^{\eta_V}_\mathcal{T}.\]
        Because $\sqcup_{\bullet} = \cup$ this can be combined to
        \[\emph{q} \in \llbracket F
        (\Psi) \rrbracket^{\eta_V}_\mathcal{T} \sqcup_\bullet \llbracket F
        (\Psi') \rrbracket^{\eta_V}_\mathcal{T},\]
        which is as desired.

        \item In case of $\Phi = \exists (x_i\colon \odot).\,\Psi$ it follows that $\mathcal{T}, \eta \models \Phi$ iff
        there exists  $\mathcal{X} \in Q$ with $\mathcal{T}, \eta' \models \Psi$, where $\eta'$ is a variable mapping with $\eta'(x) = \eta(x)$ for all variables $x \neq x_i$ and $\eta'(x_i) = \mathcal{X}$. By induction hypothesis it holds that $\mathcal{T}, \eta' \models \Psi$ is exactly the case when
        \begin{align*}
            (\emph{q}_s, \emph{q}_s, \eta'(x_1), \dots, \eta'(x_{i-1}), \eta'(x_i), \eta'(x_{i+1}), \dots, \eta(x_{q}), \emph{q}_r) \in
            \llbracket F(\Psi) \rrbracket^{{\eta'_V}}_\mathcal{T}.
        \end{align*}
        To reach the value of $\eta'(x_i)$ we have to replace the $2s+i$-th component by one of 
        the last $r$ components and move through all reachable states. By Observation~\ref{observation:existential_quantification_first} the formula defined in Definition~\ref{definition:existential_quantification_first} fulfils this behaviour. Because the first-order variable $x_i$ is represented by the $2s+i$-th component and $F(\Phi) = \exists_{2s+i} \Psi$, we replace in $F(\Phi)$ the $2s+i$-th component by one of the last $r$ components, move through all reachable states and checking if $F(\Psi)$ holds. That means it holds 
        \[\emph{q} \in \llbracket F
        (\Phi) \rrbracket^{\eta_V}_\mathcal{T}\]
        iff $\mathcal{T}, \eta \models \Phi$.

        \item In case of $\Phi = \exists (X \colon \tau).\,\Psi$ it follows that $\mathcal{T}, \eta \models \Phi$ iff
        there exists $\mathcal{X} \in D_\tau(Q)$ with $\mathcal{T}, \eta' \models \Psi$, where $\eta'$ is a variable mapping with $\eta'(x) = \eta(x)$ for all variables $x \neq X$ and $\eta'(X) = \mathcal{X}$.
        By induction hypothesis it follows that $\emph{q} \in
        \llbracket F(\Psi) \rrbracket^{{\eta'_V}}_\mathcal{T}$ iff $\mathcal{T}, \eta' \models \Psi$. By Lemma~\ref{lemma:existential_quantifier_higher} the formula $\exists^\tau X.\, \Psi(X)$ is semantically equivalent to
        $\underset{\mathcal{X} \in \llbracket \tau \rrbracket_\mathcal{T}}{\bigsqcup} \llbracket \Psi(X) \rrbracket^{\eta'}_\mathcal{T}$.
        It follows with $F(\Phi) = \exists^\tau X.\, F(\Psi)(X)$ that it holds \[\emph{q} \in
        \llbracket F(\Phi) \rrbracket^{\eta_V}_\mathcal{T}.\]

        \item In case of $\Phi = [LFP\,\Psi(X, x_{i_1}, \dots, x_{i_n})](v_{j_1}, \dots, v_{j_n})$, where $X$ is a
        free variable in $\Psi$ of HO type $(\odot, \dots, \odot)$ and $x_{i_1}, \dots, x_{i_n}$ are free first-order
        variables of $\Psi$ and $v_{j_1}, \dots, v_{j_n}$ are first-order variables of $\Phi$, then it follows that
        $\mathcal{T}, \eta \models \Phi$ exactly then if $(\eta(v_{j_1}), \dots, \eta(v_{j_n})) \in LFP
        (F_\Psi^\mathcal{T})$. By definition of LFP the tuple $(\eta(v_{j_1}), \dots, \eta(v_{j_n}))$ is in
        $LFP(F_\Psi^\mathcal{T})$ iff $X$ is the smallest $X$ such that $X = F_\Psi^\mathcal{T}(X)$ and $(\eta(v_{j_1}), \dots, \eta(v_{j_n})) \in
        F_\Psi^\mathcal{T}(X)$. By definition of $F_\Psi^\mathcal{T}(X)$ it holds $(\eta
        (v_{j_1}), \dots, \eta(v_{j_n})) \in F_\Psi^\mathcal{T}(X)$ exactly then if $\mathcal{T}, \eta' 
        \models \Psi$, where $\eta'$ is a variable mapping with $\eta'(x) = \eta(x)$ for all variables $x \neq X$ and $\eta'(X) = F_\Psi^\mathcal{T}(X)$. By induction hypothesis this is exactly the case if
        \begin{align*}
        (\emph{q}_s, &\emph{q}_s, \eta(x_1), \dots, \eta(v_{g(1)-1}), \eta(v_{g(1)}), \eta(v_{g(1)+1}), \dots \eta(v_{g(2)-1}),\\& \eta
            (v_{g(2)}), \eta(v_{g(2)+1}), \dots, \eta(v_{g(n)-1}), \eta(v_{g(n)}), \eta(v_{g(n)+1}), \dots, \eta
            (x_q), \\& \emph{q}_r) \in \llbracket
        F(\Psi) \rrbracket^{\eta'_V}_\mathcal{T},
        \end{align*}
         where $g: \{1, \dots, n\} $ $\rightarrow \{j_1, \dots, j_n\}$ is a function.
        
        The next step is to show that $\eta'_V(X)$ is also a least fixpoint of $\llbracket
         \mu(X\colon \bullet).\,$ $F(\Psi) \rrbracket^{\eta'_V}_\mathcal{T}$. By Theorem~\ref{theorem:kleene} the least fixpoint $\eta'(X)$ can be calculated by a sequence $X_0, \dots, X_m
         $ where here $X_0 = \emptyset = \eta^0(X)$ and $X_{i+1} = F_\Psi^\mathcal{T}(X_i) = \eta^{i+1}(X)$ and $\eta'(X) = X_m = \eta^m(X)$. On the other hand the least fixpoint $\eta'_V(X)$ can be calculated by a sequence 
         $Y_0, \dots, Y_{m'}$ where here $Y_0 = \emptyset = \eta^0_V(X)$ and $Y_{i+1} = \llbracket F(\Psi)\rrbracket_\mathcal{T}^{\eta'_V[X \mapsto Y_i]} = \eta^{i+1}_V(X)$ and $\eta'_V(X) = Y_{m'} = \eta^m_V(X)$. We now show by induction that $X_i$ corresponds with $Y_i$ in PHFL context for all $i$. Obviously $X_0 = \emptyset = Y_0$. By induction hypothesis it holds $X_i$ corresponds with $Y_i$ in PHFL context for 
         one $i$. Then $X_{i+1} = F_\Psi^\mathcal{T}(X_i)$ what is
         \[\{(a_1, \dots, a_n) \mid \mathcal{T}, \eta'' \models \Psi\},\] 
         where $\eta''$ is a variable mapping with $\eta''(x) = \eta(x)$ for all variables $x \neq X$ and $\eta''(x_{i_1}) = a_1, \dots,  \eta''(x_{i_n}) = a_n$ and $\eta''(X) = X_i$. 
         By using the induction hypothesis, $X_i$ corresponds with $Y_i$ in PHFL context, and the main induction hypothesis for the proof of this theorem, this is exactly the same to
                 \begin{align*}
                 \{(a_1, \dots, a_n) \mid 
        (\emph{q}_s, &\emph{q}_s, \eta''(x_1), \dots, \eta''(v_{g(1)-1}),\\& \eta''(v_{g(1)}), \eta''(v_{g(1)+1}), \dots \eta''(v_{g(2)-1}),\\& \eta''
            (v_{g(2)}), \eta''(v_{g(2)+1}), \dots, \eta''(v_{g(n)-1}), \\&\eta''(v_{g(n)}), \eta''(v_{g(n)+1}), \dots, \eta''
            (x_q), \\& \emph{q}_r) \in \llbracket
        F(\Psi) \rrbracket^{\eta''_V[X\mapsto Y_i]}_\mathcal{T}\}
        \end{align*}
		Because $\eta''_V(x) = \eta'_V(x)$ for all variable $x$ it follows $Y_{i+1} = \llbracket F(\Psi)\rrbracket_\mathcal{T}^{\eta''_V[X \mapsto Y_i]}$ that means        
        \begin{align*}
                 \{(a_1, \dots, a_n) \mid 
        (\emph{q}_s, &\emph{q}_s, \eta''(x_1), \dots, \eta''(v_{g(1)-1}),\\& \eta''(v_{g(1)}), \eta''(v_{g(1)+1}), \dots \eta''(v_{g(2)-1}),\\& \eta''
            (v_{g(2)}), \eta''(v_{g(2)+1}), \dots, \eta''(v_{g(n)-1}), \\&\eta'(v_{g(n)}), \eta''(v_{g(n)+1}), \dots, \eta''
            (x_q), \\& \emph{q}_r) \in Y_{i+1}\}.
        \end{align*}
         and it follows $X_{i+1}$ corresponds with $Y_{i+1}$ in PHFL context. The induction holds.
        
Because of the construction of variable mapping $\eta'_V$ and $(\eta'(v_{j_1}), \dots, 
        \eta'(v_{j_n})) \in \eta'(X)$ the
        tuple $(\eta'(v_{j_1}), \dots, \eta'(v_{j_n}), q_{n+1}', \dots, q_s', \emph{q}_s, \emph{q}_q, \emph{q}_r)$ is also in $\eta'_V(X)$.         
        
       Because components $1, \dots, n$ are not set to the mappings of first-order variables 
       $v_{j_1}, \dots, v_{j_n}$, we first move the components $2s+j_1, \dots, 2s+j_n$ to components $1, \dots, n$ respectively and check then the least fixpoint operator.
        So it holds with $F(\Phi) = \{2s+j_1, \dots, 2s+j_n, n+1, \dots, d\} \mu (X).\, F(\Psi)$ and
        function $g: \{1, \dots, n\} $ $\rightarrow \{j_1, \dots, j_n\}$ that
        \begin{align*}
            (\emph{q}_s, \emph{q}_s, \eta(x_1),& \dots, \eta(v_{g(1)-1}), \eta(v_{g(1)}), \eta(v_{g(1)+1}), \dots \eta(v_{g(2)-1}), \eta
            (v_{g(2)}),\\& \eta(v_{g(2)+1}), \dots, \eta(v_{g(n)-1}), \eta(v_{g(n)}), \eta(v_{g(n)+1}), \dots, \eta
            (x_q),\\& \emph{q}_r) \in \llbracket  F(\Phi) \rrbracket^{\eta_V
            }_\mathcal{T}
        \end{align*}
        exactly then if $\mathcal{T}, \eta \models \Phi$.

        \item In case of $\Phi = [LFP\,\Psi(X, X_1, \dots, X_n)](V_1, \dots, V_n)$, where $X$ is a
        free variable in $\Psi$ of HO type $(\tau, \dots, \tau)$ and $X_1, \dots, X_n$ are free
        variables of $\Psi$ of type $\tau$ and $V_1, \dots, V_n$ are free variables of $\Phi$ also of type $\tau$, then
        it follows that $\mathcal{T}, \eta \models \Phi$ exactly then if $(\eta(V_1), \dots, \eta(V_n) \in LFP
        (F_\Psi^\mathcal{T})$. By definition of LFP the tuple $(\eta(V_1), \dots, \eta(V_n))$ is in
        $LFP(F_\Psi^\mathcal{T})$ iff $X$ is the smallest $X$ such that $X = F_\Psi^\mathcal{T}(X)$ and $(\eta(V_1), \dots, \eta(V_n)) \in
        F_\Psi^\mathcal{T}(X)$. By definition of $F_\Psi^\mathcal{T}(X)$ it holds $(\eta
        (V_1), \dots, \eta(V_n)) \in F_\Psi^\mathcal{T}(X)$ exactly then if $\mathcal{T}, \eta' 
        \models \Psi$, where $\eta'$ is a variable mapping with $\eta'(x) = \eta(x)$ for all variables $x \neq X$ and $\eta'(X) = F_\Psi^\mathcal{T}(X)$. By induction hypothesis it holds
        \[\emph{q} \in \llbracket  F(\Psi)
        \rrbracket^{\eta_V}_\mathcal{T}.\]

        
The next step is to show that $\eta'_V(X)$ is also a least fixpoint of $\llbracket
         \mu(X\colon (\tau \dots, \tau)).\,$ $F(\Psi) \rrbracket^{\eta_V}_\mathcal{T}$. By Theorem~\ref{theorem:kleene} the least fixpoint $\eta'(X)$ can be calculated by a sequence $X_0, 
         \dots, X_m$ where here $X_0 = \emptyset$ and $X_{i+1} = F_\Psi^\mathcal{T}(X_i)$ and $\eta'(X) = X_m$. On the other hand the least fixpoint $\eta'_V(X)$ can be calculated by a 
         sequence $Y_0, \dots, Y_{m'}$ where here 
         $Y_0 = \bot_{T((\tau, \dots, \tau))}$ 
         and $Y_{i+1} = \llbracket F(\Psi)\rrbracket_\mathcal{T}^{\eta'_V[X \mapsto Y_i]}$. We now show by induction that $X_i$ corresponds with $Y_i$ in PHFL context for all $i$. For the induction basis it holds $X_0 = \emptyset$. Because $X$ in HO context is of type $(\tau, \dots \tau)$ the same $X$ has in PHFL context the type $T((\tau, \dots, \tau))$. An empty set of type $(\tau, \dots \tau)$ is mapped by variable mapping of Definition~\ref{definition:variable_mapping_function} to  $\lambda (X_1 \colon T(\tau)).\, \dots \lambda (X_n \colon T(\tau)).\,\bot$ and this is equal to $Y_0$. By induction hypothesis it holds $X_i$ corresponds with $Y_i$  in PHFL context for 
         one $i$. Then $X_{i+1} = F_\Psi^\mathcal{T}(X_i)$ what is
         \[\{(a_1, \dots, a_n) \mid \mathcal{T}, \eta'' \models \Psi\},\]
         where $\eta''$ is a variable mapping with $\eta''(x) = \eta(x)$ for all variables $x \neq X$ and $\eta''(X_{1}) = a_1, \dots,  \eta''(X_{n}) = a_n$ and $\eta''(X) = X_i$.  
         By using the small induction hypothesis and the main induction hypothesis at once this is exactly the case when
                 \begin{align*}
                 \{(a_1, \dots, a_n) \mid 
        \emph{q} \in \llbracket
        F(\Psi) \rrbracket^{\eta''_V[X\mapsto Y_i]}_\mathcal{T}\}
        \end{align*}
		Because $Y_{i+1} = \llbracket F(\Psi)\rrbracket_\mathcal{T}^{\eta''_V[X \mapsto Y_i]}$ it follows        
        \begin{align*}
                 \{(a_1, \dots, a_n) \mid 
        \emph{q} \in Y_{i+1}\}.
        \end{align*}
                Because of the construction of variable mapping $\eta'_V$ and $(\eta'(V_1), \dots, \eta'(V_n)) \in \eta'(X)$ it holds
        $(\dotsb\big(\eta'_V(X)\,\eta'_V(V_1)\big) \dotsb )\,\eta'_V(V_n) = Q^d$
        and so 
        \begin{align*}
        \emph{q} \in (\dotsb \big(\eta'_V(X)\,\eta'_V(V_1)\big) \dotsb )\,\eta'_V(V_n).
        \end{align*} 
       then it follows $X_{i+1} = Y_{i+1}$. The induction holds.        

        With $F(\Phi) = (\dotsb \big(\mu (X \colon T(\tau)).\,\Phi(X)\big)\,V_1)\dotsb)\,V_n$ it follows
        \[\emph{q} \in \llbracket
         F(\Phi) \rrbracket^{\eta_V}_\mathcal{T}.\]
        exactly then if $\mathcal{T}, \eta \models \Phi$.
    \end{compactitem}
\end{proof}

\begin{remark}
In the proof of Theorem~\ref{theorem:ho_lfp_equals_phfl} the dimension $d$ of $F(\Phi)$ is $2s+q+r$ but only the components where the free variables are represented are filled with input parameters. That means by only projecting these components in the by $F(\Phi)$ defined query $\mathcal{Q}^d_{F(\Phi)}$ we get the resulting query $\mathcal{Q}^f_{\Phi}$ that is defined by $\Phi$. 
\end{remark}

The combination of Theorem~\ref{theorem:ho_lfp_equals_phfl}, Theorem~\ref{theorem:hoLfpEqualsExptime} and
Theorem~\ref{theorem:phfl_k_in_k_exptime} proves the following theorem.

\begin{theorem}
    Let $k \geq 0$. PHFL$^k$ captures \exptime{$k$} over finite labeled transition systems.
\end{theorem}



%%
%% Author: DKron
%% 17.08.2018
%%

\section{Lower Bound of PHFL$^{k + 1}_{tail}$}\label{sec:lowerBoundOfPhflTail}

The lower bound of PHFL$^{k+1}_{tail}$ can be proven similar to the lower bound of PHFL$^k$. The main idea is not to
show directly, that the lower bound of PHFL$^{k+1}_{tail}$ is \expspace{$k$} but rather by detour over the
bisimulation-invariant fragment of HO(PFP)$^{k+1}$. In the first subsection we show that a Turing Machine that is in
$k$-EXPSPACE can be encoded by a HO(PFP)$^{k+1}$ formula. The next subsection uses this statement to show that the
lower bound of PHFL$^{k+1}_{tail}$ is \expspace{$k$} by encoding formulas of HO(PFP)$^{k+1}$ in PHFL$^{k+1}_{tail}$.

\subsection{$k$-EXPSPACE in HO(PFP)$^{k+1}$}\label{subsec:kExpspaceInHopfp}

In this subsection we want to show that a run of an $exp(k, f(n))$ space bounded DTM can be encoded by some
HO$^{k+1}$ formula. The main idea of this statement is an extension of the result of Abiteboul and
Vianu~\cite{abiteboul1995computing} into higher-order. They have shown that HO(PFP)$^1$ coincides with $0$-EXPSPACE.

%Because a finite $\sigma$-structure $\mathcal{A}$ for some relational signature $\sigma$ cannot be fed to a Turing
%%machine directly, we have to encode $\mathcal{A}$ to an input word $\langle \mathcal{A} \rangle$. We refer the
%interested reader to~\ref{}

\begin{lemma}
    \label{lemma:expspace_in_ho_pfp}
    Given a DTM $M = (Q, \Sigma, \Gamma, \delta, q_0, \Box, F, R)$ in $k$-EXPSPACE there exists a formula $\Psi$ in HO(PFP)$^{k+1}$ over signature $\sigma$ such that for all variable mappings $\eta$ and all $\sigma$-structures $\mathcal{A}$ it holds $\mathcal{A}, \eta \models \Psi$ exactly then if the run of $M$ on the standard coding of $(\mathcal{A}, \eta)$ is accepting.
\end{lemma}

\begin{proof}
    Let $M = (Q, \Sigma, \Gamma, \delta, q_0, \Box, F, R)$ be a $exp(k, f(n))$ space bounded DTM. Furthermore, let $\sigma$ be a relational signature, $\sigma_< = \sigma\;\cup
    <$ an extension of $\sigma$ and $\mathcal{A}$ a finite $\sigma_<$-structure including $<^\mathcal{A}$ that
    represents a linear ordering on the universe $\mathcal{U}$ of $\mathcal{A}$. Finally, let $\tau$ be an HO type of
    order $k + 1$ and $\alpha$ a variable mapping.
    To prove this lemma we want to define a relational representation, of the final configuration of $M$ that has the standard coding of $(\mathcal{A}, \alpha)$\footnote{The standard coding of structures is a non-trivial problem. Because the description of the encoding goes beyond the scope of this thesis, we only refer to~\cite{abiteboul1995computing} for further informations about.}, abbreviated with $w$, as input word,
    as a partial fixpoint of some HO$^{k+1}$ formula. For this we set up the underlying HO$^{k+1}$ formula
    $\varphi$ such that the iterates $(F_\varphi^\mathcal{A})^{i+1}(\emptyset)$ over $D_\tau
    (\mathcal{U})$ represent the $i$-th configuration of $M$ on input standard coding of $(\mathcal{A}, \alpha)$, for every $i$ until termination.

    Before we can define the configurations of $M$ in HO we have to make some preparations. Remember that a
    configuration of $M$ is a relation between the current state, the current reading head position and
    the current tape content represented by a function. These configurations will be combined in one relation $X$.
    Because the size of the formula we build have to be polynomial and the reading head of $M$ can be on
    one of $exp(k, f(n))$ cells for example, we have to encode the number in sets of order $k + 1$. Furthermore, to do
    not exceed the bound of order $k + 1$, we have to split the tape content function in this way that in one tuple
    $x$ of $X$ is just the current state, the current head position and one position of the tape with its content. By
    syntax of HO types each component of $x \in X$ have to be of the same type, so to access the different states and
    tape symbols they have to be numerated. $\{0, \dots, |Q| - 1\}$ for states and $\{0, \dots, |\Gamma| - 1\}$ for tape
    symbols.

    The next step is to define some abbreviations that we want to use in the definitions of the configurations. The
    first and most important abbreviation is the definition of orders of any HO type. These orders are defined similarly
    to the defined formulas of Definition~\ref{definition:lower_bound_less_higher}. A tuple $x$ is smaller then a tuple $y$ if there is a position $i$ where $x_i < y_i$ and there is no position $j<i$ where $x_j > y_j$. A set $X$ is smaller then a tuple $Y$ if there is a $x \in Y$ such that $x \not\in X$ and there is no $y<x$ such that $y\in X$ but $y\not\in Y$.
    \begin{align*}
        <^\odot(x, y) \coloneqq &\,<^\mathcal{A}(x, y) \\
        <^{\tau' \times \dots \times \tau'}(x_1, y_1, \dots, x_n, y_n) \coloneqq &\,\underset{i =
        1}{\overset{n}{\bigvee}}<^{\tau'}(x_i, y_i) \wedge \underset{j = 1}{\overset{i - 1}{\bigwedge}}
        \neg <^{\tau'}(y_j, x_j)\\
        <^{(\tau', \dots, \tau')}(X, Y) \coloneqq &\,\exists (x_1 \colon {\tau'}). \,\dots \exists(x_n \colon
        {\tau'}).\, Y(x_1, \dots, x_n)
        \\&\,\wedge \neg X(x_1, \dots, x_n)\,\wedge \forall (y_1 \colon {\tau'}). \,\dots
        \forall(y_1 \colon {\tau'}).\,\\&\,<^{\tau'\times \dots \times \tau'}
        (y_1, x_1, \dots, y_n, x_n) \\&\,\Rightarrow (X(y_1, \dots, y_n) \Rightarrow Y(y_1, \dots, y_n))
    \end{align*}
    With these formulas it is possible to define another two important abbreviations. On the one hand equality
    of two variables of arbitrary type and on the other hand the successor of a given element. If $X$ and
    $Y$ are two variables of type $\tau$ then the equality of $X$ and $Y$ is given by the formula
    \[X = Y \coloneqq \neg<^\tau(X, Y) \wedge \neg <^\tau(Y, X).\]
    Finally, if $X$ and $Y$ are two variables of type $\tau$ then the prove that $Y$ is the successor of $X$ is given by
    the formula
    \[next^{\tau}(X, Y) \coloneqq\; <^\tau(X, Y) \wedge \forall (Z \colon \tau).\, <^\tau(Z, Y) \Rightarrow\;<^\tau
    (Z, X).\]

    Now we are able to define the configurations in HO(PFP)$^k$. The first configuration is the initial configuration
    $C_0^M$. For input word $w$ this is given by the formula
    \begin{align*}
        \varphi_0(q, h, i, b) \coloneqq &\,q = q_0 \wedge h = 0 \wedge (\neg <^\tau(|w|, i) \Rightarrow b = w_{i})\,
        \vee\\&\,(\neg <^\tau (i, |w|) \wedge \neg i = |w| \Rightarrow b = \Box))
    \end{align*}
    where $q$ is the current state, $h$ the current head position, $i$ a tape index and $b$ the symbol on $i$. $q_0$,
    $0$, $|w|$, $w_{i}$ and $\Box$ are the numerical representations as sets of the same elements in $Q$ and $\Gamma$.

    To iterate through all configurations of $M$ on input $w$, we need a variable $X$ of type $\tau =
    (\tau', \tau', \tau', \tau')$ where $\tau'$ has order $k + 1$, so $X$ has order $k + 2$. On an iteration
    $(F_\varphi^\mathcal{A})^{i+1}(\emptyset)$ for the following formula $\varphi$ the variable $X$ includes
    all configurations that will be reached within $i$ transitions.
    \begin{align*}
        \varphi(X, q, h, i, b) \coloneqq &\,(\neg \exists (q_{old} \colon \tau').\, \neg\exists
        (h_{old}
        \colon \tau').\, \neg\exists (i_{old} \colon \tau').\, \neg\exists (b_{old} \colon \tau').\,\\
        &\, X(q_{old}, h_{old}, i_{old}, b_{old}) \wedge \varphi_0(q, h, i, b)) \,\vee \\
        &\,(\exists (q_{old} \colon \tau').\, \exists (h_{old} \colon \tau').\, \exists (i_{old} \colon
        \tau').\, \exists (b_{old} \colon \tau').\,\\
        &\, X(q_{old}, h_{old}, i_{old}, b_{old}) \wedge \xi(X, q, h, i, b))
    \end{align*}
    Note that $\varphi_0$ is invoked only in the first iteration and thus provides the correct initialisation. The
    formula $\xi$ collects the transitions of those tuples in $X$ according to the transition function $\delta$ of
    $M$. In each iteration exactly one configuration will be added to $X$ because $M$ is deterministic.
    The formula $\xi$ is given by
    \begin{align*}
        \xi(X, q, h, i, b) \coloneqq &\,\exists (q_{old} \colon \tau').\, \exists (h_{old} \colon
        \tau').\, \exists (i_{old} \colon \tau').\, \exists (b_{old} \colon \tau').\, \\
        &\,\exists (q_{new} \colon \tau').\, \exists (b_{new} \colon \tau').\,X(q_{old}, h_{old}, i_{old},
        b_{old}) \,\wedge \\
        &\, (\underset{\delta(q_{old}, b_{old}) = (q_{new}, b_{new}, d)}{\bigvee} q = q_{new} \wedge h =
        h_{old} + d \wedge i = i_{old}\,\wedge\\&\, ((\neg i = h_{old} \wedge b =
        b_{old}) \vee (i = h_{old} \wedge b = b_{new})))
    \end{align*}
    where $h = h_{old} + d$ depends on $d$ and is given by
    \[h = h_{old} + d \coloneqq
    \begin{cases}
        next^\tau(h_{old}, h),  & \text{if } d = L\\
        next^\tau(h, h_{old}),  & \text{if } d = R\\
        h = h{old},  & \text{if } d = N
    \end{cases}\]

    Because $M$ terminates the formula
    \[\psi_0(q', h', i', b') \coloneqq [\mathit{PFP}\;\varphi(X, q, h, i, b)](q', h', i', b') \]
    is guaranteed to define the relational description of the final configuration of $M$ on input word $w$.
    Finally, the formula
    \[\psi \coloneqq \exists (h' \colon \tau').\, \exists (i' \colon \tau').\, \exists (b' \colon \tau').\,
    \underset{q' \in F}{\bigvee} \psi_0(q', h', i', b')\]
    defines the acceptance of $M$ on input $w$.
\end{proof}

\subsection{Bisimulation Invariant HO(PFP)$^{k+1}$ to PHFL$^{k+1}_{tail}$}\label{subsec:bisimulationInvariantHopfptoPhfl}

As mentioned in the introduction of this section the main idea is not to show directly, that the 
lower bound of PHFL$^{k+1}_{tail}$ is \expspace{$k$} but rather by detour over the 
bisimulation-invariant fragment of HO(PFP)$^{k+1}$. In the previous subsection we have seen 
 that a DTM in $k$-EXPSPACE can be performed by an HO(PFP)$^{k+1}$
formula. By encoding the bisimulation-invariant fragment of HO(PFP)$^{k+1}$ into
PHFL$^{k+1}_{tail}$ combined with the knowledge that $k$-EXPSPACE is captured by 
HO(PFP)$^{k+1}$ leads to the lower bound of PHFL$^{k+1}_{tail}$. In
Section~\ref{sec:existential_quantifiers_in_phfl} and Section~\ref{sec:lowerBoundOfPhfl} we 
have shown that the HO$^{k+1}$ part can be encoded in PHFL$^k$. It is easy to prove that the 
encoded formulas are all tail-recursive. It follows that the HO$^{k+1}$ part can also be 
encoded in  PHFL$^{k+1}_{tail}$. The PFP operator is the only kind of HO(PFP)$^{k+1}$ formula 
that we have to encode in this subsection to get the lower bound of PHFL$^{k+1}_{tail}$.

Before we give the definition of the transforming function, we define a PHFL formula for HO formulas that uses the PFP operator.

\begin{definition}
Let $\Psi$ be a HO$^k$ formula with free first-order variables $x_1, \dots, x_f$, quantified first-order variables $x_{f+1}, \dots,
    x_q$ and let be $X$ a HO variable of HO type $\tau = (\tau', \dots, \tau')$ where $\tau' \neq \odot$ and any HO variables
    $X_1, V_1, \dots, $ $X_n, V_n$ HO type $\tau'$. Furthermore, let
    $\Phi$ be a PHFL$^k$
    formula and let $s$ be the maximal arity
    of all second order variables of $\Psi$, then $PFP^\tau X.\,\Phi$
    is a PHFL$^k$ formula with dimension $d = 2 * s + q + r$, where $r$ is the number of states that defines the reduced LTS,  defined as:
    \begin{align*}
     PFP^\tau X. \, \Phi \coloneqq \big(\mu (F \colon T(\tau) \rightarrow \bullet).\,&\lambda (X \colon T(\tau)).\, \big(X\,\wedge \forall^{\tau'}X_1.\, \dotsb \forall^{\tau'}X_n.\, \\&\big( (\dotsb (X X_1) \dotsb) X_n \Leftrightarrow \Phi(X, X_1, \dots, X_n) \big)\big)\\& \vee F(\Phi(X))\big)\lambda (V_1 \colon T(\tau')).\, \dotsb \lambda (V_n \colon T(\tau')).\,\bot
\end{align*}    
    In case of $\tau = (\odot, \dots, \odot)$ and $v_1, \dots, v_n$ are first-order variables with index $i_1, \dots, i_n$ respectively let $PFP^{(\odot, \dots, \odot)} X.\,\Phi$ defined as:
    \begin{align*}
    PFP^{(\odot, \dots, \odot)} X.\,\Phi \coloneqq \big(\mu (F \colon \bullet \rightarrow \bullet).\,&\lambda (X \colon \bullet).\, \big(X \wedge \forall_1 \dotsb \forall_n \\&\big(X \Leftrightarrow \Phi(X)\big)\big) \vee F(\Phi(X))\big)\,\bot
    \end{align*}
\end{definition}

Now we are able to define the function to translate a bisimulation invariant HO(PFP)$^{k+1}$
formula to a PHFL$^{k+1}_{tail}$. 

\begin{definition}
    \label{definition:lower_bounds_phfl_formula_function_pfp}
    Let $F$ be the function of Definition~\ref{definition:lower_bounds_phfl_formula_function} we extend this function in this way that it maps a bisimulation invariant HO(PFP)$^{k+1}$ formula $\Phi$ with free first-order variables $x_1, \dots, x_f$, quantified first-order variables $x_{f+1}, \dots,
    x_q$ to a PHFL$^{k+1}_{tail}$ formula with dimension $d = 2 * s + q + r$, where $r$ is the number of states that defines the reduced LTS and $s$ is the maximal arity
    of all second order variables of $\Phi$, defined
    inductive on $\Phi$ as follows:
    \begin{align*}
        F([PFP\;\Phi(X, x_{i_1}, \dots, x_{i_n})](v_{j_1}, \dots, v_{j_n})) \coloneqq &\,PFP^{(\odot, \dots, \odot)} X.\, F(\Phi) \\
        F([PFP\;\Phi(X, X_1, \dots, X_n)](V_1, \dots, V_n)) \coloneqq &\,PFP^\tau X.\, F(\Phi) 
    \end{align*}
\end{definition}

The last step is to show that the semantics of a given HO(PFP)$^{k+1}$ formula coincides with the semantics of the by function $F$ encoded PHFL$^{k+1}_{tail}$ formula. As mentioned in Section~\ref{sec:lower_bounds_preparation} without loss of generality the statement can be proven by consider only  reduced LTS. 

\begin{lemma}
    \label{lemma:ho_pfp_equals_phfl_tail}
    Let $f \geq 1$ and $k \geq 0$. For every bisimulation-invariant formula $\Phi$ of HO(PFP)$^{k + 1}$ there is a
    PHFL$^{k+1}_{tail}$ formula $\Psi$ such that the $f$-adic query $\mathcal{Q}_\Phi^f$ that belongs to $\Phi$ is equal to the $f$-adic query  $\mathcal{Q}_\Psi^f$ that belongs to $\Psi$.
\end{lemma}

\begin{proof}
    This lemma can be proven by showing for all HO(PFP)$^{k+1}$ formulas $\Phi$ with free first-order variables $x_1,
    \dots, x_f$ and quantified first-order variables $x_{f+1}, \dots, x_q$, all reduced LTS $\mathcal{T} = (Q, \Sigma, P,
    \Delta, v)$ with respect to $\emph{q}_r = q_1, \dots, q_r$ and all variable mappings $\eta$ that it holds that $\mathcal{T}, \eta \models \Phi$ iff $\emph{q} =
    (\emph{q}_s, \emph{q}_s, \emph{q}_f, \emph{q}_q, \emph{q}_r)$ and $\emph{q} \in \llbracket
   F(\Phi)\rrbracket^{\eta_V}_{\mathcal{T}}$, where $\emph{q}_s = q_1',\dots,q_s'$ is a sequence of $s$ placeholders used for the interaction of second-order variables, $\emph{q}_f = \eta(x_1), \dots, \eta(x_f)$ is a sequence of the by $\eta$ mapped free first-order variables, $\emph{q}_q = q_1'', \dots, q_q''$ is a sequence of $q$ placeholders used for the quantified first-order variables, $q_1', \dots, q_s', q_1'', \dots, q_q'' \in Q$ are arbitrary states, $F$ is the formula function of
    Definition~\ref{definition:lower_bounds_phfl_formula_function_pfp} and $\eta_V$ the variable mapping of
    Definition~\ref{definition:lower_bound_variable_function}. This statement can be proven by induction over formula
    $\Phi$.
    As mentioned in the introduction we have to show only correctness of the PFP operators.
    \begin{compactitem}
    \item In case of $\Phi = [PFP\,\Psi(X, x_{i_1}, \dots, x_{i_n})](v_{j_1}, \dots, v_{j_n})$, where $X$ is a
        free variable in $\Psi$ of HO type $(\odot, \dots, \odot)$ and $x_{i_1}, \dots, x_{i_n}$ are free first-order
        variables of $\Psi$ and $v_{j_1}, \dots, v_{j_n}$ are first-order variables of $\Phi$, then it follows that
        $\mathcal{T}, \eta \models \Phi$ exactly then if $(\eta(v_{j_1}), \dots, \eta(v_{j_n})) \in PFP
        (F_\Psi^\mathcal{T})$. 
        
        ...
        TODO
        ...
        
        exactly then if $\mathcal{T}, \eta \models \Phi$.

        \item In case of $\Phi = [PFP\,\Psi(X, X_1, \dots, X_n)](V_1, \dots, V_n)$, where $X$ is a
        free variable in $\Psi$ of HO type $(\tau, \dots, \tau)$ and $X_1, \dots, X_n$ are free first-order
        variables of $\Psi$ of type $\tau$ and $V_1, \dots, V_n$ are free variables of $\Phi$ also of type $\tau$, then
        it follows that $\mathcal{T}, \eta \vdash \Phi$ exactly then if $(\eta(X_1), \dots, \eta(X_n) \in PFP
        (F_\Psi^\mathcal{T})$. 
        
...
TODO
...        
        
        exactly then if $\mathcal{T}, \eta \models \Phi$.
    \end{compactitem}
\end{proof}

The combination of Lemma~\ref{lemma:ho_pfp_equals_phfl_tail}, Lemma~\ref{lemma:expspace_in_ho_pfp} and
Theorem~\ref{theorem:phfl_k_plus_1_tail_in_k_expspace} proves the following theorem.

\begin{theorem}
    Let $k \geq 0$. PHFL$^{k+1}_{tail}$ captures \expspace{$k$} over labeled transition systems.
\end{theorem}


\chapter{Conclusion}

In this thesis, we contributed to descriptive complexity theory by relating any order of PHFL to the corresponding complexity class. In detail, we showed that PHFL$^k$ captures the complexity class \exptime{$k$} for any $k > 1$ over finite labelled transition systems. Due to the fact that the statement above is also true for $k = 0$~\cite{otto1999bisimulation} and $k = 1$~\cite{lange2014capturing} we were able to verify that PHFL$^k$ captures \exptime{$k$} for any $k \geq 0$ on finite labelled transition systems. Furthermore, it was showed that the logic PHFL$^{k+1}_{tail}$ captures the complexity class \expspace{$k$} for any $k > 1$. In analogy to the exponential time classes, it was also proven that PHFL$^{k+1}_{tail}$ captures \expspace{$k$} for any $k \geq 0$ on finite labelled transition systems~\cite{otto1999bisimulation}~\cite{lange2014capturing}.

Since PHFL does not have existential and universal quantification over arbitrary higher-order relations a lot of effort had to be spent into the development of the encoding of the existential quantifiers of any order.
To obtain higher-order quantification in PHFL we used the existential quantifiers of type $\tau = (\odot, \dots, \odot)$ to define the order of domains of kind
$D_{(\tau, \dots, \tau)}(Q)$. This order was then be used to define a formula that returns the successor
of a given element of $D_{(\tau, \dots, \tau)}(Q)$ in respect to this order. Finally, we used this
formula to define the existential quantifier of type $(\tau, \dots, \tau)$. This procedure was applied to all
possible types of HO. In this way we got higher-order quantification of any type in PHFL.

The presented results contribute to the understanding on these complexity classes, which opens the possibilities for additional research, especially for further characterization of $k$-EXPTIME and $k$-EXPSPACE. That could lead to a further research on the characterization of classes \nexptime{$k$}. Those characterizations cannot be mapped to the encodings of \exptime{$k$} presented in this thesis. Another possibility may be the characterization of the polynomial hierarchy~\cite{stockmeyer1976polynomial}.

\section*{Acknowledgements}

First of all I would like to thank my thesis advisor Prof. Dr. Martin Lange for transferring the topic to me. I am very thankful for the help of Florian Bruse. In a twentyfour-seven service he gave me a patient guidance and lot of proof reading. I also like to thank Andreas and Michael for their English proof readings. Furthermore, I want to express my gratitude to my wife Lisa, who supported me in this exhausting time with much patience, understanding and the care of our daughter. A special thanks goes to Richard who not only did a good English proof reading, but furthermore backed me up morally. 

\bibliographystyle{unsrt}
\bibliography{Master_Thesis_David_Kronenberger_29110994}

\end{document}