%%
%% Author: DKron
%% 24.07.2018
%%

\chapter{Lower Bounds}\label{ch:lowerBounds}

In this chapter we want to establish the lower bounds of the expressive power of PHFL$^k$ and PHFL$^k_{tail}$. The
lower bounds of PHFL$^k$ and PHFL$^k_{tail}$ can be proven by encoding the run of a Turing Machine by queries.
Another possibility for the lower bounds can be get by detour over HO(LFP)$^{k+1}$ and HO(PFP)$^{k+1}$. In more detail
we encode the bisimulation-invariant fragment of HO(LFP)$^{k+1}$ and HO(PFP)$^{k+1}$ into PHFL$^k$ and
PHFL$^k_{tail}$ respectively. These fragments are enough because PHFL cannot distinguish between bisimilar structures~\cite{lange2014capturing}. That means PHFL formulas can only define bisimulation-invariant
graph problems. To encode the bisimulation-invariant fragment of HO(LF\-P)$^{k+1}$ and HO(PFP)$^{k+1}$ into PHFL$^k$ and
PHFL$^{k}_{tail}$ respectively we want to translate HO(LFP)$^{k+1}$ formulas into a PHFL$^k$ formula and 
HO(PFP)$^{k+1}$ into a PHFL$^k_{tail}$ formula. The costliest part of these encodings is that of existential quantifiers. In the 
first section we consider some preparations that are necessary to encode the formulas. After this we show that existential 
quantifiers with bound variable $X$ of order $k \geq 1$ can be expressed by a PHFL$^{k-1}$ formula. In the next section we 
use this formula to show that the bisimulation invariant fragment of HO(LFP)$^{k+1}$ can be encoded into PHFL$^k$ and so 
that the lower bound of the expressive power of PHFL$^k$ is \exptime{$k$}. In the last section we show that the lower bound 
of the expressive power of PHFL$^k_{tail}$ is \expspace{$k$}.

\section{Preparation}\label{sec:lower_bounds_preparation}

Before we can start with the encodings there are some important steps that we have to consider.  Let $\mathcal{T} = (Q, 
\Sigma, P, \Delta, v)$ an LTS, $q_1, \dots, q_n, p_1, \dots, p_m \in Q$ some states of $\mathcal{T}$ and $\mathcal{T}'$ the 
reduced LTS of $\mathcal{T}$ with respect to $p_1, \dots, p_m$.  
As mentioned in the introdution of this chapter it is known from~\cite{lange2014capturing} that PHFL cannot distinguish 
between bisimilar structures. What means PHFL formulas can only define bisimulation-invariant graph problems. We use this 
knowledge to make the model checking of HO and PHFL formulas easier. That means without loss of generality instead of checking if a tuple of states 
$(q_1, \dots, q_n)$ satisfies a formula $\Phi$ in respect to $\mathcal{T}$ we can also check if $(q_1, \dots, q_n)$ satisfies 
$\Phi$ in respect to $\mathcal{T}'$. 

From now all LTS are reduced LTS with respect to some states of their state sets. In more detail, let $\mathcal{T} = (Q, \Sigma, P, \Delta, v)$ in this chapter a reduced LTS with respect to $q_1, \dots, q_r$, where $q_1, \dots, q_r \in \Sigma$. Because each state of those LTS can be reached by at least one $q_i$ it is possible to define a total order on their states. 

\begin{remark} \label{remark:transitive_relation}
    In~\cite{otto1999bisimulation} was shown that it is possible to define a $2$-adic formula $\Phi_<$ that defines a
    transitive relation $<$ such that $< \cap > = \emptyset$ and $< \cup > = \not\sim$. In this thesis $<$ defines a total
    order on states of a reduced LTS.
\end{remark}


%%
%% Author: DKron
%% 17.08.2018
%%

\section{Existential Quantifiers in PHFL}\label{sec:existential_quantifiers_in_phfl}

In this section we define PHFL$^{k}$ formulas that describes existential quantification over HO domains of types of
order $k \geq 1$. But before we can define these formulas we have to translate the types.


Note that the most types in HO$^{k + 1}$ do not exist in PHFL$^k$. While HO$^{k +
1}$ includes variables for example of kind set of sets, PHFL$^k$ does not support this kind of type.
But each set $X$ in HO$^{k+1}$ can be described by the characteristic function of $X$ in PHFL$^k$.

The following definition translates all HO types of order $2$ or greater to types in PHFL. The base type of HO
has to be encode differently, and will be establish after this definition.

\begin{definition}
    \label{definition:lower_bound_type_function}
    $T$ is a function that maps any type of HO of order $2$ or greater to a type of PHFL defined inductive over the
    type of HO as follows:
    \begin{align*}
        T((\odot, \dots, \odot)) &= \bullet\\
        T((\tau', \dots, \tau')) &= T(\tau')^+ \rightarrow (T(\tau')^+ \rightarrow \dotsb \rightarrow (T(\tau')^+
        \rightarrow \bullet) \dotsb ),
    \end{align*}
    where $\tau' \neq \odot$.
\end{definition}

Note that the orders of HO types and PHFL types are defined differently. So it holds $ord(T(\tau)) = order(\tau) - 2$
for all HO types $\tau$ with order $2$ or greater.

\begin{example}
    Let $\tau$ be a type of HO
    \[\tau = (\tau', \tau')\]
    with
    \[\tau' = (((\odot)), ((\odot)))\]
    then by Definition~\ref{definition:lower_bound_type_function} of type function $T$
    \[T(\tau) = T(\tau') \rightarrow (T(\tau') \rightarrow \bullet).\]
    with
    \[T(\tau') = (\bullet \rightarrow \bullet) \rightarrow ((\bullet \rightarrow \bullet) \rightarrow \bullet)\]
\end{example}

With this type function $T$ a HO$^{k + 1}$ variable $X$ of type $\tau$ can be translated to a PHFL$^k$ variable
of type $T(\tau)$. Intuitively the variable $X$ which is a set of $D_\tau(\mathcal{U})$ in HO$^{k+1}$ is represented
in PHFL$^k$ as the characteristic function of $X$ over $D_\tau(\mathcal{U})$. Note that the domain of HO types
of order $2$ are similar to the domain of base type of PHFL.

As mentioned above the base type of HO has to be encoded differently. The reason is that the base type in PHFL is a
set of tuples of states and a single state can not be depict directly by a variable. In this thesis the
idea is to use the polyadic logic of PHFL to represent the different first-order variables of an HO$^{k+1}$ formula
$\Psi$. Each first-order variable of $\Psi$ represents one component in the corresponding PHFL$^k$ formula $\Phi$, that
means each variable increases the dimension of $\Phi$. The assignment of a first-order variable $x_i$ in $\Psi$ is
then the current state of the $i$-th component in $\Phi$.

Let $\Phi$ be an HO$^{k+1}$ formula without loss of generality the first-order variables of $\Phi$ are enumerated as
$x_1, \dots, x_f, x_{f + 1}, \dots, x_q$, where $x_1, \dots, x_f$ are the free and $x_{f+1}, \dots, x_q$ are the
quantified variables of $\Phi$.

\subsection{First-Order and Second-Order Quantification}\label{subsec:existentialQuantifiers}

After we know how to interpret the different HO types and variables we are now able to consider the existential
quantification. Before we establish higher-order quantification we look at first-order and second-order quantification.

As mentioned in the introduction of this chapter we encode the bisimulation-invariant fragment of HO(LFP)$^{k + 1}$ and HO(PFP)$^{k+1}$. In Section~\ref{sec:lower_bounds_preparation}  we explained that as a result we can consider reduced LTS, where any state is reachable by at least one of the states $q_1, \dots, q_r$. Because of the total order on states of $\mathcal{T}$ explained in Remark~\ref{remark:transitive_relation} the first-order
quantification can be encoded by moving through all states reachable from one $q_1, \dots, q_r$ and check if we
reach a state tuple where the bounded formula holds.

To access the states $q_1, \dots, q_r$ in the PHFL formulas that we get by encoding HO formulas we use the polyadicity and store $q_1, \dots, q_r$ in components that will be never influenced by the PHFL formulas. The following remark explains that the PHFL formulas has a dimension that is big enough to fulfil all the requirements. 

\begin{remark}
    The PHFL formula $\Phi$ that we get through the encoding of a given HO formula $\Psi$ has dimension
    $d$ that is always big enough to translate all second-order variables of $\Psi$ to an order $0$ variable in
    $\Phi$. In more detail $s$ is the maximal arity of second-order variables in $\Psi$ and $d > s$. To compare
    two  elements of $Q^{s}$, where $Q$ is the state set of an LTS, the dimension $d$ of 
    $\Phi$ is  at leas twice as big as the maximum of $s$. To distinguish all
    different first-order variables in $\Psi$, the dimension $d$ of $\Phi$ has to be extend by $q$, where $q$ is the number of different first-order variables. That means $d >= 2 * s + q$. Finally, to access the 
    states $q_1, \dots, q_r$ described in Section~\ref{sec:lower_bounds_preparation}, 
    we extend $d$ additionally with $r$ components. That means the dimension of $\Phi$ is $d = 2 * s + q + r$.
\end{remark}

If we consider $\exists (x_i \colon \odot).\,\Phi$ then it can be understand as, check if we reach a state tuple where $
\Phi$ holds once the $i$-th component is replaced by one of the last $r$ components. These where $q_1, \dots, q_r$ are stored.

\begin{definition}
    Let $\Psi$ be a HO$^k$ formula with free variables $x_1, \dots, x_f$ and quantified variables $x_{f+1}, \dots,
    x_q$ and $s$ be the maximal arity of all second order variables of $\Psi$, then $\exists_i \Phi$ is a PHFL$^k$
    formula with dimension $d = 2 * s + q+ r$, where $r$ is the number of states that defines the 
    reduced LTS, defined as
    \[\exists_i \Phi \coloneqq \bigvee^{j=d}_{2*s+q+1} \{(1, \dots, i-1, j, i + 1, \dots, d)\} \mu (X
    \colon \bullet).\,\Phi \vee \bigvee_{a \in \Sigma} \langle a \rangle_{i} X.\]
    The formula $\forall_i \Phi$ is also a PHL$^k$ formula with dimension $d$ and is defined as
    \[\forall_i \Phi \coloneqq \neg \exists_i \neg \Phi.\]
\end{definition}

Now we consider the second-order quantification. Let $\tau = (\odot, \dots, \odot)$ a HO type, $\sigma$ a signature and
$\mathcal{U}$ the universe of a $\sigma$ structure. Because the idea we use for first-order quantification is
impossible to adapt to the second-order quantification, we use another encoding. The idea to obtain second-order
quantification in PHFL is that we have to iterate through all possible elements in a domain $D_\tau(\mathcal{U})$ and
check if the given formula is fulfilled. The first thing that we need for iteration over any element of a domain
$D_\tau(\mathcal{U})$ is an order on $D_\tau(\mathcal{U})$. If we have the order of $D_\tau(\mathcal{U})$ we can use
this order to define a formula that returns us the successor of a given element of $D_\tau(\mathcal{U})$ in the scope
of this order. Finally, this formula can be used to iterate through all elements and check if a given formula is
fulfilled.
So the first thing we need is the order of domains of type $(\odot, \dots, \odot)$.

To get the order of type $\tau = (\odot, \dots, \odot)$ we define two formulas. The first tells us given two
sets of type $\tau$ which one is the smaller one. The other formula tells us the same for two tuples in those sets
of type $\tau$. We say that a tuple $x$ is smaller in respect to $\tau$ than a tuple $y$ if there is an index $i$ such
that the element in $x$ on $i$ is smaller in respect to $\odot$ then the element in $y$ on $i$ and there is no position
$j$ left of $i$ such that the element in $x$ on $j$ is bigger in respect to $\odot$ then the element in $y$ on $j$. We
say that a set $X$ is smaller in respect to $\tau$ than a set $Y$ if there is an element $s_1$ in $X$ that is not in $Y$
and all smaller elements $s_2$ to $s_1$ in respect to $\odot$ are only in $X$ if $s_2$ is also in $Y$. This is
formalized in PHFL in the following definition.

\begin{definition}
    \label{definition:lower_bound_less_second}
    Let $\Psi$ be a HO$^k$ formula with free variables $x_1, \dots, x_f$ and quantified variables $x_{f+1}, \dots,
    x_q$ and $s$ be the maximal arity of all second order variables of $\Psi$, then $<^\odot$, $<^{\odot \times \dots
    \times \odot}$ and $<^{(\odot, \dots, \odot)}(X, Y)$ are PHFL$^k$ formulas with dimension $d = 2 * s + q + r$, where $r$ is the number of states that defines the reduced LTS, defined as:

    \begin{align*}
        <^\odot \coloneqq &\,\Phi_< \\
        <^{\odot \times \dots \times \odot} \coloneqq
            &\,\underset{i = 1}{\overset{s}{\bigvee}}\{(i, i + s, 3, \dots, d)\} <^\odot \wedge \\
            &\,\underset{j = 1}{\overset{i - 1}{\bigwedge}}\{(j + s, j, 3, \dots, d)\} \neg <^\odot \\
        <^{(\odot, \dots, \odot)}(X, Y) \coloneqq
            &\,\exists_{i_1}.\, \dots \exists_{i_n}. \{(i_1, \dots, i_n, n + 1,\dots, d)\}Y \wedge \\
            &\,\neg \{(i_1, \dots, i_n, n + 1, \dots, d)\} X\,\wedge \\
            &\, \forall_{j_1}. \,\dots \forall_{j_n}. \{(j_1, \dots, j_n, n+1, \dots, s, \\
            &\,i_1, \dots, i_n, s + n + 1, \dots,  d)\}<^{\odot \times \dots \times \odot} \Rightarrow \\
            &\,(\{(j_1,\dots, j_n, n + 1, \dots, d)\} X \Rightarrow \\
            &\,\neg \{(j_1, \dots, j_n, n + 1, \dots, d)\} Y)
    \end{align*}
\end{definition}

After we have now orders of the HO types $(\odot, \dots, \odot)$ we can define formulas that returns the successor of
an input element in respect to the order of $(\odot, \dots, \odot)$. The idea of the following formula is based on
binary incrementation. Let $\tau = (\odot, \dots, \odot)$ an HO type and $\mathcal{U}$ the universe of a
$\sigma$-structure. Remember that a set $X \in D_\tau(\mathcal{U})$ can be represented by its characteristic
function. This can be transformed to a binary string where each position of this string represents an element of
$D_{\odot}(\mathcal{U})^n$. Because each position in the binary string represents an element of $D_{\odot}
(\mathcal{U})^n$ and a position have always to represent the same element in $D_{\odot}(\mathcal{U})^n$, the elements
in $D_{\odot}(\mathcal{U})^n$ have to be ordered. The order of the elements of $D_{\odot}(\mathcal{U})^n$ is given by
the formula $<^{\odot \times \dots \times \odot}$ of Definition~\ref{definition:lower_bound_less_second}. If the position
$i$ in the binary string is $1$ then this means that the element with index $i$ in $D_{\odot}(\mathcal{U})^n$ is
also in $X$ and if the position $i$ in the binary string is $0$ then the element with index $i$ in $D_{\odot}
(\mathcal{U})^n$ is not in $X$. Note that this binary string is just a visualization of the idea of the data and is
not a directly visible part of the formula. The user of that formula have to enter a set and its elements are
represented in that binary string way. The binary representation of $X$ in regard to $D_{\odot}(\mathcal{U})^n$ can be
extended to a function $f \colon D_\tau(\mathcal{U}) \rightarrow {0, \dots, |D_\tau(\mathcal{U})| - 1}$ such that
each element $X$ of $D_\tau(\mathcal{U})$ will be mapped to its binary string in regard to $D_{\odot}(\mathcal{U})^n$
. With this knowledge we can say that $Y \in D_\tau(\mathcal{U})$ is the direct successor of $X \in D_\tau
(\mathcal{U})$ if $f(Y) = (f(X) + 1)$ modulo $|D_\tau(\mathcal{U})|$. In detail that means that the $i$-th bit is $1$
in $f(Y)$ if it is either $0$ in $X$ and all lower bits are $1$ in $X$ or it is $1$ in $X$ and there is a bit lower
then $i$ that is $0$ in $X$.

\begin{definition}
    \label{definition:lower_bounds_next_second}
    Let $\Psi$ be a HO$^k$ formula with free variables $x_1, \dots, x_f$, quantified variables $x_{f+1}, \dots,
    x_q$ and $s$ be the maximal arity of all second order variables of $\Psi$, then $next^{(\odot, \dots, \odot)}$
    is a PHFL$^k$ formula with dimension $d = 2 * s + q + r$, where  $r$ is the number of states that defines the reduced LTS, defined as:

    \begin{align*}
        next^{(\odot, \dots, \odot)} \coloneqq &\,\lambda (X \colon \bullet).\, (\neg X \wedge \forall_{s +
        1}\dots\forall_{s + s}<^{\odot \times \dots \times \odot}\, \Rightarrow \\&\,\{(s +
        1, \dots, s + s, s + 1, \dots, d)\} X) \,\vee \\&\,(X \wedge \exists_{s + 1}\dots\exists_{s + s} <^{\odot
        \times \dots \times \odot} \,\wedge \\&\,\{(s + 1, \dots, s + s, s + 1, \dots, d)\}
        \neg X)
    \end{align*}
\end{definition}

With the previous definition we are now able to define the second-order quantification in PHFL.

\begin{definition}
    \label{definition:existential_quantification_second}
    Let $\Psi$ be a HO$^k$ formula with free variables $x_1, \dots, x_f$, quantified variables $x_{f+1}, \dots,
    x_q$ and $\Phi$ be a PHFL$^k$ formula with free order $0$ variable $X$. Furthermore, let $s$ be the maximal arity
    of all second order variables of $\Psi$, then $\exists^{(\odot, \dots, \odot)}X .\,\Phi(X)$
    is a PHFL$^k$ formula with dimension $d = 2 * s + q + r$, where $r$ is the number of states that defines the reduced LTS, defined as:
    \[\exists^{(\odot, \dots, \odot)}X.\, \Phi(X) \coloneqq (\mu (F \colon \bullet \rightarrow \bullet).\, \lambda (X
    \colon \bullet).\, \Phi(X) \vee F(next^{(\odot, \dots, \odot)} X)) \bot
    \]
    The formula $\forall^{(\odot, \dots, \odot)}X.\,\Phi$ is also a PHL$^k$ formula with dimension $d$ and is defined as
    \[\forall^{(\odot, \dots, \odot)}X.\,\Phi(X) \coloneqq \neg \exists^{(\odot, \dots, \odot)}X .\,\neg\Phi(X).\]
\end{definition}

The last step is to show that the given formula of Definition~\ref{definition:existential_quantification_second} defines
second-order existential quantification in PHFL.

\begin{lemma}
    \label{lemma:existential_quantifier_second}
    For all HO types $\tau = (\odot, \dots, \odot)$, all variable mappings $\eta$ and all LTS $\mathcal{T}$ holds
    \[\llbracket \exists^\tau X.\,\Phi(X)\rrbracket^\eta_\mathcal{T} \equiv \underset{\mathcal{X} \in \llbracket \tau
    \rrbracket_\mathcal{T}}{\bigsqcup} \llbracket \Phi(X) \rrbracket^{\eta[X\rightarrow \mathcal{X}]}_\mathcal{T}.\]
\end{lemma}

\begin{proof}
    By fixpoint unfolding and $\beta$-reduction the formula 
    \[\exists^{(\odot, \dots, \odot)}X.\, \Phi(X) = (\mu (F \colon \bullet \rightarrow \bullet).\, \lambda (X
    \colon \bullet).\, \Phi(X) \vee F(next^{(\odot, \dots, \odot)} X)) \bot\] is equivalent to 
    \[\Phi(\bot) \vee \Phi(next^{(\odot, \dots, \odot)}\bot) \vee \Phi(next^{(\odot, \dots, \odot)} next^{(\odot, \dots, \odot)} \bot) \vee \dotsb. \]
    This can be simplified with
    \[\underset{i\geq0}{\bigvee} \Phi({next^{(\odot, \dots, \odot)}}^i \bot)\]
Note that $\bot$ represents always the empty set. If we put $\bot$ into $next^{(\odot, \dots, \odot)}$ we check all elements that can be in $D_{(\odot, \dots, \odot)}$ if those fulfil one of the disjuncts. We will see that only the smallest element fulfills the first disjunct. Also the smallest element is not in the empty set and there are no smaller elements to the smallest. So the first disjunct is true and it is the only element that fulfil one of the disjuncts. So the formula ${next^{(\odot, \dots, \odot)}}^1 \bot$ returns the set that includes only the smallest element with respect to $\odot \times \dots \times \odot$. If we look at the formula ${next^{(\odot, \dots, \odot)}}^2 \bot$, we put the set that includes the smallest element into $next^{(\odot, \dots, \odot)}$. If we look into this formula we will see that now the formula returns a set that includes only the second smallest element. The smallest element does not fulfil any disjunct, but the second smallest fulfills the second disjunct. As described in the introduction of Definition~\ref{definition:lower_bounds_next_second} formula $next^{(\odot, \dots, \odot)}$ is some kind of binary incrementation. In this manner each possible set of type $(\odot, \dots, \odot)$ will be reached. Finally, because these sets are checked each for itself in the scope of $\Phi$, the lemma holds.
\end{proof}

\subsection{Higher-Order Quantification}\label{subsec:higher-orderQuantification}

In this subsection we use the second-order existential quantification to define the higher-order
quantification. The idea to obtain higher-order quantification in PHFL is similar to the higher-order quantifaction.
We use the existential quantifier of type $\tau = (\odot, \dots, \odot)$ to define the order of domains of kind
$D_{(\tau, \dots, \tau)}(\mathcal{U})$. This order can then be used to define a formula that returns us the successor
of a given element of $D_{(\tau, \dots, \tau)}(\mathcal{U})$ in respect to this order. Finally, we use this
formula to define the existential quantifier of type $(\tau, \dots, \tau)$. This procedure will be lift up to all
possible types of HO. In this way we get higher-order quantification of any type in PHFL.

So the first thing we need is the order of any domain. Note that a order for type $\tau = (\tau', \dots, \tau')$ always
depends on the existential quantifiers of type $\tau'$. As in the second-order case we define two formulas. The first
tells us given two sets of type $\tau$ which one is the smaller one. The other formula tells us the same for two
tuples in those sets of type $\tau$.

\begin{definition}
    \label{definition:lower_bound_less_higher}
    Let $\Psi$ be a HO$^k$ formula with free variables $x_1, \dots, x_f$ and quantified variables $x_{f+1}, \dots,
    x_q$. Furthermore, let $s$ be the maximal arity of all second order variables of $\Psi$ and $\tau \neq \odot$ a HO
    type, then $<^{(\tau, \dots, \tau)}(X, Y)$ and $<^{\tau \times \dots \times \tau}(X, Y)$ are PHFL$^k$ formulas
    with dimension $d = 2 * s + q + r$, where $r$ is the number of states that defines the reduced LTS, defined as:
    \begin{align*}
        <^{\tau \times \dots \times \tau}(X_1, Y_1, \dots, X_n, Y_n) \coloneqq &\,\underset{i =
        1}{\overset{n}{\bigvee}}<^{\tau}(X_i, Y_i) \wedge \underset{j = 1}{\overset{i - 1}{\bigwedge}}
        \neg <^{\tau}(Y_j, X_j)\\
        <^{(\tau, \dots, \tau)}(X, Y) \coloneqq &\,\exists^{\tau}X_1. \,\dots \exists^{\tau}X_n.\,(\dotsb(Y\,X_1)\dotsb)\,X_n\\
        &\,\wedge \neg (\dotsb(X\,X_1) \dotsb)\,X_n\,\wedge \\&\,\forall^{\tau}Y_1. \,\dots
        \forall^{\tau}Y_n.\,<^{\tau \times \dots \times \tau}
        (Y_1, X_1, \dots, Y_n, X_n) \\&\,\Rightarrow \big((\dotsb(X\,Y_1) \dotsb)\,Y_n \Rightarrow (\dotsb(Y\,Y_1)
         \dotsb)\,Y_n\big)
    \end{align*}
\end{definition}

After we have now orders of any HO type, always depending on the lower type existential quantifier, we can define
formulas that returns the successor of an input element in respect to the order of the HO type. The idea of the
following formula is similar to the successor formula of Definition~\ref{definition:lower_bounds_next_second} and is
based on binary incrementation. Where here the binary string representation is also not directly visible in the
formula. The user of that formula have to enter a function and the image of the elements of the origin set of this
function represents $0$ or $1$ in that binary string.

\begin{definition}
    \label{definition:lower_bounds_next_higher}
    Let $\Psi$ be a HO$^k$ formula with free variables $x_1, \dots, x_f$ and quantified variables $x_{f+1}, \dots,
    x_q$. Furthermore, let $s$ be the maximal arity of all second order variables of $\Psi$ and $\tau \neq \odot$ a
    HO type, then $next^{(\tau, \dots, \tau)}$ is a PHFL$^k$ formula with dimension $d = 2 * s + q + r$, where $r$ is the number of states that defines the reduced LTS, defined as:

    \begin{align*}
        next^{(\tau, \dots, \tau)} \coloneqq &\,\lambda (X \colon T ((\tau, \dots, \tau))).\,\big(\neg (\dotsb(X\,X_1)\dotsb)\,X_n \\&\, \wedge \forall^{\tau}Y_1.\, \dots \forall^{\tau}Y_n.\,<^{\tau \times
        \dots \times \tau}(Y_1, X_1, \dots, Y_n, X_n) \\&\,\Rightarrow  (\dotsb(X\,Y_1)\dotsb)\,Y_n\big) \,\vee
        \\&\,\big((\dotsb (X\,X_1) \dotsb)\,X_n \wedge \exists^{\tau}Y_1.\, \dots \exists^{\tau}Y_n.\, \\&\,
        <^{\tau \times \dots \times \tau}
        (Y_1, X_1, \dots, Y_n, X_n)\,\wedge \neg (\dotsb(X\,Y_1)\dotsb)\,Y_n\big)
    \end{align*}
\end{definition}

With the previous definition we are now able to define the higher-order quantification in PHFL.

\begin{definition}
    \label{definition:existential_quantification_higher}
    Let $\Psi$ be a HO$^k$ formula with free variables $x_1, \dots, x_f$, quantified variables $x_{f+1}, \dots,
    x_q$ and let $\tau = (\tau', \dots, \tau')$ be a HO type of order $l$ where $\tau' \neq \odot$. Furthermore, let
    $\Phi$ be a PHFL$^k$
    formula with free variable $X$ of order $l - 2$ and let $s$ be the maximal arity
    of all second order variables of $\Psi$, then $\exists^{\tau}X .\,\Phi(X)$
    is a PHFL$^k$ formula with dimension $d = 2 * s + q + r$, where $r$ is the number of states that defines the reduced LTS,  defined as:
    \begin{align*}
        \exists^{\tau}X.\, \Phi(X) \coloneqq &\,(\mu (F \colon T(\tau) \rightarrow \bullet).\, \lambda (X \colon T(\tau)
        ).\,
        \Phi(X)
        \vee F(next^\tau X))\\&\,(\lambda (X_1 \colon T(\tau')).\, \dots \lambda (X_n \colon T(\tau')).\,\bot).
    \end{align*}
    The formula $\forall^{\tau}X.\,\Phi$ is also a PHL$^k$ formula with dimension $d$ and is defined as
    \[\forall^{\tau}X.\,\Phi(X) \coloneqq \neg \exists^{\tau}X .\,\neg\Phi(X).\]
\end{definition}

The last step is to show that the given formula of Definition~\ref{definition:existential_quantification_higher} defines
higher-order existential quantification.

\begin{lemma}
    \label{lemma:existential_quantifier_higher}
    For all HO types $\tau$ of order $3$ or greater, all variable mappings $\eta$ and all reduced LTS $\mathcal{T} = (Q, \Sigma, P, \Delta, v)$ holds
    \[\llbracket \exists^\tau X.\,\Phi(X)\rrbracket^\eta_\mathcal{T} \equiv \underset{\mathcal{X} \in \llbracket \tau
    \rrbracket_\mathcal{T}}{\bigsqcup} \llbracket \Phi(X) \rrbracket^{\eta[X\rightarrow \mathcal{X}]}_\mathcal{T}.\]
\end{lemma}

\begin{proof}
    This lemma is proven by induction over the order of type $\tau$. The induction basis $\tau = (\odot, \dots, \odot)$ is given by Lemma~\ref{lemma:existential_quantifier_second}. By induction hypothesis it holds for any HO type $\tau$ of order $k$, all variable mappings $\eta$ and all reduced LTS $\mathcal{T} = (Q, \Sigma, P, \Delta, v)$
    \[\llbracket \exists^\tau X.\,\Phi(X)\rrbracket^\eta_\mathcal{T} \equiv \underset{\mathcal{X} \in \llbracket \tau
    \rrbracket_\mathcal{T}}{\bigsqcup} \llbracket \Phi(X) \rrbracket^{\eta[X\rightarrow \mathcal{X}]}_\mathcal{T}.\]
    We have to show that for any HO type $\tau'=(\tau, \dots, \tau)$ of order $k+1$, all variable mappings $\eta$ and all reduced LTS $\mathcal{T} = (Q, \Sigma, P, \Delta, v)$ it holds
    \[\llbracket \exists^{\tau'} X.\,\Phi(X)\rrbracket^\eta_\mathcal{T} \equiv \underset{\mathcal{X} \in \llbracket \tau'
    \rrbracket_\mathcal{T}}{\bigsqcup} \llbracket \Phi(X) \rrbracket^{\eta[X\rightarrow \mathcal{X}]}_\mathcal{T}.\]    
     By fixpoint unfolding and $\beta$-reduction the formula
    \begin{align*}
        \exists^{\tau'}X.\, \Phi(X) \coloneqq &\,(\mu (F \colon T(\tau') \rightarrow \bullet).\, \lambda (X \colon T(\tau')).\,
        \Phi(X)
        \vee F(next^{\tau'} X))\\&\,(\lambda (X_1 \colon T(\tau)).\, \dots \lambda (X_n \colon T(\tau)).\,\bot)
    \end{align*}    
    is equivalent to
    \begin{align*}
        &\Phi((\lambda (X_1 \colon T(\tau)).\, \dots \lambda (X_n \colon T(\tau)).\,\bot))\, \vee \\
        &\Phi(next^{\tau'} (\lambda (X_1 \colon T(\tau)).\, \dots \lambda (X_n \colon T(\tau)).\,\bot))\, \vee \\
        & \Phi(next^{\tau'} next^{\tau'} (\lambda (X_1 \colon T(\tau)).\, \dots \lambda (X_n \colon T(\tau)).\,\bot)) \vee \dotsb
    \end{align*}    
    This can be simplified with
    \[\underset{i\geq0}{\bigvee} \Phi({next^{\tau}}^i (\lambda (X_1 \colon T(\tau)).\, \dots \lambda 
    (X_n \colon T(\tau)).\,\bot))\]
    Note that $(\lambda (X_1 \colon T(\tau)).\, \dots \lambda (X_n \colon T(\tau)).\,\bot))$ 
    represents a function that maps any input for $X_1, \dots, X_n$ to $\bot$ and $\bot$ 
    represents the empty set. This function is interpreted as the smallest function while $
    (\lambda (X_1 \colon T(\tau)).\, \dots \lambda (X_n \colon T(\tau)).\,\top))$ represents the 
    biggest function that maps any input to $Q^d$, where $d$ is a dimension of PHFL formula $
    \llbracket \exists^\tau X.\,\Phi(X)\rrbracket^\eta_\mathcal{T}$.
    
    If we put $(\lambda (X_1 \colon T(\tau)).\, \dots \lambda (X_n \colon T(\tau)).\,\bot))$ into $next^{\tau}$ we check all elements that can be in $D_{\tau}$ if those fulfil one of 
    the disjuncts. Note that the quantifiers formulas used in $next^\tau$ and $<^{\tau'\times\dots\times\tau'}$ respectively are all for types of order $k$. By induction hypothesis those formulas are defining quantification of types of order $k$. Using those quantifiers we will see that only the smallest element with respect to $\tau'\times\dots\times\tau'$ fulfils the first disjunct. Also the 
    smallest element is mapped in the input function to the empty set and there are no smaller elements to the smallest. 
    That means the first disjunct is true and it is the only element that fulfil one of the disjuncts. So the 
    formula ${next^{\tau}}^1 (\lambda (X_1 \colon T(\tau)).\, \dots \lambda (X_n \colon T(\tau)).\,\bot)$ returns the function that maps the smallest element with respect to $\tau'\times\dots\times\tau'$ to $Q^d$ and all other elements to $\emptyset$. If we look at the formula $
    {next^{\tau}}^2 (\lambda (X_1 \colon T(\tau)).\, \dots \lambda (X_n \colon T(\tau)).\,\bot)$, we put the function that maps only the smallest element to $Q^d$ into 
    $next^{\tau}$. If we look into this formula we will see that now the formula 
    returns a function that maps only the second smallest element to $Q^d$. The smallest element does not 
    fulfil any disjunct, but the second smallest fulfills the second disjunct. As described in the 
    introduction of Definition~\ref{definition:lower_bounds_next_second} formula $next^{(\odot, 
    \dots, \odot)}$ is some kind of binary incrementation. The same holds for the formula $next^\tau$ of Definition~\ref{definition:lower_bounds_next_higher}. In this manner each possible function of 
    type $\tau$ will be reached. Finally, because these functions are checked each for 
    itself in the scope of $\Phi$, the induction is complete and the lemma holds.
\end{proof}

%%
%% Author: DKron
%% 17.08.2018
%%

\section{Lower Bound of PHFL$^k$}\label{sec:lowerBoundOfPhfl}

As mentioned in the introduction of this chapter we can show that the lower bound of PHFL$^k$ is \exptime{$k$} by
make a detour over HO(LFP)$^{k+1}$. This and following ideas are oriented on~\cite{lange2014capturing} where was
shown that PHFL$^1$ captures \exptime{$1$}. The first thing we will see is that it was proven that HO(LFP)$^{k +
1}$ coincides with $k$-fold exponential time over finite and ordered structures. To use this we have to encode the
bisimulation invariant fragment of HO$^{k+1}$ into PHFL$^k$. For this we first define an abbreviation for the HO(LFP)
formulas that uses the LFP operator. Next, we define a function that uses this abbreviation and these of
Section~\ref{sec:existential_quantifiers_in_phfl} to map a HO(LFP)$^{k+1}$ formula to a PHFL$^k$ formula. Finally, we
show that the semantics of the HO(LFP)$^{k+1}$ formula coincides with the transformed PHFL$^k$ formula.

\begin{theorem}{\cite{freireMartins2011descriptive}}\label{theorem:hoLfpEqualsExptime}
    For all $k \geq 1$, HO(LFP)$^{k + 1} = k$-EXPTIME over finite and ordered structures.
\end{theorem}

The proof follows the idea to encode the run of a $k$-EXPTIME Turing Machine $M$ by a formula $\phi$ of HO(LFP)$^{k +
1}$ in this way such that $M$ accepts $\mathcal{A}$ iff $\mathcal{A} \models \phi$. On the other hand each HO(LFP)$^{k +
1}$ formula $\phi$ can be evaluated by a $k$-EXPTIME Turing Machine $M_\phi$.

Because Theorem~\ref{theorem:hoLfpEqualsExptime} holds it is also possible to prove that the lower bound of PHFL$^k$
is in \exptime{$k$} by encoding the bisimulation invariant fragment of HO(LFP)$^{k + 1}$ into PHFL$^k$. To encode the
bisimulation invariant fragment of HO(LFP)$^{k + 1}$ into PHFL$^k$ we have to define a function that transforms a HO
(LFP)$^{k + 1}$ formula to a PHFL$^k$ formula. Note that the types and variables of an HO formula have to be also
transformed. See Section~\ref{sec:existential_quantifiers_in_phfl} for further details.

Before we give the definition of the transforming function, we define a PHFL formula for HO formulas that uses the
LFP operator.

\begin{definition}
    For any HO variable $X$ of HO type $\tau = (\tau', \dots, \tau')$ where $\tau' \neq \odot$ and any HO variables
    $x_1, v_1, \dots, $ $x_n, v_n$ HO type $\tau'$ and any PHFL formula $\Phi$ let
    \[LFP^\tau X.\,\Phi \coloneqq (\dots ((\mu (X \colon T(\tau)).\,\Phi(X, x_1, \dots, x_n))(v_1))(v_2))\dots)(v_n).\]
    In case of $\tau = (\odot, \dots, \odot)$ and $v_1, \dots, v_n$ have index $i_1, \dots, i_n$ respectively let
    \[LFP^{\tau} X.\,\Phi \coloneqq \{(i_1, \dots, i_n, n + 1, \dots, d)\} \mu (X \colon \bullet).\,\Phi(X).\]
\end{definition}

Remember that we have defined the signatures of HO(LFP) relational. Because we are working on LTS, the relations in
the signatures have either arity one or two. Relations with arity one represents the propostions and those with arity
two the actions of an LTS. Now we are able to define the function to translate a bisimulation invariant HO(LFP)$^{k+1}$
formula to a PHFL$^k$.

\begin{definition}
    \label{definition:lower_bounds_phfl_formula_function}
    Let $F$ be the function that maps a bisimulation invariant HO(LFP)$^{k+1}$ formula to a PHFL$^k$ formula defined
    inductive on bisimulation invariant HO(LFP)$^{k+1}$ formula $\Phi$ as follows:
    \begin{align*}
        F(p(x_i)) \coloneqq &\, p_i \\
        F(a(x_i, x_j)) \coloneqq &\, \langle a \rangle_i \{(i, j, 3, \dots, d)\} \Phi_\sim \\
        F(\Phi \vee \Psi) \coloneqq &\, F(\Phi) \vee F(\Psi) \\
        F(\neg \Phi) \coloneqq &\, \neg F(\Phi) \\
        F(\exists (x_i \colon \odot).\,\Phi) \coloneqq &\, \exists_i F(\Phi) \\
        F(\exists (X \colon \tau).\,\Phi) \coloneqq &\, \exists^\tau X.\,F(\Phi(X)) \\
        F([LFP\;\Phi(X, x_1, \dots, x_n)](v_1, \dots, v_n)) \coloneqq &\,LFP^\tau X.\, F(\Phi) \\
        F(X(x_1, \dots, x_n)) \coloneqq &\, (\dots ((X(x_1))(x_2))\dots)(x_n)\\
        F(X(x_{i_1}, \dots, x_{i_n})) \coloneqq &\, \{(i_1, \dots, i_n, n + 1, \dots, d)\}X
    \end{align*}
\end{definition}

The last step is to show that the semantics of a given HO(LFP)$^{k+1}$ formula coincides with the semantics of the by
function $F$ encoded PHFL$^k$ formula. For this we need a last definition for the semantics of the formulas. As
described in Section~\ref{sec:existential_quantifiers_in_phfl} the variables in HO of types with order $3$ or higher
are not supported in PHFL. Also first-order variables are not supported. For this we define a function that maps a
given variable mapping for a HO formula to the correct variable mapping for PHFL semantics. This function ignores the
mapping of first-order variables, maps second-order variables also to sets and higher-order variables to the
corresponding characteristic function. Note, that the sets of order $2$ in HO have order $0$ in PHFL. The first-order
variables of an HO formula that are marked as ignored in the following function, can be mapped to anything because we
do not use them in PHFL directly.

\begin{definition}
    \label{definition:lower_bound_variable_function}
    Let $d$ a dimension of PHFL and $\eta$ a variable mapping for a HO(LFP)$^{k+1}$ formula, then function
    $V$ maps $\eta$ to a variable mapping that can be used for a PHFL$^k$ formula with dimension $d$ where
    \[V(\eta)(X)=
    \begin{cases}
        \text{ignored}, & \text{if } X \text{ is of type } \odot \\
        A,  & \text{if } X \text{ is of type } (\odot, \dots, \odot)\\
        F, & \text{if } X \text{ is of type } (\tau, \dots, \tau) \text{ and } \tau \neq \odot,
    \end{cases}\]
    where $A \subseteq Q^d$ such that $(q_1, \dots, q_n, q_{n + 1}, \dots, q_d) \in A$ iff $(q_1, \dots, q_n) \in
    \eta(X)$ and $F$ is a function of type $T((\tau, \dots, \tau))$ defined as follows:
    \begin{align*}
        (\dots((F(V(\eta)(X_1)))(V(\eta)(X_2)))\dots)(V(\eta)(X_n)) &= Q^d &\text{ iff } (X_1, \dots, X_n) \in \eta(X)\\
        (\dots((F(V(\eta)(X_1)))(V(\eta)(X_2)))\dots)(V(\eta)(X_n)) &= \emptyset &\text{ iff } (X_1, \dots, X_n)
        \not\in \eta(X)
    \end{align*}
\end{definition}

The following example shows how a set of higher type will be translate to its characteristic function via the
function $V$ of Definition~\ref{definition:lower_bound_variable_function}.

\begin{example}
    Let $\mathcal{A}$ a $\sigma$-structure over universe $\mathcal{U} = \{1, 2, 3\}$ and $X$ a HO(LFP)$^{k + 1}$
    variable of type $((\odot, \odot), (\odot, \odot))$ mapped through variable mapping $\eta$ to
    \[\eta(X) = \{(\{(1, 1)\}, \{(2, 2)\}), (\{(1, 1), (2, 2)\}, \{(3, 3)\})\}\]
    then $V(\eta)(X)$ is a PHFL$^k$ function of type $\bullet \rightarrow (\bullet \rightarrow \bullet)$ such
    that $V(\eta)(X)$ $(\{(1, 1)\}) = f$, $V(\eta)(X)(\{(1, 1), (2, 2)\}) = g$ and $V(\eta)(X)(z) = h$ for $z \in
    Q^d$ and $z \neq \{(1, 1)\}$ and $z \neq \{(1, 1), (2, 2)\}$. $f$, $g$ and $h$ are functions of type $\bullet
    \rightarrow \bullet$ where $f(\{(2, 2)\}) = g(\{(3, 3)\}) = Q^d$ and $f(z) = g(z') = h(z'') = \emptyset$ for $z,
    z', z'' \in Q^d$ and $z \neq \{(2, 2)\}$ and $z' \neq \{(3, 3)\}$.
\end{example}

Because we are interested in the bisimulation-invariant fragment of HO( LFP)$^{k+1}$ we give a further definition that
makes the proof of the following lemma more comfortable. This definition gives for a given LTS and tuple of states a
reduced LTS of all those states that are reachable by at least one state of the given state tuple.

Finally we have to show the following lemma.

\begin{lemma}
    \label{lemma:ho_lfp_equals_phfl}
    Let $r \geq 1$ and $k \geq 0$. For every bisimulation invariant formula $\Phi$ of HO(LFP)$^{k + 1}$ there is a
    PHFL$^k$ formula $\Psi$ such that $\mathcal{Q}_\Phi^r = \mathcal{Q}_\Psi^r$.
\end{lemma}

\begin{proof}
    This lemma can be proven by showing for all HO(LFP)$^{k+1}$ formulas $\Phi$ with free first-order variables $x_1,
    \dots, x_f$ and quantified first-order variables $x_{f+1}, \dots, x_q$, all LTS $\mathcal{T} = (Q, \Sigma, P,
    \Delta, v)$ and all variable mappings $\eta$ that it holds that $\mathcal{T}, \eta \models \Phi$ iff $\emph{q} =
    (\eta(x_1), \dots, \eta(x_f), q_0, \dots, q_0)$ and $\emph{q} \in \llbracket
    \Gamma \vdash F(\Phi)\colon \bullet \rrbracket^{V(\eta)}_{\mathcal{T}}$, where $q_0
    \in Q$ is an arbitrary state, $F$ is the formula function of
    Definition~\ref{definition:lower_bounds_phfl_formula_function} and $V$ the variable mapping function of
    Definition~\ref{definition:lower_bound_variable_function}. This statement can be proven by induction over formula
    $\Phi$.
    \begin{compactitem}
        \item In case of $\Phi = p(x_i)$ then $\mathcal{T}, \eta \models \Phi$ holds exactly then if $\eta(x_i) \in
        p^\mathcal{T}$. Translated to the normal LTS definition of $\mathcal{T}$ it is the same as $p \in v(\eta(x_i))$.
        With $F(\Phi) = p_i$ this is exactly
        \begin{align*}
            (\eta(x_1),& \dots, \eta(x_{i-1}), \eta(x_{i}), \eta(x_{i+1}), \dots, \eta(x_f),\\& q_0, \dots, q_0) \in
            \llbracket \Gamma \vdash F(\Phi) \colon \bullet \rrbracket^{V(\eta)}_\mathcal{T}.
        \end{align*}
        \item In case of $\Phi = a(x_i, x_j)$ then $\mathcal{T}, \eta \models \Phi$ holds exactly then if $(\eta(x_i)
        , \eta(x_j))$ $ \in a^\mathcal{T}$. Translated to the normal LTS definition of $\mathcal{T}$ it is the same as $
        \eta(x_i) \overset{a}{\rightarrow} \eta(x_j)$. Let $g: \{1, 2\} \rightarrow \{i, j\}$ be a function.
        By  definition of the semantics of $\langle a \rangle_i \Phi$ the tuple
        \begin{align*}
            (\eta(x_1),& \dots, \eta(x_{g(1) - 1}), \eta(x_{g(1)}), \eta(x_{g(1)+1}), \dots, \eta(x_{g(2)-1}), \eta
            (x_{g(2)}),\\& \eta(x_{g(2)+1}), \dots, \eta(x_f), q_0, \dots, q_0))
        \end{align*}
        is an element of the semantics of $\langle a \rangle_i \Phi$ if
        \begin{align*}
            (\eta(x_1),& \dots, \eta(x_{g(1) - 1}), q_m, \eta(x_{g(1)+1}), \dots, \eta(x_{g(2)-1}), q_n,\\& \eta(x_{g(2)
            +1}), \dots, \eta(x_f), q_0, \dots, q_0)
        \end{align*}
        is an element of the semantics of $\Phi$ where $\eta(x_i)$ can
        move by an $a$ action to $q_m$ if $g(1) = i$ or to $q_n$ if $g(2) = i$. Because $\eta(x_j)$ is the state that
        have to be reached via an $a$ action from $\eta(x_i)$ we have to check if $q_m = \eta(x_j)$ in case of $g(1)
        = j$ and in case of $g(2) = j$ we have to check if $q_n = \eta(x_i)$. If two states are equal in an LTS is
        similar to check if those states are bisimilar. $\sim$ is given by formula $\Phi_\sim$ of
        Example~\ref{example:phfl_order_0}.
        Because this formula returns those $d$-tuples where the first and second component are bisimilar, we have to
        move the $i$-th and $j$-th component to the first and second component. This is given by $\{(i, j, 3, \dots, d
        )\} \Phi_\sim$. Summarizing all this steps with $F(\Phi) = \langle a \rangle_i \{(i, j, 2, \dots, d)\}
        \Phi_\sim$ it follows
        \begin{align*}
            (\eta(x_1),& \dots, \eta(x_{g(1) - 1}), \eta(x_{g(1)}), \eta(x_{g(1)+1}), \dots, \eta(x_{g(2)-1}), \eta
            (x_{g(2)}),\\& \eta(x_{g(2)+1}), \dots, \eta(x_f), q_0, \dots, q_0)) \in \llbracket \Gamma \vdash F(\Phi)
            \colon \bullet \rrbracket
        \end{align*}
        if $(\eta(x_i), \eta(x_j))$ $ \in a^\mathcal{T}$ and
        \begin{align*}
            (\eta(x_1),& \dots, \eta(x_{g(1) - 1}), \eta(x_{g(1)}), \eta(x_{g(1)+1}), \dots, \eta(x_{g(2)-1}), \eta
            (x_{g(2)}),\\& \eta(x_{g(2)+1}), \dots, \eta(x_f), q_0, \dots, q_0)) \not\in \llbracket \Gamma \vdash
            F(\Phi) \colon \bullet \rrbracket
        \end{align*}
        if $(\eta(x_i), \eta(x_j))$ $ \not\in a^\mathcal{T}$.

        \item In case of $\Phi = X(x_{i_1}, \dots, x_{i_n})$ where $X$ is a free variable of HO type $(\odot, \dots,
        \odot)$ and $x_{i_1}, \dots, x_{i_n}$ are free first-order variables then $\mathcal{T}, \eta \models \Phi$
        holds exactly then if $(\eta(x_{i_1}), \dots \eta(x_{i_n})) \in \eta(X)$. Because of definition of $V$ the
        tuple $(\eta(x_{i_1}), \dots \eta(x_{i_n}), \eta(x_{n + 1}), \dots, \eta(x_{f}), q_0, \dots, q_0)$ is in $V
        (\eta)(X)$ if $(\eta(x_{i_1}),\dots \eta(x_{i_n})) \in \eta(X)$ and is not in $V(\eta)(X)$ otherwise. Because
        the components $i_1, \dots, i_n$ for first-order variables $x_{i_1}, \dots, x_{i_n}$ have not to be $1,
        \dots, n$ respectively, we first move them to this components and check then if
        \[(\eta(x_{i_1}), \dots \eta(x_{i_n}), \eta(x_{n + 1}), \dots, \eta(x_{f}), q_0, \dots, q_0) \in V(\eta)(X).\]
        So it holds with $F(\Phi) = {(i_1, \dots, i_n, n+1, \dots, d)}X$ and function $g: \{1, \dots, n\}
        \rightarrow \{i_1, \dots, i_n\}$ that
        \begin{align*}
            (\eta(x_1),& \dots, \eta(x_{g(1)-1}), \eta(x_{g(1)}), \eta(x_{g(1)+1}), \dots \eta(x_{g(2)-1}), \eta
            (x_{g(2)}),\\& \eta(x_{g(2)+1}), \dots, \eta(x_{g(n)-1}), \eta(x_{g(n)}), \eta(x_{g(n)+1}), \dots, \eta
            (x_f),\\& q_0, \dots, q_0) \in \llbracket \Gamma \vdash F(\Phi) \colon \bullet \rrbracket^{V(\eta)
            }_\mathcal{T}
        \end{align*}
        if $(\eta(x_{i_1}), \dots \eta(x_{i_n})) \in \eta(X)$ and
        \begin{align*}
            (\eta(x_1),& \dots, \eta(x_{g(1)-1}), \eta(x_{g(1)}), \eta(x_{g(1)+1}), \dots \eta(x_{g(2)-1}), \eta
            (x_{g(2)}),\\& \eta(x_{g(2)+1}), \dots, \eta(x_{g(n)-1}), \eta(x_{g(n)}), \eta(x_{g(n)+1}), \dots, \eta
            (x_f),\\& q_0, \dots, q_0) \not\in \llbracket \Gamma \vdash F(\Phi) \colon \bullet \rrbracket^{V(\eta)
            }_\mathcal{T}
        \end{align*}
        if $(\eta(x_{i_1}), \dots \eta(x_{i_n})) \not\in \eta(X)$.

        \item In case of $\Phi = X(X_1, \dots, X_n)$ where $X$ is a free variable of HO type $(\tau, \dots,
        \tau)$ and $X_1, \dots, X_n$ are free variables of HO type $\tau$ then $\mathcal{T}, \eta \models \Phi$
        holds exactly then if $(\eta(X_1), \dots \eta(X_n)) \in \eta(X)$. Because of definition of $V$ it follows
        \[(\dots((V(\eta)(X)(V(\eta)(X_1)))(V(\eta)(X_2)))\dots)(V(\eta)(X_n)) = Q^d\]
        if $(\eta(X_1), \dots \eta(X_n)) \in \eta(X)$ and
        \[(\dots((V(\eta)(X)(V(\eta)(X_1)))(V(\eta)(X_2)))\dots)(V(\eta)(X_n)) = \emptyset\]
        if $(\eta(X_1), \dots \eta(X_n)) \not\in \eta(X)$. With $F(\Phi) = (\dots ((X(X_1))(X_2))$ $\dots)(X_n)$
        it follows
        \[ (\eta(x_1), \dots, \eta(x_f), q_0, \dots, q_0) \in \llbracket \Gamma \vdash F(\Phi) \colon
        \bullet \rrbracket^{V(\eta)}_\mathcal{T} = Q^d.\]
        if $(\eta(X_1), \dots \eta(X_n))\in \eta(X)$ and
        \[ (\eta(x_1), \dots, \eta(x_f), q_0, \dots, q_0) \not\in \llbracket \Gamma \vdash F(\Phi) \colon \bullet
        \rrbracket^{V(\eta)}_\mathcal{T} = \emptyset.\]
        if $(\eta(X_1), \dots \eta(X_n)) \not\in \eta(X)$.
    \end{compactitem}
    By induction hypothesis it holds for HO(LFP)$^{k+1}$ formulas $\Psi$ and $\Psi'$ with free variables $x_1, \dots,
    x_f$ and quantified first-order variables $x_{f+1}, \dots, x_q$ that for all LTS $\mathcal{T}$ and all variable
    mappings $\eta$ that $\mathcal{T}, \eta \models \Psi$ iff $ (\eta(x_1), \dots, \eta(x_f),$ $ q_0, \dots, q_0) \in
    \llbracket \Gamma \vdash F(\Psi) \colon \bullet^{V(\eta)}_\mathcal{T}$ and $\mathcal{T}, \eta \models \Psi'$ iff $
    (\eta(x_1), \dots, \eta(x_f), q_0, \dots, $ $q_0) \in \llbracket \Gamma \vdash F(\Psi') \colon \bullet^{V(\eta)
    }_\mathcal{T}$.
    \begin{compactitem}
        \item In case of $\Phi = \neg \Psi$ it follows that $\mathcal{T}, \eta \models \Phi$ exactly then if
        $\mathcal{T}, \eta \not\models \Psi$. By induction hypothesis that is exactly then the case when
        \[ (\eta(x_1), \dots, \eta(x_f), q_0, \dots, q_0) \not\in \llbracket \Gamma \vdash F(\Psi) \colon \bullet
        \rrbracket^{V(\eta)}_\mathcal{T}.\]
        This is exactly the case if
        \[ (\eta(x_1), \dots, \eta(x_f), q_0, \dots, q_0) \in Q^d \setminus \llbracket \Gamma^- \vdash F(\Psi) \colon
        \bullet \rrbracket^{V(\eta)}_\mathcal{T}.\]
        And this is exactly the semantics of $F(\Phi) = \neg F(\Psi)$.

        \item In case of $\Phi = \Psi \vee \Psi'$ it follows that $\mathcal{T}, \eta \models \Phi$ exactly then if
        $\mathcal{T}, \eta \models \Psi$ or $\mathcal{T}, \eta \models \Psi'$. By induction hypothesis that is
        exactly then the case when
        \[(\eta(x_1), \dots, \eta(x_f), q_0, \dots, q_0) \in \llbracket \Gamma \vdash F
        (\Psi) \colon \bullet \rrbracket^{V(\eta)}_\mathcal{T}\]
        or
        \[(\eta(x_1), \dots, \eta(x_f), q_0, \dots, q_0) \in \llbracket \Gamma \vdash F
        (\Psi') \colon \bullet \rrbracket^{V(\eta)}_\mathcal{T}.\]
        Because $\sqcup_{\bullet} = \cup$ this can be combined to
        \[(\eta(x_1), \dots, \eta(x_f), q_0, \dots, q_0) \in \llbracket \Gamma \vdash F
        (\Psi) \colon \bullet \rrbracket^{V(\eta)}_\mathcal{T} \sqcup_\bullet \llbracket \Gamma \vdash F
        (\Psi') \colon \bullet \rrbracket^{V(\eta)}_\mathcal{T},\]
        what is the semantics of $F(\Phi) = F(\Psi) \vee F(\Psi')$.

        \item In case of $\Phi = \exists (x_i\colon \odot).\,\Psi$ it follows that $\mathcal{T}, \eta \vdash \Phi$ iff
        there exists  $\mathcal{X} \in Q$ with $\mathcal{T}, \eta[x_i \rightarrow \mathcal{X}] \models \Psi$. Because
        $x_i$ is a free variable in $\Psi$ it follows by induction hypothesis that $\mathcal{T}, \eta[x_i
        \rightarrow \mathcal{X}] \models \Psi$ is exactly the case when
        \begin{align*}
            (\eta(x_1),& \dots, \eta(x_{i-1}), \eta(x_i), \eta(x_{i+1}), \dots, \eta(x_{f+1}),\\& q_0, \dots, q_0) \in
            \llbracket \Gamma \vdash F(\Psi) \colon \bullet \rrbracket^{V(\eta)}_\mathcal{T}.
        \end{align*}
        Because $x_i$ is bounded in $\Phi$ we have to replace the $i$-th component by one of the free. This is
        denoted by $\overset{f}{\underset{j=1}{\bigvee}} \{(1, \dots, i-1, j, i+1, \dots, d)\}$.
        Because of bisimilarity all states of $\mathcal{T}$ have an order and are reachable by moving
        By replacing the $i$-th component by one of the free first-order variables and check all reachable state
        tuples, we move over the

        \item In case of $\Phi = \exists (X \colon \tau).\,\Psi$ it follows that $\mathcal{T}, \eta \vdash \Phi$ iff
        there exists $\mathcal{X} \in D_\tau(Q)$ with $\mathcal{T}, \eta[X \rightarrow \mathcal {X}] \models \Psi$.
        By induction hypothesis it follows that $(\eta(x_1), \dots, \eta(x_{f}), q_0, \dots, q_0) \in
        \llbracket \Gamma \vdash F(\Psi) \colon \bullet \rrbracket^{V(\eta)}_\mathcal{T}$ iff $\mathcal{T}, \eta[X
        \rightarrow \mathcal {X}] \models \Psi$. By Lemma~\ref{lemma:existential_quantifier} the formula
        $\exists^\tau X.\, \Psi(X)$ iterates through all elements of $D_\tau(Q)$ and checks if one fulfills $\Psi(X)$
        by returning $Q^d$.
        With $F(\Phi) = \exists^\tau X.\, F(\Psi)(X)$ it holds \[(\eta(x_1), \dots, \eta(x_{f}), q_0, \dots, q_0) \in
        \llbracket \Gamma \vdash F(\Phi) \colon \bullet \rrbracket^{V(\eta)}_\mathcal{T}.\]

        \item In case of $\Phi = [LFP\,\Psi(X, x_{i_1}, \dots, x_{i_n})](v_{j_1}, \dots, v_{j_n})$, where $X$ is a
        free variable in $\Psi$ of HO type $(\odot, \dots, \odot)$ and $x_{i_1}, \dots, x_{i_n}$ are free first-order
        variables of $\Psi$ and $v_{j_1}, \dots, v_{j_n}$ are first-order variables of $\Phi$, then it follows that
        $\mathcal{T}, \eta \vdash \Phi$ exactly then if $(\eta(v_{j_1}), \dots, \eta(v_{j_n})) \in LFP
        (F_\Psi^\mathcal{T})$. By definition of LFP $(\eta(v_{j_1}), \dots, \eta(v_{j_n})) \in
        LFP(F_\Psi^\mathcal{T})$ iff $X = F_\Psi^\mathcal{T}(X)$ and $(\eta(v_{j_1}), \dots, \eta(v_{j_n})) \in
        F_\Psi^\mathcal{T}(X)$. By definition of $F_\Psi^\mathcal{T}(X)$ it holds $(\eta
        (v_{j_1}), \dots, \eta(v_{j_n})) \in F_\Psi^\mathcal{T}(X)$ exactly then if $\mathcal{T}, \eta \models \Psi
        (\eta(X), \eta(v_{j_1}), \dots, \eta(v_{j_n}))$. By induction hypothesis
        \[(\eta(v_{j_1}), \dots, \eta(v_{j_n}), \eta(x_{n+1}), \dots, \eta(x_f), q_0, \dots, q_0) \in \llbracket
        \Gamma \vdash F(\Psi) \colon \bullet \rrbracket^{V(\eta)}_\mathcal{T}.\]
        Because of construction of variable mapping $V$ and $(\eta(v_{j_1}), \dots, \eta(v_{j_n})) \in \eta(X)$ the
        tuple $(\eta(v_{j_1}), \dots, \eta(v_{j_n}), \eta(x_{n+1}), \dots, \eta(x_f), q_0, \dots, q_0)$ is also in $V
        (\eta)(X)$. By definition of the least fixpoint operator in PHFL this set $X$ is reached if it includes all
        elements of $\llbracket \Gamma \vdash F(\Psi) \colon \bullet \rrbracket^{V(\eta)}_\mathcal{T}$. Because of
        construction of $F_\Psi^\mathcal{T}(X)$ and $X = F_\Psi^\mathcal{T}(X)$ it holds that $(\eta(v_{j_1}),
        \dots, \eta(v_{j_n})) \in X$ iff
        \[(\eta(v_{j_1}), \dots, \eta(v_{j_n}), \eta(x_{n+1}), \dots, \eta(x_f), q_0, \dots, q_0) \in \llbracket
        \Gamma \vdash F(\Psi) \colon \bullet \rrbracket^{V(\eta)}_\mathcal{T}.\]
        Because $X$ is the least fixpoint of $F_\Psi^\mathcal{T}$ the resulting set $X'$ is the smallest set that
        fulfills $X' \subseteq X$ and so
        \[(\eta(v_{j_1}), \dots, \eta(v_{j_n}), \eta(x_{n+1}), \dots, \eta(x_f), q_0, \dots, q_0) \in \llbracket
        \Gamma \vdash \mu(X\colon \bullet).\,F(\Psi) \colon \bullet \rrbracket^{V(\eta)}_\mathcal{T}.\]
        Because the components $j_1, \dots, j_n$ for first-order variables $v_{j_1}, \dots, v_{j_n}$ have not to be $1,
        \dots, n$ respectively, we first move them to this components and check then the least fixpoint operator.
        With $F(Phi) = $ it follows
        \[(\eta(v_{j_1}), \dots, \eta(v_{j_n}), \eta(x_{n+1}), \dots, \eta(x_f), q_0, \dots, q_0) \in \llbracket
        \Gamma \vdash \mu(X\colon \bullet).\,F(\Phi) \colon \bullet \rrbracket^{V(\eta)}_\mathcal{T}.\]
        So it holds with $F(\Phi) = \{j_1, \dots, j_n, n+1, \dots, d\} \mu (X \colon \bullet).\, F(\Psi)$ and
        function $g: \{1, \dots, n\} \rightarrow \{j_1, \dots, j_n\}$ that
        \begin{align*}
            (\eta(x_1),& \dots, \eta(v_{g(1)-1}), \eta(v_{g(1)}), \eta(v_{g(1)+1}), \dots \eta(v_{g(2)-1}), \eta
            (v_{g(2)}),\\& \eta(v_{g(2)+1}), \dots, \eta(v_{g(n)-1}), \eta(v_{g(n)}), \eta(v_{g(n)+1}), \dots, \eta
            (x_f),\\& q_0, \dots, q_0) \in \llbracket \Gamma \vdash F(\Phi) \colon \bullet \rrbracket^{V(\eta)
            }_\mathcal{T}
        \end{align*}
        exactly then if $\mathcal{T}, \eta \vdash \Phi$.

        \item In case of $\Phi = [LFP\,\Psi(X, X_1, \dots, X_n)](V_1, \dots, V_n)$, where $X$ is a
        free variable in $\Psi$ of HO type $(\tau, \dots, \tau)$ and $X_1, \dots, X_n$ are free first-order
        variables of $\Psi$ of type $\tau$ and $V_1, \dots, V_n$ are free variables of $\Phi$ also of type $\tau$, then
        it follows that $\mathcal{T}, \eta \vdash \Phi$ exactly then if $(\eta(X_1), \dots, \eta(X_n) \in LFP
        (F_\Psi^\mathcal{T})$. By definition of LFP $(\eta(V_1), \dots, \eta(V_n)) \in
        LFP(F_\Psi^\mathcal{T})$ iff $X = F_\Psi^\mathcal{T}(X)$ and $(\eta(V_1), \dots, \eta(V_n)) \in
        F_\Psi^\mathcal{T}(X)$. By definition of $F_\Psi^\mathcal{T}(X)$ it holds $(\eta
        (V_1), \dots, \eta(V_n)) \in F_\Psi^\mathcal{T}(X)$ exactly then if $\mathcal{T}, \eta \models \Psi
        (\eta(X), \eta(V_1), \dots, \eta(V_n))$. By induction hypothesis
        \[(\eta(x_1), \dots, \eta(x_f), q_0, \dots, q_0) \in \llbracket \Gamma \vdash F(\Psi) \colon \bullet
        \rrbracket^{V(\eta)}_\mathcal{T}.\]
        Because of construction of variable mapping $V$ and $(\eta(V_1), \dots, \eta(V_n)) \in \eta(X)$ it holds
        $(\dots((V(\eta)(X)(V(\eta)(V_1)))(V(\eta)(V_2))) \dots (V(\eta)(V_n)) = Q^d$
        and so the tuple $(\eta(x_1), \dots, \eta(x_f), q_0, \dots, q_0)$ is also in $(\dots((V(\eta)(X)(V(\eta)(V_1)
        ))(V(\eta)(V_2))) \dots (V(\eta)(V_n))$. By definition of the least fixpoint operator in PHFL this function
        $V(\eta)(X)$ is reached if it includes all
        elements of $\llbracket \Gamma \vdash F(\Psi) \colon \bullet \rrbracket^{V(\eta)}_\mathcal{T}$. Because of
        construction of $F_\Psi^\mathcal{T}(X)$ and $X = F_\Psi^\mathcal{T}(X)$ it holds that $(\eta(V_{1}),
        \dots, \eta(V_{n})) \in X$ iff
        \[(\eta(x_1), \dots, \eta(x_f), q_0, \dots, q_0) \in \llbracket
        \Gamma \vdash F(\Psi) \colon \bullet \rrbracket^{V(\eta)}_\mathcal{T}.\]
        Because $X$ is the least fixpoint of $F_\Psi^\mathcal{T}$ the resulting set $X'$ is the smallest set that
        fulfills $X' \subseteq X$ and so
        \[(\eta(x_{1}), \dots, \eta(x_f), q_0, \dots, q_0) \in \llbracket
        \Gamma \vdash \mu(X\colon \bullet).\,F(\Psi) \colon \bullet \rrbracket^{V(\eta)}_\mathcal{T}.\]
        With $F(Phi) = (\dots ((\mu (X \colon T(\tau)).\,\Phi(X, X_1, \dots, X_n))(V_1))(V_2))\dots)(V_n)$ it follows
        \[(\eta(x_{1}), \dots, \eta(x_f), q_0, \dots, q_0) \in \llbracket
        \Gamma \vdash \mu(X\colon \bullet).\,F(\Phi) \colon \bullet \rrbracket^{V(\eta)}_\mathcal{T}.\]
        exactly then if $\mathcal{T}, \eta \vdash \Phi$.
    \end{compactitem}
\end{proof}

The combination of Lemma~\ref{lemma:ho_lfp_equals_phfl}, Theorem~\ref{theorem:hoLfpEqualsExptime} and
Theorem~\ref{theorem:phfl_k_in_k_exptime} proves the following theorem.

\begin{theorem}
    Let $k \geq 0$. PHFL$^k$ captures \exptime{$k$} over labeled transition systems.
\end{theorem}



%%
%% Author: DKron
%% 17.08.2018
%%

\section{Lower Bound of PHFL$^{k + 1}_{tail}$}\label{sec:lowerBoundOfPhflTail}

The lower bound of PHFL$^{k+1}_{tail}$ can be proven similar to the lower bound of PHFL$^k$. The main idea is not to
show directly, that the lower bound of PHFL$^{k+1}_{tail}$ is \expspace{$k$} but rather by detour over the
bisimulation-invariant fragment of HO(PFP)$^{k+1}$. In the first subsection we show that a Turing Machine that is in
$k$-EXPSPACE can be encoded by a HO(PFP)$^{k+1}$ formula. The next subsection uses this statement to show that the
lower bound of PHFL$^{k+1}_{tail}$ is \expspace{$k$} by encoding formulas of HO(PFP)$^{k+1}$ in PHFL$^{k+1}_{tail}$.

\subsection{$k$-EXPSPACE and HO(PFP)$^{k+1}$}\label{subsec:kExpspaceInHopfp}

In this subsection we want to show that a run of an $exp(k, f(n))$ space bounded DTM can be encoded by some
HO$^{k+1}$ formula. The main idea of this statement is an extension of the result of Abiteboul and
Vianu~\cite{abiteboul1995computing} into higher-order. They have shown that HO(PFP)$^1$ coincides with $0$-EXPSPACE.

%Because a finite $\sigma$-structure $\mathcal{T}$ for some relational signature $\sigma$ cannot be fed to a Turing
%%machine directly, we have to encode $\mathcal{T}$ to an input word $\langle \mathcal{T} \rangle$. We refer the
%interested reader to~\ref{}

\begin{lemma}
    \label{lemma:expspace_in_ho_pfp}
    Given a $k$-EXPSPACE-bounded DTM $M = (Q, \Sigma, \Gamma, \delta, q_0, \Box,$ $ F, R)$ there exists a formula $\Psi$ in HO(PFP)$^{k+1}$ over signature $\sigma$ such that for all suitable variable mappings $\eta$ and all LTS $\mathcal{T}$ it holds $\mathcal{T}, \eta \models \Psi$ exactly then if the run of $M$ on the standard coding of $(\mathcal{T}, \eta)$ is accepting.
\end{lemma}

\begin{proof}
    Let $M = (Q, \Sigma, \Gamma, \delta, q_0, \Box, F, R)$ be a $exp(k, f(n))$ space bounded DTM, $\mathcal{T}$ an LTS and $Q'$ the set of states of $\mathcal{T}$. Furthermore, we can advise an linear ordering $\exists (< \colon (\odot, \odot)).\, \varphi$ on $Q'$ of $\mathcal{T}$, where $\varphi$ describes that $<$ is an order~\cite{fagin1974generalized}. Finally, let $\tau$ be an HO type of
    order $k + 1$ and $\eta$ a variable mapping.
    To prove this lemma we want to define a relational representation, of the final configuration of $M$ that has the standardized encoding of $(\mathcal{T}, \eta)$\footnote{The standardized encoding of structures is a non-trivial problem. Because the description of the encoding goes beyond the scope of this thesis, we only refer to~\cite{abiteboul1995computing} for further information about.}, abbreviated with $w$, as input word,
    as a partial fixpoint of some HO(PFP)$^{k+1}$ formula. 
    In order to do this, we construct a partial 
fixpoint of order $k+2$ such that for all $i$, the $i$-th approximation of 
this fixpoint encodes the $i$-th configuration in the run of $M$ with 
input word $w$.    

    Before we can define the configurations of $M$ in HO we have to make some preparations. Remember that a
    configuration of $M$ comprises the current state, the current reading head position and
    the current tape content represented by a function. These configurations will be combined in one relation $X$.
    Because the size of the formula we build have to be polynomial and the reading head of $M$ can be on
    one of $exp(k, f(n))$ cells for example, we have to encode the number in sets of order $k + 1$. Furthermore, in order to
    not exceed the bound of order $k + 1$, we have to split the tape content function in such a way that in one tuple
    $x$ of $X$ is just the current state, the current head position and one position of the tape with its content. By
    syntax of HO types each component of $x \in X$ has to be of the same type, so to access the different states and
    tape symbols they have to be numerated. $\{0, \dots, |Q| - 1\}$ for states and $\{0, \dots, |\Gamma| - 1\}$ for tape
    symbols.

    The next step is to define some abbreviations that we want to use in the definitions of the configurations. The
    first and most important abbreviation is the definition of orders of any HO type. These orders are defined similarly
    to the defined formulas of Definition~\ref{definition:lower_bound_less_higher}. A tuple $x$ is smaller then a tuple $y$ if there is a position $i$ where $X_i < y_i$ and there is no position $j<i$ where $X_j > y_j$. A set $X$ is smaller then a tuple $Y$ if there is a $x \in Y$ such that $x \not\in X$ and there is no $y<x$ such that $y\in X$ but $y\not\in Y$.
    \begin{align*}
        <^\odot(x, y) \coloneqq &\,<(x, y) \\
        <^{\tau' \times \dots \times \tau'}(X_1, y_1, \dots, X_n, y_n) \coloneqq &\,\underset{i =
        1}{\overset{n}{\bigvee}}<^{\tau'}(X_i, y_i) \wedge \underset{j = 1}{\overset{i - 1}{\bigwedge}}
        \neg <^{\tau'}(y_j, X_j)\\
        <^{(\tau', \dots, \tau')}(X, Y) \coloneqq &\,\exists (X_1 \colon {\tau'}). \,\dots \exists(X_n \colon
        {\tau'}).\, Y(X_1, \dots, X_n)
        \\&\,\wedge \neg X(X_1, \dots, X_n)\,\wedge \forall (y_1 \colon {\tau'}). \,\dots
        \forall(y_1 \colon {\tau'}).\,\\&\,<^{\tau'\times \dots \times \tau'}
        (y_1, X_1, \dots, y_n, X_n) \\&\,\Rightarrow (X(y_1, \dots, y_n) \Rightarrow Y(y_1, \dots, y_n))
    \end{align*}
    With these formulas it is possible to define another two important abbreviations. On the one hand equality
    of two variables of arbitrary type and on the other hand the successor of a given element. If $X$ and
    $Y$ are two variables of type $\tau$ then the equality of $X$ and $Y$ is given by the formula
    \[X = Y \coloneqq \neg<^\tau(X, Y) \wedge \neg <^\tau(Y, X).\]
    Finally, if $X$ and $Y$ are two variables of type $\tau$ then the prove that $Y$ is the successor of $X$ is given by
    the formula
    \[next^{\tau}(X, Y) \coloneqq\; <^\tau(X, Y) \wedge \forall (Z \colon \tau).\, <^\tau(Z, Y) \Rightarrow\;<^\tau
    (Z, X).\]

    Now we are able to define the configurations in HO(PFP)$^k$. The first configuration is the initial configuration. For input word $w$ this is given by the formula
    \begin{align*}
        \varphi_0(q, h, i, b) \coloneqq &\,q = q_0 \wedge h = 0 \wedge (\neg <^\tau(|w|, i) \Rightarrow b = w_{i})\,
        \vee\\&\,(\neg <^\tau (i, |w|) \wedge \neg i = |w| \Rightarrow b = \Box))
    \end{align*}
    where $q$ is the current state, $h$ the current head position, $i$ a tape index and $b$ the symbol on $i$. $q_0$,
    $0$, $|w|$, $w_{i}$ and $\Box$ are the numerical representations as sets of the same elements in $Q$ and $\Gamma$.

    To iterate through all configurations of $M$ on input $w$, we need a variable $X$ of type $\tau =
    (\tau', \tau', \tau', \tau')$ where $\tau'$ has order $k + 1$, so $X$ has order $k + 2$. On an iteration
    $(F_\varphi^{\mathcal{T},\eta})^{i+1}(\varnothing)$ for the following formula $\varphi$ the variable $X$ includes
    all configurations that will be reached within $i$ transitions.
    \begin{align*}
        \varphi(X, q, h, j, b) \coloneqq &\,(\forall (q_{old} \colon \tau').\, \forall
        (h_{old}
        \colon \tau').\, \forall (j_{old} \colon \tau').\, \forall (b_{old} \colon \tau').\,\\
        &\, \neg X(q_{old}, h_{old}, i_{old}, b_{old}) \vee \neg \varphi_0(q, h, i, b)) \vee \xi(X, q, h, i, b)
    \end{align*}
    Note that $\varphi_0$ is invoked only in the first iteration and thus provides the correct initialisation. The
    formula $\xi$ collects the transitions of those tuples in $X$ according to the transition function $\delta$ of
    $M$. In each iteration exactly one configuration will be added to $X$ because $M$ is deterministic.
    The formula $\xi$ is given by
    \begin{align*}
        \xi(X, q, h, i, b) \coloneqq &\,\exists (q_{old} \colon \tau').\, \exists (h_{old} \colon
        \tau').\, \exists (i_{old} \colon \tau').\, \exists (b_{old} \colon \tau').\, \\
        &\,\exists (q_{new} \colon \tau').\, \exists (b_{new} \colon \tau').\,X(q_{old}, h_{old}, i_{old},
        b_{old}) \,\wedge \\
        &\, \Big(\underset{\delta(q_{old}, b_{old}) = (q_{new}, b_{new}, d)}{\bigvee} q = q_{new} \wedge h =
        h_{old} + d \wedge i = i_{old}\,\wedge\\&\, \big((\neg i = h_{old} \wedge b =
        b_{old}) \vee (i = h_{old} \wedge b = b_{new})\big)\Big)
    \end{align*}
    where $h = h_{old} + d$ depends on $d$ and is given by
    \[h = h_{old} + d \coloneqq
    \begin{cases}
        next^\tau(h_{old}, h),  & \text{if } d = L\\
        next^\tau(h, h_{old}),  & \text{if } d = R\\
        h = h{old},  & \text{if } d = N
    \end{cases}\]

    Because $M$ terminates, the formula
    \[\psi_0(q', h', i', b') \coloneqq [\mathit{PFP}\;\varphi(X, q, h, i, b)](q', h', i', b') \]
    is guaranteed to define the relational description of the final configuration of $M$ on input word $w$.
    Finally, the formula
    \[\psi \coloneqq \exists (h' \colon \tau').\, \exists (i' \colon \tau').\, \exists (b' \colon \tau').\,
    \underset{q' \in F}{\bigvee} \psi_0(q', h', i', b')\]
    defines the acceptance of $M$ on input $w$.
\end{proof}

\subsection{Encoding of Bisimulation Invariant HO(PFP)$^{k+1}$ in PHFL$^{k+1}_{tail}$}\label{subsec:bisimulationInvariantHopfptoPhfl}

As mentioned in the introduction of this section the main idea is not to show directly that the 
lower bound of PHFL$^{k+1}_{tail}$ is \expspace{$k$} but rather by detour over the 
bisimulation-invariant fragment of HO(PFP)$^{k+1}$. In the previous subsection we have seen 
 that a $k$-EXPSPACE-bounded DTM can be performed by an HO(PFP)$^{k+1}$
formula. By encoding the bisimulation-invariant fragment of HO(PFP)$^{k+1}$ into
PHFL$^{k+1}_{tail}$ combined with the knowledge that $k$-EXPSPACE is captured by 
HO(PFP)$^{k+1}$ leads to the lower bound of PHFL$^{k+1}_{tail}$. In
Section~\ref{sec:existential_quantifiers_in_phfl} and Section~\ref{sec:lowerBoundOfPhfl} we 
have shown that the HO$^{k+1}$ part can be encoded in PHFL$^k$. It is easy to prove that the 
encoded formulas are all tail-recursive\footnote{The PHFL$^0$ formula $\Phi_\sim$ (Example~\ref{example:phfl_order_0}) is, indeed, not tail-recursive, but over finite LTS it is equivalent to a tail-recursive PHFL$^1$ formula~\cite{lange2014capturing_long}.}. It follows that the HO$^{k+1}$ part can also be 
encoded in  PHFL$^{k+1}_{tail}$. The PFP operator is the only kind of HO(PFP)$^{k+1}$ formula 
that we have to encode in this subsection to get the lower bound of PHFL$^{k+1}_{tail}$.

Before we give the definition of the transforming function, we define a PHFL formula for HO formulas that uses the PFP operator.

\begin{definition}
Let $d$ be the constant as described in Remark~\ref{remark:dimension} and $X$ a HO variable of HO type $\tau = (\tau', \dots, \tau')$ where $\tau' \neq \odot$. Furthermore, let
    $\Phi$ be a PHFL$^k_{tail}$
    formula, then $PFP^\tau X.\,\Phi$
    is a PHFL$^k_{tail}$ formula with dimension $d$ defined as:
    \begin{align*}
     PFP^\tau X. \, \Phi \coloneqq &\Big(\mu (F \colon T(\tau) \rightarrow \bullet).\,\lambda (X \colon T(\tau)).\, \big(X\,\wedge \forall^{\tau'}X_1.\, \dotsb \forall^{\tau'}X_n.\, \\&( (\dotsb (X X_1) \dotsb) X_n \Leftrightarrow \Phi(X, X_1, \dots, X_n) ) \vee F(\Phi(X)\big)\Big)\bot_{T(\tau)}
\end{align*}    
    In case of $\tau = (\odot, \dots, \odot)$ let $PFP^{(\odot, \dots, \odot)} X.\,\Phi$ defined as:
    \begin{align*}
    PFP^{(\odot, \dots, \odot)} X.\,\Phi \coloneqq & \Big(\mu (F \colon \bullet \rightarrow \bullet).\,\lambda (X \colon \bullet).\, \big(X \wedge \forall_1 \dotsb \forall_n \\&(X \Leftrightarrow \Phi(X)) \vee F(\Phi(X)\big)\Big)\,\bot 
    \end{align*}
\end{definition}

Now we are able to define the function that translates a bisimulation invariant HO(PFP)$^{k+1}$
formula to a PHFL$^{k+1}_{tail}$ formula. Note that the encoding function defined in the following definition differs only in the fixpoint operators from the encoding function from Definition~\ref{definition:lower_bounds_phfl_formula_function}.

\begin{definition}
    \label{definition:lower_bounds_phfl_formula_function_pfp}
   Define $F$ as the function that maps a bisimulation invariant HO(PFP)$^{k+1}$ formula $\varphi$ to a PHFL$^{k+1}_{tail}$ formula with dimension $d$, where $d$ and $s$ are the constants as described in Remark~\ref{remark:dimension} and $\Phi_\sim$ is the formula of Example~\ref{example:phfl_order_0}, then $F$ is defined
    inductive on $\varphi$ as follows:
    \begin{align*}
        F(p(X_i)) \coloneqq &\, p_{2s+i} \\
        F(a(X_i, X_j)) \coloneqq &\, \langle a \rangle_{2s+i} \{(2s+i, 2s+j, \\
        &\,3, \dots, d)\} \Phi_\sim \\
        F(\Phi \vee \Psi) \coloneqq &\, F(\Phi) \vee F(\Psi) \\
        F(\neg \Phi) \coloneqq &\, \neg F(\Phi) \\
        F(\exists (X_i \colon \odot).\,\Phi) \coloneqq &\, \exists_{2s+i} F(\Phi) \\
        F(\exists (X \colon \tau).\,\Phi) \coloneqq &\, \exists^\tau X.\,F(\Phi(X)) \\
        F([PFP\;\Phi(X, X_{i_1}, \dots, X_{i_n})](V_{j_1}, \dots, V_{j_n})) \coloneqq &\,\{(j_1, \dots, j_n, n + 1, \dots, d)\} \\
        &\,PFP^{(\odot, \dots, \odot)} X.\, F(\Phi) \\
        F([PFP\;\Phi(X, X_1, \dots, X_n)](V_1, \dots, V_n)) \coloneqq &\,(\dotsb \big(PFP^\tau X.\, F(\Phi)\big)\,V_1)\dotsb)\,V_n \\
        F(X(X_{i_1}, \dots, X_{i_n})) \coloneqq &\, \{(2s+i_1, \dots, 2s+i_n, \\
        &\,n + 1, \dots, d)\}X\\
        F(X(X_1, \dots, X_n)) \coloneqq &\, (\dotsb (X\,X_1)\dotsb)\,X_n
    \end{align*}
\end{definition}

The last step is to show that the semantics of a given HO(PFP)$^{k+1}$ formula coincides with the semantics of the translated PHFL$^{k+1}_{tail}$ formula. As mentioned in Section~\ref{sec:lower_bounds_preparation} without loss of generality the statement can be proven by consider only  reduced LTS. 

\begin{lemma}
    \label{lemma:ho_pfp_equals_phfl_tail}
    Let $f \geq 1$ and $k \geq 0$. For every bisimulation-invariant formula $\Phi$ of HO(PFP)$^{k + 1}$ there is a
    PHFL$^{k+1}_{tail}$ formula $\Psi$ such that the $f$-adic query $\mathcal{Q}_\Phi^f$ defined by $\Phi$ is equal to the $f$-adic query  $\mathcal{Q}_\Psi^f$ defined by $\Psi$.
\end{lemma}

\begin{proof}
    This lemma can be proven by showing for all HO(PFP)$^{k+1}$ formulas $\Phi$ with first-order variables $X_1,
    \dots, X_q$, all reduced LTS $\mathcal{T} = (Q, \Sigma, P,
    \Delta, v)$ with respect to $\emph{q}_r = q_1, \dots, q_r$ and all variable mappings $\eta$ that it holds that $\mathcal{T}, \eta \models \Phi$ iff $\emph{q} =
    (\emph{q}_s, \emph{q}_s, \emph{q}_q, \emph{q}_r)$ and $\emph{q} \in \llbracket
   F(\Phi)\rrbracket^{\eta_V}_{\mathcal{T}}$. Here $\emph{q}_s = q_1',\dots,q_s'$ is a sequence of $s$ placeholders used for the interaction of second-order variables, $\emph{q}_q = \eta(X_1), \dots, \eta(X_q)$ is a sequence of first-order variables that are mapped by $\eta$ where $\eta(X_i) = q_0$ if $X_i$ is bound to a quantifier, $q_0, q_1', \dots, q_s' \in Q$ are arbitrary states, $F$ is the formula function of
    Definition~\ref{definition:lower_bounds_phfl_formula_function} and $\eta_V$ the variable mapping of
    Definition~\ref{definition:lower_bound_variable_function}. This statement can be proven by induction over formula
    $\Phi$.
    Because the correctness proof of the non-fixpoint formulas is very similar to the correctness proof of Theorem~\ref{theorem:ho_lfp_equals_phfl} we concentrate us on showing correctness of the PFP operators.
    \begin{compactitem}
    \item In case of $\Phi = [PFP\,\Psi(X, X_{i_1}, \dots, X_{i_n})](V_{j_1}, \dots, V_{j_n})$, where $X$ is a
        free variable in $\Psi$ of HO type $(\odot, \dots, \odot)$ and $X_{i_1}, \dots, X_{i_n}$ are free first-order
        variables of $\Psi$ and $V_{j_1}, \dots, V_{j_n}$ are first-order variables of $\Phi$ such that, without loss of generality, $i_1 < i_2, j_1 < j_2, \dots, i_{n-1} < i_n, j_{n-1} < j_n$. All other cases working similar. Then it follows that
        $\mathcal{T}, \eta \models \Phi$ exactly then if $(\eta(V_{j_1}), \dots, \eta(V_{j_n})) \in PFP
        (F_\Psi^{\mathcal{T},\eta})$. 
        By Definition~\ref{definition:pfp} this is the 
        case when there is an $m$ such that ${F_\Psi^{\mathcal{T},\eta}}^m(\varnothing) = {F_\Psi^		
       {\mathcal{T},\eta}}^{m+1}(\varnothing)$ and $(\eta(V_{j_1}), \dots, \eta(V_{j_n})) \in {F_\Psi^{\mathcal{T},\eta}}^m(\varnothing)$. By Definition~\ref{definition:induced_operator} $(\eta(V_{j_1}), \dots, 		
        \eta(V_{j_n})) \in {F_\Psi^{\mathcal{T},\eta}}^m(\varnothing)$ iff $\mathcal{T}, \eta \models \Psi({F_\Psi^
       {\mathcal{T},\eta}}^{m}(\varnothing), \eta(V_{j_1}), \dots, $ $\eta(V_{j_n}))$. By induction hypothesis 
        this is exactly the case when 
\begin{align*}
        (\emph{q}_s, &\emph{q}_s, \eta(X_1), \dots, \eta(X_{i_1-1}), \eta(X_{i_1}), \eta(X_{i_1+1}), \dots, \eta(X_{i_n-1}), \\&\eta(X_{i_n}), \eta(X_{i_n+1}), \dots, \eta
            (X_q), \emph{q}_r) \in \llbracket
        F(\Psi) \rrbracket^{\eta_V}_\mathcal{T}.
        \end{align*}       
        Note that $X$ is set to ${F_\Psi^{\mathcal{T},\eta}}^{m}(\varnothing)$.
		         
         If we use $\lambda$-approximation and $\beta$-reduction on PFP$^{(\odot, \dots, \odot)} X.\,F(\Psi)$ we can 
         see that it can be summarized to
         \begin{align*}
         \varphi \coloneqq \overset{m'}{\underset{i=0}{\bigvee}} \Big(&F(\Psi)^i \bot \wedge \forall_1 \dotsb \forall_n
         \big(F(\Psi)^i \bot \Leftrightarrow F(\Psi)^{i+1} \bot \big)\Big)
         \end{align*}
		Note that 
		\[{F_\Psi^{\mathcal{T},\eta}}^m(\varnothing) = F(\Psi)^{m'} \bot.\] 
		This holds because by induction over $i$ obviously it holds
		$\varnothing = \bot$
		and 
		$F_\Psi^{\mathcal{T},\eta}(\varnothing) = F(\Psi) \bot.$
		That means 
		${F_\Psi^{\mathcal{T},\eta}}^{i+1}(\varnothing) = F(\Psi^{i+1}) \bot $
		holds because by induction hypothesis it holds that
		${F_\Psi^{\mathcal{T},\eta}}^{i}
		(\varnothing) = F(\Psi^{i}) \bot$
		and 
		$F_\Psi^{\mathcal{T},\eta}({F_\Psi^{\mathcal{T},\eta}}^{i}(\varnothing))= F(\Psi)\big(F(\Psi)^{i} \bot \big).$
		
		If it exists an $i$ such that ${F_\Psi^{\mathcal{T},\eta}}^i(\varnothing) = {F_\Psi^		
        {\mathcal{T},\eta}}^{i+1}(\varnothing)$ this is exactly the case when the right conjunct of $\varphi$
		\begin{align*}
		\forall_1 \dotsb \forall_n \big(\Psi^i(\bot) \Leftrightarrow \Psi^{i+1}(\bot)\big)
         \end{align*}
		holds. The left conjunct of $\varphi$ returns then the set that we get through  
		the $i$-th application of $F(\Psi)$ on the empty set.
		
		Because of the construction of variable mapping $\eta_V$ and $(\eta(V_{j_1}), \dots, 
        \eta(V_{j_n}))$ $ \in \eta(X)$ the
        tuple $(\eta(V_{j_1}), \dots, \eta(V_{j_n}), q_{n+1}', \dots, q_s', \emph{q}_s, 
        \emph{q}_q,  \emph{q}_r)$ is also in $\eta_V(X)$.
         That means it holds $(\eta(V_{j_1}), \dots, \eta(V_{j_n})) \in {F_\Psi^{\mathcal{T},\eta}}^i(\varnothing)$ with
        \[F(\Phi) = \{(i_1, \dots, i_n)\} (PFP^{(\odot, \dots, \odot)} X.\, F(\Psi)) \]
        exactly then if
        \begin{align*}
        (\emph{q}_s, &\emph{q}_s, \eta(X_1), \dots, \eta(V_{j_1-1}), \eta(V_{j_1}), \eta(V_{j_1+1}), \dots \eta(V_{j_n-1}),\\& \eta(V_{j_n}), \eta(V_{j_n+1}), \dots, \eta
            (X_q), \emph{q}_r) \in \llbracket F(\Phi) \rrbracket_\mathcal{T}^{\eta_V}.
        \end{align*}
        
        In the case that $PFP(F_\Psi^{\mathcal{T},\eta})$ returns the empty set because there is no $m$ such that ${F_\Psi^{\mathcal{T},\eta}}^m(\varnothing) = {F_\Psi^		
       {\mathcal{T},\eta}}^{m+1}(\varnothing)$, the right conjunct of $\varphi$ is always false. Because of the least fixpoint operator in $PFP^{(\odot, \dots, \odot)} X.\, F(\Psi)$ the iteration is finite and it also will return the empty set.

        \item In case of $\Phi = [PFP\,\Psi(X, X_1, \dots, X_n)](V_1, \dots, V_n)$, where $X$ is a
        free variable in $\Psi$ of HO type $(\tau, \dots, \tau)$ and $X_1, \dots, X_n$ are free 
        variables of $\Psi$ of type $\tau$ and $V_1, \dots, V_n$ are free variables of $\Phi$ also of 
        type $\tau$, then it follows that $\mathcal{T}, \eta \models \Phi$ exactly then if $(\eta(V_1), 
        \dots, \eta(V_n)) \in PFP(F_\Psi^{\mathcal{T},\eta})$. By Definition~\ref{definition:pfp} this is exactly the 
        case when there is an $m$ such that ${F_\Psi^{\mathcal{T},\eta}}^m(\varnothing) = {F_\Psi^		
        \mathcal{T}}^{m+1}(\varnothing)$ and $(\eta(V_1), \dots, \eta(V_n)) \in {F_\Psi^{\mathcal{T},\eta}}
        ^m(\varnothing)$. By Definition~\ref{definition:induced_operator} $(\eta(V_1), \dots, 		
        \eta(V_n)) \in {F_\Psi^{\mathcal{T},\eta}}^m(\varnothing)$ iff $\mathcal{T}, \eta \models \Psi({F_\Psi^
        \mathcal{T}}^{m}(\varnothing), \eta(V_1), \dots, $ $\eta(V_n))$. By induction hypothesis 
        this is exactly the case when $\emph{q} \in \llbracket F(\Psi) \rrbracket_\mathcal{T}
        ^{\eta_V}$. Note that $X$ is set to ${F_\Psi^{\mathcal{T},\eta}}^{m}(\varnothing)$.
		         
         If we use $\lambda$-approximation and $\beta$-reduction on PFP$^\tau X.\,F(\Psi)$ we can 
         see that it can be summarized to
         \begin{align*}
         \varphi \coloneqq \overset{m'}{\underset{i= 0}{\bigvee}} \Big(&F(\Psi)^i \bot_{T((\tau, \dots, \tau))} \wedge \forall^{\tau'} X_1.\, \dotsb \forall^{\tau'} X_n.\,\\& 
         \big((\dotsb(F(\Psi)^i \bot_{T((\tau, \dots, \tau))} X_1)\dotsb)X_n \Leftrightarrow \\&\;(\dotsb(F(\Psi)^{i+1}(\bot_{T((\tau, \dots, \tau))}) X_1)\dotsb)X_n\big)\Big)
         \end{align*}
		Note that 
		\[{F_\Psi^{\mathcal{T},\eta}}^m(\varnothing) = F(\Psi)^{m'} \bot_{T((\tau, \dots, \tau))}.\] 
		This holds because by induction over $i$ obviously it holds
		\[\varnothing = \bot_{T((\tau, \dots, \tau))}\] 
		and 
		\[F_\Psi^{\mathcal{T},\eta}(\varnothing) = F(\Psi)\bot_{T((\tau, \dots, \tau))}.\] 
		That means 
		\[{F_\Psi^{\mathcal{T},\eta}}^{i+1}(\varnothing) = F(\Psi)^{i+1}\bot_{T((\tau, \dots, \tau))}\] 
		holds because by induction hypothesis it holds 
		\[{F_\Psi^{\mathcal{T},\eta}}^{i}
		(\varnothing) = F(\Psi)^{i}\bot_{T((\tau, \dots, \tau))}\] 
		and 
		\[F_\Psi^{\mathcal{T},\eta}({F_\Psi^{\mathcal{T},\eta}}^{i}(\varnothing))= F(\Psi)\big(F(\Psi)^{i}\bot_{T((\tau, \dots, \tau))})\big.\]
		If it exists an $i$ such that ${F_\Psi^{\mathcal{T},\eta}}^i(\varnothing) = {F_\Psi^		
        \mathcal{T}}^{i+1}(\varnothing)$ this is exactly the case when the right conjunct of $\varphi$
		\begin{align*}
		&\forall^{\tau'} X_1.\, \dotsb \forall^{\tau'} X_n.\,\\& 
         \big((\dotsb(\Psi^i(\bot_{T((\tau, \dots, \tau))}) X_1)\dotsb)X_n \Leftrightarrow \\&\;(\dotsb(\Psi^{i+1}(\bot_{T((\tau, \dots, \tau))}) X_1)\dotsb)X_n\big)
         \end{align*}
		holds. The left conjunct of $\varphi$ returns then the function that we get through  
		the $i$-th application of $\Psi$ on $\bot_{T((\tau, \dots, \tau))}$.
		
		Because of the construction of variable mapping $\eta_V$ and $(\eta(V_1), \dots, \eta(V_n)) 
		\in \eta(X)$ it holds $(\dotsb\big(\eta_V(X)\,\eta_V(V_1)\big) \dotsb )\,\eta_V(V_n) = Q^d$
        and so 
         \[\emph{q} \in (\dotsb\big(\eta_V(X)\,\eta_V(V_1)\big) \dotsb )\,\eta_V(V_n).\]
         That means it holds $(\eta(V_1), \dots, \eta(V_n)) \in {F_\Psi^{\mathcal{T},\eta}}^i(\varnothing)$ with 
        \[F(\Phi) = (\dotsb((PFP^T((\tau, \dots, \tau)) X.\, F(\Phi))\, V_1)\dotsb )\, V_n \]
        exactly then if \[\emph{q} \in \llbracket F(\Phi) \rrbracket_\mathcal{T}^{\eta_V}.\]
        
        In the case that $PFP(F_\Psi^{\mathcal{T},\eta})$ returns the empty set because there is no $m$ such that ${F_\Psi^{\mathcal{T},\eta}}^m(\varnothing) = {F_\Psi^		
       {\mathcal{T},\eta}}^{m+1}(\varnothing)$, the right conjunct of $\varphi$ is always false. Because of the least fixpoint operator in $PFP^{T((\tau, \dots, \tau))} X.\, F(\Psi)$ the iteration is finite and it  will return $\bot_{T((\tau, \dots, \tau))}$.
    \end{compactitem}
\end{proof}

The combination of Lemma~\ref{lemma:ho_pfp_equals_phfl_tail}, Lemma~\ref{lemma:expspace_in_ho_pfp} and 
Theorem~\ref{theorem:phfl_k_plus_1_tail_in_k_expspace} proves the following theorem for $k>0$. For $k = 0$ and $k = 1$ this statement was proven by M. Otto in~\cite{otto1999bisimulation} and by M. Lange and E. Lozes in~\cite{lange2014capturing}.

\begin{theorem}
    Let $k \geq 0$. PHFL$^{k+1}_{tail}$ captures \expspace{$k$} over labelled transition systems.
\end{theorem}
