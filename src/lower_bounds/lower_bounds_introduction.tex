%%
%% Author: DKron
%% 24.07.2018
%%

\chapter{Lower Bounds}\label{ch:lowerBounds}

In this chapter we want to establish that the lower bounds of the expressive power of PHFL$^k$ and PHFL$^k_{tail}$ are \exptime{$k$} and \expspace{$k$}, respectively. The
lower bounds of PHFL$^k$ and PHFL$^k_{tail}$ can be proven by encoding the run of a Turing Machine as query.
As another possibility one can use intermediate logics HO(LFP)$^{k+1}$ and HO(PFP)$^{k+1}$ and encode the bisimulation-invariant fragments of these as PHFL$^k$ and
PHFL$^k_{tail}$, respectively. Note that PHFL cannot distinguish between bisimilar structures~\cite{viswanathan2004higher}. This means that PHFL formulas can only define bisimulation-invariant
graph problems. Moreover, PHFL is sufficient to encode the bisimulation-invariant fragments of HO(LFP)$^{k+1}$ and HO(PFP)$^{k+1}$. To encode the bisimulation-invariant fragments of HO(LF\-P)$^{k+1}$ and HO(PFP)$^{k+1}$ as PHFL$^k$ and
PHFL$^{k}_{tail}$ respectively we want to translate HO(LFP)$^{k+1}$ formulas into PHFL$^k$ formulas and 
HO(PFP)$^{k+1}$ into a PHFL$^k_{tail}$ formula. 

Encoding existential quantifiers is the most complex part of the translation. In the 
first section we consider some preparations that are necessary to model them in PHFL. Thereafter,  we show that existential 
quantifiers that bind a variable of order $k \geq 1$ can be expressed by a PHFL$^{k-1}$ formula. In the subsequent section we 
use this formula to show that the bisimulation invariant fragment of HO(LFP)$^{k+1}$ can be encoded into PHFL$^k$ and so 
that the lower bound of the expressive power of PHFL$^k$ is \exptime{$k$}. And finally we show that the lower bound 
of the expressive power of PHFL$^k_{tail}$ is \expspace{$k$}.

\section{Preparation}\label{sec:lower_bounds_preparation}

Before we can start with the encodings there are some important steps that we have to consider.  Let $\mathcal{T} = (Q, 
\Sigma, P, \Delta, v)$ be an LTS, $q_1, \dots, q_n, p_1, \dots, p_m \in Q$ some states of $\mathcal{T}$ and let $\mathcal{T}'$ be the 
reduced LTS of $\mathcal{T}$ with respect to $p_1, \dots, p_m$.  
As mentioned in the introduction of this chapter it is known from~\cite{lange2014capturing} that PHFL cannot distinguish 
between bisimilar structures. That means PHFL formulas can only define bisimulation-invariant graph problems. That means that, without loss of generality, we can check if $([q_1]_\sim, \dots, [q_n]_\sim)$ satisfies a formula $\Phi$ with respect to $\mathcal{T}'$ instead of checking if $(q_1, \dots, q_n)$ satisfies 
$\Phi$ with respect to $\mathcal{T}$. With $[q_i]_\sim$ we denote the equivalence class of $q_i$ with respect to $\sim$.

From this point all LTS are reduced LTS with respect to some states of their state sets. In more detail, let $\mathcal{T} = (Q, \Sigma, P, \Delta, v)$ be a reduced LTS with respect to $(q_1, \dots, q_r)$, where $q_1, \dots, q_r \in Q$. On those LTS it is possible to define a total order on their states. 

\begin{remark} \label{remark:transitive_relation}
    In~\cite{otto1999bisimulation} it was shown that it is possible to define a $2$-adic PHFL$^0$ formula $\Phi_<$ over reduced LTS that defines a
    transitive relation $<$ such that $< \cap > = \emptyset$ and $< \cup > = \not\sim$. This relation $<$ defines a total
    order on states of a reduced LTS.
\end{remark}


%%
%% Author: DKron
%% 17.08.2018
%%

\section{Existential Quantifiers in PHFL}\label{sec:existential_quantifiers_in_phfl}

In this section we define PHFL$^{k}$ formulas that describes existential quantification over HO variables of order $k
\geq 1$. But before we can define these formulas we have to translate the types.

The types of HO have to be translated because the most types in HO$^{k + 1}$ do not exist in PHFL$^k$. While HO$^{k +
1}$ includes variables for example of kind set of sets, PHFL$^k$ does not support this kind of type.
But each set $A$ in HO$^{k+1}$ can be described by the characteristic function of $A$ in PHFL$^k$.

The following definition translates all HO types with order $2$ or greater to types in PHFL. The base type of HO
has to be encode differently, and will be regarded after this definition.

\begin{definition}
    \label{definition:lower_bound_type_function}
    Let $T$ be a function that maps a type of HO of order $2$ or greater to a type of PHFL defined inductive over the
    type of HO as follows:

    \begin{align*}
        T((\odot, \dots, \odot)) &= \bullet\\
        T((\tau', \dots, \tau')) &= T(\tau')^+ \rightarrow (T(\tau')^+ \rightarrow \dots \rightarrow (T(\tau')^+
        \rightarrow \bullet) \dots )
    \end{align*}
\end{definition}

Note that $ord(T(\tau)) = order(\tau) - 2$ for all $\tau$ with order $2$ or greater.

\begin{example}
    Let $\tau$ be a type of HO
    \[\tau = (\tau', \tau')\]
    with
    \[\tau' = (((\odot)), ((\odot)))\]
    then by Definition~\ref{definition:lower_bound_type_function} of type function $T$
    \[T(\tau) = T(\tau') \rightarrow (T(\tau') \rightarrow \bullet).\]
    with
    \[T(\tau') = (\bullet \rightarrow \bullet) \rightarrow ((\bullet \rightarrow \bullet) \rightarrow \bullet)\]
\end{example}

With this type function $T$ a HO$^{k + 1}$ variable $X$ of type $\tau$ can be translated to a PHFL$^k$ variable
with type $T(\tau)$. Please note that $\tau$ has order $2$ or greater. Intuitively the variable $X$ which represents a
set in HO$^{k+1}$ is represented in PHFL$^k$ as the characteristic function of $X$. Please note that $X$ for type
$\tau = (\odot, \dots, \odot)$ is also a set in PHFL$^k$ of PHFL type $\bullet$.

As mentioned above the base type of HO have to be encoded differently. The reason is that the base type in PHFL is at
least a set of tuples of states. So single states can not be depict directly by a variable. In this thesis the
idea is to use the polyadic fragment of a PHFL$^k$ formula $\Phi$ to represent the different first-order variables of an
HO$^{k+1}$ formula $\Psi$. Each first-order variable of $\Psi$ represents one component in $\Phi$, that means each
variable
increases the dimension of $\Phi$. The assignment of a first-order variable $x_i$ in $\Psi$ is then the current state
of the $i$-th component in $\Phi$. This is mainly from~\cite{lange2014capturing}.

Without loss of generality the first-order variables are enumerated as $x_1, \dots, x_f, x_{f + 1}, \dots, x_q$,
where $x_1, \dots, x_f$ are the free and $x_{f+1}, \dots, x_q$ are the quantified variables.

\begin{remark}
    The PHFL formula $\Phi$ that we get through the encoding of a given HO formula $\Psi$ has dimension
    $d$ that is always big enough to translate all second-order variables of $\Psi$ to an order $0$ variable in
    $\Phi$. In more detail $s$ is the maximal arity of second-order variables in $\Psi$ and $d > s$. To distiguish all
    different first-order variables in $\Psi$, the dimension $d$ of $\Phi$ is also bigger then $q$.  Finally, to compare
    two
    elements of $Q^{m}$, where $Q$ is the state set of an LTS and $m = max({s, q})$, the dimension $d$ of $\Phi$ is
    twice as big as the maximum of $s$ and $q$. That means $d = 2 * m$.
\end{remark}

After we know how to interpret the different HO types and variables we are now able to consider the existential
quantification. Before we regard higher-order quantification we look at first-order quantification.

Because we encode the bisimulation-invariant fragment of HO(LFP)$^{k + 1}$, the first-order quantification can be
encoded by moving through all states reachable by one of the free variables and check if we reach a state tuple where
the bounded formula holds. If we regard $\exists (x_i \colon \odot).\,\Phi$ then it can be understand as, check if we
reach a state tuple where $\Phi$ holds once the $i$-th component is replaced by one of the free variables. This can
be formalized in PHFL as
\[\exists_i \Phi \coloneqq \bigvee_{j=1}^f \{(1, \dots, i-1, j, i + 1, \dots, d)\} \mu (X
\colon \bullet).\,\Phi \vee \bigvee_{a \in \Sigma} \langle a \rangle_i X.\]
Now we regard the higher-order quantification. Let $\tau$ a HO type, $\sigma$ a signature and $\mathcal{U}$ the
universe of a $\sigma$ structure. Because the idea we use for first-order quantification is not obvious to adapt to
the higher-order quantification, we use another encoding. The idea to obtain higher-order quantification in PHFL is that
we have to iterate through all possible elements in a domain $D_\tau(\mathcal{U})$ and check if the given formula is
fulfilled. To make it possible to iterate through $D_\tau(\mathcal{U})$ we need a formula that returns us the
successor of a given element in $D_\tau(\mathcal{U})$. Eventually, to get the successor of a given element we need some
order on $D_\tau(\mathcal{U})$. So the first thing we need is the order of any domain.

\begin{remark}
    In~\cite{otto1999bisimulation} was shown that it is possible to define a $2$-adic formula $\Phi_<$ that defines a
    transitive relation $<$ such that $< \cap > = \emptyset$ and $< \cup > = \not\sim$. In this thesis $<$ is a total
    order on states of an LTS.
\end{remark}

To get an order of any type we define formulas that tells us given two elements of same type which one is the smaller
one. We say that a tuple $x$ is smaller than a tuple $y$ if there is an index $i$ such that the element in $x$ on $i$
is smaller then the element in $y$ on $i$ and there is no position $j$ left of $i$ such that the element in $x$ on
$j$ is bigger then the element in $y$ on $j$. We say that a set $x$ is smaller than a set $y$ if there is an element
$e$ in $x$ that is not in $y$ and all smaller elements $f$ to $e$ are only in $x$ if $f$ is also in $y$. This is
formalized in PHFL in the following definition.

\begin{definition}
    \label{definition:lower_bound_less}
    The PHFL$^k$ formula $<^\tau$ where $k = ord(T(\tau))$ is defined inductive over the type $\tau$ as follows:

    \begin{align*}
        <^\odot \coloneqq &\,\Phi_< \\
        <^{\odot \times \dots \times \odot} \coloneqq &\,\underset{i = 1}{\overset{m}{\bigvee}}\{(i, i + m, 3, \dots,
        d)\} <^\odot \wedge \\
        &\,\underset{j = 1}{\overset{i - 1}{\bigwedge}}\{(j + m, j, 3, \dots, d)\} \neg
        <^\odot\\
        <^{\tau' \times \dots \times \tau'}(x_1, y_1, \dots, x_1, y_n) \coloneqq &\,\underset{i =
        1}{\overset{n}{\bigvee}}<^{\tau'}(x_i, y_i) \wedge \underset{j = 1}{\overset{i - 1}{\bigwedge}}
        \neg <^{\tau'}(y_j, x_j)\\
        <^{(\odot, \dots, \odot)}(X, Y) \coloneqq &\,\exists_{i_1}.\, \dots \exists_{i_n}. \{(i_1, \dots, i_n, n
        + 1, \dots, d)\}Y \wedge \\&\,\neg \{(i_1, \dots,
        i_n, n + 1, \dots, d)\} X\,\wedge\\&\, \forall_{j_1}. \,\dots \forall_{j_n}. \{(j_1, \dots, j_n, n+1,
        \dots, m, \\&\,i_1, \dots, i_n, m + n + 1, \dots, d)\}<^{\odot
        \times \dots \times \odot} \Rightarrow \\&\,(\{(j_1, \dots, j_n, n + 1, \dots, d)\}X
        \Rightarrow \\&\,\neg \{(j_1, \dots, j_n, n + 1, \dots, d)\} Y)
        \\
        <^{(\tau', \dots, \tau')}(X, Y) \coloneqq &\,\exists^{\tau'}x_1. \,\dots \exists^{\tau'}x_n.\,(\dots((Y
        (x_1))(x_2))
        \dots) (x_n)
         \\&\,\wedge \neg (\dots((X(x_1))(x_2)) \dots)(x_n)\,\wedge \\&\,\forall^{\tau'}y_1. \,\dots
        \forall^{\tau'}y_n.
        \,<^{\tau'
        \times \dots \times \tau'}
        (y_1, x_1, \dots, y_n, x_n) \\&\,\Rightarrow ((\dots((X(y_1))(y_2)) \dots)(y_n) \\&\,\Rightarrow (\dots((Y(y_1))
        (y_2)) \dots(y_n))
    \end{align*}
    where the formulas of kind $\exists^{\tau'} x.\,\phi$ are the in PHFL$^{k-1}$ defined higher-order quantification
    for HO
    type $\tau'$ and formulas of kind $\forall^{\tau'} x.\,\phi$ are the counterparts.
\end{definition}

After we have now orders of any HO type we can define formulas that returns the successor of the input element.

\begin{definition}
    The PHFL$^k$ formula $next^\tau$ where $k = ord(T(\tau))$ is defined by the type $\tau$ as follows:
    \begin{align*}
        next^{(\odot, \dots, \odot)} \coloneqq &\,\lambda (X \colon \bullet).\, (\neg X \wedge \forall_{m +
        1}\dots\forall_{m + m}<^{\odot \times \dots \times \odot}\, \Rightarrow \\&\,\{(m +
        1, \dots, m + m, m + 1, \dots, d)\} X) \,\vee \\&\,(X \wedge \exists_{m + 1}\dots\exists_{m + m} <^{\odot
        \times \dots \times \odot} \,\wedge \\&\,\{(m + 1, \dots, m + m, m + 1, \dots, d)\}
        \neg X)\\
        next^{(\tau', \dots, \tau')} \coloneqq &\,\lambda (X \colon T ((\tau', \dots, \tau'))).\,(\neg (\dots((X
        (x_1))(x_2))\dots) (x_n) \\&\, \wedge \forall^{\tau'}y_1.\, \dots \forall^{\tau'}y_n.\,<^{\tau' \times
        \dots \times \tau'}(y_1, x_1, \dots, y_n, x_n) \\&\,\Rightarrow  (\dots((X(y_1))(y_2))\dots)(y_n)) \,\vee
        \\&\,((\dots ((X(x_1))(x_2)) \dots)(x_n) \wedge \exists^{\tau'}y_1.\, \dots \exists^{\tau'}y_n.\, \\&\,
        <^{\tau' \times \dots \times \tau'}
        (y_1, x_1,
        \dots, y_n, x_n)\,\wedge \neg (\dots((X(y_1))(y_2))\dots)(y_n))
    \end{align*}
\end{definition}

The idea of the two formulas are based on binary incrementation. Let $\tau = (\tau', \dots, \tau')$ an HO type and
$\mathcal{U}$ the universe of a $\sigma$-structure. Remember that a set $X \in D_\tau(\mathcal{U})$ can be
represented by its characteristic function. This can be transformed to a binary string where each position of
this string represents an element of $D_{\tau'}(\mathcal{U})^n$. Because each position in the binary string represents
an element of $D_{\tau'}(\mathcal{U})^n$ and a position have always to represent the same element in $D_{\tau'}
(\mathcal{U})^n$, the elements in $D_{\tau'}(\mathcal{U})^n$ have to be ordered. The order of the elements of
$D_{\tau'}(\mathcal{U})^n$ is given by the formula $<^{\tau' \times \dots \times \tau'}$ of
Definition~\ref{definition:lower_bound_less}. If the position $i$ in the binary string is $1$ then this means that
the element with index $i$ in $D_{\tau'}(\mathcal{U})^n$ is also in $X$ and if the position $i$ in the binary string
is $0$ then the element with index $i$ in $D_{\tau'}(\mathcal{U})^n$ is not in $X$. This binary representation of $X$
in regard to $D_{\tau'}(\mathcal{U})^n$ can be extended to a function $f\colon D_\tau(\mathcal{U}) \rightarrow {0,
\dots, |D_\tau(\mathcal{U})| - 1}$ such that each element $X$ of $D_\tau(\mathcal{U})$ will be mapped to its binary
string in regard to $D_{\tau'}(\mathcal{U})^n$. With this knowledge we can say that $Y \in D_\tau(\mathcal{U})$ is the
direct successor of $X \in D_\tau(\mathcal{U})$ if $f(Y) = (f(X) + 1)$ modulo $|D_\tau(\mathcal{U})|$. In detail that
means that the $i$-th bit is $1$ in $f(Y)$ if it is either $0$ in $X$ and all lower bits are $1$ in $X$ or it is $1$
in $X$ and there is a bit lower then $i$ that is $0$ in $X$.

With the previous definitions we are now able to define the higher-order quantification in PHFL.
\begin{definition}
    \label{definition:existential_quantification}
    Let $\tau = (\tau', \dots, \tau')$ be an HO type where $\tau' \neq \odot$ then $\exists^{\tau}X .\,\Phi(X)$ is
    defined as follows:
    \begin{align*}
        \exists^{\tau}X.\, \Phi(X) \coloneqq &\,(\mu (F \colon T(\tau) \rightarrow \bullet).\, \lambda (X \colon T(\tau)
        ).\,
        \Phi(X)
        \vee F(next^\tau X))\\&\,(\lambda (x_1 \colon \tau').\, \dots \lambda (x_n \colon \tau').\,\bot)
    \end{align*}
    In case of $\tau = (\odot, \dots, \odot)$, $\exists^{\tau}X .\,\Phi(X)$ is defined as
    \[        \exists^{\tau}X.\, \Phi(X) \coloneqq (\mu (F \colon \bullet \rightarrow \bullet).\, \lambda (X
    \colon \bullet).\, \Phi(X) \vee F(next^{\tau} X)) \bot
    \]
\end{definition}

The last step is to show that the given formula of Definition~\ref{definition:existential_quantification} defines
existential quantification.

\begin{lemma}
    \label{lemma:existential_quantifier}
    For all HO types $\tau$ of order $2$ or greater, all variable mappings $\eta$ and all LTS $\mathcal{T}$ holds
    \[\llbracket \exists^\tau X.\,\Phi(X)\rrbracket^\eta_\mathcal{T} \equiv \underset{\mathcal{X} \in \llbracket \tau
    \rrbracket_\mathcal{T}}{\bigsqcup} \llbracket \Phi(X) \rrbracket^{\eta[X\rightarrow \mathcal{X}]}_\mathcal{T}.\]
\end{lemma}

\begin{proof}
    TODO
\end{proof}

%%
%% Author: DKron
%% 17.08.2018
%%

\section{Lower Bound of PHFL$^k$}\label{sec:lowerBoundOfPhfl}

As mentioned in the introduction of this chapter we can show that the lower bound of PHFL$^k$ is \exptime{$k$} by
make a detour over HO(LFP)$^{k+1}$. This and following ideas are oriented on~\cite{lange2014capturing} where it was
shown that PHFL$^1$ captures \exptime{$1$}. At first we will see that it was proven that HO(LFP)$^{k +
1}$ coincides with $k$-fold exponential time over finite and ordered structures. To use this we have to encode the
bisimulation invariant fragment of HO$^{k+1}$ into PHFL$^k$. Therefore, we define an abbreviation for the HO(LFP)
formulas that uses the LFP operator first and then a function that uses this abbreviation and those of
Section~\ref{sec:existential_quantifiers_in_phfl} to map an HO(LFP)$^{k+1}$ formula to a PHFL$^k$ formula. Finally, we
show that for any bisimulation-invariant HO(LFP)$^{k+1}$ formula $\Phi$ there is a transformed PHFL$^k$ formula $\Psi$ such that the query $\mathcal{Q}^d_\Psi$ defined by $\Psi$ can be obtained from the query $\mathcal{Q}^f_\Phi$ defined by $\Phi$ via projection to the relevant components. Hence, PHFL$^k$ can define all queries defined by the bisimulation-invariant fragment of HO(LFP)$^{k+1}$.

\begin{theorem}{\cite{freireMartins2011descriptive}}\label{theorem:hoLfpEqualsExptime}
    For all $k \geq 1$, HO(LFP)$^{k + 1}$ captures $k$-EXPTIME over finite and ordered structures.
\end{theorem}

The proof follows the idea to encode the run of a $k$-EXPTIME Turing Machine $M$ by a formula $\phi$ of HO(LFP)$^{k +
1}$ in such a way that $M$ accepts the standard coding of $\mathcal{A}$ iff $\mathcal{A} \models \phi$. On the other hand each HO(LFP)$^{k +
1}$ formula $\phi$ can be evaluated by a $k$-EXPTIME Turing Machine $M_\phi$.

Because Theorem~\ref{theorem:hoLfpEqualsExptime} holds, it is also possible to prove that the lower bound of PHFL$^k$
is in \exptime{$k$} by encoding the bisimulation invariant fragment of HO(LFP)$^{k + 1}$ into PHFL$^k$. To encode the
bisimulation invariant fragment of HO(LFP)$^{k + 1}$ into PHFL$^k$ we have to define a function that transforms a HO
(LFP)$^{k + 1}$ formula into a PHFL$^k$ formula. Note that the types and variables of an HO formula also need to be
transformed. See Section~\ref{sec:existential_quantifiers_in_phfl} for further details.

Before we consider the definition of the transforming function, we define a PHFL formula for HO formulas that uses the
LFP operator.

\begin{definition}
\label{definition:lfp_in_phfl}
   Let $d$ be the constant as described in Remark~\ref{remark:dimension} and let $X$ be an HO variable of HO type $\tau = (\tau', \dots, \tau')$ where $\tau' \neq \odot$. Furthermore, let
    $\Phi$ be a PHFL$^k$ formula, then $LFP^\tau X.\,\Phi$ is a PHFL$^k$ formula with dimension $d$, defined as:
    \[LFP^\tau X.\,\Phi \coloneqq \mu (X \colon T(\tau)).\,\Phi(X).\]
    In case of $\tau = (\odot, \dots, \odot)$ let
    \[LFP^{\tau} X.\,\Phi \coloneqq \mu (X \colon \bullet).\,\Phi(X).\]
\end{definition}

Now we are able to define the function in the following definition by using the abbreviations of Definitions~\ref{definition:existential_quantification_second},~\ref{definition:existential_quantification_higher},~\ref{definition:existential_quantification_first} and~\ref{definition:lfp_in_phfl}. The function translates a bisimulation invariant HO(LFP)$^{k+1}$ formula to a PHFL$^k$ formula.

\begin{definition}
    \label{definition:lower_bounds_phfl_formula_function}
   Define $F$ as the function that maps a bisimulation invariant HO(LFP)$^{k+1}$ formula $\varphi$ to a PHFL$^k$ formula with dimension $d$, where $d$ and $s$ are the constants as described in Remark~\ref{remark:dimension} and $\Phi_\sim$ is the formula of Example~\ref{example:phfl_order_0}, then $F$ is defined
    inductive over $\varphi$ as follows:
    \begin{align*}
        F(p(x_i)) \coloneqq &\, p_{2s+i} \\
        F(a(x_i, x_j)) \coloneqq &\, \langle a \rangle_{2s+i} \{(2s+i, 2s+j, \\
        &\,3, \dots, d)\} \Phi_\sim \\
        F(\Phi \vee \Psi) \coloneqq &\, F(\Phi) \vee F(\Psi) \\
        F(\neg \Phi) \coloneqq &\, \neg F(\Phi) \\
        F(\exists (x_i \colon \odot).\,\Phi) \coloneqq &\, \exists_{2s+i} F(\Phi) \\
        F(\exists (X \colon \tau).\,\Phi) \coloneqq &\, \exists^\tau X.\,F(\Phi(X)) \\
        F([LFP\;\Phi(X, x_{i_1}, \dots, x_{i_n})](v_{j_1}, \dots, v_{j_n})) \coloneqq &\,\{(j_1, \dots, j_n, n + 1, \dots, d)\} \\
        &\,LFP^{(\odot, \dots, \odot)} X.\, F(\Phi) \\
        F([LFP\;\Phi(X, X_1, \dots, X_n)](V_1, \dots, V_n)) \coloneqq &\,(\dotsb \big(LFP^\tau X.\, F(\Phi)\big)\,V_1)\dotsb)\,V_n \\
        F(X(x_{i_1}, \dots, x_{i_n})) \coloneqq &\, \{(2s+i_1, \dots, 2s+i_n, \\
        &\,n + 1, \dots, d)\}X\\
        F(X(X_1, \dots, X_n)) \coloneqq &\, (\dotsb (X\,X_1)\dotsb)\,X_n
    \end{align*}
\end{definition}
Keep in mind that we are working on LTS, that means that the relations in the signatures for HO(LFP) formulas have either arity one or two. Relations with arity one represent the 
propositions and those with arity two the actions of an LTS.

\subsection{Variables}\label{subsec:lower_bounds_variables}

After we encoded HO(LFP)$^{k+1}$ syntactically in PHFL$^k$ formulas, the last step is to translate the 
interpretation of variables. As described in Section~\ref{sec:existential_quantifiers_in_phfl} the variables in HO of 
types with order $3$ or higher are not supported in PHFL. Also first-order variables are not supported. Therefore, we 
define a function that maps a given variable mapping for a HO formula to the correct variable mapping for PHFL 
semantics. This function ignores the mapping of first-order variables, maps second-order variables to sets and 
higher-order variables to the corresponding characteristic function. Note, that the sets of order $2$ in HO have order 
$0$ in PHFL. The first-order variables of an HO formula that are marked as undefined in the following function, can be 
mapped to any arbitrary value because we do not use them in PHFL directly.

    Note that HO variables of order $2$ or higher are syntactically equivalent to the variables used in the translated PHFL formulas. To distinguish them in this definition an HO variable $X$ is denoted by $\hat{X}$ for the usage in PHFL context. 

\begin{definition}
    \label{definition:lower_bound_variable_function}
    Let $d$ be the constant as described in Remark~\ref{remark:dimension}, let $\eta$ be a
    variable mapping for a HO$^k$ formula and let $\mathcal{T} = (Q, \Sigma, P, \Delta, v)$ be a reduced LTS, then $\eta_V$ is a 
    variable mapping for a PHFL$^k$ formula with dimension $d$. 
    
    Variable mapping $\eta_V$ is defined as:
    \[\eta_V(\hat{X})\coloneqq
    \begin{cases}
        \text{undefined}, & \text{if } X \text{ is of type } \odot \\
        A,  & \text{if } X \text{ is of type } (\odot, \dots, \odot)\\
        G, & \text{if } X \text{ is of type } (\tau, \dots, \tau) \text{ and } \tau \neq \odot,
    \end{cases}\]
    where $A \subseteq Q^d$ such that $(q_1, \dots, q_n, q_{n + 1}, \dots, q_d) \in A$ iff $(q_1, \dots, q_n) \in
    \eta(X)$ and $G$ is a function of type $T((\tau, \dots, \tau))$ defined as follows:
    \begin{align*}
        (\dotsb\big(G\,\eta_V(\hat{X_1})\big)\dotsb)\,\eta_V(\hat{X_n}) &= Q^d &\text{ iff } (\eta(X_1), \dots, \eta(X_n)) \in \eta(X)\\
        (\dotsb\big(G\,\eta_V(\hat{X_1})\big)\dotsb)\,\eta_V(\hat{X_n}) &= \emptyset &\text{ iff } (\eta(X_1), \dots, \eta(X_n))
        \not\in \eta(X)
    \end{align*}
\end{definition}

The following example shows how a set of higher type variables will be translated into the characteristic function via the
variable mapping $\eta_V$ of Definition~\ref{definition:lower_bound_variable_function}.

\begin{example}
    Let $\mathcal{A}$ be a $\sigma$-structure over universe $\mathcal{U} = \{1, 2, 3\}$, let $X$ be a HO(LFP)$^{k + 1}$
    variable of type $((\odot, \odot), (\odot, \odot))$ mapped through variable mapping $\eta$ to
    \[\eta(X) = \{(\{(1, 1)\}, \{(2, 2)\}), (\{(1, 1), (2, 2)\}, \{(3, 3)\})\},\]
    let $\mathcal{T} = (\mathcal{U}, \Sigma, P, \Delta, v)$ be a reduced LTS and $d$ a dimension of PHFL, then $\eta_V(X)$ is a PHFL$^k$ function of type $\bullet \rightarrow (\bullet \rightarrow \bullet)$ such
    that $\eta_V(X)$ $(\{(1, 1)\}) = f$, $\eta_V(X)(\{(1, 1), (2, $ $2)\}) = g$ and $\eta_V(X)(z) = h$ for $z \in
    \mathcal{U}^d$ where $z \neq \{(1, 1)\}$ and $z \neq \{(1, 1), (2, 2)\}$. Moreover, $f$, $g$ and $h$ are functions of type $\bullet
    \rightarrow \bullet$ where $f(\{(2, 2)\}) = g(\{(3, 3)\}) = \mathcal{U}^d$ and $f(z) = g(z') = h(z'') = \emptyset$ for $z,
    z', z'' \in \mathcal{U}^d$ where $z \neq \{(2, 2)\}$ and $z' \neq \{(3, 3)\}$.
\end{example}

\subsection{Correctness Proof}\label{subsec:lower_bounds_correctness_lfp}

The last step is to show such that the query that can be defined by the PHFL$^k$ formula $\Psi$ that is obtained by transforming a HO(LFP)$^{k+1}$ formula $\Phi$ can be achieved from the query defined by $\Phi$ via projection to the relevant components. that for any query that can be associated to a HO(LFP)$^{k+1}$ formula $\Phi$ coincides with the query that can be associated to the PHFL$^k$ formula that is obtained by transforming $\Phi$. As mentioned in Section~\ref{sec:lower_bounds_preparation} without loss of generality the statement can be proven by consider only  reduced LTS. 

To make the correctness proof clearer there is one last remark.

\begin{remark}
    It holds for any HO(LFP)$^{k+1}$ formula $\Phi$ that for PHFL$^k$ formula $F(\Phi)$ and some type environment $\Gamma$ the type judgment $\Gamma \vdash
    \Phi \colon \bullet$ is derivable. This statement
    is easy proved by induction over the structure of formula $\Phi$. 
\end{remark}

Because the type judgement is always derivable we ignore the type environment in the following proof and write just $\llbracket \Phi \rrbracket^\eta_\mathcal{T}$ instead of $\llbracket \Gamma \vdash F(\Phi) \colon \tau \rrbracket^\eta_\mathcal{T}$, where $\Phi$ is a PHFL formula, $\eta$ is a variable mapping, $\mathcal{T}$ is an LTS, $\Gamma$ is a type environment and $\tau$ is a PHFL type.

\begin{theorem}
    \label{theorem:ho_lfp_equals_phfl}
    Let $d, f \geq 1$ and $k \geq 0$. For every bisimulation-invariant formula $\Phi$ of HO(LFP)$^{k + 1}$ there is a
    PHFL$^k$ formula $\Psi$ such that a projection on the $d$-adic query $\mathcal{Q}_\Phi^d$ that is defined by $\Phi$ is equal to the $f$-adic query $\mathcal{Q}_\Psi^f$ that is defined by $\Psi$.
\end{theorem}

\begin{proof}
    This lemma can be proven by showing for all HO(LFP)$^{k+1}$ formulas $\Phi$ with first-order variables $x_1,
    \dots, x_q$, all reduced LTS $\mathcal{T} = (Q, \Sigma, P,
    \Delta, v)$ with respect to $\emph{q}_r = q_1, \dots, q_r$ and all variable mappings $\eta$ that it holds that $\mathcal{T}, \eta \models \Phi$ iff $\emph{q} =
    (\emph{q}_s, \emph{q}_s, \emph{q}_q, \emph{q}_r)$ and $\emph{q} \in \llbracket
   F(\Phi)\rrbracket^{\eta_V}_{\mathcal{T}}$, where $\emph{q}_s = q_1',\dots,q_s'$ is a sequence of $s$ placeholders used for the interaction of second-order variables, $\emph{q}_q = \eta(x_1), \dots, \eta(x_q)$ is a sequence of first-order variables that are mapped by $\eta$ where $\eta(x_i) = q_0$ if $x_i$ is bound to a quantifier, $q_0, q_1', \dots, q_s' \in Q$ are arbitrary states, $F$ is the formula function of
    Definition~\ref{definition:lower_bounds_phfl_formula_function} and $\eta_V$ the variable mapping of
    Definition~\ref{definition:lower_bound_variable_function}. This statement can be proven by induction over formula
    $\Phi$.
    \begin{compactitem}
        \item In case of $\Phi = p(x_i)$ where $x_i$ is a free first-order variable then $\mathcal{T}, \eta \models \Phi$ holds exactly then if $\eta(x_i) \in
        p^\mathcal{T}$. Translated to the normal LTS definition of $\mathcal{T}$ it is the same as $p \in v(\eta(x_i))$.
        With $F(\Phi) = p_{2s+i}$ this is exactly
        \begin{align*}
            (\emph{q}_s, \emph{q}_s, \eta(x_1),& \dots, \eta(x_{i-1}), \eta(x_{i}), \eta(x_{i+1}), \dots, \eta(x_q), \emph{q}_r) \in
            \llbracket F(\Phi) \rrbracket^{\eta_V}_\mathcal{T}.
        \end{align*}
        
        \item In case of $\Phi = a(x_i, x_j)$ where $x_i$ and $x_j$ are free first-order variables then $\mathcal{T}, \eta \models \Phi$ holds exactly then if $(\eta(x_i)
        , \eta(x_j))$ $ \in a^\mathcal{T}$. Translated to the normal LTS definition of $\mathcal{T}$ it is the same as $
        \eta(x_i) \overset{a}{\rightarrow} \eta(x_j)$. Let $g: \{1, 2\} \rightarrow \{i, j\}$ be a function.
        By  definition of the semantics of $\langle a \rangle_{2s+i}$ the tuple
        \begin{align*}
            (\emph{q}_s, &\emph{q}_s, \eta(x_1), \dots, \eta(x_{g(1) - 1}), \eta(x_{g(1)}), \eta(x_{g(1)+1}), \dots, \eta(x_{g(2)-1}), \eta
            (x_{g(2)}),\\& \eta(x_{g(2)+1}), \dots, \eta(x_q), \emph{q}_r)
        \end{align*}
        is an element of the semantics of $\langle a \rangle_{2s+i} \Phi_\sim$ if
        \begin{align*}
            (\emph{q}_s, &\emph{q}_s, \eta(x_1), \dots, \eta(x_{g(1) - 1}), q_m, \eta(x_{g(1)+1}), \dots, \eta(x_{g(2)-1}), q_n,\\& \eta(x_{g(2)
            +1}), \dots, \eta(x_q), \emph{q}_r)
        \end{align*}
        is an element of the semantics of $\Phi_\sim$ where $\eta(x_i)$ can
        move by an $a$ action to $q_m$ if $g(1) = i$ or to $q_n$ if $g(2) = i$. Because $\eta(x_j)$ is the state that
        have to be reached via an $a$ action from $\eta(x_i)$ we have to check if $q_m = \eta(x_j)$ in case of $g(1)
        = j$ and in case of $g(2) = j$ we have to check if $q_n = \eta(x_i)$. If two states are equal in a reduced LTS is
        the same to check if those states are bisimilar. $\sim$ is given by formula $\Phi_\sim$ of
        Example~\ref{example:phfl_order_0}.
        Because this formula returns those $d$-tuples where the first and second component are bisimilar, we have to
        move the $2s+i$-th and $2s+j$-th component to the first and second component. This is given by $\{(2s+i, 2s+j, 3, \dots, d
        )\} \Phi_\sim$. Summarizing all these steps with $F(\Phi) = \langle a \rangle_{2s+i} \{(2s+i, 2s+j, 3, \dots, d)\}
        \Phi_\sim$ it follows
        \begin{align*}
            (\emph{q}_s, &\emph{q}_s, \eta(x_1), \dots, \eta(x_{g(1) - 1}), \eta(x_{g(1)}), \eta(x_{g(1)+1}), \dots, \eta(x_{g(2)-1}), \eta
            (x_{g(2)}),\\& \eta(x_{g(2)+1}), \dots, \eta(x_q), \emph{q}_r) \in \llbracket F(\Phi) \rrbracket^{\eta_V}_\mathcal{T}
        \end{align*}
        if $(\eta(x_i), \eta(x_j))$ $ \in a^\mathcal{T}$ and
        \begin{align*}
            (\emph{q}_s, &\emph{q}_s, \eta(x_1), \dots, \eta(x_{g(1) - 1}), \eta(x_{g(1)}), \eta(x_{g(1)+1}), \dots, \eta(x_{g(2)-1}), \eta
            (x_{g(2)}),\\& \eta(x_{g(2)+1}), \dots, \eta(x_q), \emph{q}_r) \not\in \llbracket F(\Phi) \rrbracket^{\eta_V}_\mathcal{T}
        \end{align*}
        if $(\eta(x_i), \eta(x_j))$ $ \not\in a^\mathcal{T}$.

        \item In case of $\Phi = X(x_{i_1}, \dots, x_{i_n})$ where $X$ is a free variable of HO type $(\odot, \dots,
        \odot)$ and $x_{i_1}, \dots, x_{i_n}$ are free first-order variables then $\mathcal{T}, \eta \models \Phi$
        holds exactly then if $(\eta(x_{i_1}), \dots \eta(x_{i_n})) \in \eta(X)$. Because of definition of $\eta_V$ the
        tuple $(\eta(x_{i_1}), \dots \eta(x_{i_n}), q_{n + 1}', \dots, q_{s}', \emph{q}_s, \emph{q}_q, \emph{q}_r)$ is in $
        \eta_V(X)$ if $(\eta(x_{i_1}),\dots \eta(x_{i_n})) \in \eta(X)$ and is not in $\eta_V(X)$ otherwise. Because
         components $1, \dots, n$ are not set to the mappings of first-order variables $x_{i_1}, \dots, x_{i_n}$, we first move the components $2s+i_1, \dots, 2s+i_n$ to components $1, \dots, n$ respectively and check then if
        \[(\eta(x_{i_1}), \dots \eta(x_{i_n}), q_{n + 1}', \dots, q_{s}', \emph{q}_s, \emph{q}_q, \emph{q}_r) \in \eta_V(X).\]
        So it holds with $F(\Phi) = {(2s+i_1, \dots, 2s+i_n, n+1, \dots, d)}X$ and function $g: \{1, \dots, n\}
        \rightarrow \{i_1, \dots, i_n\}$ that
        \begin{align*}
            (\emph{q}_s, &\emph{q}_s, \eta(x_1), \dots, \eta(x_{g(1)-1}), \eta(x_{g(1)}), \eta(x_{g(1)+1}), \dots \eta(x_{g(2)-1}), \eta
            (x_{g(2)}),\\& \eta(x_{g(2)+1}), \dots, \eta(x_{g(n)-1}), \eta(x_{g(n)}), \eta(x_{g(n)+1}), \dots, \eta
            (x_q),\\& \emph{q}_r) \in \llbracket F(\Phi) \rrbracket^{\eta_V
            }_\mathcal{T}
        \end{align*}
        if $(\eta(x_{i_1}), \dots \eta(x_{i_n})) \in \eta(X)$ and
        \begin{align*}
            (\emph{q}_s, &\emph{q}_s, \eta(x_1), \dots, \eta(x_{g(1)-1}), \eta(x_{g(1)}), \eta(x_{g(1)+1}), \dots \eta(x_{g(2)-1}), \eta
            (x_{g(2)}),\\& \eta(x_{g(2)+1}), \dots, \eta(x_{g(n)-1}), \eta(x_{g(n)}), \eta(x_{g(n)+1}), \dots, \eta
            (x_q),\\& \emph{q}_r) \not\in \llbracket F(\Phi) \rrbracket^{\eta_V
            }_\mathcal{T}
        \end{align*}
        if $(\eta(x_{i_1}), \dots \eta(x_{i_n})) \not\in \eta(X)$.

        \item In case of $\Phi = X(X_1, \dots, X_n)$ where $X$ is a free variable of HO type $(\tau, \dots,
        \tau)$ and $X_1, \dots, X_n$ are free variables of HO type $\tau$ then $\mathcal{T}, \eta \models \Phi$
        holds exactly then if $(\eta(X_1), \dots \eta(X_n)) \in \eta(X)$. Because of definition of $\eta_V$ it follows
        \[(\dotsb\big(\eta_V(X)\,\eta_V(X_1)\big)\dotsb)\,\eta_V(X_n) = Q^d\]
        if $(\eta(X_1), \dots \eta(X_n)) \in \eta(X)$ and
        \[(\dotsb\big(\eta_V(X)\,\eta_V(X_1)\big)\dotsb)\,\eta_V(X_n) = \emptyset\]
        if $(\eta(X_1), \dots \eta(X_n)) \not\in \eta(X)$. With $F(\Phi) = (\dotsb (X\,X_1)\dotsb)\,X_n$
        it follows
        \[ \emph{q} \in \llbracket F(\Phi) \rrbracket^{\eta_V}_\mathcal{T} = Q^d.\]
        if $(\eta(X_1), \dots \eta(X_n))\in \eta(X)$ and
        \[ \emph{q} \not\in \llbracket F(\Phi)
        \rrbracket^{\eta_V}_\mathcal{T} = \emptyset.\]
        if $(\eta(X_1), \dots \eta(X_n)) \not\in \eta(X)$.
    \end{compactitem}
    
    By induction hypothesis it holds for HO(LFP)$^{k+1}$ formulas $\Psi$ and $\Psi'$ with first-order variables $x_1,
    \dots, x_q$, all reduced LTS $\mathcal{T} = (Q, \Sigma, P,
    \Delta, v)$ with respect to $\emph{q}_r$ and all variable mappings $\eta$  that $\mathcal{T}, \eta \models \Psi$ iff $\emph{q} \in \llbracket
   F(\Psi)\rrbracket^{\eta_V}_{\mathcal{T}}$ and $\mathcal{T}, \eta \models \Psi'$ iff $\emph{q} \in \llbracket
   F(\Psi')\rrbracket^{\eta_V}_{\mathcal{T}}$, where $\emph{q} =
    (\emph{q}_s, \emph{q}_s, \emph{q}_q, \emph{q}_r)$.
    
    \begin{compactitem}
        \item In case of $\Phi = \neg \Psi$ it follows that $\mathcal{T}, \eta \models \Phi$ exactly then if
        $\mathcal{T}, \eta \not\models \Psi$. By induction hypothesis that is exactly then the case when
        \[ \emph{q} \not\in \llbracket F(\Psi) \rrbracket^{\eta_V}_\mathcal{T}.\]
        This is exactly the case if
        \[ \emph{q} \in Q^d \setminus \llbracket F(\Psi) \rrbracket^{\eta_V}_\mathcal{T}.\]
        And this is exactly the semantics of $F(\Phi) = \neg F(\Psi)$.

        \item In case of $\Phi = \Psi \vee \Psi'$ it follows that $\mathcal{T}, \eta \models \Phi$ exactly then if
        $\mathcal{T}, \eta \models \Psi$ or $\mathcal{T}, \eta \models \Psi'$. By induction hypothesis that is
        exactly then the case when
        \[\emph{q} \in \llbracket F
        (\Psi) \rrbracket^{\eta_V}_\mathcal{T}\]
        or
        \[\emph{q} \in \llbracket F
        (\Psi') \rrbracket^{\eta_V}_\mathcal{T}.\]
        Because $\sqcup_{\bullet} = \cup$ this can be combined to
        \[\emph{q} \in \llbracket F
        (\Psi) \rrbracket^{\eta_V}_\mathcal{T} \sqcup_\bullet \llbracket F
        (\Psi') \rrbracket^{\eta_V}_\mathcal{T},\]
        which is as desired.

        \item In case of $\Phi = \exists (x_i\colon \odot).\,\Psi$ it follows that $\mathcal{T}, \eta \models \Phi$ iff
        there exists  $\mathcal{X} \in Q$ with $\mathcal{T}, \eta' \models \Psi$, where $\eta'$ is a variable mapping with $\eta'(x) = \eta(x)$ for all variables $x \neq x_i$ and $\eta'(x_i) = \mathcal{X}$. By induction hypothesis it holds that $\mathcal{T}, \eta' \models \Psi$ is exactly the case when
        \begin{align*}
            (\emph{q}_s, \emph{q}_s, \eta'(x_1), \dots, \eta'(x_{i-1}), \eta'(x_i), \eta'(x_{i+1}), \dots, \eta(x_{q}), \emph{q}_r) \in
            \llbracket F(\Psi) \rrbracket^{{\eta'_V}}_\mathcal{T}.
        \end{align*}
        To reach the value of $\eta'(x_i)$ we have to replace the $2s+i$-th component by one of 
        the last $r$ components and move through all reachable states. By Observation~\ref{observation:existential_quantification_first} the formula defined in Definition~\ref{definition:existential_quantification_first} fulfils this behaviour. Because the first-order variable $x_i$ is represented by the $2s+i$-th component and $F(\Phi) = \exists_{2s+i} \Psi$, we replace in $F(\Phi)$ the $2s+i$-th component by one of the last $r$ components, move through all reachable states and checking if $F(\Psi)$ holds. That means it holds 
        \[\emph{q} \in \llbracket F
        (\Phi) \rrbracket^{\eta_V}_\mathcal{T}\]
        iff $\mathcal{T}, \eta \models \Phi$.

        \item In case of $\Phi = \exists (X \colon \tau).\,\Psi$ it follows that $\mathcal{T}, \eta \models \Phi$ iff
        there exists $\mathcal{X} \in D_\tau(Q)$ with $\mathcal{T}, \eta' \models \Psi$, where $\eta'$ is a variable mapping with $\eta'(x) = \eta(x)$ for all variables $x \neq X$ and $\eta'(X) = \mathcal{X}$.
        By induction hypothesis it follows that $\emph{q} \in
        \llbracket F(\Psi) \rrbracket^{{\eta'_V}}_\mathcal{T}$ iff $\mathcal{T}, \eta' \models \Psi$. By Lemma~\ref{lemma:existential_quantifier_higher} the formula $\exists^\tau X.\, \Psi(X)$ is semantically equivalent to
        $\underset{\mathcal{X} \in \llbracket \tau \rrbracket_\mathcal{T}}{\bigsqcup} \llbracket \Psi(X) \rrbracket^{\eta'}_\mathcal{T}$.
        It follows with $F(\Phi) = \exists^\tau X.\, F(\Psi)(X)$ that it holds \[\emph{q} \in
        \llbracket F(\Phi) \rrbracket^{\eta_V}_\mathcal{T}.\]

        \item In case of $\Phi = [LFP\,\Psi(X, x_{i_1}, \dots, x_{i_n})](v_{j_1}, \dots, v_{j_n})$, where $X$ is a
        free variable in $\Psi$ of HO type $(\odot, \dots, \odot)$ and $x_{i_1}, \dots, x_{i_n}$ are free first-order
        variables of $\Psi$ and $v_{j_1}, \dots, v_{j_n}$ are first-order variables of $\Phi$, then it follows that
        $\mathcal{T}, \eta \models \Phi$ exactly then if $(\eta(v_{j_1}), \dots, \eta(v_{j_n})) \in LFP
        (F_\Psi^\mathcal{T})$. By definition of LFP the tuple $(\eta(v_{j_1}), \dots, \eta(v_{j_n}))$ is in
        $LFP(F_\Psi^\mathcal{T})$ iff $X$ is the smallest $X$ such that $X = F_\Psi^\mathcal{T}(X)$ and $(\eta(v_{j_1}), \dots, \eta(v_{j_n})) \in
        F_\Psi^\mathcal{T}(X)$. By definition of $F_\Psi^\mathcal{T}(X)$ it holds $(\eta
        (v_{j_1}), \dots, \eta(v_{j_n})) \in F_\Psi^\mathcal{T}(X)$ exactly then if $\mathcal{T}, \eta' 
        \models \Psi$, where $\eta'$ is a variable mapping with $\eta'(x) = \eta(x)$ for all variables $x \neq X$ and $\eta'(X) = F_\Psi^\mathcal{T}(X)$. By induction hypothesis this is exactly the case if
        \begin{align*}
        (\emph{q}_s, &\emph{q}_s, \eta(x_1), \dots, \eta(v_{g(1)-1}), \eta(v_{g(1)}), \eta(v_{g(1)+1}), \dots \eta(v_{g(2)-1}),\\& \eta
            (v_{g(2)}), \eta(v_{g(2)+1}), \dots, \eta(v_{g(n)-1}), \eta(v_{g(n)}), \eta(v_{g(n)+1}), \dots, \eta
            (x_q), \\& \emph{q}_r) \in \llbracket
        F(\Psi) \rrbracket^{\eta'_V}_\mathcal{T},
        \end{align*}
         where $g: \{1, \dots, n\} $ $\rightarrow \{j_1, \dots, j_n\}$ is a function.
        
        The next step is to show that $\eta'_V(X)$ is also a least fixpoint of $\llbracket
         \mu(X\colon \bullet).\,$ $F(\Psi) \rrbracket^{\eta'_V}_\mathcal{T}$. By Theorem~\ref{theorem:kleene} the least fixpoint $\eta'(X)$ can be calculated by a sequence $X_0, \dots, X_m
         $ where here $X_0 = \emptyset = \eta^0(X)$ and $X_{i+1} = F_\Psi^\mathcal{T}(X_i) = \eta^{i+1}(X)$ and $\eta'(X) = X_m = \eta^m(X)$. On the other hand the least fixpoint $\eta'_V(X)$ can be calculated by a sequence 
         $Y_0, \dots, Y_{m'}$ where here $Y_0 = \emptyset = \eta^0_V(X)$ and $Y_{i+1} = \llbracket F(\Psi)\rrbracket_\mathcal{T}^{\eta'_V[X \mapsto Y_i]} = \eta^{i+1}_V(X)$ and $\eta'_V(X) = Y_{m'} = \eta^m_V(X)$. We now show by induction that $X_i$ corresponds with $Y_i$ in PHFL context for all $i$. Obviously $X_0 = \emptyset = Y_0$. By induction hypothesis it holds $X_i$ corresponds with $Y_i$ in PHFL context for 
         one $i$. Then $X_{i+1} = F_\Psi^\mathcal{T}(X_i)$ what is
         \[\{(a_1, \dots, a_n) \mid \mathcal{T}, \eta'' \models \Psi\},\] 
         where $\eta''$ is a variable mapping with $\eta''(x) = \eta(x)$ for all variables $x \neq X$ and $\eta''(x_{i_1}) = a_1, \dots,  \eta''(x_{i_n}) = a_n$ and $\eta''(X) = X_i$. 
         By using the induction hypothesis, $X_i$ corresponds with $Y_i$ in PHFL context, and the main induction hypothesis for the proof of this theorem, this is exactly the same to
                 \begin{align*}
                 \{(a_1, \dots, a_n) \mid 
        (\emph{q}_s, &\emph{q}_s, \eta''(x_1), \dots, \eta''(v_{g(1)-1}),\\& \eta''(v_{g(1)}), \eta''(v_{g(1)+1}), \dots \eta''(v_{g(2)-1}),\\& \eta''
            (v_{g(2)}), \eta''(v_{g(2)+1}), \dots, \eta''(v_{g(n)-1}), \\&\eta''(v_{g(n)}), \eta''(v_{g(n)+1}), \dots, \eta''
            (x_q), \\& \emph{q}_r) \in \llbracket
        F(\Psi) \rrbracket^{\eta''_V[X\mapsto Y_i]}_\mathcal{T}\}
        \end{align*}
		Because $\eta''_V(x) = \eta'_V(x)$ for all variable $x$ it follows $Y_{i+1} = \llbracket F(\Psi)\rrbracket_\mathcal{T}^{\eta''_V[X \mapsto Y_i]}$ that means        
        \begin{align*}
                 \{(a_1, \dots, a_n) \mid 
        (\emph{q}_s, &\emph{q}_s, \eta''(x_1), \dots, \eta''(v_{g(1)-1}),\\& \eta''(v_{g(1)}), \eta''(v_{g(1)+1}), \dots \eta''(v_{g(2)-1}),\\& \eta''
            (v_{g(2)}), \eta''(v_{g(2)+1}), \dots, \eta''(v_{g(n)-1}), \\&\eta'(v_{g(n)}), \eta''(v_{g(n)+1}), \dots, \eta''
            (x_q), \\& \emph{q}_r) \in Y_{i+1}\}.
        \end{align*}
         and it follows $X_{i+1}$ corresponds with $Y_{i+1}$ in PHFL context. The induction holds.
        
Because of the construction of variable mapping $\eta'_V$ and $(\eta'(v_{j_1}), \dots, 
        \eta'(v_{j_n})) \in \eta'(X)$ the
        tuple $(\eta'(v_{j_1}), \dots, \eta'(v_{j_n}), q_{n+1}', \dots, q_s', \emph{q}_s, \emph{q}_q, \emph{q}_r)$ is also in $\eta'_V(X)$.         
        
       Because components $1, \dots, n$ are not set to the mappings of first-order variables 
       $v_{j_1}, \dots, v_{j_n}$, we first move the components $2s+j_1, \dots, 2s+j_n$ to components $1, \dots, n$ respectively and check then the least fixpoint operator.
        So it holds with $F(\Phi) = \{2s+j_1, \dots, 2s+j_n, n+1, \dots, d\} \mu (X).\, F(\Psi)$ and
        function $g: \{1, \dots, n\} $ $\rightarrow \{j_1, \dots, j_n\}$ that
        \begin{align*}
            (\emph{q}_s, \emph{q}_s, \eta(x_1),& \dots, \eta(v_{g(1)-1}), \eta(v_{g(1)}), \eta(v_{g(1)+1}), \dots \eta(v_{g(2)-1}), \eta
            (v_{g(2)}),\\& \eta(v_{g(2)+1}), \dots, \eta(v_{g(n)-1}), \eta(v_{g(n)}), \eta(v_{g(n)+1}), \dots, \eta
            (x_q),\\& \emph{q}_r) \in \llbracket  F(\Phi) \rrbracket^{\eta_V
            }_\mathcal{T}
        \end{align*}
        exactly then if $\mathcal{T}, \eta \models \Phi$.

        \item In case of $\Phi = [LFP\,\Psi(X, X_1, \dots, X_n)](V_1, \dots, V_n)$, where $X$ is a
        free variable in $\Psi$ of HO type $(\tau, \dots, \tau)$ and $X_1, \dots, X_n$ are free
        variables of $\Psi$ of type $\tau$ and $V_1, \dots, V_n$ are free variables of $\Phi$ also of type $\tau$, then
        it follows that $\mathcal{T}, \eta \models \Phi$ exactly then if $(\eta(V_1), \dots, \eta(V_n) \in LFP
        (F_\Psi^\mathcal{T})$. By definition of LFP the tuple $(\eta(V_1), \dots, \eta(V_n))$ is in
        $LFP(F_\Psi^\mathcal{T})$ iff $X$ is the smallest $X$ such that $X = F_\Psi^\mathcal{T}(X)$ and $(\eta(V_1), \dots, \eta(V_n)) \in
        F_\Psi^\mathcal{T}(X)$. By definition of $F_\Psi^\mathcal{T}(X)$ it holds $(\eta
        (V_1), \dots, \eta(V_n)) \in F_\Psi^\mathcal{T}(X)$ exactly then if $\mathcal{T}, \eta' 
        \models \Psi$, where $\eta'$ is a variable mapping with $\eta'(x) = \eta(x)$ for all variables $x \neq X$ and $\eta'(X) = F_\Psi^\mathcal{T}(X)$. By induction hypothesis it holds
        \[\emph{q} \in \llbracket  F(\Psi)
        \rrbracket^{\eta_V}_\mathcal{T}.\]

        
The next step is to show that $\eta'_V(X)$ is also a least fixpoint of $\llbracket
         \mu(X\colon (\tau \dots, \tau)).\,$ $F(\Psi) \rrbracket^{\eta_V}_\mathcal{T}$. By Theorem~\ref{theorem:kleene} the least fixpoint $\eta'(X)$ can be calculated by a sequence $X_0, 
         \dots, X_m$ where here $X_0 = \emptyset$ and $X_{i+1} = F_\Psi^\mathcal{T}(X_i)$ and $\eta'(X) = X_m$. On the other hand the least fixpoint $\eta'_V(X)$ can be calculated by a 
         sequence $Y_0, \dots, Y_{m'}$ where here 
         $Y_0 = \bot_{T((\tau, \dots, \tau))}$ 
         and $Y_{i+1} = \llbracket F(\Psi)\rrbracket_\mathcal{T}^{\eta'_V[X \mapsto Y_i]}$. We now show by induction that $X_i$ corresponds with $Y_i$ in PHFL context for all $i$. For the induction basis it holds $X_0 = \emptyset$. Because $X$ in HO context is of type $(\tau, \dots \tau)$ the same $X$ has in PHFL context the type $T((\tau, \dots, \tau))$. An empty set of type $(\tau, \dots \tau)$ is mapped by variable mapping of Definition~\ref{definition:variable_mapping_function} to  $\lambda (X_1 \colon T(\tau)).\, \dots \lambda (X_n \colon T(\tau)).\,\bot$ and this is equal to $Y_0$. By induction hypothesis it holds $X_i$ corresponds with $Y_i$  in PHFL context for 
         one $i$. Then $X_{i+1} = F_\Psi^\mathcal{T}(X_i)$ what is
         \[\{(a_1, \dots, a_n) \mid \mathcal{T}, \eta'' \models \Psi\},\]
         where $\eta''$ is a variable mapping with $\eta''(x) = \eta(x)$ for all variables $x \neq X$ and $\eta''(X_{1}) = a_1, \dots,  \eta''(X_{n}) = a_n$ and $\eta''(X) = X_i$.  
         By using the small induction hypothesis and the main induction hypothesis at once this is exactly the case when
                 \begin{align*}
                 \{(a_1, \dots, a_n) \mid 
        \emph{q} \in \llbracket
        F(\Psi) \rrbracket^{\eta''_V[X\mapsto Y_i]}_\mathcal{T}\}
        \end{align*}
		Because $Y_{i+1} = \llbracket F(\Psi)\rrbracket_\mathcal{T}^{\eta''_V[X \mapsto Y_i]}$ it follows        
        \begin{align*}
                 \{(a_1, \dots, a_n) \mid 
        \emph{q} \in Y_{i+1}\}.
        \end{align*}
                Because of the construction of variable mapping $\eta'_V$ and $(\eta'(V_1), \dots, \eta'(V_n)) \in \eta'(X)$ it holds
        $(\dotsb\big(\eta'_V(X)\,\eta'_V(V_1)\big) \dotsb )\,\eta'_V(V_n) = Q^d$
        and so 
        \begin{align*}
        \emph{q} \in (\dotsb \big(\eta'_V(X)\,\eta'_V(V_1)\big) \dotsb )\,\eta'_V(V_n).
        \end{align*} 
       then it follows $X_{i+1} = Y_{i+1}$. The induction holds.        

        With $F(\Phi) = (\dotsb \big(\mu (X \colon T(\tau)).\,\Phi(X)\big)\,V_1)\dotsb)\,V_n$ it follows
        \[\emph{q} \in \llbracket
         F(\Phi) \rrbracket^{\eta_V}_\mathcal{T}.\]
        exactly then if $\mathcal{T}, \eta \models \Phi$.
    \end{compactitem}
\end{proof}

\begin{remark}
In the proof of Theorem~\ref{theorem:ho_lfp_equals_phfl} the dimension $d$ of $F(\Phi)$ is $2s+q+r$ but only the components where the free variables are represented are filled with input parameters. That means by only projecting these components in the by $F(\Phi)$ defined query $\mathcal{Q}^d_{F(\Phi)}$ we get the resulting query $\mathcal{Q}^f_{\Phi}$ that is defined by $\Phi$. 
\end{remark}

The combination of Theorem~\ref{theorem:ho_lfp_equals_phfl}, Theorem~\ref{theorem:hoLfpEqualsExptime} and
Theorem~\ref{theorem:phfl_k_in_k_exptime} proves the following theorem.

\begin{theorem}
    Let $k \geq 0$. PHFL$^k$ captures \exptime{$k$} over finite labeled transition systems.
\end{theorem}



%%
%% Author: DKron
%% 17.08.2018
%%

\section{Lower Bound of PHFL$^{k + 1}_{tail}$}\label{sec:lowerBoundOfPhflTail}

The lower bound of PHFL$^{k+1}_{tail}$ can be proven similar to the lower bound of PHFL$^k$. The main idea is not to
show directly, that the lower bound of PHFL$^{k+1}_{tail}$ is \expspace{$k$} but rather by detour over the
bisimulation-invariant fragment of HO(PFP)$^{k+1}$. In the first subsection we show that a Turing Machine that is in
$k$-EXPSPACE can be encoded by a HO(PFP)$^{k+1}$ formula. The next subsection uses this statement to show that the
lower bound of PHFL$^{k+1}_{tail}$ is \expspace{$k$} by encoding formulas of HO(PFP)$^{k+1}$ in PHFL$^{k+1}_{tail}$.

\subsection{$k$-EXPSPACE in HO(PFP)$^{k+1}$}\label{subsec:kExpspaceInHopfp}

In this subsection we want to show that a run of an $exp(k, f(n))$ space bounded DTM can be encoded by some
HO$^{k+1}$ formula. The main idea of this statement is an extension of the result of Abiteboul and
Vianu~\cite{abiteboul1995computing} into higher-order. They have shown that HO(PFP)$^1$ coincides with $0$-EXPSPACE.

%Because a finite $\sigma$-structure $\mathcal{A}$ for some relational signature $\sigma$ cannot be fed to a Turing
%%machine directly, we have to encode $\mathcal{A}$ to an input word $\langle \mathcal{A} \rangle$. We refer the
%interested reader to~\ref{}

\begin{lemma}
    \label{lemma:expspace_in_ho_pfp}
    Given a DTM $M = (Q, \Sigma, \Gamma, \delta, q_0, \Box, F, R)$ in $k$-EXPSPACE there exists a formula $\Psi$ in HO(PFP)$^{k+1}$ over signature $\sigma$ such that for all variable mappings $\eta$ and all $\sigma$-structures $\mathcal{A}$ it holds $\mathcal{A}, \eta \models \Psi$ exactly then if the run of $M$ on the standard coding of $(\mathcal{A}, \eta)$ is accepting.
\end{lemma}

\begin{proof}
    Let $M = (Q, \Sigma, \Gamma, \delta, q_0, \Box, F, R)$ be a $exp(k, f(n))$ space bounded DTM. Furthermore, let $\sigma$ be a relational signature, $\sigma_< = \sigma\;\cup
    <$ an extension of $\sigma$ and $\mathcal{A}$ a finite $\sigma_<$-structure including $<^\mathcal{A}$ that
    represents a linear ordering on the universe $\mathcal{U}$ of $\mathcal{A}$. Finally, let $\tau$ be an HO type of
    order $k + 1$ and $\alpha$ a variable mapping.
    To prove this lemma we want to define a relational representation, of the final configuration of $M$ that has the standard coding of $(\mathcal{A}, \alpha)$\footnote{The standard coding of structures is a non-trivial problem. Because the description of the encoding goes beyond the scope of this thesis, we only refer to~\cite{abiteboul1995computing} for further informations about.}, abbreviated with $w$, as input word,
    as a partial fixpoint of some HO$^{k+1}$ formula. For this we set up the underlying HO$^{k+1}$ formula
    $\varphi$ such that the iterates $(F_\varphi^\mathcal{A})^{i+1}(\emptyset)$ over $D_\tau
    (\mathcal{U})$ represent the $i$-th configuration of $M$ on input standard coding of $(\mathcal{A}, \alpha)$, for every $i$ until termination.

    Before we can define the configurations of $M$ in HO we have to make some preparations. Remember that a
    configuration of $M$ is a relation between the current state, the current reading head position and
    the current tape content represented by a function. These configurations will be combined in one relation $X$.
    Because the size of the formula we build have to be polynomial and the reading head of $M$ can be on
    one of $exp(k, f(n))$ cells for example, we have to encode the number in sets of order $k + 1$. Furthermore, to do
    not exceed the bound of order $k + 1$, we have to split the tape content function in this way that in one tuple
    $x$ of $X$ is just the current state, the current head position and one position of the tape with its content. By
    syntax of HO types each component of $x \in X$ have to be of the same type, so to access the different states and
    tape symbols they have to be numerated. $\{0, \dots, |Q| - 1\}$ for states and $\{0, \dots, |\Gamma| - 1\}$ for tape
    symbols.

    The next step is to define some abbreviations that we want to use in the definitions of the configurations. The
    first and most important abbreviation is the definition of orders of any HO type. These orders are defined similarly
    to the defined formulas of Definition~\ref{definition:lower_bound_less_higher}. A tuple $x$ is smaller then a tuple $y$ if there is a position $i$ where $x_i < y_i$ and there is no position $j<i$ where $x_j > y_j$. A set $X$ is smaller then a tuple $Y$ if there is a $x \in Y$ such that $x \not\in X$ and there is no $y<x$ such that $y\in X$ but $y\not\in Y$.
    \begin{align*}
        <^\odot(x, y) \coloneqq &\,<^\mathcal{A}(x, y) \\
        <^{\tau' \times \dots \times \tau'}(x_1, y_1, \dots, x_n, y_n) \coloneqq &\,\underset{i =
        1}{\overset{n}{\bigvee}}<^{\tau'}(x_i, y_i) \wedge \underset{j = 1}{\overset{i - 1}{\bigwedge}}
        \neg <^{\tau'}(y_j, x_j)\\
        <^{(\tau', \dots, \tau')}(X, Y) \coloneqq &\,\exists (x_1 \colon {\tau'}). \,\dots \exists(x_n \colon
        {\tau'}).\, Y(x_1, \dots, x_n)
        \\&\,\wedge \neg X(x_1, \dots, x_n)\,\wedge \forall (y_1 \colon {\tau'}). \,\dots
        \forall(y_1 \colon {\tau'}).\,\\&\,<^{\tau'\times \dots \times \tau'}
        (y_1, x_1, \dots, y_n, x_n) \\&\,\Rightarrow (X(y_1, \dots, y_n) \Rightarrow Y(y_1, \dots, y_n))
    \end{align*}
    With these formulas it is possible to define another two important abbreviations. On the one hand equality
    of two variables of arbitrary type and on the other hand the successor of a given element. If $X$ and
    $Y$ are two variables of type $\tau$ then the equality of $X$ and $Y$ is given by the formula
    \[X = Y \coloneqq \neg<^\tau(X, Y) \wedge \neg <^\tau(Y, X).\]
    Finally, if $X$ and $Y$ are two variables of type $\tau$ then the prove that $Y$ is the successor of $X$ is given by
    the formula
    \[next^{\tau}(X, Y) \coloneqq\; <^\tau(X, Y) \wedge \forall (Z \colon \tau).\, <^\tau(Z, Y) \Rightarrow\;<^\tau
    (Z, X).\]

    Now we are able to define the configurations in HO(PFP)$^k$. The first configuration is the initial configuration
    $C_0^M$. For input word $w$ this is given by the formula
    \begin{align*}
        \varphi_0(q, h, i, b) \coloneqq &\,q = q_0 \wedge h = 0 \wedge (\neg <^\tau(|w|, i) \Rightarrow b = w_{i})\,
        \vee\\&\,(\neg <^\tau (i, |w|) \wedge \neg i = |w| \Rightarrow b = \Box))
    \end{align*}
    where $q$ is the current state, $h$ the current head position, $i$ a tape index and $b$ the symbol on $i$. $q_0$,
    $0$, $|w|$, $w_{i}$ and $\Box$ are the numerical representations as sets of the same elements in $Q$ and $\Gamma$.

    To iterate through all configurations of $M$ on input $w$, we need a variable $X$ of type $\tau =
    (\tau', \tau', \tau', \tau')$ where $\tau'$ has order $k + 1$, so $X$ has order $k + 2$. On an iteration
    $(F_\varphi^\mathcal{A})^{i+1}(\emptyset)$ for the following formula $\varphi$ the variable $X$ includes
    all configurations that will be reached within $i$ transitions.
    \begin{align*}
        \varphi(X, q, h, i, b) \coloneqq &\,(\neg \exists (q_{old} \colon \tau').\, \neg\exists
        (h_{old}
        \colon \tau').\, \neg\exists (i_{old} \colon \tau').\, \neg\exists (b_{old} \colon \tau').\,\\
        &\, X(q_{old}, h_{old}, i_{old}, b_{old}) \wedge \varphi_0(q, h, i, b)) \,\vee \\
        &\,(\exists (q_{old} \colon \tau').\, \exists (h_{old} \colon \tau').\, \exists (i_{old} \colon
        \tau').\, \exists (b_{old} \colon \tau').\,\\
        &\, X(q_{old}, h_{old}, i_{old}, b_{old}) \wedge \xi(X, q, h, i, b))
    \end{align*}
    Note that $\varphi_0$ is invoked only in the first iteration and thus provides the correct initialisation. The
    formula $\xi$ collects the transitions of those tuples in $X$ according to the transition function $\delta$ of
    $M$. In each iteration exactly one configuration will be added to $X$ because $M$ is deterministic.
    The formula $\xi$ is given by
    \begin{align*}
        \xi(X, q, h, i, b) \coloneqq &\,\exists (q_{old} \colon \tau').\, \exists (h_{old} \colon
        \tau').\, \exists (i_{old} \colon \tau').\, \exists (b_{old} \colon \tau').\, \\
        &\,\exists (q_{new} \colon \tau').\, \exists (b_{new} \colon \tau').\,X(q_{old}, h_{old}, i_{old},
        b_{old}) \,\wedge \\
        &\, (\underset{\delta(q_{old}, b_{old}) = (q_{new}, b_{new}, d)}{\bigvee} q = q_{new} \wedge h =
        h_{old} + d \wedge i = i_{old}\,\wedge\\&\, ((\neg i = h_{old} \wedge b =
        b_{old}) \vee (i = h_{old} \wedge b = b_{new})))
    \end{align*}
    where $h = h_{old} + d$ depends on $d$ and is given by
    \[h = h_{old} + d \coloneqq
    \begin{cases}
        next^\tau(h_{old}, h),  & \text{if } d = L\\
        next^\tau(h, h_{old}),  & \text{if } d = R\\
        h = h{old},  & \text{if } d = N
    \end{cases}\]

    Because $M$ terminates the formula
    \[\psi_0(q', h', i', b') \coloneqq [\mathit{PFP}\;\varphi(X, q, h, i, b)](q', h', i', b') \]
    is guaranteed to define the relational description of the final configuration of $M$ on input word $w$.
    Finally, the formula
    \[\psi \coloneqq \exists (h' \colon \tau').\, \exists (i' \colon \tau').\, \exists (b' \colon \tau').\,
    \underset{q' \in F}{\bigvee} \psi_0(q', h', i', b')\]
    defines the acceptance of $M$ on input $w$.
\end{proof}

\subsection{Bisimulation Invariant HO(PFP)$^{k+1}$ to PHFL$^{k+1}_{tail}$}\label{subsec:bisimulationInvariantHopfptoPhfl}

As mentioned in the introduction of this section the main idea is not to show directly, that the 
lower bound of PHFL$^{k+1}_{tail}$ is \expspace{$k$} but rather by detour over the 
bisimulation-invariant fragment of HO(PFP)$^{k+1}$. In the previous subsection we have seen 
 that a DTM in $k$-EXPSPACE can be performed by an HO(PFP)$^{k+1}$
formula. By encoding the bisimulation-invariant fragment of HO(PFP)$^{k+1}$ into
PHFL$^{k+1}_{tail}$ combined with the knowledge that $k$-EXPSPACE is captured by 
HO(PFP)$^{k+1}$ leads to the lower bound of PHFL$^{k+1}_{tail}$. In
Section~\ref{sec:existential_quantifiers_in_phfl} and Section~\ref{sec:lowerBoundOfPhfl} we 
have shown that the HO$^{k+1}$ part can be encoded in PHFL$^k$. It is easy to prove that the 
encoded formulas are all tail-recursive. It follows that the HO$^{k+1}$ part can also be 
encoded in  PHFL$^{k+1}_{tail}$. The PFP operator is the only kind of HO(PFP)$^{k+1}$ formula 
that we have to encode in this subsection to get the lower bound of PHFL$^{k+1}_{tail}$.

Before we give the definition of the transforming function, we define a PHFL formula for HO formulas that uses the PFP operator.

\begin{definition}
Let $\Psi$ be a HO$^k$ formula with free first-order variables $x_1, \dots, x_f$, quantified first-order variables $x_{f+1}, \dots,
    x_q$ and let be $X$ a HO variable of HO type $\tau = (\tau', \dots, \tau')$ where $\tau' \neq \odot$ and any HO variables
    $X_1, V_1, \dots, $ $X_n, V_n$ HO type $\tau'$. Furthermore, let
    $\Phi$ be a PHFL$^k$
    formula and let $s$ be the maximal arity
    of all second order variables of $\Psi$, then $PFP^\tau X.\,\Phi$
    is a PHFL$^k$ formula with dimension $d = 2 * s + q + r$, where $r$ is the number of states that defines the reduced LTS,  defined as:
    \begin{align*}
     PFP^\tau X. \, \Phi \coloneqq \big(\mu (F \colon T(\tau) \rightarrow \bullet).\,&\lambda (X \colon T(\tau)).\, \big(X\,\wedge \forall^{\tau'}X_1.\, \dotsb \forall^{\tau'}X_n.\, \\&\big( (\dotsb (X X_1) \dotsb) X_n \Leftrightarrow \Phi(X, X_1, \dots, X_n) \big)\big)\\& \vee F(\Phi(X))\big)\lambda (V_1 \colon T(\tau')).\, \dotsb \lambda (V_n \colon T(\tau')).\,\bot
\end{align*}    
    In case of $\tau = (\odot, \dots, \odot)$ and $v_1, \dots, v_n$ are first-order variables with index $i_1, \dots, i_n$ respectively let $PFP^{(\odot, \dots, \odot)} X.\,\Phi$ defined as:
    \begin{align*}
    PFP^{(\odot, \dots, \odot)} X.\,\Phi \coloneqq \big(\mu (F \colon \bullet \rightarrow \bullet).\,&\lambda (X \colon \bullet).\, \big(X \wedge \forall_1 \dotsb \forall_n \\&\big(X \Leftrightarrow \Phi(X)\big)\big) \vee F(\Phi(X))\big)\,\bot
    \end{align*}
\end{definition}

Now we are able to define the function to translate a bisimulation invariant HO(PFP)$^{k+1}$
formula to a PHFL$^{k+1}_{tail}$. 

\begin{definition}
    \label{definition:lower_bounds_phfl_formula_function_pfp}
    Let $F$ be the function of Definition~\ref{definition:lower_bounds_phfl_formula_function} we extend this function in this way that it maps a bisimulation invariant HO(PFP)$^{k+1}$ formula $\Phi$ with free first-order variables $x_1, \dots, x_f$, quantified first-order variables $x_{f+1}, \dots,
    x_q$ to a PHFL$^{k+1}_{tail}$ formula with dimension $d = 2 * s + q + r$, where $r$ is the number of states that defines the reduced LTS and $s$ is the maximal arity
    of all second order variables of $\Phi$, defined
    inductive on $\Phi$ as follows:
    \begin{align*}
        F([PFP\;\Phi(X, x_{i_1}, \dots, x_{i_n})](v_{j_1}, \dots, v_{j_n})) \coloneqq &\,PFP^{(\odot, \dots, \odot)} X.\, F(\Phi) \\
        F([PFP\;\Phi(X, X_1, \dots, X_n)](V_1, \dots, V_n)) \coloneqq &\,PFP^\tau X.\, F(\Phi) 
    \end{align*}
\end{definition}

The last step is to show that the semantics of a given HO(PFP)$^{k+1}$ formula coincides with the semantics of the by function $F$ encoded PHFL$^{k+1}_{tail}$ formula. As mentioned in Section~\ref{sec:lower_bounds_preparation} without loss of generality the statement can be proven by consider only  reduced LTS. 

\begin{lemma}
    \label{lemma:ho_pfp_equals_phfl_tail}
    Let $f \geq 1$ and $k \geq 0$. For every bisimulation-invariant formula $\Phi$ of HO(PFP)$^{k + 1}$ there is a
    PHFL$^{k+1}_{tail}$ formula $\Psi$ such that the $f$-adic query $\mathcal{Q}_\Phi^f$ that belongs to $\Phi$ is equal to the $f$-adic query  $\mathcal{Q}_\Psi^f$ that belongs to $\Psi$.
\end{lemma}

\begin{proof}
    This lemma can be proven by showing for all HO(PFP)$^{k+1}$ formulas $\Phi$ with free first-order variables $x_1,
    \dots, x_f$ and quantified first-order variables $x_{f+1}, \dots, x_q$, all reduced LTS $\mathcal{T} = (Q, \Sigma, P,
    \Delta, v)$ with respect to $\emph{q}_r = q_1, \dots, q_r$ and all variable mappings $\eta$ that it holds that $\mathcal{T}, \eta \models \Phi$ iff $\emph{q} =
    (\emph{q}_s, \emph{q}_s, \emph{q}_f, \emph{q}_q, \emph{q}_r)$ and $\emph{q} \in \llbracket
   F(\Phi)\rrbracket^{\eta_V}_{\mathcal{T}}$, where $\emph{q}_s = q_1',\dots,q_s'$ is a sequence of $s$ placeholders used for the interaction of second-order variables, $\emph{q}_f = \eta(x_1), \dots, \eta(x_f)$ is a sequence of the by $\eta$ mapped free first-order variables, $\emph{q}_q = q_1'', \dots, q_q''$ is a sequence of $q$ placeholders used for the quantified first-order variables, $q_1', \dots, q_s', q_1'', \dots, q_q'' \in Q$ are arbitrary states, $F$ is the formula function of
    Definition~\ref{definition:lower_bounds_phfl_formula_function_pfp} and $\eta_V$ the variable mapping of
    Definition~\ref{definition:lower_bound_variable_function}. This statement can be proven by induction over formula
    $\Phi$.
    As mentioned in the introduction we have to show only correctness of the PFP operators.
    \begin{compactitem}
    \item In case of $\Phi = [PFP\,\Psi(X, x_{i_1}, \dots, x_{i_n})](v_{j_1}, \dots, v_{j_n})$, where $X$ is a
        free variable in $\Psi$ of HO type $(\odot, \dots, \odot)$ and $x_{i_1}, \dots, x_{i_n}$ are free first-order
        variables of $\Psi$ and $v_{j_1}, \dots, v_{j_n}$ are first-order variables of $\Phi$, then it follows that
        $\mathcal{T}, \eta \models \Phi$ exactly then if $(\eta(v_{j_1}), \dots, \eta(v_{j_n})) \in PFP
        (F_\Psi^\mathcal{T})$. 
        
        ...
        TODO
        ...
        
        exactly then if $\mathcal{T}, \eta \models \Phi$.

        \item In case of $\Phi = [PFP\,\Psi(X, X_1, \dots, X_n)](V_1, \dots, V_n)$, where $X$ is a
        free variable in $\Psi$ of HO type $(\tau, \dots, \tau)$ and $X_1, \dots, X_n$ are free first-order
        variables of $\Psi$ of type $\tau$ and $V_1, \dots, V_n$ are free variables of $\Phi$ also of type $\tau$, then
        it follows that $\mathcal{T}, \eta \vdash \Phi$ exactly then if $(\eta(X_1), \dots, \eta(X_n) \in PFP
        (F_\Psi^\mathcal{T})$. 
        
...
TODO
...        
        
        exactly then if $\mathcal{T}, \eta \models \Phi$.
    \end{compactitem}
\end{proof}

The combination of Lemma~\ref{lemma:ho_pfp_equals_phfl_tail}, Lemma~\ref{lemma:expspace_in_ho_pfp} and
Theorem~\ref{theorem:phfl_k_plus_1_tail_in_k_expspace} proves the following theorem.

\begin{theorem}
    Let $k \geq 0$. PHFL$^{k+1}_{tail}$ captures \expspace{$k$} over labeled transition systems.
\end{theorem}
