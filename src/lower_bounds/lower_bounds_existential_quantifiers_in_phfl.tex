
%%
%% Author: DKron
%% 17.08.2018
%%

\section{Existential Quantifiers in PHFL}\label{sec:existential_quantifiers_in_phfl}

In this section we define PHFL$^{k}$ formulas that describes existential quantification over HO variables of order $k
\geq 1$. But before we can define these formulas we have to translate the types.

The types of HO have to be translated because the most types in HO$^{k + 1}$ do not exist in PHFL$^k$. While HO$^{k +
1}$ includes variables for example of kind set of sets, PHFL$^k$ does not support this kind of type.
But each set $A$ in HO$^{k+1}$ can be described by the characteristic function of $A$ in PHFL$^k$.

The following definition translates all HO types with order $2$ or greater to types in PHFL. The base type of HO
has to be encode differently, and will be regarded after this definition.

\begin{definition}
    \label{definition:lower_bound_type_function}
    Let $T$ be a function that maps a type of HO of order $2$ or greater to a type of PHFL defined inductive over the
    type of HO as follows:

    \begin{align*}
        T((\odot, \dots, \odot)) &= \bullet\\
        T((\tau', \dots, \tau')) &= T(\tau')^+ \rightarrow (T(\tau')^+ \rightarrow \dots \rightarrow (T(\tau')^+
        \rightarrow \bullet) \dots )
    \end{align*}
\end{definition}

Note that $ord(T(\tau)) = order(\tau) - 2$ for all $\tau$ with order $2$ or greater.

\begin{example}
    Let $\tau$ be a type of HO
    \[\tau = (\tau', \tau')\]
    with
    \[\tau' = (((\odot)), ((\odot)))\]
    then by Definition~\ref{definition:lower_bound_type_function} of type function $T$
    \[T(\tau) = T(\tau') \rightarrow (T(\tau') \rightarrow \bullet).\]
    with
    \[T(\tau') = (\bullet \rightarrow \bullet) \rightarrow ((\bullet \rightarrow \bullet) \rightarrow \bullet)\]
\end{example}

With this type function $T$ a HO$^{k + 1}$ variable $X$ of type $\tau$ can be translated to a PHFL$^k$ variable
with type $T(\tau)$. Please note that $\tau$ has order $2$ or greater. Intuitively the variable $X$ which represents a
set in HO$^{k+1}$ is represented in PHFL$^k$ as the characteristic function of $X$. Please note that $X$ for type
$\tau = (\odot, \dots, \odot)$ is also a set in PHFL$^k$ of PHFL type $\bullet$.

As mentioned above the base type of HO have to be encoded differently. The reason is that the base type in PHFL is at
least a set of tuples of states. So single states can not be depict directly by a variable. In this thesis the
idea is to use the polyadic fragment of a PHFL$^k$ formula $\Phi$ to represent the different first-order variables of an
HO$^{k+1}$ formula $\Psi$. Each first-order variable of $\Psi$ represents one component in $\Phi$, that means each
variable
increases the dimension of $\Phi$. The assignment of a first-order variable $x_i$ in $\Psi$ is then the current state
of the $i$-th component in $\Phi$. This is mainly from~\cite{lange2014capturing}.

Without loss of generality the first-order variables are enumerated as $x_1, \dots, x_f, x_{f + 1}, \dots, x_q$,
where $x_1, \dots, x_f$ are the free and $x_{f+1}, \dots, x_q$ are the quantified variables.

\begin{remark}
    The PHFL formula $\Phi$ that we get through the encoding of a given HO formula $\Psi$ has dimension
    $d$ that is always big enough to translate all second-order variables of $\Psi$ to an order $0$ variable in
    $\Phi$. In more detail $s$ is the maximal arity of second-order variables in $\Psi$ and $d > s$. To distiguish all
    different first-order variables in $\Psi$, the dimension $d$ of $\Phi$ is also bigger then $q$.  Finally, to compare
    two
    elements of $Q^{m}$, where $Q$ is the state set of an LTS and $m = max({s, q})$, the dimension $d$ of $\Phi$ is
    twice as big as the maximum of $s$ and $q$. That means $d = 2 * m$.
\end{remark}

After we know how to interpret the different HO types and variables we are now able to consider the existential
quantification. Before we regard higher-order quantification we look at first-order quantification.

Because we encode the bisimulation-invariant fragment of HO(LFP)$^{k + 1}$, the first-order quantification can be
encoded by moving through all states reachable by one of the free variables and check if we reach a state tuple where
the bounded formula holds. If we regard $\exists (x_i \colon \odot).\,\Phi$ then it can be understand as, check if we
reach a state tuple where $\Phi$ holds once the $i$-th component is replaced by one of the free variables. This can
be formalized in PHFL as
\[\exists_i \Phi \coloneqq \bigvee_{j=1}^f \{(1, \dots, i-1, j, i + 1, \dots, d)\} \mu (X
\colon \bullet).\,\Phi \vee \bigvee_{a \in \Sigma} \langle a \rangle_i X.\]
Now we regard the higher-order quantification. Let $\tau$ a HO type, $\sigma$ a signature and $\mathcal{U}$ the
universe of a $\sigma$ structure. Because the idea we use for first-order quantification is not obvious to adapt to
the higher-order quantification, we use another encoding. The idea to obtain higher-order quantification in PHFL is that
we have to iterate through all possible elements in a domain $D_\tau(\mathcal{U})$ and check if the given formula is
fulfilled. To make it possible to iterate through $D_\tau(\mathcal{U})$ we need a formula that returns us the
successor of a given element in $D_\tau(\mathcal{U})$. Eventually, to get the successor of a given element we need some
order on $D_\tau(\mathcal{U})$. So the first thing we need is the order of any domain.

\begin{remark}
    In~\cite{otto1999bisimulation} was shown that it is possible to define a $2$-adic formula $\Phi_<$ that defines a
    transitive relation $<$ such that $< \cap > = \emptyset$ and $< \cup > = \not\sim$. In this thesis $<$ is a total
    order on states of an LTS.
\end{remark}

To get an order of any type we define formulas that tells us given two elements of same type which one is the smaller
one. We say that a tuple $x$ is smaller than a tuple $y$ if there is an index $i$ such that the element in $x$ on $i$
is smaller then the element in $y$ on $i$ and there is no position $j$ left of $i$ such that the element in $x$ on
$j$ is bigger then the element in $y$ on $j$. We say that a set $x$ is smaller than a set $y$ if there is an element
$e$ in $x$ that is not in $y$ and all smaller elements $f$ to $e$ are only in $x$ if $f$ is also in $y$. This is
formalized in PHFL in the following definition.

\begin{definition}
    \label{definition:lower_bound_less}
    The PHFL$^k$ formula $<^\tau$ where $k = ord(T(\tau))$ is defined inductive over the type $\tau$ as follows:

    \begin{align*}
        <^\odot \coloneqq &\,\Phi_< \\
        <^{\odot \times \dots \times \odot} \coloneqq &\,\underset{i = 1}{\overset{m}{\bigvee}}\{(i, i + m, 3, \dots,
        d)\} <^\odot \wedge \\
        &\,\underset{j = 1}{\overset{i - 1}{\bigwedge}}\{(j + m, j, 3, \dots, d)\} \neg
        <^\odot\\
        <^{\tau' \times \dots \times \tau'}(x_1, y_1, \dots, x_1, y_n) \coloneqq &\,\underset{i =
        1}{\overset{n}{\bigvee}}<^{\tau'}(x_i, y_i) \wedge \underset{j = 1}{\overset{i - 1}{\bigwedge}}
        \neg <^{\tau'}(y_j, x_j)\\
        <^{(\odot, \dots, \odot)}(X, Y) \coloneqq &\,\exists_{i_1}.\, \dots \exists_{i_n}. \{(i_1, \dots, i_n, n
        + 1, \dots, d)\}Y \wedge \\&\,\neg \{(i_1, \dots,
        i_n, n + 1, \dots, d)\} X\,\wedge\\&\, \forall_{j_1}. \,\dots \forall_{j_n}. \{(j_1, \dots, j_n, n+1,
        \dots, m, \\&\,i_1, \dots, i_n, m + n + 1, \dots, d)\}<^{\odot
        \times \dots \times \odot} \Rightarrow \\&\,(\{(j_1, \dots, j_n, n + 1, \dots, d)\}X
        \Rightarrow \\&\,\neg \{(j_1, \dots, j_n, n + 1, \dots, d)\} Y)
        \\
        <^{(\tau', \dots, \tau')}(X, Y) \coloneqq &\,\exists^{\tau'}x_1. \,\dots \exists^{\tau'}x_n.\,(\dots((Y
        (x_1))(x_2))
        \dots) (x_n)
         \\&\,\wedge \neg (\dots((X(x_1))(x_2)) \dots)(x_n)\,\wedge \\&\,\forall^{\tau'}y_1. \,\dots
        \forall^{\tau'}y_n.
        \,<^{\tau'
        \times \dots \times \tau'}
        (y_1, x_1, \dots, y_n, x_n) \\&\,\Rightarrow ((\dots((X(y_1))(y_2)) \dots)(y_n) \\&\,\Rightarrow (\dots((Y(y_1))
        (y_2)) \dots(y_n))
    \end{align*}
    where the formulas of kind $\exists^{\tau'} x.\,\phi$ are the in PHFL$^{k-1}$ defined higher-order quantification
    for HO
    type $\tau'$ and formulas of kind $\forall^{\tau'} x.\,\phi$ are the counterparts.
\end{definition}

After we have now orders of any HO type we can define formulas that returns the successor of the input element.

\begin{definition}
    The PHFL$^k$ formula $next^\tau$ where $k = ord(T(\tau))$ is defined by the type $\tau$ as follows:
    \begin{align*}
        next^{(\odot, \dots, \odot)} \coloneqq &\,\lambda (X \colon \bullet).\, (\neg X \wedge \forall_{m +
        1}\dots\forall_{m + m}<^{\odot \times \dots \times \odot}\, \Rightarrow \\&\,\{(m +
        1, \dots, m + m, m + 1, \dots, d)\} X) \,\vee \\&\,(X \wedge \exists_{m + 1}\dots\exists_{m + m} <^{\odot
        \times \dots \times \odot} \,\wedge \\&\,\{(m + 1, \dots, m + m, m + 1, \dots, d)\}
        \neg X)\\
        next^{(\tau_1, \dots, \tau_n)} \coloneqq &\,\lambda (X \colon T ((\tau_1, \dots, \tau_n))).\,(\neg (\dots((X
        (x_1))(x_2))\dots) (x_n) \\&\, \wedge \forall^{\tau_1}y_1.\, \dots \forall^{\tau_n}y_n.\,<^{\tau_1 \times
        \dots \times \tau_n}(y_1, x_1, \dots, y_n, x_n) \\&\,\Rightarrow  (\dots((X(y_1))(y_2))\dots)(y_n)) \,\vee
        \\&\,((\dots ((X(x_1))(x_2)) \dots)(x_n) \wedge \exists^{\tau_1}y_1.\, \dots \exists^{\tau_n}y_n.\, \\&\,
        <^{\tau_1 \times \dots \times \tau_n}
        (y_1, x_1,
        \dots, y_n, x_n)\,\wedge \neg (\dots((X(y_1))(y_2))\dots)(y_n))
    \end{align*}
\end{definition}

The idea of the two formulas are based on binary incrementation. Let $\tau = (\tau', \dots, \tau')$ an HO type and
$\mathcal{U}$ the universe of a $\sigma$-structure. Remember that a set $X \in D_\tau(\mathcal{U})$ can be
represented by its characteristic function. This can be transformed to a binary string where each position of
this string represents an element of $D_{\tau'}(\mathcal{U})^n$. Because each position in the binary string represents
an element of $D_{\tau'}(\mathcal{U})^n$ and a position have always to represent the same element in $D_{\tau'}
(\mathcal{U})^n$, the elements in $D_{\tau'}(\mathcal{U})^n$ have to be ordered. The order of the elements of
$D_{\tau'}(\mathcal{U})^n$ is given by the formula $<^{\tau' \times \dots \times \tau'}$ of
Definition~\ref{definition:lower_bound_less}. If the position $i$ in the binary string is $1$ then this means that
the element with index $i$ in $D_{\tau'}(\mathcal{U})^n$ is also in $X$ and if the position $i$ in the binary string
is $0$ then the element with index $i$ in $D_{\tau'}(\mathcal{U})^n$ is not in $X$. This binary representation of $X$
in regard to $D_{\tau'}(\mathcal{U})^n$ can be extended to a function $f\colon D_\tau(\mathcal{U}) \rightarrow {0,
\dots, |D_\tau(\mathcal{U})| - 1}$ such that each element $X$ of $D_\tau(\mathcal{U})$ will be mapped to its binary
string in regard to $D_{\tau'}(\mathcal{U})^n$. With this knowledge we can say that $Y \in D_\tau(\mathcal{U})$ is the
direct successor of $X \in D_\tau(\mathcal{U})$ if $f(Y) = (f(X) + 1)$ modulo $|D_\tau(\mathcal{U})|$. In detail that
means that the $i$-th bit is $1$ in $f(Y)$ if it is either $0$ in $X$ and all lower bits are $1$ in $X$ or it is $1$
in $X$ and there is a bit lower then $i$ that is $0$ in $X$.

With the previous definitions we are now able to define the higher-order quantification in PHFL.
\begin{definition}
    \label{definition:existential_quantification}
    Let $\tau = (\tau', \dots, \tau')$ be an HO type where $\tau' \neq \odot$ then $\exists^{\tau}X .\,\Phi(X)$ is
    defined as follows:
    \begin{align*}
        \exists^{\tau}X.\, \Phi(X) \coloneqq &\,(\mu (F \colon T(\tau) \rightarrow \bullet).\, \lambda (X \colon T(\tau)
        ).\,
        \Phi(X)
        \vee F(next^\tau X))\\&\,(\lambda (x_1 \colon \tau').\, \dots \lambda (x_n \colon \tau').\,\bot)
    \end{align*}
    In case of $\tau = (\odot, \dots, \odot)$, $\exists^{\tau}X .\,\Phi(X)$ is defined as
    \[        \exists^{\tau}X.\, \Phi(X) \coloneqq (\mu (F \colon \bullet \rightarrow \bullet).\, \lambda (X
    \colon \bullet).\, \Phi(X) \vee F(next^{\tau} X)) \bot
    \]
\end{definition}

The last step is to show that the given formula of Definition~\ref{definition:existential_quantification} defines
existential quantification.

\begin{lemma}
    \label{lemma:existential_quantifier}
    For all HO types $\tau$ of order $2$ or greater, all variable mappings $\eta$ and all LTS $\mathcal{T}$ holds
    \[\llbracket \exists^\tau X.\,\Phi(X)\rrbracket^\eta_\mathcal{T} \equiv \underset{\mathcal{X} \in \llbracket \tau
    \rrbracket_\mathcal{T}}{\bigsqcup} \llbracket \Phi(X) \rrbracket^{\eta[X\rightarrow \mathcal{X}]}_\mathcal{T}.\]
\end{lemma}

\begin{proof}
    TODO
\end{proof}