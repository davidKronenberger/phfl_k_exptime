
%%
%% Author: DKron
%% 17.08.2018
%%

\section{Existential Quantifiers in PHFL}\label{sec:existential_quantifiers_in_phfl}

In this section we define PHFL$^{k}$ formulas that describe existential quantification over HO domains of types of
order $k \geq 1$. But before we can define these formulas we have to translate the types.


Note that most types in HO$^{k + 1}$ do not exist in PHFL$^k$. For example, while HO$^{k +
1}$ includes sets of sets, PHFL$^k$ does not support this kind of type.
But each set $X$ in HO$^{k+1}$ can be described by the characteristic function of $X$ in PHFL$^k$.

The following definition translates all HO types of order $2$ or greater to types in PHFL. The base type of HO
has to be encoded differently, and will be established after this definition.

\begin{definition}
    \label{definition:lower_bound_type_function}
    $T$ is a function that maps any type of HO of order $2$ or greater to a type of PHFL defined inductive over the
    type of HO as follows:
    \begin{align*}
        T((\odot, \dots, \odot)) &= \bullet\\
        T((\tau', \dots, \tau')) &= T(\tau')^+ \rightarrow (T(\tau')^+ \rightarrow \dotsb \rightarrow (T(\tau')^+
        \rightarrow \bullet) \dotsb ),
    \end{align*}
    where $\tau' \neq \odot$.
\end{definition}

Note that the orders of HO types and PHFL types are defined differently. It holds that $ord(T(\tau)) = order(\tau) - 2$
for all HO types $\tau$ with order $2$ or greater.

\begin{example}
    Let $\tau = (\tau', \tau')$ be a type of HO with
    \[\tau' = ((\odot), (\odot))\]
    then by Definition~\ref{definition:lower_bound_type_function} of type function $T$
    \[T(\tau) = T(\tau') \rightarrow (T(\tau') \rightarrow \bullet)\]
    where
    \[T(\tau') = \bullet \rightarrow (\bullet \rightarrow \bullet).\]
\end{example}

With this type function $T$ an HO$^{k + 1}$ variable $X$ of type $\tau$ can be translated to a PHFL$^k$ variable
of type $T(\tau)$. Intuitively, the variable $X$ which is a set of $D_\tau(Q)$ in HO$^{k+1}$ is represented
in PHFL$^k$ as the characteristic function of $X$ over $D_\tau(Q)$. This function maps $x$ to $Q^d$ if $x \in X$ and $x$ to $\varnothing$ if $x\not\in X$, where $d > 0$ is a natural number. Note that the domain of HO types
of order $2$ is similar to the domain of base type of PHFL.

As mentioned above the base type of HO has to be encoded differently. The reason for that is that the base type in PHFL is a
set of tuples of states and a single state cannot be depicted directly by a variable. This thesis uses the
idea of~\cite{lange2014capturing} to use the polyadicity of PHFL to represent the different first-order variables of an HO formula
$\Psi$. Each first-order variable of $\Psi$ represents one component in the dimension of the 
corresponding PHFL$^k$ formula $\Phi$ that
means each variable increases the dimension of $\Phi$. The assignment of a first-order 
variable $X_i$ in $\Psi$ is then the current state of the $i$-th component in $\Phi$.

Let $\Phi$ be an HO formula. From now on, we assume that for every HO formula $\Phi$, the first-order variables occurring in $\Phi$ are 
$X_1, \dots, X_q$ for some $q$ depending on $\Phi$.

\subsection{First-Order and Second-Order Quantification}\label{subsec:existentialQuantifiers}

After we know how to interpret the different HO types and variables we can now take a look at the existential quantification. Before we establish higher-order quantification we start with first-order and second-order quantification.

As mentioned in the introduction of this chapter we encode the bisimula\-tion-invariant fragment of HO(LFP)$^{k + 1}$ and HO(PFP)$^{k+1}$. In Section~\ref{sec:lower_bounds_preparation}  we explained that we can treat the semantics of them by using only reduced LTS, where any state is reachable by at least one of the states $q_1, \dots, q_r$. Because of the total order on states of $\mathcal{T}$ explained in Remark~\ref{remark:transitive_relation} the first-order
quantification can be encoded by going over all states reachable from one $q_1, \dots, q_r$ and check whether we
reached a state tuple where the bound formula holds.

To access the states $q_1, \dots, q_r$ in the PHFL formulas that we get by encoding HO 
formulas we use the polyadicity and store $q_1, \dots, q_r$ in components of a dimension that 
will be never influenced by the PHFL formulas. The following remark explains that the PHFL 
formulas has a dimension that is large enough to satisfy all the requirements. 

\begin{remark}
\label{remark:dimension}
    The PHFL formula $\Phi$ that we get through the encoding of a given HO formula $\Psi$ has dimension
    $d$ that is always large enough to translate all second-order variables of $\Psi$ to an order $0$ variable in
    $\Phi$. In more detail $s$ is the maximal arity of second-order variables in $\Psi$ and $d > s$. To compare
    two  elements of $Q^{s}$, where $Q$ is the set of states of an LTS, the dimension $d$ of 
    $\Phi$ is at least twice as big as the maximum of $s$. Because the higher-order variables can be handled otherwise they do not influence the dimension $d$. To distinguish all
    different first-order variables in $\Psi$, the dimension $d$ of $\Phi$ has to be extended by $q$, where $q$ is the number of different first-order variables. That means $d > 2s + q$. Finally, to access the 
    states $q_1, \dots, q_r$ described in Section~\ref{sec:lower_bounds_preparation}, 
    we extend $d$ additionally with $r$ components. That means the dimension of $\Phi$ is $d = 2s + q + r$.
\end{remark}

If we consider $\exists (X_i \colon \odot).\,\Phi$ then it can be understood as the check whether we reach a state tuple where $
\Phi$ holds once the $i$-th component is replaced by one of the last $r$ components. This is where $q_1, \dots, q_r$ are stored.

\begin{definition}
\label{definition:existential_quantification_first}
    Let $d$ and $r$ be the constants as described in Remark~\ref{remark:dimension} where $r < d$, and $1 \leq i \leq d$ then $\exists_i \Phi$ is a PHFL$^k$ formula of dimension $d$ defined as
    \[\exists_i \Phi \coloneqq \bigvee^{d}_{j=d-r+1} \{(1, \dots, i-1, j, i + 1, \dots, d)\} \mu (X
    \colon \bullet).\,\Phi \vee \bigvee_{a \in \Sigma} \langle a \rangle_{i} X.\]
    The formula $\forall_i \Phi$ is also a PHFL$^k$ formula of dimension $d$ and is defined as
    \[\forall_i \Phi \coloneqq \neg \exists_i \neg \Phi.\]
\end{definition}

Note that the constant $i$ of Definition~\ref{definition:existential_quantification_first} can be between $1$ and $d$. Because we use $\exists_i$ later in this thesis only for such $i$ that are in the range of $1$ and $d - r$ the formula works as expected.

\begin{observation}
\label{observation:existential_quantification_first}
		Let $\mathcal{T} = (Q, \Sigma, P, \Delta, v)$ be a reduced LTS with respect to $(q_1, \dots, q_r)$, where $q_1, \dots, q_r \in Q$. The formula given in Definition~
		\ref{definition:existential_quantification_first} defines first-order quantification in PHFL, because it replaces the $i$-th component by one of the last $r$ components and moves 
		through all reachable states. This is enough because any element of $Q$ is reachable in reduced LTS $\mathcal{T}$ from at least one state of 
        $q_1, \dots, q_r$ and those are stored in the last $r$ components of any $d$ tuple. The movement at the $i$-th component and checking if $\Phi$ 
        holds is a consequence of $\mu (X \colon \bullet).\,\Phi \vee \bigvee_{a \in \Sigma} \langle a 
        \rangle_{i} X$. To replace the $i$-th component by one of $q_1, \dots, q_r$ 
        we use this part of the formula $\bigvee_{d}^{j=d-r+1} \{(1, \dots, i-1, j, i + 1, \dots, d)\}$.
\end{observation}

Now we consider second-order quantification. Let $\tau = (\odot, \dots, \odot)$ be an HO type and
$Q$ the set of states of an LTS $\mathcal{T}$. Because the transformation for first-order quantification cannot be adapted to the second-order quantification, we use a different encoding. To obtain second-order
quantification in PHFL we have to iterate over all possible elements of a given domain $D_\tau(Q)$ and
check if the given formula is satisfied. The first thing that we need to iterate over any element of a domain
$D_\tau(Q)$ is an order on $D_\tau(Q)$. If we have the order of $D_\tau(Q)$ we can use
this order to define a formula that returns the successor of a given element of $D_\tau(Q)$ in the scope
of this order. Finally, this formula can be used to iterate over all elements and check if a given formula is
satisfied.
At first, we need the order of domains of type $(\odot, \dots, \odot)$.

To get the order of type $\tau = (\odot, \dots, \odot)$ we define two formulas. The first one describes which is the smaller one of two given sets of type $\tau$. The other one describes the smaller of two tuples of sets of type $\tau$. We say that a tuple $x$ is smaller than $y$ with respect to $
\tau$ if there is an index $i$ such that the element in $x$ on $i$ is smaller than the element in $y$ on $i$ in 
respect to $\odot$, and such that there is no position $j$ on the left-hand side of $i$ such 
that the element in $x$ on $j$ is larger than the element in $y$ on $j$ with respect to $\odot$. 
We say that a set $X$ is smaller than a set $Y$ with respect to $\tau$ if there is an element 
$s_1$ in $X$ that is not in $Y$, and such that all elements $s_2$ that are smaller than $s_1$ with respect to $\odot$ 
are only in $X$ if $s_2$ is also in $Y$. This is formalized in PHFL in the following definition.
\begin{definition}
    \label{definition:lower_bound_less_second}
    Let $d$ and $s$ be the constants as described in Remark~\ref{remark:dimension} where $2*s < d$, then $<^\odot$, $<^{\odot \times \dots
    \times \odot}$ and $<^{(\odot, \dots, \odot)}(X, Y)$ are PHFL$^k$ formulas of dimension $d$ defined as:
    \begin{align*}
        <^\odot \coloneqq &\,\Phi_< \\
        <^{\odot \times \dots \times \odot} \coloneqq
            &\,\underset{i = 1}{\overset{s}{\bigvee}}\{(i, i + s, 3, \dots, d)\} <^\odot \wedge \\
            &\,\underset{j = 1}{\overset{i - 1}{\bigwedge}}\{(j + s, j, 3, \dots, d)\} \neg <^\odot 
    \end{align*}
    \begin{align*}
        <^{(\odot, \dots, \odot)}(X, Y) \coloneqq
            &\,\exists_{i_1}.\, \dots \exists_{i_n}. \{(i_1, \dots, i_n, n + 1,\dots, d)\}Y \wedge \\
            &\,\neg \{(i_1, \dots, i_n, n + 1, \dots, d)\} X\,\wedge \\
            &\, \forall_{j_1}. \,\dots \forall_{j_n}. \{(j_1, \dots, j_n, n+1, \dots, s, \\
            &\,i_1, \dots, i_n, s + n + 1, \dots,  d)\}<^{\odot \times \dots \times \odot} \Rightarrow \\
            &\,(\{(j_1,\dots, j_n, n + 1, \dots, d)\} X \Rightarrow \\
            &\,\neg \{(j_1, \dots, j_n, n + 1, \dots, d)\} Y)
    \end{align*}
\end{definition}
We can see that  $<^{\odot \times \dots \times \odot}$ defines the lexicographic order of HO type $\odot \times \dots \times \odot$ and $<^{(\odot, \dots, \odot)}(X, Y)$ the orders of HO types $(\odot, \dots, \odot)$.

After we have now orders of the HO types $(\odot, \dots, \odot)$ we can define formulas that return the successor of
an input element with respect to the order of $(\odot, \dots, \odot)$. The idea of the following formula is based on
binary incrementation. Let $\tau = (\odot, \dots, \odot)$ be an HO type and $Q$ the set of states of an LTS
$\mathcal{T}$. Remember that a set $X \in D_\tau(Q)$ can be represented by its characteristic
function. This can be transformed to a binary string where each position of this string represents an element of
$D_{\odot}(Q)^n$. Because each position in the binary string represents an element of $D_{\odot}
(Q)^n$ and a position always has to represent the same element in $D_{\odot}(Q)^n$, the elements
in $D_{\odot}(Q)^n$ have to be ordered. The order of the elements of $D_{\odot}(Q)^n$ is a consequence of
the formula $<^{\odot \times \dots \times \odot}$ of Definition~\ref{definition:lower_bound_less_second}. If the position
$i$ in the binary string is $1$, the element with index $i$ in $D_{\odot}(Q)^n$ is
also in $X$. And if the position $i$ in the binary string is $0$, the element with index $i$ in $D_{\odot}
(Q)^n$ is not in $X$. This binary representation of $X$ in regard to $D_{\odot}
(Q)^n$ can be thought of as a function $f \colon D_\tau(Q) \rightarrow 
{0, \dots, |D_\tau(Q)| - 1}$ such that
each element $X$ of $D_\tau(Q)$ will be mapped to its binary string in regard to 
$D_{\odot}(Q)^n$. Similar to operator $F$ of Example~\ref{example:pfp} or operator 
$F_\Phi^\mathcal{T}$ of Example~\ref{example:ho_pfp} the following formula identifies $Y \in 
D_\tau(Q)$ as the successor of $X \in D_\tau(Q)$ if $f(Y) = f(X) + 1 
\mod |D_\tau(Q)|$. In detail that means that the $i$-th bit is $1$ in $f(Y)$ if 
the $i$-th bit is either $1$ in $X$ and there is a bit on the left of $i$ that is $0$ in 
$X$ or the $i$-th bit is $1$ in $f(Y)$ if it is $0$ in $X$ and all bits to the left of $i$ are $1$ in $X
$. 

\begin{definition}
    \label{definition:lower_bounds_next_second}
    Let $d$ and $s$ be the constants as described in Remark~\ref{remark:dimension} where $2*s < d$, then $next^{(\odot, \dots, \odot)}$
    is a PHFL$^k$ formula of dimension $d$ defined as:
    \begin{align*}
        next^{(\odot, \dots, \odot)} \coloneqq &\,\lambda (X \colon \bullet).\, (\neg X \wedge \forall_{s +
        1}\dots\forall_{s + s}<^{\odot \times \dots \times \odot}\, \Rightarrow \\&\,\{(s +
        1, \dots, s + s, s + 1, \dots, d)\} X) \,\vee \\&\,(X \wedge \exists_{s + 1}\dots\exists_{s + s} <^{\odot
        \times \dots \times \odot} \,\wedge \\&\,\{(s + 1, \dots, s + s, s + 1, \dots, d)\}
        \neg X)
    \end{align*}
\end{definition}

\begin{remark}
\label{remark:lower_bounds_next_second}
	Iterated application of $next^{(\odot, \dots, \odot)}$ to $\bot$ cycles through all elements in the domain of $(\odot, \dots, \odot)$. The PHFL formula $\bot$ represents the empty set. If we put $\bot$ into $next^{(\odot, \dots, \odot)}$ we can check whether all elements of $D_{(\odot, \dots, \odot)}$ satisfy one 
	of the subformulas of the disjunction. We will see that only the smallest element with respect to $<^{\odot\times\dots\times\odot}$ satisfies the first subformula. The reason is that the smallest element is not in the empty set and there are no smaller elements than the smallest.
	So the first subformula is true and it is the only element that satisfies the disjunction. So the formula ${next^{(\odot, \dots, \odot)}}^1 \bot$ returns the set that includes only the smallest 
	element with respect to $\odot \times \dots \times \odot$. If we take a look at the formula ${next^{(\odot, \dots, \odot)}}^2 \bot$, we put the set that only contains the smallest element into 
	$next^{(\odot, \dots, \odot)}$. If we dive deeper into this formula we will see that the formula returns a set that includes only the second smallest element. The smallest element does not 
	satisfy the disjunction, but the second smallest satisfies the second subformula. As described in the introduction of Definition~\ref{definition:lower_bounds_next_second}, formula $next^{(\odot, 
	\dots, \odot)}$ can be thought of as binary incrementation. In this manner each possible set of type $(\odot, \dots, \odot)$ will be reached. Note that if we put the full set into $next^{(\odot, 
	\dots, \odot)}$ it returns the empty set.
\end{remark}

With this definition we are now able to define the second-order quantification in PHFL.

\begin{definition}
    \label{definition:existential_quantification_second}
    Let $d$ be the constant as described in Remark~\ref{remark:dimension} and $\Phi$ be a PHFL$^k$ formula with order $0$ variable $X$, then $\exists^{(\odot, \dots, \odot)}X .\,\Phi(X)$
    is a PHFL$^k$ formula of dimension $d$ defined as:
    \[\exists^{(\odot, \dots, \odot)}X.\, \Phi(X) \coloneqq (\mu (F \colon \bullet \rightarrow \bullet).\, \lambda (X
    \colon \bullet).\, \Phi(X) \vee F(next^{(\odot, \dots, \odot)} X)) \bot
    \]
    The formula $\forall^{(\odot, \dots, \odot)}X.\,\Phi$ is also a PHL$^k$ formula of dimension $d$ and is defined as
    \[\forall^{(\odot, \dots, \odot)}X.\,\Phi(X) \coloneqq \neg \exists^{(\odot, \dots, \odot)}X .\,\neg\Phi(X).\]
\end{definition}

In the last step we show that the given formula of Definition~\ref{definition:existential_quantification_second} defines
second-order existential quantification in PHFL.

\begin{lemma}
    \label{lemma:existential_quantifier_second}
    For all HO types $\tau = (\odot, \dots, \odot)$, all variable mappings $\eta$ and all LTS $\mathcal{T}$ it holds that
    \[\llbracket \exists^\tau X.\,\Phi(X)\rrbracket^\eta_\mathcal{T} \equiv \underset{\mathcal{X} \in \llbracket \tau
    \rrbracket_\mathcal{T}}{\bigsqcup} \llbracket \Phi(X) \rrbracket^{\eta[X\rightarrow \mathcal{X}]}_\mathcal{T}.\]
\end{lemma}

\begin{proof}
    By fixpoint unfolding and $\beta$-reduction the formula 
    \[\exists^{(\odot, \dots, \odot)}X.\, \Phi(X) = (\mu (F \colon \bullet \rightarrow \bullet).\, \lambda (X
    \colon \bullet).\, \Phi(X) \vee F(next^{(\odot, \dots, \odot)} X)) \bot\] is equivalent to 
    \[\Phi(\bot) \vee \Phi(next^{(\odot, \dots, \odot)}\bot) \vee \Phi(next^{(\odot, \dots, \odot)} (next^{(\odot, \dots, \odot)} \bot)) \vee \dotsb. \]
    This can be simplified to
    \[\underset{i\geq0}{\bigvee} \Phi({next^{(\odot, \dots, \odot)}}^i \bot)\]
	Because these sets, reached as explained in Remark~\ref{remark:lower_bounds_next_second}, are checked one after another in the scope of $\Phi$ and this iteration is finite 
	because of the least fixpoint operator, the lemma holds.
\end{proof}

\subsection{Higher-Order Quantification}\label{subsec:higher-orderQuantification}

In this subsection we use the encoding of second-order quantification  we just defined to lift this encoding to higher-order
quantification. The idea to obtain higher-order quantification in PHFL is similar to the second-order quantification.
We use the existential quantifier of type $\tau = (\odot, \dots, \odot)$ to define the order of domains of kind
$D_{(\tau, \dots, \tau)}(Q)$. This order can then be used to define a formula that returns the successor
of a given element of $D_{(\tau, \dots, \tau)}(Q)$ with respect to this order. Finally, we use this
formula to define the existential quantifier of type $(\tau, \dots, \tau)$. This procedure will be applied to all
possible types of HO. In this way we get higher-order quantification of any type in PHFL.

The first thing we need is the order for every domain. Note that an order for a type $\tau = (\tau', \dots, \tau')$ always
depends on the quantifiers of type $\tau'$ and the orders of type $\tau' \times \dots \times \tau'$. The orders of type $\tau' \times \dots \times \tau'$, however, depends on the order of type $\tau'$. In case of $\tau' = (\odot, \dots, \odot)$ we use the quantifier formulas of Definition~\ref{definition:existential_quantification_second}. On the other hand if $\tau'$ is a type of an order bigger then $2$, the quantifier formulas of Definition~\ref{definition:existential_quantification_higher} are used. Like we did in the second-order case we define two formulas. The first
one tells us which set of two given sets of type $\tau$ is the smaller one. The other formula tells us the same for two
tuples in the same sets of type $\tau$.

\begin{definition}
    \label{definition:lower_bound_less_higher}
    Let $d$ be the constant as described in Remark~\ref{remark:dimension} and $\tau \neq \odot$ an HO
    type, then $<^{(\tau, \dots, \tau)}(X, Y)$ and $<^{\tau \times \dots \times \tau}(X, Y)$ are PHFL$^k$ formulas
    of dimension $d$ defined as:
    \begin{align*}
        <^{\tau \times \dots \times \tau}(X_1, Y_1, \dots, X_n, Y_n) \coloneqq &\,\underset{i =
        1}{\overset{n}{\bigvee}}<^{\tau}(X_i, Y_i) \wedge \underset{j = 1}{\overset{i - 1}{\bigwedge}}
        \neg <^{\tau}(Y_j, X_j)\\
        <^{(\tau, \dots, \tau)}(X, Y) \coloneqq &\,\exists^{\tau}X_1. \,\dots \exists^{\tau}X_n.\,(\dotsb(Y\,X_1)\dotsb)\,X_n\\
        &\,\wedge \neg (\dotsb(X\,X_1) \dotsb)\,X_n\,\wedge \\&\,\forall^{\tau}Y_1. \,\dots
        \forall^{\tau}Y_n.\,<^{\tau \times \dots \times \tau}
        (Y_1, X_1, \dots, Y_n, X_n) \\&\,\Rightarrow \big((\dotsb(X\,Y_1) \dotsb)\,Y_n \Rightarrow (\dotsb(Y\,Y_1)
         \dotsb)\,Y_n\big).
    \end{align*}
    The formulas $\exists^\tau$ and $\forall^\tau$ are defined in Definition~\ref{definition:existential_quantification_second} or Definition~\ref{definition:existential_quantification_higher} if $ord(\tau) = 2$ or $ord(\tau) > 2$, respectively. In case of $\tau = (\odot, \dots, \odot)$ the formula $<^{(\odot, \dots, \odot)}(X, Y)$ is defined in Definition~\ref{definition:lower_bound_less_second}.
\end{definition}
We can see that $<^{\tau \times \dots \times \tau}$ defines the lexicographic order of HO type $\tau$ and $<^{(\tau, \dots, \tau)}(X, Y)$ the orders of HO types $(\tau, \dots, \tau)$, where $\tau \neq \odot$. Note that $<^{(\tau, \dots, \tau)}(X, Y)$ is well defined because the formula uses the quantifier formulas of a lower type and the lowest possible type is already defined in Definition~\ref{definition:existential_quantification_second}. The same holds for $<^{\tau \times \dots \times \tau}$, where the lowest used order formula is defined in Definition~\ref{definition:lower_bound_less_second}. 

The following definition defines an abbreviation for the smallest function of an arbitrary PHFL type that have an PHFL order of at least $1$.

\begin{definition} 
\label{definition:smallest}
	Let $(\tau, \dots, \tau)$ be an HO type with $\tau \neq \odot$ then $\bot_{T((\tau, \dots, \tau))}$ is PHFL formula defined as
	\[\bot_{T((\tau, \dots, \tau))} \coloneqq (\lambda (X_1 \colon T(\tau)).\, \dots \lambda (X_n \colon T(\tau)).\,\bot)).\] 
    This formula represents a function that maps any input for $X_1, \dots, X_n$ to $\bot$ and $\bot$ 
    represents the empty set. That means this function is the smallest element with respect to $<^{(\tau, \dots, \tau)}$.
\end{definition}

With this definition of order of any HO type, always depending on the lower type existential quantifier, we can define
formulas that return the successor of an input element with respect to the order of the HO type. The idea of the
following formula is similar to the successor formula of Definition~\ref{definition:lower_bounds_next_second} and can be thought of as binary incrementation. Note that here the binary string represents functions. That means that one configuration of a function, represents one bit in the binary string. The smallest function, see Definition~\ref{definition:smallest}, is this where any element is mapped to $\varnothing$ and the greatest function where any element is mapped to $Q^d$, where $Q$ is a state set of an LTS and $d$ a dimension of PHFL. 

\begin{definition}
    \label{definition:lower_bounds_next_higher}
    Let $d$ be the constant as described in Remark~\ref{remark:dimension} and $\tau \neq \odot$ an
    HO type, then $next^{(\tau, \dots, \tau)}$ is a PHFL$^k$ formula of dimension $d$ defined as:

    \begin{align*}
        next^{(\tau, \dots, \tau)} \coloneqq &\,\lambda (X \colon T ((\tau, \dots, \tau))).\,\lambda (X_1 \colon T(\tau)).\, \dotsb \lambda (X_n \colon T(\tau)).\, \\&\,
        \big(\neg (\dotsb(X\,X_1)\dotsb)\,X_n \wedge \forall^{\tau}Y_1.\, \dots \forall^{\tau}Y_n.\,\\&\,<^{\tau \times
        \dots \times \tau}(Y_1, X_1, \dots, Y_n, X_n) \Rightarrow  (\dotsb(X\,Y_1)\dotsb)\,Y_n\big) \,\vee
        \\&\,\big((\dotsb (X\,X_1) \dotsb)\,X_n \wedge \exists^{\tau}Y_1.\, \dots \exists^{\tau}Y_n.\, \\&\,
        <^{\tau \times \dots \times \tau}
        (Y_1, X_1, \dots, Y_n, X_n)\,\wedge \neg (\dotsb(X\,Y_1)\dotsb)\,Y_n\big)
    \end{align*}
\end{definition}

\begin{remark}
\label{remark:lower_bounds_next_higher}
Iterated application of $next^{(\tau, \dots, \tau)}$ to $\bot_{T(\tau)}$ cycles through all elements in the domain of $(\tau, \dots, \tau)$. Note that this works similar to the iterated application of $next^{(\odot, \dots, \odot)}$ to $\bot$ in Remark~\ref{remark:lower_bounds_next_second}.
    If we put $\bot_{T(\tau)}$ into $next^{\tau}$, where $\tau = (\tau', \dots, \tau')$ is an HO type of order $k+1$ and $\tau'\neq \odot$, we can check whether all elements of $D_{\tau}$  satisfy one of the subformulas of the disjunction. Note that the formulas for the quantifiers used in $next^\tau$ and $<^{\tau'\times\dots\times\tau'}$ respectively are all for types of order $k$. By induction hypothesis those formulas define quantification of types of order $k$. Using these quantifiers we will see that only the smallest element with respect to $\tau'\times\dots\times\tau'$ satisfies the first subformula. 
The reason is that the 
    smallest element is mapped in the input function to the empty set and there are no smaller elements to the smallest. 
    That means the first subformula is true and it is the only element that satisfies the disjunction. So the 
    formula ${next^{\tau}}^1 \bot_{T((\tau, \dots, \tau))}$ returns the function that maps the smallest element with respect to $\tau'\times\dots\times\tau'$ to $Q^d$ and all other elements to $\emptyset$ where $d$ is a dimension of PHFL formula $
    \llbracket \exists^\tau X.\,\Phi(X)\rrbracket^\eta_\mathcal{T}$. If we take a look at the formula $
    {next^{\tau}}^2 \bot_{T((\tau, \dots, \tau))}$, the function that maps only the smallest element to $Q^d$ is set as input of
    $next^{\tau}$. If we dive deeper into this formula we will see that now the formula 
    returns a function that maps only the second smallest element to $Q^d$. The smallest element does not 
    satisfy the disjunction, but the second smallest satisfies the second subformula. As described in the 
    introduction of Definition~\ref{definition:lower_bounds_next_second} formula $next^{(\odot, 
    \dots, \odot)}$ is some kind of binary incrementation. The same holds for the formula $next^\tau$ of Definition~\ref{definition:lower_bounds_next_higher}. In this manner each possible function of 
    type $\tau$ will be reached. Note that if we put the function that maps any input to the full set into $next^{(\odot, 
	\dots, \odot)}$ it returns the function that maps any input to the empty set.
\end{remark}

With the previous definition we are now able to define the higher-order quantification in PHFL.

\begin{definition}
    \label{definition:existential_quantification_higher}
    Let $d$ be the constant as described in Remark~\ref{remark:dimension} and let $\tau = (\tau', \dots, \tau')$ be an HO type of order $l$ where $\tau' \neq \odot$. Furthermore, let
    $\Phi$ be a PHFL$^k$ formula with free variable $X$ of order $l - 2$, then $\exists^{\tau}X .\,\Phi(X)$
    is a PHFL$^k$ formula of dimension $d$ defined as
    \begin{align*}
        \exists^{\tau}X.\, \Phi(X) \coloneqq &\,\big(\mu (F \colon T(\tau) \rightarrow \bullet).\, \lambda (X \colon T(\tau)
        ).\,
        \Phi(X)
        \vee F(next^\tau X)\big)\bot_{T(\tau)}.
    \end{align*}
    The formula $\forall^{\tau}X.\,\Phi$ is also a PHFL$^k$ formula of dimension $d$ and is defined as
    \[\forall^{\tau}X.\,\Phi(X) \coloneqq \neg \exists^{\tau}X .\,\neg\Phi(X).\]
\end{definition}

The last step is to show that the given formula of Definition~\ref{definition:existential_quantification_higher} defines
higher-order existential quantification.

\begin{lemma}
    \label{lemma:existential_quantifier_higher}
    For all HO types $\tau$ of order $3$ or greater, all variable mappings $\eta$ and all reduced LTS $\mathcal{T} = (Q, \Sigma, P, \Delta, v)$ holds
    \[\llbracket \exists^\tau X.\,\Phi(X)\rrbracket^\eta_\mathcal{T} \equiv \underset{\mathcal{X} \in \llbracket \tau
    \rrbracket_\mathcal{T}}{\bigsqcup} \llbracket \Phi(X) \rrbracket^{\eta[X\rightarrow \mathcal{X}]}_\mathcal{T}.\]
\end{lemma}

\begin{proof}
    This lemma is proven by induction over the order of type $\tau$. The induction basis $\tau = (\odot, \dots, \odot)$ is given by Lemma~\ref{lemma:existential_quantifier_second}. Based on the induction hypothesis, for any HO type $\tau$ of order $k$, all variable mappings $\eta$ and all reduced LTS $\mathcal{T} = (Q, \Sigma, P, \Delta, v)$ it holds that
    \[\llbracket \exists^\tau X.\,\Phi(X)\rrbracket^\eta_\mathcal{T} \equiv \underset{\mathcal{X} \in \llbracket \tau
    \rrbracket_\mathcal{T}}{\bigsqcup} \llbracket \Phi(X) \rrbracket^{\eta[X\rightarrow \mathcal{X}]}_\mathcal{T}.\]
    We have to show that for any HO type $\tau'=(\tau, \dots, \tau)$ of order $k+1$, all variable mappings $\eta$ and all reduced LTS $\mathcal{T} = (Q, \Sigma, P, \Delta, v)$ the following statement holds
    \[\llbracket \exists^{\tau'} X.\,\Phi(X)\rrbracket^\eta_\mathcal{T} \equiv \underset{\mathcal{X} \in \llbracket \tau'
    \rrbracket_\mathcal{T}}{\bigsqcup} \llbracket \Phi(X) \rrbracket^{\eta[X\rightarrow \mathcal{X}]}_\mathcal{T}.\]    
     By fixpoint unfolding and $\beta$-reduction the formula
    \begin{align*}
        \exists^{\tau'}X.\, \Phi(X) \coloneqq &\,\big(\mu (F \colon T(\tau') \rightarrow \bullet).\, \lambda (X \colon T(\tau')).\,
        \Phi(X)
        \vee F(next^{\tau'} X)\big)\bot_{T(\tau')}
    \end{align*}    
    is equivalent to
    \begin{align*}
        \Phi(\bot_{T(\tau')})\, \vee 
        \Phi(next^{\tau'} \bot_{T(\tau')})\, \vee 
        \Phi(next^{\tau'} next^{\tau'} \bot_{T(\tau')}) \vee \dotsb.
    \end{align*}    
    This can be simplified to
    \[\underset{i\geq0}{\bigvee} \Phi({next^{\tau}}^i \bot_{T(\tau')}).\]
    
	Because these functions, reached as explained in Remark~\ref{remark:lower_bounds_next_higher}, are checked consecutively in the scope of $\Phi$ and this iteration is finite 
	because of the least fixpoint operator, the lemma holds.
\end{proof}