
%%
%% Author: DKron
%% 17.08.2018
%%

\section{Existential Quantifiers in PHFL}\label{sec:existential_quantifiers_in_phfl}

In this section we define PHFL$^{k}$ formulas that describes existential quantification over HO domains of types of
order $k \geq 1$. But before we can define these formulas we have to translate the types.


Note that the most types in HO$^{k + 1}$ do not exist in PHFL$^k$. While HO$^{k +
1}$ includes variables for example of kind set of sets, PHFL$^k$ does not support this kind of type.
But each set $X$ in HO$^{k+1}$ can be described by the characteristic function of $X$ in PHFL$^k$.

The following definition translates all HO types of order $2$ or greater to types in PHFL. The base type of HO
has to be encode differently, and will be establish after this definition.

\begin{definition}
    \label{definition:lower_bound_type_function}
    $T$ is a function that maps any type of HO of order $2$ or greater to a type of PHFL defined inductive over the
    type of HO as follows:
    \begin{align*}
        T((\odot, \dots, \odot)) &= \bullet\\
        T((\tau', \dots, \tau')) &= T(\tau')^+ \rightarrow (T(\tau')^+ \rightarrow \dotsb \rightarrow (T(\tau')^+
        \rightarrow \bullet) \dotsb ),
    \end{align*}
    where $\tau' \neq \odot$.
\end{definition}

Note that the orders of HO types and PHFL types are defined differently. So it holds $ord(T(\tau)) = order(\tau) - 2$
for all HO types $\tau$ with order $2$ or greater.

\begin{example}
    Let $\tau$ be a type of HO
    \[\tau = (\tau', \tau')\]
    with
    \[\tau' = (((\odot)), ((\odot)))\]
    then by Definition~\ref{definition:lower_bound_type_function} of type function $T$
    \[T(\tau) = T(\tau') \rightarrow (T(\tau') \rightarrow \bullet).\]
    with
    \[T(\tau') = (\bullet \rightarrow \bullet) \rightarrow ((\bullet \rightarrow \bullet) \rightarrow \bullet)\]
\end{example}

With this type function $T$ a HO$^{k + 1}$ variable $X$ of type $\tau$ can be translated to a PHFL$^k$ variable
of type $T(\tau)$. Intuitively the variable $X$ which is a set of $D_\tau(\mathcal{U})$ in HO$^{k+1}$ is represented
in PHFL$^k$ as the characteristic function of $X$ over $D_\tau(\mathcal{U})$. Note that the domain of HO types
of order $2$ are similar to the domain of base type of PHFL.

As mentioned above the base type of HO has to be encoded differently. The reason is that the base type in PHFL is a
set of tuples of states and a single state can not be depict directly by a variable. In this thesis the
idea is to use the polyadic logic of PHFL to represent the different first-order variables of an HO$^{k+1}$ formula
$\Psi$. Each first-order variable of $\Psi$ represents one component in the corresponding PHFL$^k$ formula $\Phi$, that
means each variable increases the dimension of $\Phi$. The assignment of a first-order variable $x_i$ in $\Psi$ is
then the current state of the $i$-th component in $\Phi$.

Let $\Phi$ be an HO$^{k+1}$ formula without loss of generality the first-order variables of $\Phi$ are enumerated as
$x_1, \dots, x_f, x_{f + 1}, \dots, x_q$, where $x_1, \dots, x_f$ are the free and $x_{f+1}, \dots, x_q$ are the
quantified variables of $\Phi$.

\subsection{First-Order and Second-Order Quantification}\label{subsec:existentialQuantifiers}

After we know how to interpret the different HO types and variables we are now able to consider the existential
quantification. Before we establish higher-order quantification we look at first-order and second-order quantification.

As mentioned in the introduction of this chapter we encode the bisimulation-invariant fragment of HO(LFP)$^{k + 1}$ and HO(PFP)$^{k+1}$. In Section~\ref{sec:lower_bounds_preparation}  we explained that as a result we can consider reduced LTS, where any state is reachable by at least one of the states $q_1, \dots, q_r$. Because of the total order on states of $\mathcal{T}$ explained in Remark~\ref{remark:transitive_relation} the first-order
quantification can be encoded by moving through all states reachable from one $q_1, \dots, q_r$ and check if we
reach a state tuple where the bounded formula holds.

To access the states $q_1, \dots, q_r$ in the PHFL formulas that we get by encoding HO formulas we use the polyadicity and store $q_1, \dots, q_r$ in components that will be never influenced by the PHFL formulas. The following remark explains that the PHFL formulas has a dimension that is big enough to fulfil all the requirements. 

\begin{remark}
    The PHFL formula $\Phi$ that we get through the encoding of a given HO formula $\Psi$ has dimension
    $d$ that is always big enough to translate all second-order variables of $\Psi$ to an order $0$ variable in
    $\Phi$. In more detail $s$ is the maximal arity of second-order variables in $\Psi$ and $d > s$. To compare
    two  elements of $Q^{s}$, where $Q$ is the state set of an LTS, the dimension $d$ of 
    $\Phi$ is  at leas twice as big as the maximum of $s$. To distinguish all
    different first-order variables in $\Psi$, the dimension $d$ of $\Phi$ has to be extend by $q$, where $q$ is the number of different first-order variables. That means $d >= 2 * s + q$. Finally, to access the 
    states $q_1, \dots, q_r$ described in Section~\ref{sec:lower_bounds_preparation}, 
    we extend $d$ additionally with $r$ components. That means the dimension of $\Phi$ is $d = 2 * s + q + r$.
\end{remark}

If we consider $\exists (x_i \colon \odot).\,\Phi$ then it can be understand as, check if we reach a state tuple where $
\Phi$ holds once the $i$-th component is replaced by one of the last $r$ components. These where $q_1, \dots, q_r$ are stored.

\begin{definition}
    Let $\Psi$ be a HO$^k$ formula with free variables $x_1, \dots, x_f$ and quantified variables $x_{f+1}, \dots,
    x_q$ and $s$ be the maximal arity of all second order variables of $\Psi$, then $\exists_i \Phi$ is a PHFL$^k$
    formula with dimension $d = 2 * s + q+ r$, where $r$ is the number of states that defines the 
    reduced LTS, defined as
    \[\exists_i \Phi \coloneqq \bigvee^{j=d}_{2*s+q+1} \{(1, \dots, i-1, j, i + 1, \dots, d)\} \mu (X
    \colon \bullet).\,\Phi \vee \bigvee_{a \in \Sigma} \langle a \rangle_{i} X.\]
    The formula $\forall_i \Phi$ is also a PHL$^k$ formula with dimension $d$ and is defined as
    \[\forall_i \Phi \coloneqq \neg \exists_i \neg \Phi.\]
\end{definition}

Now we consider the second-order quantification. Let $\tau = (\odot, \dots, \odot)$ a HO type, $\sigma$ a signature and
$\mathcal{U}$ the universe of a $\sigma$ structure. Because the idea we use for first-order quantification is
impossible to adapt to the second-order quantification, we use another encoding. The idea to obtain second-order
quantification in PHFL is that we have to iterate through all possible elements in a domain $D_\tau(\mathcal{U})$ and
check if the given formula is fulfilled. The first thing that we need for iteration over any element of a domain
$D_\tau(\mathcal{U})$ is an order on $D_\tau(\mathcal{U})$. If we have the order of $D_\tau(\mathcal{U})$ we can use
this order to define a formula that returns us the successor of a given element of $D_\tau(\mathcal{U})$ in the scope
of this order. Finally, this formula can be used to iterate through all elements and check if a given formula is
fulfilled.
So the first thing we need is the order of domains of type $(\odot, \dots, \odot)$.

To get the order of type $\tau = (\odot, \dots, \odot)$ we define two formulas. The first tells us given two
sets of type $\tau$ which one is the smaller one. The other formula tells us the same for two tuples in those sets
of type $\tau$. We say that a tuple $x$ is smaller in respect to $\tau$ than a tuple $y$ if there is an index $i$ such
that the element in $x$ on $i$ is smaller in respect to $\odot$ then the element in $y$ on $i$ and there is no position
$j$ left of $i$ such that the element in $x$ on $j$ is bigger in respect to $\odot$ then the element in $y$ on $j$. We
say that a set $X$ is smaller in respect to $\tau$ than a set $Y$ if there is an element $s_1$ in $X$ that is not in $Y$
and all smaller elements $s_2$ to $s_1$ in respect to $\odot$ are only in $X$ if $s_2$ is also in $Y$. This is
formalized in PHFL in the following definition.

\begin{definition}
    \label{definition:lower_bound_less_second}
    Let $\Psi$ be a HO$^k$ formula with free variables $x_1, \dots, x_f$ and quantified variables $x_{f+1}, \dots,
    x_q$ and $s$ be the maximal arity of all second order variables of $\Psi$, then $<^\odot$, $<^{\odot \times \dots
    \times \odot}$ and $<^{(\odot, \dots, \odot)}(X, Y)$ are PHFL$^k$ formulas with dimension $d = 2 * s + q + r$, where $r$ is the number of states that defines the reduced LTS, defined as:

    \begin{align*}
        <^\odot \coloneqq &\,\Phi_< \\
        <^{\odot \times \dots \times \odot} \coloneqq
            &\,\underset{i = 1}{\overset{s}{\bigvee}}\{(i, i + s, 3, \dots, d)\} <^\odot \wedge \\
            &\,\underset{j = 1}{\overset{i - 1}{\bigwedge}}\{(j + s, j, 3, \dots, d)\} \neg <^\odot \\
        <^{(\odot, \dots, \odot)}(X, Y) \coloneqq
            &\,\exists_{i_1}.\, \dots \exists_{i_n}. \{(i_1, \dots, i_n, n + 1,\dots, d)\}Y \wedge \\
            &\,\neg \{(i_1, \dots, i_n, n + 1, \dots, d)\} X\,\wedge \\
            &\, \forall_{j_1}. \,\dots \forall_{j_n}. \{(j_1, \dots, j_n, n+1, \dots, s, \\
            &\,i_1, \dots, i_n, s + n + 1, \dots,  d)\}<^{\odot \times \dots \times \odot} \Rightarrow \\
            &\,(\{(j_1,\dots, j_n, n + 1, \dots, d)\} X \Rightarrow \\
            &\,\neg \{(j_1, \dots, j_n, n + 1, \dots, d)\} Y)
    \end{align*}
\end{definition}

After we have now orders of the HO types $(\odot, \dots, \odot)$ we can define formulas that returns the successor of
an input element in respect to the order of $(\odot, \dots, \odot)$. The idea of the following formula is based on
binary incrementation. Let $\tau = (\odot, \dots, \odot)$ an HO type and $\mathcal{U}$ the universe of a
$\sigma$-structure. Remember that a set $X \in D_\tau(\mathcal{U})$ can be represented by its characteristic
function. This can be transformed to a binary string where each position of this string represents an element of
$D_{\odot}(\mathcal{U})^n$. Because each position in the binary string represents an element of $D_{\odot}
(\mathcal{U})^n$ and a position have always to represent the same element in $D_{\odot}(\mathcal{U})^n$, the elements
in $D_{\odot}(\mathcal{U})^n$ have to be ordered. The order of the elements of $D_{\odot}(\mathcal{U})^n$ is given by
the formula $<^{\odot \times \dots \times \odot}$ of Definition~\ref{definition:lower_bound_less_second}. If the position
$i$ in the binary string is $1$ then this means that the element with index $i$ in $D_{\odot}(\mathcal{U})^n$ is
also in $X$ and if the position $i$ in the binary string is $0$ then the element with index $i$ in $D_{\odot}
(\mathcal{U})^n$ is not in $X$. Note that this binary string is just a visualization of the idea of the data and is
not a directly visible part of the formula. The user of that formula have to enter a set and its elements are
represented in that binary string way. The binary representation of $X$ in regard to $D_{\odot}(\mathcal{U})^n$ can be
extended to a function $f \colon D_\tau(\mathcal{U}) \rightarrow {0, \dots, |D_\tau(\mathcal{U})| - 1}$ such that
each element $X$ of $D_\tau(\mathcal{U})$ will be mapped to its binary string in regard to $D_{\odot}(\mathcal{U})^n$
. With this knowledge we can say that $Y \in D_\tau(\mathcal{U})$ is the direct successor of $X \in D_\tau
(\mathcal{U})$ if $f(Y) = (f(X) + 1)$ modulo $|D_\tau(\mathcal{U})|$. In detail that means that the $i$-th bit is $1$
in $f(Y)$ if it is either $0$ in $X$ and all lower bits are $1$ in $X$ or it is $1$ in $X$ and there is a bit lower
then $i$ that is $0$ in $X$.

\begin{definition}
    \label{definition:lower_bounds_next_second}
    Let $\Psi$ be a HO$^k$ formula with free variables $x_1, \dots, x_f$, quantified variables $x_{f+1}, \dots,
    x_q$ and $s$ be the maximal arity of all second order variables of $\Psi$, then $next^{(\odot, \dots, \odot)}$
    is a PHFL$^k$ formula with dimension $d = 2 * s + q + r$, where  $r$ is the number of states that defines the reduced LTS, defined as:

    \begin{align*}
        next^{(\odot, \dots, \odot)} \coloneqq &\,\lambda (X \colon \bullet).\, (\neg X \wedge \forall_{s +
        1}\dots\forall_{s + s}<^{\odot \times \dots \times \odot}\, \Rightarrow \\&\,\{(s +
        1, \dots, s + s, s + 1, \dots, d)\} X) \,\vee \\&\,(X \wedge \exists_{s + 1}\dots\exists_{s + s} <^{\odot
        \times \dots \times \odot} \,\wedge \\&\,\{(s + 1, \dots, s + s, s + 1, \dots, d)\}
        \neg X)
    \end{align*}
\end{definition}

With the previous definition we are now able to define the second-order quantification in PHFL.

\begin{definition}
    \label{definition:existential_quantification_second}
    Let $\Psi$ be a HO$^k$ formula with free variables $x_1, \dots, x_f$, quantified variables $x_{f+1}, \dots,
    x_q$ and $\Phi$ be a PHFL$^k$ formula with free order $0$ variable $X$. Furthermore, let $s$ be the maximal arity
    of all second order variables of $\Psi$, then $\exists^{(\odot, \dots, \odot)}X .\,\Phi(X)$
    is a PHFL$^k$ formula with dimension $d = 2 * s + q + r$, where $r$ is the number of states that defines the reduced LTS, defined as:
    \[\exists^{(\odot, \dots, \odot)}X.\, \Phi(X) \coloneqq (\mu (F \colon \bullet \rightarrow \bullet).\, \lambda (X
    \colon \bullet).\, \Phi(X) \vee F(next^{(\odot, \dots, \odot)} X)) \bot
    \]
    The formula $\forall^{(\odot, \dots, \odot)}X.\,\Phi$ is also a PHL$^k$ formula with dimension $d$ and is defined as
    \[\forall^{(\odot, \dots, \odot)}X.\,\Phi(X) \coloneqq \neg \exists^{(\odot, \dots, \odot)}X .\,\neg\Phi(X).\]
\end{definition}

The last step is to show that the given formula of Definition~\ref{definition:existential_quantification_second} defines
second-order existential quantification in PHFL.

\begin{lemma}
    \label{lemma:existential_quantifier_second}
    For all HO types $\tau = (\odot, \dots, \odot)$, all variable mappings $\eta$ and all LTS $\mathcal{T}$ holds
    \[\llbracket \exists^\tau X.\,\Phi(X)\rrbracket^\eta_\mathcal{T} \equiv \underset{\mathcal{X} \in \llbracket \tau
    \rrbracket_\mathcal{T}}{\bigsqcup} \llbracket \Phi(X) \rrbracket^{\eta[X\rightarrow \mathcal{X}]}_\mathcal{T}.\]
\end{lemma}

\begin{proof}
    By fixpoint unfolding and $\beta$-reduction the formula 
    \[\exists^{(\odot, \dots, \odot)}X.\, \Phi(X) = (\mu (F \colon \bullet \rightarrow \bullet).\, \lambda (X
    \colon \bullet).\, \Phi(X) \vee F(next^{(\odot, \dots, \odot)} X)) \bot\] is equivalent to 
    \[\Phi(\bot) \vee \Phi(next^{(\odot, \dots, \odot)}\bot) \vee \Phi(next^{(\odot, \dots, \odot)} next^{(\odot, \dots, \odot)} \bot) \vee \dotsb. \]
    This can be simplified with
    \[\underset{i\geq0}{\bigvee} \Phi({next^{(\odot, \dots, \odot)}}^i \bot)\]
Note that $\bot$ represents always the empty set. If we put $\bot$ into $next^{(\odot, \dots, \odot)}$ we check all elements that can be in $D_{(\odot, \dots, \odot)}$ if those fulfil one of the disjuncts. We will see that only the smallest element fulfills the first disjunct. Also the smallest element is not in the empty set and there are no smaller elements to the smallest. So the first disjunct is true and it is the only element that fulfil one of the disjuncts. So the formula ${next^{(\odot, \dots, \odot)}}^1 \bot$ returns the set that includes only the smallest element with respect to $\odot \times \dots \times \odot$. If we look at the formula ${next^{(\odot, \dots, \odot)}}^2 \bot$, we put the set that includes the smallest element into $next^{(\odot, \dots, \odot)}$. If we look into this formula we will see that now the formula returns a set that includes only the second smallest element. The smallest element does not fulfil any disjunct, but the second smallest fulfills the second disjunct. As described in the introduction of Definition~\ref{definition:lower_bounds_next_second} formula $next^{(\odot, \dots, \odot)}$ is some kind of binary incrementation. In this manner each possible set of type $(\odot, \dots, \odot)$ will be reached. Finally, because these sets are checked each for itself in the scope of $\Phi$, the lemma holds.
\end{proof}

\subsection{Higher-Order Quantification}\label{subsec:higher-orderQuantification}

In this subsection we use the second-order existential quantification to define the higher-order
quantification. The idea to obtain higher-order quantification in PHFL is similar to the higher-order quantifaction.
We use the existential quantifier of type $\tau = (\odot, \dots, \odot)$ to define the order of domains of kind
$D_{(\tau, \dots, \tau)}(\mathcal{U})$. This order can then be used to define a formula that returns us the successor
of a given element of $D_{(\tau, \dots, \tau)}(\mathcal{U})$ in respect to this order. Finally, we use this
formula to define the existential quantifier of type $(\tau, \dots, \tau)$. This procedure will be lift up to all
possible types of HO. In this way we get higher-order quantification of any type in PHFL.

So the first thing we need is the order of any domain. Note that a order for type $\tau = (\tau', \dots, \tau')$ always
depends on the existential quantifiers of type $\tau'$. As in the second-order case we define two formulas. The first
tells us given two sets of type $\tau$ which one is the smaller one. The other formula tells us the same for two
tuples in those sets of type $\tau$.

\begin{definition}
    \label{definition:lower_bound_less_higher}
    Let $\Psi$ be a HO$^k$ formula with free variables $x_1, \dots, x_f$ and quantified variables $x_{f+1}, \dots,
    x_q$. Furthermore, let $s$ be the maximal arity of all second order variables of $\Psi$ and $\tau \neq \odot$ a HO
    type, then $<^{(\tau, \dots, \tau)}(X, Y)$ and $<^{\tau \times \dots \times \tau}(X, Y)$ are PHFL$^k$ formulas
    with dimension $d = 2 * s + q + r$, where $r$ is the number of states that defines the reduced LTS, defined as:
    \begin{align*}
        <^{\tau \times \dots \times \tau}(X_1, Y_1, \dots, X_n, Y_n) \coloneqq &\,\underset{i =
        1}{\overset{n}{\bigvee}}<^{\tau}(X_i, Y_i) \wedge \underset{j = 1}{\overset{i - 1}{\bigwedge}}
        \neg <^{\tau}(Y_j, X_j)\\
        <^{(\tau, \dots, \tau)}(X, Y) \coloneqq &\,\exists^{\tau}X_1. \,\dots \exists^{\tau}X_n.\,(\dotsb(Y\,X_1)\dotsb)\,X_n\\
        &\,\wedge \neg (\dotsb(X\,X_1) \dotsb)\,X_n\,\wedge \\&\,\forall^{\tau}Y_1. \,\dots
        \forall^{\tau}Y_n.\,<^{\tau \times \dots \times \tau}
        (Y_1, X_1, \dots, Y_n, X_n) \\&\,\Rightarrow \big((\dotsb(X\,Y_1) \dotsb)\,Y_n \Rightarrow (\dotsb(Y\,Y_1)
         \dotsb)\,Y_n\big)
    \end{align*}
\end{definition}

After we have now orders of any HO type, always depending on the lower type existential quantifier, we can define
formulas that returns the successor of an input element in respect to the order of the HO type. The idea of the
following formula is similar to the successor formula of Definition~\ref{definition:lower_bounds_next_second} and is
based on binary incrementation. Where here the binary string representation is also not directly visible in the
formula. The user of that formula have to enter a function and the image of the elements of the origin set of this
function represents $0$ or $1$ in that binary string.

\begin{definition}
    \label{definition:lower_bounds_next_higher}
    Let $\Psi$ be a HO$^k$ formula with free variables $x_1, \dots, x_f$ and quantified variables $x_{f+1}, \dots,
    x_q$. Furthermore, let $s$ be the maximal arity of all second order variables of $\Psi$ and $\tau \neq \odot$ a
    HO type, then $next^{(\tau, \dots, \tau)}$ is a PHFL$^k$ formula with dimension $d = 2 * s + q + r$, where $r$ is the number of states that defines the reduced LTS, defined as:

    \begin{align*}
        next^{(\tau, \dots, \tau)} \coloneqq &\,\lambda (X \colon T ((\tau, \dots, \tau))).\,\big(\neg (\dotsb(X\,X_1)\dotsb)\,X_n \\&\, \wedge \forall^{\tau}Y_1.\, \dots \forall^{\tau}Y_n.\,<^{\tau \times
        \dots \times \tau}(Y_1, X_1, \dots, Y_n, X_n) \\&\,\Rightarrow  (\dotsb(X\,Y_1)\dotsb)\,Y_n\big) \,\vee
        \\&\,\big((\dotsb (X\,X_1) \dotsb)\,X_n \wedge \exists^{\tau}Y_1.\, \dots \exists^{\tau}Y_n.\, \\&\,
        <^{\tau \times \dots \times \tau}
        (Y_1, X_1, \dots, Y_n, X_n)\,\wedge \neg (\dotsb(X\,Y_1)\dotsb)\,Y_n\big)
    \end{align*}
\end{definition}

With the previous definition we are now able to define the higher-order quantification in PHFL.

\begin{definition}
    \label{definition:existential_quantification_higher}
    Let $\Psi$ be a HO$^k$ formula with free variables $x_1, \dots, x_f$, quantified variables $x_{f+1}, \dots,
    x_q$ and let $\tau = (\tau', \dots, \tau')$ be a HO type of order $l$ where $\tau' \neq \odot$. Furthermore, let
    $\Phi$ be a PHFL$^k$
    formula with free variable $X$ of order $l - 2$ and let $s$ be the maximal arity
    of all second order variables of $\Psi$, then $\exists^{\tau}X .\,\Phi(X)$
    is a PHFL$^k$ formula with dimension $d = 2 * s + q + r$, where $r$ is the number of states that defines the reduced LTS,  defined as:
    \begin{align*}
        \exists^{\tau}X.\, \Phi(X) \coloneqq &\,(\mu (F \colon T(\tau) \rightarrow \bullet).\, \lambda (X \colon T(\tau)
        ).\,
        \Phi(X)
        \vee F(next^\tau X))\\&\,(\lambda (X_1 \colon T(\tau')).\, \dots \lambda (X_n \colon T(\tau')).\,\bot).
    \end{align*}
    The formula $\forall^{\tau}X.\,\Phi$ is also a PHL$^k$ formula with dimension $d$ and is defined as
    \[\forall^{\tau}X.\,\Phi(X) \coloneqq \neg \exists^{\tau}X .\,\neg\Phi(X).\]
\end{definition}

The last step is to show that the given formula of Definition~\ref{definition:existential_quantification_higher} defines
higher-order existential quantification.

\begin{lemma}
    \label{lemma:existential_quantifier_higher}
    For all HO types $\tau$ of order $3$ or greater, all variable mappings $\eta$ and all reduced LTS $\mathcal{T} = (Q, \Sigma, P, \Delta, v)$ holds
    \[\llbracket \exists^\tau X.\,\Phi(X)\rrbracket^\eta_\mathcal{T} \equiv \underset{\mathcal{X} \in \llbracket \tau
    \rrbracket_\mathcal{T}}{\bigsqcup} \llbracket \Phi(X) \rrbracket^{\eta[X\rightarrow \mathcal{X}]}_\mathcal{T}.\]
\end{lemma}

\begin{proof}
    This lemma is proven by induction over the order of type $\tau$. The induction basis $\tau = (\odot, \dots, \odot)$ is given by Lemma~\ref{lemma:existential_quantifier_second}. By induction hypothesis it holds for any HO type $\tau$ of order $k$, all variable mappings $\eta$ and all reduced LTS $\mathcal{T} = (Q, \Sigma, P, \Delta, v)$
    \[\llbracket \exists^\tau X.\,\Phi(X)\rrbracket^\eta_\mathcal{T} \equiv \underset{\mathcal{X} \in \llbracket \tau
    \rrbracket_\mathcal{T}}{\bigsqcup} \llbracket \Phi(X) \rrbracket^{\eta[X\rightarrow \mathcal{X}]}_\mathcal{T}.\]
    We have to show that for any HO type $\tau'=(\tau, \dots, \tau)$ of order $k+1$, all variable mappings $\eta$ and all reduced LTS $\mathcal{T} = (Q, \Sigma, P, \Delta, v)$ it holds
    \[\llbracket \exists^{\tau'} X.\,\Phi(X)\rrbracket^\eta_\mathcal{T} \equiv \underset{\mathcal{X} \in \llbracket \tau'
    \rrbracket_\mathcal{T}}{\bigsqcup} \llbracket \Phi(X) \rrbracket^{\eta[X\rightarrow \mathcal{X}]}_\mathcal{T}.\]    
     By fixpoint unfolding and $\beta$-reduction the formula
    \begin{align*}
        \exists^{\tau'}X.\, \Phi(X) \coloneqq &\,(\mu (F \colon T(\tau') \rightarrow \bullet).\, \lambda (X \colon T(\tau')).\,
        \Phi(X)
        \vee F(next^{\tau'} X))\\&\,(\lambda (X_1 \colon T(\tau)).\, \dots \lambda (X_n \colon T(\tau)).\,\bot)
    \end{align*}    
    is equivalent to
    \begin{align*}
        &\Phi((\lambda (X_1 \colon T(\tau)).\, \dots \lambda (X_n \colon T(\tau)).\,\bot))\, \vee \\
        &\Phi(next^{\tau'} (\lambda (X_1 \colon T(\tau)).\, \dots \lambda (X_n \colon T(\tau)).\,\bot))\, \vee \\
        & \Phi(next^{\tau'} next^{\tau'} (\lambda (X_1 \colon T(\tau)).\, \dots \lambda (X_n \colon T(\tau)).\,\bot)) \vee \dotsb
    \end{align*}    
    This can be simplified with
    \[\underset{i\geq0}{\bigvee} \Phi({next^{\tau}}^i (\lambda (X_1 \colon T(\tau)).\, \dots \lambda 
    (X_n \colon T(\tau)).\,\bot))\]
    Note that $(\lambda (X_1 \colon T(\tau)).\, \dots \lambda (X_n \colon T(\tau)).\,\bot))$ 
    represents a function that maps any input for $X_1, \dots, X_n$ to $\bot$ and $\bot$ 
    represents the empty set. This function is interpreted as the smallest function while $
    (\lambda (X_1 \colon T(\tau)).\, \dots \lambda (X_n \colon T(\tau)).\,\top))$ represents the 
    biggest function that maps any input to $Q^d$, where $d$ is a dimension of PHFL formula $
    \llbracket \exists^\tau X.\,\Phi(X)\rrbracket^\eta_\mathcal{T}$.
    
    If we put $(\lambda (X_1 \colon T(\tau)).\, \dots \lambda (X_n \colon T(\tau)).\,\bot))$ into $next^{\tau}$ we check all elements that can be in $D_{\tau}$ if those fulfil one of 
    the disjuncts. Note that the quantifiers formulas used in $next^\tau$ and $<^{\tau'\times\dots\times\tau'}$ respectively are all for types of order $k$. By induction hypothesis those formulas are defining quantification of types of order $k$. Using those quantifiers we will see that only the smallest element with respect to $\tau'\times\dots\times\tau'$ fulfils the first disjunct. Also the 
    smallest element is mapped in the input function to the empty set and there are no smaller elements to the smallest. 
    That means the first disjunct is true and it is the only element that fulfil one of the disjuncts. So the 
    formula ${next^{\tau}}^1 (\lambda (X_1 \colon T(\tau)).\, \dots \lambda (X_n \colon T(\tau)).\,\bot)$ returns the function that maps the smallest element with respect to $\tau'\times\dots\times\tau'$ to $Q^d$ and all other elements to $\emptyset$. If we look at the formula $
    {next^{\tau}}^2 (\lambda (X_1 \colon T(\tau)).\, \dots \lambda (X_n \colon T(\tau)).\,\bot)$, we put the function that maps only the smallest element to $Q^d$ into 
    $next^{\tau}$. If we look into this formula we will see that now the formula 
    returns a function that maps only the second smallest element to $Q^d$. The smallest element does not 
    fulfil any disjunct, but the second smallest fulfills the second disjunct. As described in the 
    introduction of Definition~\ref{definition:lower_bounds_next_second} formula $next^{(\odot, 
    \dots, \odot)}$ is some kind of binary incrementation. The same holds for the formula $next^\tau$ of Definition~\ref{definition:lower_bounds_next_higher}. In this manner each possible function of 
    type $\tau$ will be reached. Finally, because these functions are checked each for 
    itself in the scope of $\Phi$, the induction is complete and the lemma holds.
\end{proof}