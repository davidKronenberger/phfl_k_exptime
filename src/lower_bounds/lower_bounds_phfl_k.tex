%%
%% Author: DKron
%% 17.08.2018
%%

\section{Lower Bound of PHFL$^k$}\label{sec:lowerBoundOfPhfl}

As mentioned in the introduction of this chapter we can show that the lower bound of PHFL$^k$ is \exptime{$k$} by
making a detour over HO(LFP)$^{k+1}$. This and following ideas are oriented on~\cite{lange2014capturing} where it was
shown that PHFL$^1$ captures \exptime{$1$}. At first we will see that it was proven that HO(LFP)$^{k +
1}$ coincides with $k$-fold exponential time over finite and ordered structures. To use this we have to encode the
bisimulation invariant fragment of HO$^{k+1}$ into PHFL$^k$. Therefore, we define an abbreviation for the HO(LFP)
formulas that uses the LFP operator first and then a function that use this abbreviation and those of
Section~\ref{sec:existential_quantifiers_in_phfl} to map an HO(LFP)$^{k+1}$ formula to a PHFL$^k$ formula. Finally, we
show that for any bisimulation-invariant HO(LFP)$^{k+1}$ formula $\Phi$ there is a transformed PHFL$^k$ formula $\Psi$ such that the query $\mathcal{Q}^d_\Psi$ defined by $\Psi$ can be obtained from the query $\mathcal{Q}^f_\Phi$ defined by $\Phi$ via projection to the relevant components. Hence, PHFL$^k$ can define all queries defined by the bisimulation-invariant fragment of HO(LFP)$^{k+1}$.

\begin{theorem}{\cite{freireMartins2011descriptive}}\label{theorem:hoLfpEqualsExptime}
    For all $k \geq 1$, HO(LFP)$^{k + 1}$ captures $k$-EXPTIME over finite and ordered structures.
\end{theorem}

The proof follows the idea to encode the run of a $k$-EXPTIME Turing Machine $M$ by a formula $\Phi$ of HO(LFP)$^{k +
1}$ in such a way that $M$ accepts the standardized encoding of $\mathcal{A}$ and $\eta$ iff $\mathcal{A}, \eta \models \Phi$. On the other hand, such a model-checking problem can be solved by a $k$-EXPTIME Turing Machine $M_\Phi$.

Because Theorem~\ref{theorem:hoLfpEqualsExptime} holds, it is also possible to prove that the lower bound of PHFL$^k$
is in \exptime{$k$} by encoding the bisimulation invariant fragment of HO(LFP)$^{k + 1}$ into PHFL$^k$. To encode the
bisimulation invariant fragment of HO(LFP)$^{k + 1}$ into PHFL$^k$ we have to define a function that transforms a HO
(LFP)$^{k + 1}$ formula into a PHFL$^k$ formula. Note that the types and variables of an HO formula also need to be
transformed. See Section~\ref{sec:existential_quantifiers_in_phfl} for further details.

Before we consider the definition of the transforming function, we define a PHFL formula for HO formulas that uses the
LFP operator.

\begin{definition}
\label{definition:lfp_in_phfl}
   Let $d$ be the constant as described in Remark~\ref{remark:dimension} and let $X$ be an HO variable of HO type $\tau = (\tau', \dots, \tau')$ where $\tau' \neq \odot$. Furthermore, let
    $\Phi$ be a PHFL$^k$ formula, then $LFP^\tau X.\,\Phi$ is a PHFL$^k$ formula with dimension $d$, defined as:
    \[LFP^\tau X.\,\Phi \coloneqq \mu (X \colon T(\tau)).\,\Phi(X).\]
    In case of $\tau = (\odot, \dots, \odot)$ let
    \[LFP^{\tau} X.\,\Phi \coloneqq \mu (X \colon \bullet).\,\Phi(X).\]
\end{definition}

Now we are able to define the function in the following definition by using the abbreviations of Definitions~\ref{definition:existential_quantification_second},~\ref{definition:existential_quantification_higher},~\ref{definition:existential_quantification_first} and~\ref{definition:lfp_in_phfl}. The function translates a bisimulation invariant HO(LFP)$^{k+1}$ formula to a PHFL$^k$ formula.

\begin{definition}
    \label{definition:lower_bounds_phfl_formula_function}
   Define $F$ as the function that maps a bisimulation invariant HO(LFP)$^{k+1}$ formula $\varphi$ to a PHFL$^k$ formula with dimension $d$, where $d$ and $s$ are the constants as described in Remark~\ref{remark:dimension} and $\Phi_\sim$ is the formula of Example~\ref{example:phfl_order_0}. In detail, $F$ is defined
    inductive over $\varphi$ as follows:
    \begin{align*}
        F(p(X_i)) \coloneqq &\, p_{2s+i} \\
        F(a(X_i, X_j)) \coloneqq &\, \langle a \rangle_{2s+i} \{(2s+i, 2s+j, \\
        &\,3, \dots, d)\} \Phi_\sim \\
        F(\Phi \vee \Psi) \coloneqq &\, F(\Phi) \vee F(\Psi) \\
        F(\neg \Phi) \coloneqq &\, \neg F(\Phi) \\
        F(\exists (X_i \colon \odot).\,\Phi) \coloneqq &\, \exists_{2s+i} F(\Phi) \\
        F(\exists (X \colon \tau).\,\Phi) \coloneqq &\, \exists^\tau X.\,F(\Phi(X)) \\
        F([LFP\;\Phi(X, X_{i_1}, \dots, X_{i_n})](V_{j_1}, \dots, V_{j_n})) \coloneqq &\,\{(j_1, \dots, j_n, n + 1, \dots, d)\} \\
        &\,LFP^{(\odot, \dots, \odot)} X.\, F(\Phi) \\
        F([LFP\;\Phi(X, X_1, \dots, X_n)](V_1, \dots, V_n)) \coloneqq &\,(\dotsb \big(LFP^\tau X.\, F(\Phi)\big)\,V_1)\dotsb)\,V_n \\
        F(X(X_{i_1}, \dots, X_{i_n})) \coloneqq &\, \{(2s+i_1, \dots, 2s+i_n, \\
        &\,n + 1, \dots, d)\}X\\
        F(X(X_1, \dots, X_n)) \coloneqq &\, (\dotsb (X\,X_1)\dotsb)\,X_n
    \end{align*}
\end{definition}
Keep in mind that we are working on LTS, that means that the relations in the signatures for HO(LFP) formulas have either arity one or two. Relations with arity one represent the 
propositions and those with arity two the actions of an LTS.

\subsection{Variables}\label{subsec:lower_bounds_variables}

After we encoded HO(LFP)$^{k+1}$ syntactically into PHFL$^k$ formulas, the last step is to translate the 
interpretation of variables. As described in Section~\ref{sec:existential_quantifiers_in_phfl} the variables in HO of 
types with order $3$ or higher are not supported in PHFL. Also first-order variables are not supported. Therefore, we 
define a function that maps a given variable mapping for a HO formula to the correct variable mapping for PHFL 
semantics. This function ignores the mapping of first-order variables, maps second-order variables to sets and 
higher-order variables to the corresponding characteristic function. Note, that the sets of order $2$ in HO have order 
$0$ in PHFL. The first-order variables of an HO formula that are marked as undefined in the following function, can be 
mapped to any arbitrary value because we do not use them in PHFL directly.

    Normally, the HO variables of order $2$ or higher are syntactically equivalent to the variables used in the translated PHFL formulas. To distinguish them in this definition an HO variable $X$ is denoted by $\hat{X}$ for the usage in PHFL context. 

\begin{definition}
    \label{definition:lower_bound_variable_function}
    Let $d$ be the constant as described in Remark~\ref{remark:dimension}, let $\eta$ be a
    variable mapping for a HO$^k$ formula and let $\mathcal{T} = (Q, \Sigma, P, \Delta, v)$ be a reduced LTS, then $\eta_V$ is a 
    variable mapping for a PHFL$^k$ formula with dimension $d$. Then the variable mapping $\eta_V$ is defined as:
    \[\eta_V(\hat{X})\coloneqq
    \begin{cases}
        \text{undefined}, & \text{if } X \text{ is of type } \odot \\
        A,  & \text{if } X \text{ is of type } (\odot, \dots, \odot)\\
        G, & \text{if } X \text{ is of type } (\tau, \dots, \tau) \text{ and } \tau \neq \odot,
    \end{cases}\]
    where $A \subseteq Q^d$ such that $(q_1, \dots, q_n, q_{n + 1}, \dots, q_d) \in A$ iff $(q_1, \dots, q_n) \in
    \eta(X)$ and $G$ is a function of type $T((\tau, \dots, \tau))$ defined as follows:
    \begin{align*}
        (\dotsb\big(G\,\eta_V(\hat{X_1})\big)\dotsb)\,\eta_V(\hat{X_n}) &= Q^d &\text{ iff } (\eta(X_1), \dots, \eta(X_n)) \in \eta(X)\\
        (\dotsb\big(G\,\eta_V(\hat{X_1})\big)\dotsb)\,\eta_V(\hat{X_n}) &= \emptyset &\text{ iff } (\eta(X_1), \dots, \eta(X_n))
        \not\in \eta(X)
    \end{align*}
\end{definition}

The following example shows how a set of higher type variables will be translated into the characteristic function via the
variable mapping $\eta_V$ of Definition~\ref{definition:lower_bound_variable_function}.

\begin{example}
    Let $\mathcal{T} = (Q, \Sigma, P, \Delta, v)$ be a reduced LTS with the set of states $Q = \{1, 2, 3\}$, let $X$ be a HO(LFP)$^{k + 1}$
    variable of type $((\odot, \odot), (\odot, \odot))$ mapped through variable mapping $\eta$ to
    \[\eta(X) = \{(\{(1, 1)\}, \{(2, 2)\}), (\{(1, 1), (2, 2)\}, \{(3, 3)\})\},\]
    let $d$ be a dimension of PHFL, then $\eta_V(X)$ is a PHFL$^k$ function of type $\bullet \rightarrow (\bullet \rightarrow \bullet)$ such
    that $\eta_V(X)$ $(\{(1, 1)\}) = f$, $\eta_V(X)(\{(1, 1), (2, $ $2)\}) = g$ and $\eta_V(X)(z) = h$ for $z \in
    Q^d$ where $z \neq \{(1, 1)\}$ and $z \neq \{(1, 1), (2, 2)\}$. Moreover, $f$, $g$ and $h$ are functions of type $\bullet
    \rightarrow \bullet$ where $f(\{(2, 2)\}) = g(\{(3, 3)\}) = Q^d$ and $f(z) = g(z') = h(z'') = \emptyset$ for $z,
    z', z'' \in Q^d$ where $z \neq \{(2, 2)\}$ and $z' \neq \{(3, 3)\}$.
\end{example}

\subsection{Correctness Proof}\label{subsec:lower_bounds_correctness_lfp}

The last step is to show such that the query that can be defined by the PHFL$^k$ formula $\Psi$ that is obtained by transforming a HO(LFP)$^{k+1}$ formula $\Phi$ can be obtained from the query defined by $\Phi$ via projection to the relevant components. As mentioned in Section~\ref{sec:lower_bounds_preparation}, without loss of generality, the statement can be proven by considering only reduced LTS. 

To make the correctness proof clearer there is one last remark.

\begin{remark}
    It holds for any HO(LFP)$^{k+1}$ formula $\Phi$ that for PHFL$^k$ formula $F(\Phi)$ the type judgment $\varnothing \vdash
    F(\Phi) \colon \bullet$ is derivable. This statement
    is easily proved by induction over the structure of formula $\Phi$. 
\end{remark}

Because the type judgement is always derivable we ignore the type environment in the following proof and write just $\llbracket \Phi \rrbracket^\eta_\mathcal{T}$ instead of $\llbracket \Gamma \vdash F(\Phi) \colon \tau \rrbracket^\eta_\mathcal{T}$, where $\Phi$ is a PHFL formula, $\eta$ is a variable mapping, $\mathcal{T}$ is an LTS, $\Gamma$ is a type environment and $\tau$ is a PHFL type.

\begin{theorem}
    \label{theorem:ho_lfp_equals_phfl}
    Let $\Phi$ be a bisimulation-invariant formula of HO(LFP)$^{k + 1}$, $\Psi$ a formula of PHFL$^k$ and let $d, f \geq 1$ and $k > 0$. For every $\Phi$ there is a $\Psi$ such that a projection on the $d$-adic query $\mathcal{Q}_\Phi^d$ that is defined by $\Phi$ is equal to the $f$-adic query $\mathcal{Q}_\Psi^f$ that is defined by $\Psi$.
\end{theorem}

\begin{proof}
    This lemma can be proven by showing for all HO(LFP)$^{k+1}$ formulas $\Phi$ with first-order variables $X_1,
    \dots, X_q$, all reduced LTS $\mathcal{T} = (Q, \Sigma, P,
    \Delta, v)$ with respect to $\emph{q}_r = q_1, \dots, q_r$ and all variable mappings $\eta$ that it holds that $\mathcal{T}, \eta \models \Phi$ iff $\emph{q} =
    (\emph{q}_s, \emph{q}_s, \emph{q}_q, \emph{q}_r)$ and $\emph{q} \in \llbracket
   F(\Phi)\rrbracket^{\eta_V}_{\mathcal{T}}$. Here $\emph{q}_s = q_1',\dots,q_s'$ is a sequence of $s$ placeholders used for the interaction of second-order variables, $\emph{q}_q = \eta(X_1), \dots, \eta(X_q)$ is a sequence of first-order variables that are mapped by $\eta$ where $\eta(X_i) = q_0$ if $X_i$ is bound to a quantifier, $q_0, q_1', \dots, q_s' \in Q$ are arbitrary states, $F$ is the formula function of
    Definition~\ref{definition:lower_bounds_phfl_formula_function} and $\eta_V$ the variable mapping of
    Definition~\ref{definition:lower_bound_variable_function}. This statement can be proven by induction over $\Phi$.
    \begin{compactitem}
        \item In case of $\Phi = p(X_i)$ where $X_i$ is a free first-order variable then $\mathcal{T}, \eta \models \Phi$ holds exactly then if $\eta(X_i) \in
        p^\mathcal{T}$. Translated to the normal LTS definition of $\mathcal{T}$ it is the same as $p \in v(\eta(X_i))$.
        With $F(\Phi) = p_{2s+i}$ this is exactly
        \begin{align*}
            (\emph{q}_s, \emph{q}_s, \eta(X_1),& \dots, \eta(X_{i-1}), \eta(X_{i}), \eta(X_{i+1}), \dots, \eta(X_q), \emph{q}_r) \in
            \llbracket F(\Phi) \rrbracket^{\eta_V}_\mathcal{T}.
        \end{align*}
        
        \item In case of $\Phi = a(X_i, X_j)$ where $X_i$ and $X_j$ are free first-order variables such that, without loss of generality, $i < j$. All other cases working similar. Then $\mathcal{T}, \eta \models \Phi$ holds exactly then if $(\eta(X_i)
        , \eta(X_j))$ $ \in a^\mathcal{T}$. Translated to the normal LTS definition of $\mathcal{T}$ it is the same as $
        \eta(X_i) \overset{a}{\rightarrow} \eta(X_j)$.
        By  definition of the semantics of $\langle a \rangle_{2s+i}$ the tuple
        \begin{align*}
            (\emph{q}_s, &\emph{q}_s, \eta(X_1), \dots, \eta(X_{i- 1}), \eta(X_{i}), \eta(X_{i+1}), \dots, \eta(X_{j-1}), \eta
            (X_{j}),\\& \eta(X_{j+1}), \dots, \eta(X_q), \emph{q}_r)
        \end{align*}
        is an element of the semantics of $\langle a \rangle_{2s+i} \Phi_\sim$ iff
        \begin{align*}
            (\emph{q}_s, &\emph{q}_s, \eta(X_1), \dots, \eta(X_{i - 1}), q_m, \eta(X_{i+1}), \dots, \eta(X_{j-1}), q_n,\\& \eta(X_{j
            +1}), \dots, \eta(X_q), \emph{q}_r)
        \end{align*}
        is an element of the semantics of $\Phi_\sim$ where from $\eta(X_i)$ there is an $a$-action to $q_m$. Because $\eta(X_j)$ is the state that
        has to be reachable via an $a$ action from $\eta(X_i)$ we have to check if $q_n = \eta(X_j)$. If two states are equal in a reduced LTS it is
        the same to check if these two states are bisimilar. $\sim$ is given by formula $\Phi_\sim$ of
        Example~\ref{example:phfl_order_0}.
        Because this formula returns those $d$-tuples where the first and second component are bisimilar, we have to
        move the $2s+i$-th and $2s+j$-th component to the first and second component. This is given by $\{(2s+i, 2s+j, 3, \dots, d
        )\} \Phi_\sim$. Summarizing all these steps with $F(\Phi) = \langle a \rangle_{2s+i} \{(2s+i, 2s+j, 3, \dots, d)\}
        \Phi_\sim$ it follows
        \begin{align*}
            (\emph{q}_s, &\emph{q}_s, \eta(X_1), \dots, \eta(X_{i - 1}), \eta(X_{i}), \eta(X_{i+1}), \dots, \eta(X_{j-1}), \eta
            (X_{j}),\\& \eta(X_{j+1}), \dots, \eta(X_q), \emph{q}_r) \in \llbracket F(\Phi) \rrbracket^{\eta_V}_\mathcal{T}
        \end{align*}
        if $(\eta(X_i), \eta(X_j))$ $ \in a^\mathcal{T}$ and
        \begin{align*}
            (\emph{q}_s, &\emph{q}_s, \eta(X_1), \dots, \eta(X_{i - 1}), \eta(X_{i}), \eta(X_{i+1}), \dots, \eta(X_{j-1}), \eta
            (X_{j}),\\& \eta(X_{j+1}), \dots, \eta(X_q), \emph{q}_r) \not\in \llbracket F(\Phi) \rrbracket^{\eta_V}_\mathcal{T}
        \end{align*}
        if $(\eta(X_i), \eta(X_j))$ $ \not\in a^\mathcal{T}$.

        \item In case of $\Phi = X(X_{i_1}, \dots, X_{i_n})$ where $X$ is a free variable of HO type $(\odot, \dots,
        \odot)$ and $X_{i_1}, \dots, X_{i_n}$ are free first-order variables such that, without loss of generality, $i_1 < i_2, \dots, i_{n-1} < i_n$. All other cases working similar. Then $\mathcal{T}, \eta \models \Phi$
        holds exactly then if $(\eta(X_{i_1}), \dots \eta(X_{i_n})) \in \eta(X)$. Because of definition of $\eta_V$ the
        tuple $(\eta(X_{i_1}), \dots \eta(X_{i_n}), q_{n + 1}', \dots,$ $ q_{s}', \emph{q}_s, \emph{q}_q, \emph{q}_r)$ is in $
        \eta_V(X)$ if $(\eta(X_{i_1}),\dots \eta(X_{i_n})) \in \eta(X)$ and is not in $\eta_V(X)$ otherwise. Because
         components $1, \dots, n$ are not set to the mappings of first-order variables $X_{i_1}, \dots, X_{i_n}$, we first move the components $2s+i_1, \dots, 2s+i_n$ to components $1, \dots, n$ respectively and check then if
        \[(\eta(X_{i_1}), \dots \eta(X_{i_n}), q_{n + 1}', \dots, q_{s}', \emph{q}_s, \emph{q}_q, \emph{q}_r) \in \eta_V(X).\]
        So it holds with $F(\Phi) = {(2s+i_1, \dots, 2s+i_n, n+1, \dots, d)}X$ that
        \begin{align*}
            (\emph{q}_s, &\emph{q}_s, \eta(X_1), \dots, \eta(X_{i_1-1}), \eta(X_{i_1}), \eta(X_{i_1+1}), \dots, \eta(X_{i_n-1}), \eta(X_{i_n}), \\&\eta(X_{i_n+1}), \dots, \eta
            (X_q), \emph{q}_r) \in \llbracket F(\Phi) \rrbracket^{\eta_V
            }_\mathcal{T}
        \end{align*}
        if $(\eta(X_{i_1}), \dots \eta(X_{i_n})) \in \eta(X)$ and
        \begin{align*}
            (\emph{q}_s, &\emph{q}_s, \eta(X_1), \dots, \eta(X_{i_1-1}), \eta(X_{i_1}), \eta(X_{i_1+1}), \dots, \eta(X_{i_n-1}), \eta(X_{i_n}), \\&\eta(X_{i_n+1}), \dots, \eta
            (X_q), \emph{q}_r) \not \in \llbracket F(\Phi) \rrbracket^{\eta_V
            }_\mathcal{T}
        \end{align*}
        if $(\eta(X_{i_1}), \dots \eta(X_{i_n})) \not\in \eta(X)$.

        \item In case of $\Phi = X(X_1, \dots, X_n)$ where $X$ is a free variable of HO type $(\tau, \dots,
        \tau)$ and $X_1, \dots, X_n$ are free variables of HO type $\tau$ then $\mathcal{T}, \eta \models \Phi$
        holds exactly then if $(\eta(X_1), \dots \eta(X_n)) \in \eta(X)$. Because of definition of $\eta_V$ it follows
        \[(\dotsb\big(\eta_V(X)\,\eta_V(X_1)\big)\dotsb)\,\eta_V(X_n) = Q^d\]
        if $(\eta(X_1), \dots \eta(X_n)) \in \eta(X)$ and
        \[(\dotsb\big(\eta_V(X)\,\eta_V(X_1)\big)\dotsb)\,\eta_V(X_n) = \emptyset\]
        if $(\eta(X_1), \dots \eta(X_n)) \not\in \eta(X)$. With $F(\Phi) = (\dotsb (X\,X_1)\dotsb)\,X_n$
        it follows
        \[ \emph{q} \in \llbracket F(\Phi) \rrbracket^{\eta_V}_\mathcal{T} = Q^d.\]
        if $(\eta(X_1), \dots \eta(X_n))\in \eta(X)$ and
        \[ \emph{q} \not\in \llbracket F(\Phi)
        \rrbracket^{\eta_V}_\mathcal{T} = \emptyset.\]
        if $(\eta(X_1), \dots \eta(X_n)) \not\in \eta(X)$.
    \end{compactitem}
    
    By induction hypothesis it holds for HO(LFP)$^{k+1}$ formulas $\Psi$ and $\Psi'$ with first-order variables $X_1,
    \dots, X_q$, all reduced LTS $\mathcal{T} = (Q, \Sigma, P,
    \Delta, v)$ with respect to $\emph{q}_r$ and all variable mappings $\eta$  that $\mathcal{T}, \eta \models \Psi$ iff $\emph{q} \in \llbracket
   F(\Psi)\rrbracket^{\eta_V}_{\mathcal{T}}$ and $\mathcal{T}, \eta \models \Psi'$ iff $\emph{q} \in \llbracket
   F(\Psi')\rrbracket^{\eta_V}_{\mathcal{T}}$, where $\emph{q} =
    (\emph{q}_s, \emph{q}_s, \emph{q}_q, \emph{q}_r)$.
    
    \begin{compactitem}
        \item In case of $\Phi = \neg \Psi$ it follows that $\mathcal{T}, \eta \models \Phi$ exactly then if
        $\mathcal{T}, \eta \not\models \Psi$. By induction hypothesis that is exactly then the case when
        \[ \emph{q} \not\in \llbracket F(\Psi) \rrbracket^{\eta_V}_\mathcal{T}.\]
        This is exactly the case if
        \[ \emph{q} \in Q^d \setminus \llbracket F(\Psi) \rrbracket^{\eta_V}_\mathcal{T}.\]
        And this is exactly the semantics of $F(\Phi) = \neg F(\Psi)$.

        \item In case of $\Phi = \Psi \vee \Psi'$ it follows that $\mathcal{T}, \eta \models \Phi$ exactly then if
        $\mathcal{T}, \eta \models \Psi$ or $\mathcal{T}, \eta \models \Psi'$. By induction hypothesis that is
        exactly then the case when
        \[\emph{q} \in \llbracket F
        (\Psi) \rrbracket^{\eta_V}_\mathcal{T}\]
        or
        \[\emph{q} \in \llbracket F
        (\Psi') \rrbracket^{\eta_V}_\mathcal{T}.\]
        Because $\sqcup_{\bullet} = \cup$ this can be combined to
        \[\emph{q} \in \llbracket F
        (\Psi) \rrbracket^{\eta_V}_\mathcal{T} \sqcup_\bullet \llbracket F
        (\Psi') \rrbracket^{\eta_V}_\mathcal{T},\]
        which is as desired.

        \item In case of $\Phi = \exists (X_i\colon \odot).\,\Psi$ it follows that $\mathcal{T}, \eta \models \Phi$ iff
        there exists  $\mathcal{X} \in Q$ with $\mathcal{T}, \eta' \models \Psi$, where $\eta'$ is a variable mapping with $\eta'(x) = \eta(x)$ for all variables $x \neq X_i$ and $\eta'(X_i) = \mathcal{X}$. By induction hypothesis it holds that $\mathcal{T}, \eta' \models \Psi$ is exactly the case when
        \begin{align*}
            (\emph{q}_s, \emph{q}_s, \eta'(X_1), \dots, \eta'(X_{i-1}), \eta'(X_i), \eta'(X_{i+1}), \dots, \eta(X_{q}), \emph{q}_r) \in
            \llbracket F(\Psi) \rrbracket^{{\eta'_V}}_\mathcal{T}.
        \end{align*}
        To reach the value of $\eta'(X_i)$ we have to replace the $2s+i$-th component by one of 
        the last $r$ components and move through all reachable states. By Observation~\ref{observation:existential_quantification_first} the formula defined in Definition~\ref{definition:existential_quantification_first} fulfils this behaviour. Because the first-order variable $X_i$ is represented by the $2s+i$-th component and $F(\Phi) = \exists_{2s+i} \Psi$, we replace in $F(\Phi)$ the $2s+i$-th component by one of the last $r$ components, move through all reachable states and checking if $F(\Psi)$ holds. That means it holds 
        \[\emph{q} \in \llbracket F
        (\Phi) \rrbracket^{\eta_V}_\mathcal{T}\]
        iff $\mathcal{T}, \eta \models \Phi$.

        \item In case of $\Phi = \exists (X \colon \tau).\,\Psi$ it follows that $\mathcal{T}, \eta \models \Phi$ iff
        there exists $\mathcal{X} \in D_\tau(Q)$ with $\mathcal{T}, \eta' \models \Psi$, where $\eta'$ is a variable mapping with $\eta'(x) = \eta(x)$ for all variables $x \neq X$ and $\eta'(X) = \mathcal{X}$.
        By induction hypothesis it follows that $\emph{q} \in
        \llbracket F(\Psi) \rrbracket^{{\eta'_V}}_\mathcal{T}$ iff $\mathcal{T}, \eta' \models \Psi$. By Lemma~\ref{lemma:existential_quantifier_higher} the formula $\exists^\tau X.\, \Psi(X)$ is semantically equivalent to
        $\underset{\mathcal{X} \in \llbracket \tau \rrbracket_\mathcal{T}}{\bigsqcup} \llbracket \Psi(X) \rrbracket^{\eta'}_\mathcal{T}$.
        It follows with $F(\Phi) = \exists^\tau X.\, F(\Psi)(X)$ that it holds \[\emph{q} \in
        \llbracket F(\Phi) \rrbracket^{\eta_V}_\mathcal{T}\]
        iff $\mathcal{T},\eta \models \Phi$.
        
        \item In case of $\Phi = [LFP\,\Psi(X, X_{i_1}, \dots, X_{i_n})](V_{j_1}, \dots, V_{j_n})$, where $X$ is a
        free variable in $\Psi$ of HO type $(\odot, \dots, \odot)$ and $X_{i_1}, \dots, X_{i_n}$ are free first-order
        variables of $\Psi$ and $V_{j_1}, \dots, V_{j_n}$ are first-order variables of $\Phi$ such that, without loss of generality, $i_1 < i_2, j_1 < j_2, \dots, i_{n-1} < i_ n, j_{n-1} < j_n$. All other cases working similar. Then it follows that
        $\mathcal{T}, \eta \models \Phi$ exactly if $(\eta(V_{j_1}), \dots, \eta(V_{j_n})) \in LFP
        (F_\Psi^\mathcal{T})$. By definition of LFP the tuple $(\eta(V_{j_1}), \dots, \eta(V_{j_n}))$ is in
        $LFP(F_\Psi^\mathcal{T})$ iff $X$ is the smallest $X$ such that $X = F_\Psi^\mathcal{T}(X)$ and $(\eta(V_{j_1}), \dots, \eta(V_{j_n})) \in
        F_\Psi^\mathcal{T}(X)$. By definition of $F_\Psi^\mathcal{T}(X)$ it holds $(\eta
        (V_{j_1}), \dots, \eta(V_{j_n})) \in F_\Psi^\mathcal{T}(X)$ exactly then if $\mathcal{T}, \eta' 
        \models \Psi$, where $\eta'$ is a variable mapping with $\eta'(x) = \eta(x)$ for all variables $x \neq X$ and $\eta'(X) = F_\Psi^\mathcal{T}(X)$. By induction hypothesis this is exactly the case if
        \begin{align*}
        (\emph{q}_s, &\emph{q}_s, \eta'(X_1), \dots, \eta'(X_{i_1-1}), \eta'(X_{i_1}), \eta'(X_{i_1+1}), \dots, \eta'(X_{i_n-1}), \\&\eta'(X_{i_n}), \eta'(X_{i_n+1}), \dots, \eta'
            (X_q), \emph{q}_r) \in \llbracket
        F(\Psi) \rrbracket^{\eta'_V}_\mathcal{T}.
        \end{align*}
        
        The next step is to show that it holds $\eta'(X) = LFP(F^\mathcal{T}_\Psi)$ exactly then if $\llbracket F(\Psi) \rrbracket^{\eta'_V}_\mathcal{T} = \llbracket \mu (X \colon \bullet).\,F(\Psi)\rrbracket_\mathcal{T}^{\eta_V}$. 
        By Theorem~\ref{theorem:kleene} the least fixpoint of $F^\mathcal{T}_\Psi$ can be calculated by a sequence $X_0, \dots, X_m$ where here $X_0 = \emptyset$ and $X_{i+1} 
        = F_\Psi^\mathcal{T}(X_i)$ and $\eta'(X) = X_m$. On the other hand $\llbracket \mu (X \colon \bullet).\,F(\Psi)\rrbracket_\mathcal{T}^{\eta_V}$ can be calculated by a sequence 
         $Y_0, \dots, Y_{m'}$ where here $Y_0 = \varnothing$ and $Y_{i+1} = \llbracket F(\Psi)\rrbracket_\mathcal{T}^{\eta_V[X \mapsto Y_i]}$ and $Y_{m'} = \llbracket F(\Psi)\rrbracket_\mathcal{T}^{\eta'_V}$. Furthermore, let $\eta'^0, \dots, \eta'^m$ be a sequence of variable mappings where $\eta'^i(x) = \eta(x)$ for all $x \neq X$ and $\eta'^i(X) = X_i$. It is easy to verify that $\eta'^m = \eta'$. For any $\eta'^i$ we get by Definition~\ref{definition:lower_bound_variable_function} the variable mapping $\eta'^i_V$ and so a sequence of variable mappings $\eta'^0_V, \dots, \eta'^m_V$. Moreover, $\eta^{j+1}_V$ is $\llbracket F(\Psi)\rrbracket_\mathcal{T}^{\eta'^j_V}$. Because $\eta'^0_V(X) = \varnothing$, $\eta'^1_V(X) =\llbracket F(\Psi)\rrbracket_\mathcal{T}^{\eta'^0_V}$ and so on, it follows that $\eta'^m_V(X) = \llbracket F(\Psi) \rrbracket^{\eta'_V}_\mathcal{T}$ if and only if $\eta'(X)$ is the least fixpoint of $F^\mathcal{T}_\Psi$.
         
Because of the construction of variable mapping $\eta'_V$ and $(\eta'(V_{j_1}), \dots, 
        $ $\eta'(V_{j_n})) \in \eta'(X)$ the
        tuple $(\eta'(V_{j_1}), \dots, \eta'(V_{j_n}), q_{n+1}', \dots, q_s', \emph{q}_s, \emph{q}_q, \emph{q}_r)$ is also in $\eta'_V(X)$.         
        
       Because components $1, \dots, n$ are not set to the mappings of first-order variables 
       $V_{j_1}, \dots, V_{j_n}$, we first move the components $2s+j_1, \dots, 2s+j_n$ to components $1, \dots, n$ respectively and check then the least fixpoint operator.
        So it holds with $F(\Phi) = \{2s+j_1, \dots, 2s+j_n, n+1, \dots, d\} \mu (X).\, F(\Psi)$ that
        \begin{align*}
            (\emph{q}_s, \emph{q}_s, \eta(X_1),& \dots, \eta(V_{j_1-1}), \eta(V_{j_1}), \eta(V_{j_1+1}), \dots \eta(V_{j_n-1}), \\&
             \eta(V_{j_n}), \eta(V_{j_n+1}), \dots, \eta
            (X_q), \emph{q}_r) \in \llbracket  F(\Phi) \rrbracket^{\eta_V
            }_\mathcal{T}
        \end{align*}
        exactly then if $\mathcal{T}, \eta \models \Phi$.

        \item In case of $\Phi = [LFP\,\Psi(X, X_1, \dots, X_n)](V_1, \dots, V_n)$, where $X$ is a
        free variable in $\Psi$ of HO type $(\tau, \dots, \tau)$ and $X_1, \dots, X_n$ are free
        variables of $\Psi$ of type $\tau$ and $V_1, \dots, V_n$ are free variables of $\Phi$ also of type $\tau$, then
        it follows that $\mathcal{T}, \eta \models \Phi$ exactly then if $(\eta(V_1), \dots, \eta(V_n) \in LFP
        (F_\Psi^\mathcal{T})$. By definition of LFP the tuple $(\eta(V_1), \dots, \eta(V_n))$ is in
        $LFP(F_\Psi^\mathcal{T})$ iff $X$ is the smallest $X$ such that $X = F_\Psi^\mathcal{T}(X)$ and $(\eta(V_1), \dots, \eta(V_n)) \in
        F_\Psi^\mathcal{T}(X)$. By definition of $F_\Psi^\mathcal{T}(X)$ it holds $(\eta
        (V_1), \dots, \eta(V_n)) \in F_\Psi^\mathcal{T}(X)$ exactly then if $\mathcal{T}, \eta' 
        \models \Psi$, where $\eta'$ is a variable mapping with $\eta'(x) = \eta(x)$ for all variables $x \neq X$ and $\eta'(X) = F_\Psi^\mathcal{T}(X)$. By induction hypothesis this is exactly the case if
        \[\emph{q} \in \llbracket  F(\Psi)
        \rrbracket^{\eta_V}_\mathcal{T}.\]

The next step is to show that it holds $\eta'(X) = LFP(F^\mathcal{T}_\Psi)$ exactly then if $\llbracket F(\Psi) \rrbracket^{\eta'_V}_\mathcal{T} = \llbracket \mu (X \colon T((\tau, \dots, \tau))).\,F(\Psi)\rrbracket_\mathcal{T}^{\eta_V}$. 
        By Theorem~\ref{theorem:kleene} the least fixpoint of $F^\mathcal{T}_\Psi$ can be calculated by a sequence $X_0, \dots, X_m$ where here $X_0 = \emptyset$ and $X_{i+1} 
        = F_\Psi^\mathcal{T}(X_i)$ and $\eta'(X) = X_m$. On the other hand $\llbracket \mu (X \colon T((\tau, \dots, \tau)).$ $\,F(\Psi)\rrbracket_\mathcal{T}^{\eta_V}$ can be calculated by a sequence 
         $Y_0, \dots, Y_{m'}$ where here $Y_0 = \bot_{T((\tau, \dots, \tau))}$ and $Y_{i+1} = \llbracket F(\Psi)\rrbracket_\mathcal{T}^{\eta_V[X \mapsto Y_i]}$ and $Y_{m'} = \llbracket F(\Psi)\rrbracket_\mathcal{T}^{\eta'_V}$. Furthermore, let $\eta'^0, \dots, \eta'^m$ be a sequence of variable mappings where $\eta'^i(x) = \eta(x)$ for all $x \neq X$ and $\eta'^i(X) = X_i$. It is easy to verify that $\eta'^m = \eta'$. For any $\eta'^i$ we get by Definition~\ref{definition:lower_bound_variable_function} the variable mapping $\eta'^i_V$ and so a sequence of variable mappings $\eta'^0_V, \dots, \eta'^m_V$. Moreover, $\eta^{j+1}_V$ is $\llbracket F(\Psi)\rrbracket_\mathcal{T}^{\eta'^j_V}$. Because $\eta'^0_V(X) = \bot_{T((\tau, \dots, \tau))}$, $\eta'^1_V(X) =\llbracket F(\Psi)\rrbracket_\mathcal{T}^{\eta'^0_V}$ and so on, it follows that $\eta'^m_V(X) = \llbracket F(\Psi) \rrbracket^{\eta'_V}_\mathcal{T}$ if and only if $\eta'(X)$ is the least fixpoint of $F^\mathcal{T}_\Psi$.
        
                Because of the construction of variable mapping $\eta'_V$ and $(\eta'(V_1), \dots, $ $\eta'(V_n)) \in \eta'(X)$ it holds
        $(\dotsb\big(\eta'_V(X)\,\eta'_V(V_1)\big) \dotsb )\,\eta'_V(V_n) = Q^d$
        and so 
        \begin{align*}
        \emph{q} \in (\dotsb \big(\eta'_V(X)\,\eta'_V(V_1)\big) \dotsb )\,\eta'_V(V_n).
        \end{align*}    

        With $F(\Phi) = (\dotsb \big(\mu (X \colon T((\tau, \dots, \tau))).\,\Phi(X)\big)\,V_1)\dotsb)\,V_n$ it follows
        \[\emph{q} \in \llbracket
         F(\Phi) \rrbracket^{\eta_V}_\mathcal{T}.\]
        exactly then if $\mathcal{T}, \eta \models \Phi$.
    \end{compactitem}
\end{proof}

\begin{remark}
In the proof of Theorem~\ref{theorem:ho_lfp_equals_phfl} the dimension $d$ of $F(\Phi)$ is $2s+q+r$ but only the components where the free variables are represented are filled with input parameters. That means by only projecting these components in the by $F(\Phi)$ defined query $\mathcal{Q}^d_{F(\Phi)}$ we get the resulting query $\mathcal{Q}^f_{\Phi}$ that is defined by $\Phi$. 
\end{remark}

The combination of Theorem~\ref{theorem:ho_lfp_equals_phfl}, Theorem~\ref{theorem:hoLfpEqualsExptime} and 
Theorem~\ref{theorem:phfl_k_in_k_exptime} proves the following theorem for $k>0$. For $k = 0$ and $k = 1$ this statement was proven by M. Otto in~\cite{otto1999bisimulation} and by M. Lange and E. Lozes in~\cite{lange2014capturing}.

\begin{theorem}
    Let $k \geq 0$. PHFL$^k$ captures \exptime{$k$} over finite labelled transition systems.
\end{theorem}

