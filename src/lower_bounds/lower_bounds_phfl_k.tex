
%%
%% Author: DKron
%% 17.08.2018
%%

\section{Lower Bound of PHFL$^k$}\label{sec:lowerBoundOfPhfl}

As mentioned in the introduction of this chapter we can show that the lower bound of PHFL$^k$ is \exptime{$k$} by
make a detour over HO(LFP)$^{k+1}$. This and following ideas are oriented on~\cite{lange2014capturing} where was
shown that PHFL$^1$ captures \exptime{$1$}. The first thing we will see is that it was proven that HO(LFP)$^{k +
1}$ coincides with $k$-fold exponential time over finite and ordered structures. To use this we have to encode the
bisimulation invariant fragment of HO$^{k+1}$ into PHFL$^k$. For this we first define an abbreviation for the HO(LFP)
formulas that uses the LFP operator. Next, we define a function that uses this abbreviation and these of
Section~\ref{sec:existential_quantifiers_in_phfl} to map a HO(LFP)$^{k+1}$ formula to a PHFL$^k$ formula. Finally, we
show that the semantics of the HO(LFP)$^{k+1}$ formula coincides with the transformed PHFL$^k$ formula.

\begin{theorem}{\cite{freireMartins2011descriptive}}\label{theorem:hoLfpEqualsExptime}
    For all $k \geq 1$, HO(LFP)$^{k + 1} = k$-EXPTIME over finite and ordered structures.
\end{theorem}

The proof follows the idea to encode the run of a $k$-EXPTIME Turing Machine $M$ by a formula $\phi$ of HO(LFP)$^{k +
1}$ in this way such that $M$ accepts $\mathcal{A}$ iff $\mathcal{A} \models \phi$. On the other hand each HO(LFP)$^{k +
1}$ formula $\phi$ can be evaluated by a $k$-EXPTIME Turing Machine $M_\phi$.

Because Theorem~\ref{theorem:hoLfpEqualsExptime} holds it is also possible to prove that the lower bound of PHFL$^k$
is in \exptime{$k$} by encoding the bisimulation invariant fragment of HO(LFP)$^{k + 1}$ into PHFL$^k$. To encode the
bisimulation invariant fragment of HO(LFP)$^{k + 1}$ into PHFL$^k$ we have to define a function that transforms a HO
(LFP)$^{k + 1}$ formula to a PHFL$^k$ formula. Note that the types and variables of an HO formula have to be also
transformed. See Section~\ref{sec:existential_quantifiers_in_phfl} for further details.

Before we give the definition of the transforming function, we define a PHFL formula for HO formulas that uses the
LFP operator.

\begin{definition}
    For any HO variable $X$ of HO type $\tau = (\tau', \dots, \tau')$ where $\tau' \neq \odot$ and any HO variables
    $x_1, v_1, \dots, $ $x_n, v_n$ HO type $\tau'$ and any PHFL formula $\Phi$ let
    \[LFP^\tau X.\,\Phi \coloneqq (\dots ((\mu (X \colon T(\tau)).\,\Phi(X, x_1, \dots, x_n))(v_1))(v_2))\dots)(v_n).\]
    In case of $\tau = (\odot, \dots, \odot)$ and $v_1, \dots, v_n$ have index $i_1, \dots, i_n$ respectively let
    \[LFP^{\tau} X.\,\Phi \coloneqq \{(i_1, \dots, i_n, n + 1, \dots, d)\} \mu (X \colon \bullet).\,\Phi(X).\]
\end{definition}

Remember that we have defined the signatures of HO(LFP) relational. Because we are working on LTS, the relations in
the signatures have either arity one or two. Relations with arity one represents the propostions and those with arity
two the actions of an LTS. Now we are able to define the function to translate a bisimulation invariant HO(LFP)$^{k+1}$
formula to a PHFL$^k$.

\begin{definition}
    Let $F$ be the function that maps a bisimulation invariant HO(LFP)$^{k+1}$ formula to a PHFL$^k$ formula defined
    inductive on bisimulation invariant HO(LFP)$^{k+1}$ formula $\Phi$ as follows:
    \begin{align*}
        F(p(x_i)) \coloneqq &\, p_i \\
        F(a(x_i, x_j)) \coloneqq &\, \langle a \rangle_i \{(i, j, 3, \dots, d)\} \Phi_\sim \\
        F(\Phi \vee \Psi) \coloneqq &\, F(\Phi) \vee F(\Psi) \\
        F(\neg \Phi) \coloneqq &\, \neg F(\Phi) \\
        F(\exists (x_i \colon \odot).\,\Phi) \coloneqq &\, \exists_i F(\Phi) \\
        F(\exists (X \colon \tau).\,\Phi) \coloneqq &\, \exists^\tau X.\,F(\Phi(X)) \\
        F([LFP\;\Phi(X, x_1, \dots, x_n)](v_1, \dots, v_n)) \coloneqq &\,LFP^\tau X.\, F(\Phi) \\
        F(X(x_1, \dots, x_n)) \coloneqq &\, (\dots ((X(x_1))(x_2))\dots)(x_n)\\
        F(X(x_{i_1}, \dots, x_{i_n})) \coloneqq &\, \{(i_1, \dots, i_n, n + 1, \dots, d)\}X
    \end{align*}
\end{definition}

The last step is to show that the semantics of a given HO(LFP)$^{k+1}$ formula coincides with the semantics of the by
function $F$ encoded PHFL$^k$ formula.

\begin{lemma}
    \label{lemma:ho_lfp_equals_phfl}
    Let $r \geq 1$ and $k \geq 0$. For every bisimulation invariant formula $\Phi$ of HO(LFP)$^{k + 1}$ there is a
    PHFL$^k$ formula $\Psi$ such that $\mathcal{Q}_\Phi^r = \mathcal{Q}_\Psi^r$.
\end{lemma}

\begin{proof}
    The usage of first-order variable $x_i$ in proposition $p$ in HO can be interpreted such that the variable
    assignment of $x_i$ have to fulfill $p$. In PHFL that is equal to that state where the $i$-th component is
    located fulfills $p$.

    The usage of first-order variables $x_i$ and $x_j$ in action $a$, i.e. $a(x_i, x_j)$, can be interpreted as an
    $a$-movement from assigned state of $x_i$ to assigned state of $x_j$. In PHFL that is equal to an $a$-movement from
    component $i$ to component $j$ such that the state where component $j$ is located is reachable from the state where
    component $i$ is located. That means $a(x_i, x_j)$ is equal to $\langle a \rangle_i\{(1, \dots, d) \leftarrow (i, j,
    3, \dots, d)\} \Phi_\sim$ where $\Phi_\sim$ is the formula from Example~\ref{example:phfl_order_0} that defines $\sim$.
\end{proof}

The combination of Lemma~\ref{lemma:ho_lfp_equals_phfl}, Theorem~\ref{theorem:hoLfpEqualsExptime} and
Theorem~\ref{theorem:phfl_k_in_k_exptime} proves the following theorem.

\begin{theorem}
    Let $k \geq 0$. PHFL$^k$ captures \exptime{$k$} over labeled transition systems.
\end{theorem}

