
%%
%% Author: DKron
%% 17.08.2018
%%

\section{Lower Bound of PHFL$^k$}\label{sec:lowerBoundOfPhfl}

We have to show that if a query is in \exptime{$k$}, then it can be expressed by a PHFL$^k$ formula. It is
possible to show this by encoding the run of a $k$-EXPTIME Turing Machine by a query. Because this kind of proof is
arduous we prove this by make a detour over HO(LFP)$^k$. This and following ideas are oriented
on~\cite{lange2014capturing}. In~\cite{lange2014capturing} was shown that PHFL$^1$ captures \exptime{$1$}.

It was proven that HO(LFP)$^{k + 1}$ coincides with $k$-fold exponential time over finite and ordered structures.

\begin{theorem}{\cite{freireMartins2011descriptive}}\label{theorem:hoLfpEqualsExptime}
    For all $k \geq 1$, HO(LFP)$^{k + 1} = k$-EXPTIME over finite and ordered structures.
\end{theorem}

The proof follows the idea to encode the run of a $k$-EXPTIME Turing Machine $M$ by a formula $\phi$ of HO(LFP)$^{k +
1}$ in this way such that $M$ accepts $\mathcal{A}$ iff $\mathcal{A} \models \phi$. On the other hand each HO(LFP)$^{k +
1}$ formula $\phi$ can be evaluated by a $k$-EXPTIME Turing Machine $M_\phi$.

Because Theorem~\ref{theorem:hoLfpEqualsExptime} holds it is also possible to prove that the lower bound of PHFL$^k$
is in \exptime{$k$} by encoding the bisimulation invariant fragment of HO(LFP)$^{k + 1}$ into PHFL$^k$. At first
remember that the signatures of formulas of HO(LFP) are all relational. Because we are working on LTS, the
relations in the signatures have either arity one or two. Relations with arity one represents the propostions and
those with arity two the actions of an LTS.

To encode the bisimulation invariant fragment of HO(LFP)$^{k + 1}$ into PHFL$^k$ we have to define a function that
transforms a HO(LFP)$^{k + 1}$ formula to a PHFL$^k$ formula. Furthermore, we have to translate the variables. The
reason is that a variable in HO(LFP)$^{k + 1}$ can have different types then a variable in PHFL$^k$. While HO(LFP)
$^{k + 1}$ includes variables for example of kind set of sets, PHFL$^k$ has not this kind of type. The following
definition translates all HO types with order $2$ or greater. The first-order variables have to be encode
differently, and will be regarded later.

\begin{definition}
    \label{definition:lower_bound_type_function}
    Let $T$ be a function that maps a type of HO to a type of PHFL defined inductive over the type
    of HO as follows:

    \begin{align*}
        T((\odot, \dots, \odot)) &= \bullet\\
        T((\tau', \dots, \tau')) &= T(\tau')^+ \rightarrow (T(\tau')^+ \rightarrow \dots \rightarrow (T(\tau')^+
        \rightarrow \bullet) \dots )
    \end{align*}
\end{definition}

Note that $ord(T(\tau)) = order(\tau) - 2$ for all $\tau$ with order $2$ or greater.

\begin{example}
    Let $\tau$ be a type of HO
    \[\tau = ((((\odot)), ((\odot))), (((\odot)), ((\odot))))\]
    then by Definition~\ref{definition:lower_bound_type_function} of type function $T$
    \[T(\tau) = ((\bullet \rightarrow \bullet) \rightarrow ((\bullet \rightarrow \bullet) \rightarrow \bullet))
    \rightarrow (((\bullet \rightarrow \bullet) \rightarrow ((\bullet \rightarrow \bullet) \rightarrow \bullet))
    \rightarrow \bullet).\]
\end{example}

With this type function $T$ a HO(LFP)$^{k + 1}$ variable $X$ of type $\tau$ can be translated to a PHFL$^k$ variable
with type $T(\tau)$. Please note that $\tau$ has order $2$ or greater. As mentioned above PHFL$^k$ has not HO types like
set of sets. But PHFL$^k$ has functions. Each set $A$ can be seen as a special function $f\colon X \rightarrow \{0,
1\}$. The origin set $X$ is a superset of $A$. Then function $f$ maps those elements $x$ of $X$ to $1$ if $x \in A$
and all other elements in $X$ to $0$. Those functions are called characteristic functions.

\begin{definition}
    \label{definition:lower_bound_variable_function}
    Let $Y$ a HO variable of type $\tau = (\tau', \dots, \tau')$, $y_1, \dots, y_n$ HO variables of type
    $\tau'$ respectively and $d$ the dimension of PHFL, then $V$ is a function that maps a variable
    of HO(LFP)$^{k + 1}$ to a
    variable $V(Y)$ of PHFL$^k$ defined inductive over the type of $Y$ as follows:
    \[V(Y)=
    \begin{cases}
        Y',  & \text{if }\tau = (\odot, \dots, \odot)\\
        Y'', & \text{if }\tau \neq (\odot, \dots, \odot) \text{ and } \tau \neq \odot,
    \end{cases}\]
    where $Y' \subseteq Q^d$ such that $(q_1, \dots, q_n, q_{n + 1}, \dots, q_d) \in Y'$ iff $(q_1, \dots, q_n) \in Y$
    and $Y''$ is a function of type $T(\tau)$ defined as follows:
    \begin{align*}
        (\dots((Y'(V(y_1)))(V(y_2)))\dots)(V(y_n)) &= Q^d &\text{ iff } (y_1, \dots, y_n) \in Y\\
        (\dots((Y'(V(y_1)))(V(y_2)))\dots)(V(y_n)) &= \emptyset &\text{ iff } (y_1, \dots, y_n) \not\in Y
    \end{align*}
\end{definition}

HO formulas of kind $X(x_1,\dots, x_n)$ are translated for $X$ of type $\tau =
(\tau', \dots, \tau')$ as
\[X(x_1, \dots, x_n) \coloneqq (\dots ((X(x_1))(x_2))\dots)(x_n)\]
and for $X$ of type $\tau = (\odot, \dots, \odot)$  where $x_1, \dots, x_n$ have index $i_1, \dots, i_n$ as
\[X(x_1, \dots, x_n) \coloneqq \{(1, \dots, d) \leftarrow (i_1, \dots, i_n, n + 1, \dots, d)\} X.\]
Note that the dimension $d$ of PHFL is big enough to translate all second-order variables of HO to an order 0 variable
of PHFL. In more detail $s$ is the maximal arity of second-order variables in an HO formula and $d > s$.

\begin{example}
    Let $\mathcal{A}$ a $\sigma$-structure over universe $\mathcal{U} = \{1, 2, 3\}$ and $X$ a HO(LFP)$^{k + 1}$
    variable of type $((\odot, \odot), (\odot, \odot))$ defined as
    \[X = \{(\{(1, 1)\}, \{(2, 2)\}), (\{(1, 1), (2, 2)\}, \{(3, 3)\})\}\]
    then $Y = V(X)$ is a PHFL$^k$ variable of type $\bullet \rightarrow (\bullet \rightarrow \bullet)$ such that $Y
    (\{(1, 1)\}) = f$, $Y(\{(1, 1), (2, 2)\}) = g$ and $Y(z) = h$ for $z \in Q^d$ and $z \neq \{(1, 1)\}$ and $z \neq \{
    (1, 1), (2, 2)\}$. $f$, $g$ and $h$ are functions of type $\bullet \rightarrow \bullet$ where $f(\{(2, 2)\}) = g
    (\{(3, 3)\}) = Q^d$ and $f(z) = g(z') = h(z'') = \emptyset$ for $z, z', z'' \in Q^d$ and $z \neq \{(2, 2)\}$ and
    $z' \neq \{(3, 3)\}$.
\end{example}

As mentioned above the first order variables of HO(LFP)$^{k + 1}$ have to be encoded differently. The reason is that
the base type in PHFL is at least a set of tuples of states. So single states can't be depict directly by another
variable. In this thesis the idea is to use the polyadic fragment of PHFL to represent the different first-order
variables. Each first-order variable represents one component in PHFL, that means each variable increases the
dimension of PHFL. The assignment of a first-order variable $x_i$ in HO is then the current state of the $i$-th
component in PHFL. This is mainly from~\cite{lange2014capturing}.

Without loss of generality the first-order variables are enumerated as $x_1, \dots, x_f, x_{f + 1}, \dots, x_q$,
where $x_1, \dots, x_f$ are the free and $x_{f+1}, \dots, x_q$ are the quantified variables. Note that the dimension
$d$ of PHFL is also big enough to distinguish all different first-order variables, that means $d > q$.

The usage of first-order variable $x_i$ in proposition $p$ in HO can be interpreted such that the variable
assignment of $x_i$ have to fulfill $p$. In PHFL that is equal to that the $i$-th component of a $d$-tuple
fulfills $p$, i.e. $p(x_i)$ is equal to $p_i$.

The usage of first-order variables $x_i$ and $x_j$ in action $a$, i.e. $a(x_i, x_j)$, can be interpreted as an
$a$-movement from assigned state of $x_i$ to assigned state of $x_j$. In PHFL that is equal to an $a$-movement from
component $i$ to component $j$ such that the state where component $j$ is located is reachable from the state where
component $i$ is located. That means $a(x_i, x_j)$ is equal to $\langle a \rangle_i\{(1, \dots, d) \leftarrow (i, j,
3, \dots, d)\} \Phi_\sim$ where $\Phi_\sim$ is the formula from Example~\ref{example:phfl_order_0} that defines $\sim$.

Because we encode the bisimulation-invariant fragment of HO(LFP)$^{k + 1}$, the first-order quantification can be
encoded by moving through all states reachable by one of the free variables. If we regard $\exists (x_i \colon
\odot).\,\Phi$ then it can be understand as, check if we reach a state tuple where $\Phi$ holds once the $i$-th
component is replaced by one of the free variables. This can be formalized in PHFL as \[\exists_i \Phi \coloneqq
\bigvee_{j=1}^f \{(1, \dots, d) \leftarrow (1, \dots, i-1, j, i + 1, \dots, d)\} \mu (X \colon \bullet).\,\Phi \vee
\bigvee_{a \in \Sigma} \langle a \rangle_i X.\]

Before regarding higher-order quantification we look at least fixpoints. Because of our translation of variables in
Definition~\ref{definition:lower_bound_variable_function} the replacement of LFP is straight forward. If $X$ is a
variable of HO type $\tau = (\tau', \dots, \tau')$, $x_1, v_1, \dots, x_n, v_n$ variables of HO type $\tau'$
respectively, $V(X)$ a variable of PHFL type $T(\tau)$ and $V(x_1), V(v_1), \dots, V(x_n), V(v_n)$
variables of PHFL type $T(\tau')$ respectively, then HO(LFP)$^{k + 1}$ formula
\[[LFP\;\Phi(X, x_1,\dots, x_n)](v_1, \dots, v_n)\]
can be translated to a PHFL$^k$ formula
\[LFP^\tau \coloneqq (\dots ((\mu (X \colon T(\tau).\,\Phi')(V(v_1)))(V(v_2)))\dots)(V(v_n))\]
In case of $\tau = (\odot, \dots, \odot)$ and $v_1, \dots, v_n$ have index $i_1, \dots, i_n$ respectively then
\[[LFP\;\Phi(X, x_1,\dots, x_n)](v_1, \dots, v_n)\]
can be translated to a PHFL$^k$ formula
\[LFP^{(\odot, \dots, \odot)} \coloneqq \{(1, \dots, d) \leftarrow (i_1, \dots, i_n, n + 1, \dots, d)\} \mu (X \colon
\bullet).\,\Phi\]

The last thing we have to consider is the higher-order quantification. For this we have to understand that we need an
iteration through all elements of an HO type $\tau$. That means we need a formula that tells us what is the next element
of $\tau$. This is possible if we have some order on the set $D_\tau(\mathcal{U})$.

\begin{remark}
    In~\cite{otto1999bisimulation} was shown that it is possible to define a $2$-adic formula $\Phi_<$ that defines a
    transitive relation $<$ such that $< \cap > = \emptyset$ and $< \cup > = \not\sim$. In this thesis $<$ is a total
    order on states of an LTS.
\end{remark}

We first define formulas that tells us given two elements of same type which one is the smaller one. Note that the
dimension $d$ of PHFL is twice as big as the maximum of $s$ and $q$. This is necessary to compare two elements of
$Q^{m}$, where $m = max({s, q})$ and $d = 2 * m$. The used variables in the following definition are all transformed by
variable function $V$ of Definition~\ref{definition:lower_bound_variable_function}. Also the usage of the variables
are written in HO syntax for a better readability.

\begin{definition}
    \label{definition:lower_bound_less}
    The PHFL$^k$ formula $<^\tau$ where $k = ord(T(\tau))$ is defined inductive over the type $\tau$ as follows:

    \begin{align*}
        <^\odot \coloneqq &\,\Phi_< \\
        <^{\odot \times \dots \times \odot} \coloneqq &\,\underset{i = 1}{\overset{m}{\bigvee}}\{(1, \dots, d)
        \leftarrow (i, i + m, 3, \dots, d)\} <^\odot \wedge \\
        &\,\underset{j = 1}{\overset{i - 1}{\bigwedge}}\{(1, \dots, d) \leftarrow (j + m, j, 3, \dots, d)\} \neg
        <^\odot\\
        <^{\tau' \times \dots \times \tau'}(x_1, y_1, \dots, x_1, y_n) \coloneqq &\,\underset{i =
        1}{\overset{n}{\bigvee}}<^{\tau'}(x_i, y_i) \wedge \underset{j = 1}{\overset{i - 1}{\bigwedge}}
        \neg <^{\tau'}(y_j, x_j)\\
        <^{(\odot, \dots, \odot)}(X, Y) \coloneqq &\,\exists_{i_1}.\, \dots \exists_{i_n}. \\&\,\{(1, \dots, d)
        \leftarrow  (i_1, \dots, i_n, n + 1, \dots, d)\}Y \wedge \\&\,\neg \{(1, \dots, d) \leftarrow (i_1, \dots,
        i_n, n + 1, \dots, d)\} X\,\wedge\\&\, \forall_{j_1}. \,\dots \forall_{j_n}. \\&\,\{(1, \dots, d)
        \leftarrow (j_1, \dots, j_n, n+1, \dots, m, \\&\,i_1, \dots, i_n, m + n + 1, \dots, d)\}<^{\odot
        \times \dots \times \odot} \Rightarrow (\\&\,\{(1, \dots, d) \leftarrow (j_1, \dots, j_n, n + 1, \dots, d)\}X
        \Rightarrow \\&\,\neg \{(1, \dots, d) \leftarrow (j_1, \dots, j_n, n + 1, \dots, d)\} Y)
        \\
        <^{(\tau', \dots, \tau')}(X, Y) \coloneqq &\,\exists^{\tau'}x_1. \,\dots \exists^{\tau'}x_n. \\&\,Y(x_1,
        \dots, x_n)
        \wedge \neg X(x_1, \dots, x_n)\,\wedge \\&\,\forall^{\tau'}y_1. \,\dots \forall^{\tau'}y_n. \\&\,<^{\tau'
        \times \dots \times \tau'}
        (y_1, x_1, \dots, y_n, x_n) \Rightarrow \\&\,(X(y_1, \dots, y_n) \Rightarrow Y(y_1, \dots, y_n))
    \end{align*}
    where the formulas of kind $\exists^\tau x.\phi$ are the in PHFL defined higher-order quantification for HO type
    $\tau$
    and formulas of kind $\forall^\tau x.\phi$ are the counterparts.
\end{definition}

Before we define the formulas we want to explain the defined orders. We say that a tuple $x$ is smaller than a tuple
$y$ if there is an index $i$ such that the element in $x$ on $i$ is smaller then the element in $y$ on $i$ and there
is no position $j$ left of $i$ such that the element in $x$ on $j$ is bigger then the element in $y$ on $j$. We say
that a set $x$ is smaller than a set $y$ if there is an element $e$ in $x$ that is not in $y$ and all smaller
elements $f$ to $e$ are only in $x$ if $f$ is also in $y$. After we have now orders of any HO type we can define
formulas that returns the successor of the input element.

\begin{definition}
    The PHFL$^k$ formula $next^\tau$ where $k = ord(T(\tau))$ is defined by the type $\tau$ as follows:
    \begin{align*}
        next^{(\odot, \dots, \odot)} \coloneqq &\,\lambda (X \colon \bullet).\, (\neg X \wedge \forall_{m +
        1}\dots\forall_{m + m}<^{\odot \times \dots \times \odot}\, \Rightarrow \\&\,\{(1, \dots, d) \leftarrow (m +
        1, \dots, m + m, m + 1, \dots, d)\} X) \,\vee \\&\,(X \wedge \exists_{m + 1}\dots\exists_{m + m} <^{\odot
        \times \dots \times \odot} \,\wedge \\&\,\{(1, \dots, d) \leftarrow (m + 1, \dots, m + m, m + 1, \dots, d)\}
        \neg X)\\
        next^{(\tau_1, \dots, \tau_n)} \coloneqq &\,\lambda (X \colon T ((\tau_1, \dots, \tau_n))).\,(\neg X(x_1,
        \dots, x_n)\wedge \forall^{\tau_1}y_1.\, \dots \forall^{\tau_n}y_n. \\&\, <^{\tau_1 \times \dots \times
        \tau_n}(y_1, x_1, \dots, y_n, x_n) \Rightarrow  X(y_1, \dots, y_n)) \vee \\&\,(X(x_1, \dots, x_n) \wedge
        \exists^{\tau_1}y_1.\, \dots \exists^{\tau_n}y_n.\, \\&\,<^{\tau_1 \times \dots \times \tau_n} (y_1, x_1,
        \dots, y_n, x_n)\,\wedge \neg X(y_1, \dots, y_n))
    \end{align*}
\end{definition}

The idea of the two formulas are based on binary incrementation. Remember that a set $X$ can be represented by
characteristic functions. This can be extended to a binary string where each position of this string represents an
element of the superset $A$, where $X \subseteq A$. Because each position in the binary string represents an element
of $A$ and a position have always represent the same element in $A$, the elements in $A$ have to be ordered. If the
position $i$ in the binary string is $1$ then this means that the element with index $i$ in $A$ is also in $X$ and if
the position $i$ in the binary string is $0$ then the element with index $i$ in $A$ is not in $X$. This binary
representation of $X$ in regard to $A$ can be extended to a function $f\colon \mathcal{P}(A) \rightarrow {0, \dots,
|\mathcal{P}(A)| - 1}$ such that each element $X$ of $\mathcal{P}(A)$ will be mapped to its binary string in regard
to $A$. Mapped to HO types let $\tau = (\tau', \dots, \tau')$ an HO type and $\mathcal{U}$ the universe of a
$\sigma$-structure then $f \colon D_\tau(\mathcal{U}) \rightarrow \{0, \dots, |D_\tau(\mathcal{U})| - 1\}$ is a
bijection of an element $X \in D_\tau(\mathcal{U})$ to its binary representation. The order of the elements of
$D_{\tau'}(\mathcal{U})^n$ is given by the formula $<^{\tau' \times \dots \times \tau'}$ of
Definition~\ref{definition:lower_bound_less}. We can say that $Y \in D_\tau(\mathcal{U})$ is the direct successor of
$X \in D_\tau(\mathcal{U})$ if $f(Y) = (f(X) + 1)$ modulo $|D_\tau(\mathcal{U})|$. In detail that means the
$i$-th bit is $1$ in $f(Y)$ if it is either $0$ in $X$ and all lower bits are $1$ in $X$ or it is $1$ in $X$ and
there is a bit lower then $i$ that is $0$ in $X$.

We are now able to define the higher-order quantification in PHFL.
\begin{definition}
    Let $\tau = (\tau', \dots, \tau')$ be an HO type where $\tau' \neq \odot$ then $\exists^{\tau}X .\,\Phi(X)$ is
    defined as follows:
    \begin{align*}
        \exists^{\tau}X.\, \Phi(X) \coloneqq &\,(\mu (F \colon T(\tau) \rightarrow \bullet).\, \lambda (X \colon T(\tau)
        ).\,
        \Phi(X)
        \vee F(next^\tau X))\\&\,(\lambda (x_1 \colon \tau').\, \dots \lambda (x_n \colon \tau').\,\bot)
    \end{align*}
    In case of $\tau = (\odot, \dots, \odot)$, $\exists^{\tau}X .\,\Phi(X)$ is defined as
    \[        \exists^{\tau}X.\, \Phi(X) \coloneqq (\mu (F \colon \bullet \rightarrow \bullet).\, \lambda (X
    \colon \bullet).\, \Phi(X) \vee F(next^{\tau} X)) \bot
    \]
\end{definition}

The last step is to define a function that translates bisimulation invariant HO(LFP)$^{k + 1}$ formulas in PHFL$^k$.
This is done straight forward with the definitions and explanations that we give before.

\begin{definition}
    Let $F$ be function that maps a bisimulation invariant HO(LFP)$^k$ formula to a PHFL$^k$ formula defined inductive
    on bisimulation invariant HO(LFP)$^k$ formula $\Phi$ as follows:
    \begin{align*}
        F(p(x_i)) \coloneqq &\, p_i \\
        F(a(x_i, x_j)) \coloneqq &\, \langle a \rangle_i \{(1, \dots, d) \leftarrow (i, j, 3, \dots, d)\} \Phi_\sim \\
        F(\Phi \vee \Psi) \coloneqq &\, F(\Phi) \vee F(\Psi) \\
        F(\neg \Phi) \coloneqq &\, \neg F(\Phi) \\
        F(\exists (x_i \colon \odot).\,\Phi) \coloneqq &\, \exists_i \Phi \\
        F(\exists (X \colon \tau).\,\Phi) \coloneqq &\, \exists^\tau X. \Phi(X) \\
        F([LFP\;\Phi(X, x_1, \dots, x_n)](v_1, \dots, v_n)) \coloneqq &\,LFP^\tau \\
        F(X(x_1, \dots, x_n)) \coloneqq &\, (\dots ((X(x_1))(x_2))\dots)(x_n)\\
        F(X(x_{i_1}, \dots, x_{i_n})) \coloneqq &\, \{(1, \dots, d) \leftarrow (i_1, \dots, i_n, n + 1, \dots, d)\}X
    \end{align*}
\end{definition}

With all this definitions and explanations the following lemma holds.

\begin{lemma}
    \label{lemma:ho_lfp_equals_phfl}
    Let $r \geq 1$ and $k \geq 0$. For every bisimulation invariant formula $\Phi$ of HO(LFP)$^{k + 1}$ there is a
    PHFL$^k$ formula $\Psi$ such that $\mathcal{Q}_\Phi^r = \mathcal{Q}_\Psi^r$.
\end{lemma}

The combination of Lemma~\ref{lemma:ho_lfp_equals_phfl}, Theorem~\ref{theorem:hoLfpEqualsExptime} and
Theorem~\ref{theorem:phfl_k_in_k_exptime} proves the following theorem.

\begin{theorem}
    Let $k \geq 0$. PHFL$^k$ captures \exptime{$k$} over labeled transition systems.
\end{theorem}

