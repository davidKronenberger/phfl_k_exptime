%%
%% Author: DKron
%% 17.08.2018
%%

\section{Lower Bound of PHFL$^k$}\label{sec:lowerBoundOfPhfl}

As mentioned in the introduction of this chapter we can show that the lower bound of PHFL$^k$ is \exptime{$k$} by
make a detour over HO(LFP)$^{k+1}$. This and following ideas are oriented on~\cite{lange2014capturing} where was
shown that PHFL$^1$ captures \exptime{$1$}. The first thing we will see is that it was proven that HO(LFP)$^{k +
1}$ coincides with $k$-fold exponential time over finite and ordered structures. To use this we have to encode the
bisimulation invariant fragment of HO$^{k+1}$ into PHFL$^k$. For this we first define an abbreviation for the HO(LFP)
formulas that uses the LFP operator. Next, we define a function that uses this abbreviation and these of
Section~\ref{sec:existential_quantifiers_in_phfl} to map a HO(LFP)$^{k+1}$ formula to a PHFL$^k$ formula. Finally, we
show that the semantics of the HO(LFP)$^{k+1}$ formula coincides with the transformed PHFL$^k$ formula.

\begin{theorem}{\cite{freireMartins2011descriptive}}\label{theorem:hoLfpEqualsExptime}
    For all $k \geq 1$, HO(LFP)$^{k + 1} = k$-EXPTIME over finite and ordered structures.
\end{theorem}

The proof follows the idea to encode the run of a $k$-EXPTIME Turing Machine $M$ by a formula $\phi$ of HO(LFP)$^{k +
1}$ in this way such that $M$ accepts $\mathcal{A}$ iff $\mathcal{A} \models \phi$. On the other hand each HO(LFP)$^{k +
1}$ formula $\phi$ can be evaluated by a $k$-EXPTIME Turing Machine $M_\phi$.

Because Theorem~\ref{theorem:hoLfpEqualsExptime} holds it is also possible to prove that the lower bound of PHFL$^k$
is in \exptime{$k$} by encoding the bisimulation invariant fragment of HO(LFP)$^{k + 1}$ into PHFL$^k$. To encode the
bisimulation invariant fragment of HO(LFP)$^{k + 1}$ into PHFL$^k$ we have to define a function that transforms a HO
(LFP)$^{k + 1}$ formula to a PHFL$^k$ formula. Note that the types and variables of an HO formula have to be also
transformed. See Section~\ref{sec:existential_quantifiers_in_phfl} for further details.

Before we give the definition of the transforming function, we define a PHFL formula for HO formulas that uses the
LFP operator.

\begin{definition}
    For any HO variable $X$ of HO type $\tau = (\tau', \dots, \tau')$ where $\tau' \neq \odot$ and any HO variables
    $x_1, v_1, \dots, $ $x_n, v_n$ HO type $\tau'$ and any PHFL formula $\Phi$ let
    \[LFP^\tau X.\,\Phi \coloneqq (\dots ((\mu (X \colon T(\tau)).\,\Phi(X, x_1, \dots, x_n))(v_1))(v_2))\dots)(v_n).\]
    In case of $\tau = (\odot, \dots, \odot)$ and $v_1, \dots, v_n$ have index $i_1, \dots, i_n$ respectively let
    \[LFP^{\tau} X.\,\Phi \coloneqq \{(i_1, \dots, i_n, n + 1, \dots, d)\} \mu (X \colon \bullet).\,\Phi(X).\]
\end{definition}

Remember that we have defined the signatures of HO(LFP) relational. Because we are working on LTS, the relations in
the signatures have either arity one or two. Relations with arity one represents the propostions and those with arity
two the actions of an LTS. Now we are able to define the function to translate a bisimulation invariant HO(LFP)$^{k+1}$
formula to a PHFL$^k$.

\begin{definition}
    \label{definition:lower_bounds_phfl_formula_function}
    Let $F$ be the function that maps a bisimulation invariant HO(LFP)$^{k+1}$ formula to a PHFL$^k$ formula defined
    inductive on bisimulation invariant HO(LFP)$^{k+1}$ formula $\Phi$ as follows:
    \begin{align*}
        F(p(x_i)) \coloneqq &\, p_i \\
        F(a(x_i, x_j)) \coloneqq &\, \langle a \rangle_i \{(i, j, 3, \dots, d)\} \Phi_\sim \\
        F(\Phi \vee \Psi) \coloneqq &\, F(\Phi) \vee F(\Psi) \\
        F(\neg \Phi) \coloneqq &\, \neg F(\Phi) \\
        F(\exists (x_i \colon \odot).\,\Phi) \coloneqq &\, \exists_i F(\Phi) \\
        F(\exists (X \colon \tau).\,\Phi) \coloneqq &\, \exists^\tau X.\,F(\Phi(X)) \\
        F([LFP\;\Phi(X, x_1, \dots, x_n)](v_1, \dots, v_n)) \coloneqq &\,LFP^\tau X.\, F(\Phi) \\
        F(X(x_1, \dots, x_n)) \coloneqq &\, (\dots ((X(x_1))(x_2))\dots)(x_n)\\
        F(X(x_{i_1}, \dots, x_{i_n})) \coloneqq &\, \{(i_1, \dots, i_n, n + 1, \dots, d)\}X
    \end{align*}
\end{definition}

The last step is to show that the semantics of a given HO(LFP)$^{k+1}$ formula coincides with the semantics of the by
function $F$ encoded PHFL$^k$ formula. For this we need a last definition for the semantics of the formulas. As
described in Section~\ref{sec:existential_quantifiers_in_phfl} the variables in HO of types with order $3$ or higher
are not supported in PHFL. Also first-order variables are not supported. For this we define a function that maps a
given variable mapping for a HO formula to the correct variable mapping for PHFL semantics. This function ignores the
mapping of first-order variables, maps second-order variables also to sets and higher-order variables to the
corresponding characteristic function. Note, that the sets of order $2$ in HO have order $0$ in PHFL. The first-order
variables of an HO formula that are marked as ignored in the following function, can be mapped to anything because we
do not use them in PHFL directly.

\begin{definition}
    \label{definition:lower_bound_variable_function}
    Let $d$ a dimension of PHFL and $\eta$ a variable mapping for a HO(LFP)$^{k+1}$ formula, then function
    $V$ maps $\eta$ to a variable mapping that can be used for a PHFL$^k$ formula with dimension $d$ where
    \[V(\eta)(X)=
    \begin{cases}
        \text{ignored}, & \text{if } X \text{ is of type } \odot \\
        A,  & \text{if } X \text{ is of type } (\odot, \dots, \odot)\\
        F, & \text{if } X \text{ is of type } (\tau, \dots, \tau) \text{ and } \tau \neq \odot,
    \end{cases}\]
    where $A \subseteq Q^d$ such that $(q_1, \dots, q_n, q_{n + 1}, \dots, q_d) \in A$ iff $(q_1, \dots, q_n) \in
    \eta(X)$ and $F$ is a function of type $T((\tau, \dots, \tau))$ defined as follows:
    \begin{align*}
        (\dots((F(V(\eta)(X_1)))(V(\eta)(X_2)))\dots)(V(\eta)(X_n)) &= Q^d &\text{ iff } (X_1, \dots, X_n) \in \eta(X)\\
        (\dots((F(V(\eta)(X_1)))(V(\eta)(X_2)))\dots)(V(\eta)(X_n)) &= \emptyset &\text{ iff } (X_1, \dots, X_n)
        \not\in \eta(X)
    \end{align*}
\end{definition}

The following example shows how a set of higher type will be translate to its characteristic function via the
function $V$ of Definition~\ref{definition:lower_bound_variable_function}.

\begin{example}
    Let $\mathcal{A}$ a $\sigma$-structure over universe $\mathcal{U} = \{1, 2, 3\}$ and $X$ a HO(LFP)$^{k + 1}$
    variable of type $((\odot, \odot), (\odot, \odot))$ mapped through variable mapping $\eta$ to
    \[\eta(X) = \{(\{(1, 1)\}, \{(2, 2)\}), (\{(1, 1), (2, 2)\}, \{(3, 3)\})\}\]
    then $V(\eta)(X)$ is a PHFL$^k$ function of type $\bullet \rightarrow (\bullet \rightarrow \bullet)$ such
    that $V(\eta)(X)$ $(\{(1, 1)\}) = f$, $V(\eta)(X)(\{(1, 1), (2, 2)\}) = g$ and $V(\eta)(X)(z) = h$ for $z \in
    Q^d$ and $z \neq \{(1, 1)\}$ and $z \neq \{(1, 1), (2, 2)\}$. $f$, $g$ and $h$ are functions of type $\bullet
    \rightarrow \bullet$ where $f(\{(2, 2)\}) = g(\{(3, 3)\}) = Q^d$ and $f(z) = g(z') = h(z'') = \emptyset$ for $z,
    z', z'' \in Q^d$ and $z \neq \{(2, 2)\}$ and $z' \neq \{(3, 3)\}$.
\end{example}

Because we are interested in the bisimulation-invariant fragment of HO( LFP)$^{k+1}$ we give a further definition that
makes the proof of the following lemma more comfortable. This definition gives for a given LTS and tuple of states a
reduced LTS of all those states that are reachable by at least one state of the given state tuple.

Finally we have to show the following lemma.

\begin{lemma}
    \label{lemma:ho_lfp_equals_phfl}
    Let $r \geq 1$ and $k \geq 0$. For every bisimulation invariant formula $\Phi$ of HO(LFP)$^{k + 1}$ there is a
    PHFL$^k$ formula $\Psi$ such that $\mathcal{Q}_\Phi^r = \mathcal{Q}_\Psi^r$.
\end{lemma}

\begin{proof}
    This lemma can be proven by showing for all HO(LFP)$^{k+1}$ formulas $\Phi$ with free first-order variables $x_1,
    \dots, x_f$ and quantified first-order variables $x_{f+1}, \dots, x_q$, all LTS $\mathcal{T} = (Q, \Sigma, P,
    \Delta, v)$ and all variable mappings $\eta$ that it holds that $\mathcal{T}, \eta \models \Phi$ iff $\emph{q} =
    (\eta(x_1), \dots, \eta(x_f), q_0, \dots, q_0)$ and $\emph{q} \in \llbracket
    \Gamma \vdash F(\Phi)\colon \bullet \rrbracket^{V(\eta)}_{\mathcal{T}}$, where $q_0
    \in Q$ is an arbitrary state, $F$ is the formula function of
    Definition~\ref{definition:lower_bounds_phfl_formula_function} and $V$ the variable mapping function of
    Definition~\ref{definition:lower_bound_variable_function}. This statement can be proven by induction over formula
    $\Phi$.
    \begin{compactitem}
        \item In case of $\Phi = p(x_i)$ then $\mathcal{T}, \eta \models \Phi$ holds exactly then if $\eta(x_i) \in
        p^\mathcal{T}$. Translated to the normal LTS definition of $\mathcal{T}$ it is the same as $p \in v(\eta(x_i))$.
        With $F(\Phi) = p_i$ this is exactly
        \begin{align*}
            (\eta(x_1),& \dots, \eta(x_{i-1}), \eta(x_{i}), \eta(x_{i+1}), \dots, \eta(x_f),\\& q_0, \dots, q_0) \in
            \llbracket \Gamma \vdash F(\Phi) \colon \bullet \rrbracket^{V(\eta)}_\mathcal{T}.
        \end{align*}
        \item In case of $\Phi = a(x_i, x_j)$ then $\mathcal{T}, \eta \models \Phi$ holds exactly then if $(\eta(x_i)
        , \eta(x_j))$ $ \in a^\mathcal{T}$. Translated to the normal LTS definition of $\mathcal{T}$ it is the same as $
        \eta(x_i) \overset{a}{\rightarrow} \eta(x_j)$. Let $g: \{1, 2\} \rightarrow \{i, j\}$ be a function.
        By  definition of the semantics of $\langle a \rangle_i \Phi$ the tuple
        \begin{align*}
            (\eta(x_1),& \dots, \eta(x_{g(1) - 1}), \eta(x_{g(1)}), \eta(x_{g(1)+1}), \dots, \eta(x_{g(2)-1}), \eta
            (x_{g(2)}),\\& \eta(x_{g(2)+1}), \dots, \eta(x_f), q_0, \dots, q_0))
        \end{align*}
        is an element of the semantics of $\langle a \rangle_i \Phi$ if
        \begin{align*}
            (\eta(x_1),& \dots, \eta(x_{g(1) - 1}), q_m, \eta(x_{g(1)+1}), \dots, \eta(x_{g(2)-1}), q_n,\\& \eta(x_{g(2)
            +1}), \dots, \eta(x_f), q_0, \dots, q_0)
        \end{align*}
        is an element of the semantics of $\Phi$ where $\eta(x_i)$ can
        move by an $a$ action to $q_m$ if $g(1) = i$ or to $q_n$ if $g(2) = i$. Because $\eta(x_j)$ is the state that
        have to be reached via an $a$ action from $\eta(x_i)$ we have to check if $q_m = \eta(x_j)$ in case of $g(1)
        = j$ and in case of $g(2) = j$ we have to check if $q_n = \eta(x_i)$. If two states are equal in an LTS is
        similar to check if those states are bisimilar. $\sim$ is given by formula $\Phi_\sim$ of
        Example~\ref{example:phfl_order_0}.
        Because this formula returns those $d$-tuples where the first and second component are bisimilar, we have to
        move the $i$-th and $j$-th component to the first and second component. This is given by $\{(i, j, 3, \dots, d
        )\} \Phi_\sim$. Summarizing all this steps with $F(\Phi) = \langle a \rangle_i \{(i, j, 2, \dots, d)\}
        \Phi_\sim$ it follows
        \begin{align*}
            (\eta(x_1),& \dots, \eta(x_{g(1) - 1}), \eta(x_{g(1)}), \eta(x_{g(1)+1}), \dots, \eta(x_{g(2)-1}), \eta
            (x_{g(2)}),\\& \eta(x_{g(2)+1}), \dots, \eta(x_f), q_0, \dots, q_0)) \in \llbracket \Gamma \vdash F(\Phi)
            \colon \bullet \rrbracket
        \end{align*}
        if $(\eta(x_i), \eta(x_j))$ $ \in a^\mathcal{T}$ and
        \begin{align*}
            (\eta(x_1),& \dots, \eta(x_{g(1) - 1}), \eta(x_{g(1)}), \eta(x_{g(1)+1}), \dots, \eta(x_{g(2)-1}), \eta
            (x_{g(2)}),\\& \eta(x_{g(2)+1}), \dots, \eta(x_f), q_0, \dots, q_0)) \not\in \llbracket \Gamma \vdash
            F(\Phi) \colon \bullet \rrbracket
        \end{align*}
        if $(\eta(x_i), \eta(x_j))$ $ \not\in a^\mathcal{T}$.

        \item In case of $\Phi = X(x_{i_1}, \dots, x_{i_n})$ where $X$ is a free variable of HO type $(\odot, \dots,
        \odot)$ and $x_{i_1}, \dots, x_{i_n}$ are free first-order variables then $\mathcal{T}, \eta \models \Phi$
        holds exactly then if $(\eta(x_{i_1}), \dots \eta(x_{i_n})) \in \eta(X)$. Because of definition of $V$ the
        tuple $(\eta(x_{i_1}), \dots \eta(x_{i_n}), \eta(x_{n + 1}), \dots, \eta(x_{f}), q_0, \dots, q_0)$ is in $V
        (\eta)(X)$ if $(\eta(x_{i_1}),\dots \eta(x_{i_n})) \in \eta(X)$ and is not in $V(\eta)(X)$ otherwise. Because
        the components $i_1, \dots, i_n$ for first-order variables $x_{i_1}, \dots, x_{i_n}$ have not to be $1,
        \dots, n$ respectively, we first move them to this components and check then if
        \[(\eta(x_{i_1}), \dots \eta(x_{i_n}), \eta(x_{n + 1}), \dots, \eta(x_{f}), q_0, \dots, q_0) \in V(\eta)(X).\]
        So it holds with $F(\Phi) = {(i_1, \dots, i_n, n+1, \dots, d)}X$ and function $g: \{1, \dots, n\}
        \rightarrow \{i_1, \dots, i_n\}$ that
        \begin{align*}
            (\eta(x_1),& \dots, \eta(x_{g(1)-1}), \eta(x_{g(1)}), \eta(x_{g(1)+1}), \dots \eta(x_{g(2)-1}), \eta
            (x_{g(2)}),\\& \eta(x_{g(2)+1}), \dots, \eta(x_{g(n)-1}), \eta(x_{g(n)}), \eta(x_{g(n)+1}), \dots, \eta
            (x_f),\\& q_0, \dots, q_0) \in \llbracket \Gamma \vdash F(\Phi) \colon \bullet \rrbracket^{V(\eta)
            }_\mathcal{T}
        \end{align*}
        if $(\eta(x_{i_1}), \dots \eta(x_{i_n})) \in \eta(X)$ and
        \begin{align*}
            (\eta(x_1),& \dots, \eta(x_{g(1)-1}), \eta(x_{g(1)}), \eta(x_{g(1)+1}), \dots \eta(x_{g(2)-1}), \eta
            (x_{g(2)}),\\& \eta(x_{g(2)+1}), \dots, \eta(x_{g(n)-1}), \eta(x_{g(n)}), \eta(x_{g(n)+1}), \dots, \eta
            (x_f),\\& q_0, \dots, q_0) \not\in \llbracket \Gamma \vdash F(\Phi) \colon \bullet \rrbracket^{V(\eta)
            }_\mathcal{T}
        \end{align*}
        if $(\eta(x_{i_1}), \dots \eta(x_{i_n})) \not\in \eta(X)$.

        \item In case of $\Phi = X(X_1, \dots, X_n)$ where $X$ is a free variable of HO type $(\tau, \dots,
        \tau)$ and $X_1, \dots, X_n$ are free variables of HO type $\tau$ then $\mathcal{T}, \eta \models \Phi$
        holds exactly then if $(\eta(X_1), \dots \eta(X_n)) \in \eta(X)$. Because of definition of $V$ it follows
        \[(\dots((V(\eta)(X)(V(\eta)(X_1)))(V(\eta)(X_2)))\dots)(V(\eta)(X_n)) = Q^d\]
        if $(\eta(X_1), \dots \eta(X_n)) \in \eta(X)$ and
        \[(\dots((V(\eta)(X)(V(\eta)(X_1)))(V(\eta)(X_2)))\dots)(V(\eta)(X_n)) = \emptyset\]
        if $(\eta(X_1), \dots \eta(X_n)) \not\in \eta(X)$. With $F(\Phi) = (\dots ((X(X_1))(X_2))$ $\dots)(X_n)$
        it follows
        \[ (\eta(x_1), \dots, \eta(x_f), q_0, \dots, q_0) \in \llbracket \Gamma \vdash F(\Phi) \colon
        \bullet \rrbracket^{V(\eta)}_\mathcal{T} = Q^d.\]
        if $(\eta(X_1), \dots \eta(X_n))\in \eta(X)$ and
        \[ (\eta(x_1), \dots, \eta(x_f), q_0, \dots, q_0) \not\in \llbracket \Gamma \vdash F(\Phi) \colon \bullet
        \rrbracket^{V(\eta)}_\mathcal{T} = \emptyset.\]
        if $(\eta(X_1), \dots \eta(X_n)) \not\in \eta(X)$.
    \end{compactitem}
    By induction hypothesis it holds for HO(LFP)$^{k+1}$ formulas $\Psi$ and $\Psi'$ with free variables $x_1, \dots,
    x_f$ and quantified first-order variables $x_{f+1}, \dots, x_q$ that for all LTS $\mathcal{T}$ and all variable
    mappings $\eta$ that $\mathcal{T}, \eta \models \Psi$ iff $ (\eta(x_1), \dots, \eta(x_f),$ $ q_0, \dots, q_0) \in
    \llbracket \Gamma \vdash F(\Psi) \colon \bullet^{V(\eta)}_\mathcal{T}$ and $\mathcal{T}, \eta \models \Psi'$ iff $
    (\eta(x_1), \dots, \eta(x_f), q_0, \dots, $ $q_0) \in \llbracket \Gamma \vdash F(\Psi') \colon \bullet^{V(\eta)
    }_\mathcal{T}$.
    \begin{compactitem}
        \item In case of $\Phi = \neg \Psi$ it follows that $\mathcal{T}, \eta \models \Phi$ exactly then if
        $\mathcal{T}, \eta \not\models \Psi$. By induction hypothesis that is exactly then the case when
        \[ (\eta(x_1), \dots, \eta(x_f), q_0, \dots, q_0) \not\in \llbracket \Gamma \vdash F(\Psi) \colon \bullet
        \rrbracket^{V(\eta)}_\mathcal{T}.\]
        This is exactly the case if
        \[ (\eta(x_1), \dots, \eta(x_f), q_0, \dots, q_0) \in Q^d \setminus \llbracket \Gamma^- \vdash F(\Psi) \colon
        \bullet \rrbracket^{V(\eta)}_\mathcal{T}.\]
        And this is exactly the semantics of $F(\Phi) = \neg F(\Psi)$.

        \item In case of $\Phi = \Psi \vee \Psi'$ it follows that $\mathcal{T}, \eta \models \Phi$ exactly then if
        $\mathcal{T}, \eta \models \Psi$ or $\mathcal{T}, \eta \models \Psi'$. By induction hypothesis that is
        exactly then the case when
        \[(\eta(x_1), \dots, \eta(x_f), q_0, \dots, q_0) \in \llbracket \Gamma \vdash F
        (\Psi) \colon \bullet \rrbracket^{V(\eta)}_\mathcal{T}\]
        or
        \[(\eta(x_1), \dots, \eta(x_f), q_0, \dots, q_0) \in \llbracket \Gamma \vdash F
        (\Psi') \colon \bullet \rrbracket^{V(\eta)}_\mathcal{T}.\]
        Because $\sqcup_{\bullet} = \cup$ this can be combined to
        \[(\eta(x_1), \dots, \eta(x_f), q_0, \dots, q_0) \in \llbracket \Gamma \vdash F
        (\Psi) \colon \bullet \rrbracket^{V(\eta)}_\mathcal{T} \sqcup_\bullet \llbracket \Gamma \vdash F
        (\Psi') \colon \bullet \rrbracket^{V(\eta)}_\mathcal{T},\]
        what is the semantics of $F(\Phi) = F(\Psi) \vee F(\Psi')$.

        \item In case of $\Phi = \exists (x_i\colon \odot).\,\Psi$ it follows that $\mathcal{T}, \eta \vdash \Phi$ iff
        there exists  $\mathcal{X} \in Q$ with $\mathcal{T}, \eta[x_i \rightarrow \mathcal{X}] \models \Psi$. Because
        $x_i$ is a free variable in $\Psi$ it follows by induction hypothesis that $\mathcal{T}, \eta[x_i
        \rightarrow \mathcal{X}] \models \Psi$ is exactly the case when
        \begin{align*}
            (\eta(x_1),& \dots, \eta(x_{i-1}), \eta(x_i), \eta(x_{i+1}), \dots, \eta(x_{f+1}),\\& q_0, \dots, q_0) \in
            \llbracket \Gamma \vdash F(\Psi) \colon \bullet \rrbracket^{V(\eta)}_\mathcal{T}.
        \end{align*}
        Because $x_i$ is bounded in $\Phi$ we have to replace the $i$-th component by one of the free. This is
        denoted by $\overset{f}{\underset{j=1}{\bigvee}} \{(1, \dots, i-1, j, i+1, \dots, d)\}$.
        Because of bisimilarity all states of $\mathcal{T}$ have an order and are reachable by moving
        By replacing the $i$-th component by one of the free first-order variables and check all reachable state
        tuples, we move over the

        \item In case of $\Phi = \exists (X \colon \tau).\,\Psi$ it follows that $\mathcal{T}, \eta \vdash \Phi$ iff
        there exists $\mathcal{X} \in D_\tau(Q)$ with $\mathcal{T}, \eta[X \rightarrow \mathcal {X}] \models \Psi$.
        By induction hypothesis it follows that $(\eta(x_1), \dots, \eta(x_{f}), q_0, \dots, q_0) \in
        \llbracket \Gamma \vdash F(\Psi) \colon \bullet \rrbracket^{V(\eta)}_\mathcal{T}$ iff $\mathcal{T}, \eta[X
        \rightarrow \mathcal {X}] \models \Psi$. By Lemma~\ref{lemma:existential_quantifier} the formula
        $\exists^\tau X.\, \Psi(X)$ iterates through all elements of $D_\tau(Q)$ and checks if one fulfills $\Psi(X)$
        by returning $Q^d$.
        With $F(\Phi) = \exists^\tau X.\, F(\Psi)(X)$ it holds \[(\eta(x_1), \dots, \eta(x_{f}), q_0, \dots, q_0) \in
        \llbracket \Gamma \vdash F(\Phi) \colon \bullet \rrbracket^{V(\eta)}_\mathcal{T}.\]

        \item In case of $\Phi = [LFP\,\Psi(X, x_{i_1}, \dots, x_{i_n})](v_{j_1}, \dots, v_{j_n})$, where $X$ is a
        free variable in $\Psi$ of HO type $(\odot, \dots, \odot)$ and $x_{i_1}, \dots, x_{i_n}$ are free first-order
        variables of $\Psi$ and $v_{j_1}, \dots, v_{j_n}$ are first-order variables of $\Phi$, then it follows that
        $\mathcal{T}, \eta \vdash \Phi$ exactly then if $(\eta(v_{j_1}), \dots, \eta(v_{j_n})) \in LFP
        (F_\Psi^\mathcal{T})$. By definition of LFP $(\eta(v_{j_1}), \dots, \eta(v_{j_n})) \in
        LFP(F_\Psi^\mathcal{T})$ iff $X = F_\Psi^\mathcal{T}(X)$ and $(\eta(v_{j_1}), \dots, \eta(v_{j_n})) \in
        F_\Psi^\mathcal{T}(X)$. By definition of $F_\Psi^\mathcal{T}(X)$ it holds $(\eta
        (v_{j_1}), \dots, \eta(v_{j_n})) \in F_\Psi^\mathcal{T}(X)$ exactly then if $\mathcal{T}, \eta \models \Psi
        (\eta(X), \eta(v_{j_1}), \dots, \eta(v_{j_n}))$. By induction hypothesis
        \[(\eta(v_{j_1}), \dots, \eta(v_{j_n}), \eta(x_{n+1}), \dots, \eta(x_f), q_0, \dots, q_0) \in \llbracket
        \Gamma \vdash F(\Psi) \colon \bullet \rrbracket^{V(\eta)}_\mathcal{T}.\]
        Because of construction of variable mapping $V$ and $(\eta(v_{j_1}), \dots, \eta(v_{j_n})) \in \eta(X)$ the
        tuple $(\eta(v_{j_1}), \dots, \eta(v_{j_n}), \eta(x_{n+1}), \dots, \eta(x_f), q_0, \dots, q_0)$ is also in $V
        (\eta)(X)$. By definition of the least fixpoint operator in PHFL this set $X$ is reached if it includes all
        elements of $\llbracket \Gamma \vdash F(\Psi) \colon \bullet \rrbracket^{V(\eta)}_\mathcal{T}$. Because of
        construction of $F_\Psi^\mathcal{T}(X)$ and $X = F_\Psi^\mathcal{T}(X)$ it holds that $(\eta(v_{j_1}),
        \dots, \eta(v_{j_n})) \in X$ iff
        \[(\eta(v_{j_1}), \dots, \eta(v_{j_n}), \eta(x_{n+1}), \dots, \eta(x_f), q_0, \dots, q_0) \in \llbracket
        \Gamma \vdash F(\Psi) \colon \bullet \rrbracket^{V(\eta)}_\mathcal{T}.\]
        Because $X$ is the least fixpoint of $F_\Psi^\mathcal{T}$ the resulting set $X'$ is the smallest set that
        fulfills $X' \subseteq X$ and so
        \[(\eta(v_{j_1}), \dots, \eta(v_{j_n}), \eta(x_{n+1}), \dots, \eta(x_f), q_0, \dots, q_0) \in \llbracket
        \Gamma \vdash \mu(X\colon \bullet).\,F(\Psi) \colon \bullet \rrbracket^{V(\eta)}_\mathcal{T}.\]
        Because the components $j_1, \dots, j_n$ for first-order variables $v_{j_1}, \dots, v_{j_n}$ have not to be $1,
        \dots, n$ respectively, we first move them to this components and check then the least fixpoint operator.
        With $F(Phi) = $ it follows
        \[(\eta(v_{j_1}), \dots, \eta(v_{j_n}), \eta(x_{n+1}), \dots, \eta(x_f), q_0, \dots, q_0) \in \llbracket
        \Gamma \vdash \mu(X\colon \bullet).\,F(\Phi) \colon \bullet \rrbracket^{V(\eta)}_\mathcal{T}.\]
        So it holds with $F(\Phi) = \{j_1, \dots, j_n, n+1, \dots, d\} \mu (X \colon \bullet).\, F(\Psi)$ and
        function $g: \{1, \dots, n\} \rightarrow \{j_1, \dots, j_n\}$ that
        \begin{align*}
            (\eta(x_1),& \dots, \eta(v_{g(1)-1}), \eta(v_{g(1)}), \eta(v_{g(1)+1}), \dots \eta(v_{g(2)-1}), \eta
            (v_{g(2)}),\\& \eta(v_{g(2)+1}), \dots, \eta(v_{g(n)-1}), \eta(v_{g(n)}), \eta(v_{g(n)+1}), \dots, \eta
            (x_f),\\& q_0, \dots, q_0) \in \llbracket \Gamma \vdash F(\Phi) \colon \bullet \rrbracket^{V(\eta)
            }_\mathcal{T}
        \end{align*}
        exactly then if $\mathcal{T}, \eta \vdash \Phi$.

        \item In case of $\Phi = [LFP\,\Psi(X, X_1, \dots, X_n)](V_1, \dots, V_n)$, where $X$ is a
        free variable in $\Psi$ of HO type $(\tau, \dots, \tau)$ and $X_1, \dots, X_n$ are free first-order
        variables of $\Psi$ of type $\tau$ and $V_1, \dots, V_n$ are free variables of $\Phi$ also of type $\tau$, then
        it follows that $\mathcal{T}, \eta \vdash \Phi$ exactly then if $(\eta(X_1), \dots, \eta(X_n) \in LFP
        (F_\Psi^\mathcal{T})$. By definition of LFP $(\eta(V_1), \dots, \eta(V_n)) \in
        LFP(F_\Psi^\mathcal{T})$ iff $X = F_\Psi^\mathcal{T}(X)$ and $(\eta(V_1), \dots, \eta(V_n)) \in
        F_\Psi^\mathcal{T}(X)$. By definition of $F_\Psi^\mathcal{T}(X)$ it holds $(\eta
        (V_1), \dots, \eta(V_n)) \in F_\Psi^\mathcal{T}(X)$ exactly then if $\mathcal{T}, \eta \models \Psi
        (\eta(X), \eta(V_1), \dots, \eta(V_n))$. By induction hypothesis
        \[(\eta(x_1), \dots, \eta(x_f), q_0, \dots, q_0) \in \llbracket \Gamma \vdash F(\Psi) \colon \bullet
        \rrbracket^{V(\eta)}_\mathcal{T}.\]
        Because of construction of variable mapping $V$ and $(\eta(V_1), \dots, \eta(V_n)) \in \eta(X)$ it holds
        $(\dots((V(\eta)(X)(V(\eta)(V_1)))(V(\eta)(V_2))) \dots (V(\eta)(V_n)) = Q^d$
        and so the tuple $(\eta(x_1), \dots, \eta(x_f), q_0, \dots, q_0)$ is also in $(\dots((V(\eta)(X)(V(\eta)(V_1)
        ))(V(\eta)(V_2))) \dots (V(\eta)(V_n))$. By definition of the least fixpoint operator in PHFL this function
        $V(\eta)(X)$ is reached if it includes all
        elements of $\llbracket \Gamma \vdash F(\Psi) \colon \bullet \rrbracket^{V(\eta)}_\mathcal{T}$. Because of
        construction of $F_\Psi^\mathcal{T}(X)$ and $X = F_\Psi^\mathcal{T}(X)$ it holds that $(\eta(V_{1}),
        \dots, \eta(V_{n})) \in X$ iff
        \[(\eta(x_1), \dots, \eta(x_f), q_0, \dots, q_0) \in \llbracket
        \Gamma \vdash F(\Psi) \colon \bullet \rrbracket^{V(\eta)}_\mathcal{T}.\]
        Because $X$ is the least fixpoint of $F_\Psi^\mathcal{T}$ the resulting set $X'$ is the smallest set that
        fulfills $X' \subseteq X$ and so
        \[(\eta(x_{1}), \dots, \eta(x_f), q_0, \dots, q_0) \in \llbracket
        \Gamma \vdash \mu(X\colon \bullet).\,F(\Psi) \colon \bullet \rrbracket^{V(\eta)}_\mathcal{T}.\]
        With $F(Phi) = (\dots ((\mu (X \colon T(\tau)).\,\Phi(X, X_1, \dots, X_n))(V_1))(V_2))\dots)(V_n)$ it follows
        \[(\eta(x_{1}), \dots, \eta(x_f), q_0, \dots, q_0) \in \llbracket
        \Gamma \vdash \mu(X\colon \bullet).\,F(\Phi) \colon \bullet \rrbracket^{V(\eta)}_\mathcal{T}.\]
        exactly then if $\mathcal{T}, \eta \vdash \Phi$.
    \end{compactitem}
\end{proof}

The combination of Lemma~\ref{lemma:ho_lfp_equals_phfl}, Theorem~\ref{theorem:hoLfpEqualsExptime} and
Theorem~\ref{theorem:phfl_k_in_k_exptime} proves the following theorem.

\begin{theorem}
    Let $k \geq 0$. PHFL$^k$ captures \exptime{$k$} over labeled transition systems.
\end{theorem}

