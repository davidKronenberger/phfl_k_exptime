\section{Preparation}\label{sec:lower_bounds_preparation}

Before we can start with the encodings there are some important steps that we have to consider.  Let $\mathcal{T} = (Q, 
\Sigma, P, \Delta, v)$ an LTS, $q_1, \dots, q_n, p_1, \dots, p_m \in Q$ some states of $\mathcal{T}$ and $\mathcal{T}'$ the 
reduced LTS of $\mathcal{T}$ with respect to $p_1, \dots, p_m$.  
As mentioned in the introdution of this chapter it is known from~\cite{lange2014capturing} that PHFL cannot distinguish 
between bisimilar structures. That means PHFL formulas can only define bisimulation-invariant graph problems. That means that without loss of generality we can check if $([q_1]_\sim, \dots, [q_n]_\sim)$ satisfies a formula $\Phi$ in respect of $\mathcal{T}'$ instead of checking if $(q_1, \dots, q_n)$ satisfies 
$\Phi$ in respect of $\mathcal{T}$. With $[q_i]_\sim$ we denote the equivalence class of $q_i$ with respect to $\sim$.

From this point all LTS are reduced LTS with respect to some states of their state sets. In more detail, let $\mathcal{T} = (Q, \Sigma, P, \Delta, v)$ be a reduced LTS with respect to $(q_1, \dots, q_r)$, where $q_1, \dots, q_r \in Q$. On those LTS it is possible to define a total order on their states. 

\begin{remark} \label{remark:transitive_relation}
    In~\cite{otto1999bisimulation} it was shown that it is possible to define a $2$-adic PHFL$^0$ formula $\Phi_<$ over reduced LTS that defines a
    transitive relation $<$ such that $< \cap > = \emptyset$ and $< \cup > = \not\sim$. This relation $<$ defines a total
    order on states of a reduced LTS.
\end{remark}