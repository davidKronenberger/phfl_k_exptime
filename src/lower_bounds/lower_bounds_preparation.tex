\section{Preparation}\label{sec:lower_bounds_preparation}

Before we can start with the encodings there are some important steps that we have to consider.  Let $\mathcal{T} = (Q, 
\Sigma, P, \Delta, v)$ an LTS, $q_1, \dots, q_n, p_1, \dots, p_m \in Q$ some states of $\mathcal{T}$ and $\mathcal{T}'$ the 
reduced LTS of $\mathcal{T}$ with respect to $p_1, \dots, p_m$.  
As mentioned in the introdution of this chapter it is known from~\cite{lange2014capturing} that PHFL cannot distinguish 
between bisimilar structures. What means PHFL formulas can only define bisimulation-invariant graph problems. We use this 
knowledge to make the model checking of HO and PHFL formulas easier. That means without loss of generality instead of checking if a tuple of states 
$(q_1, \dots, q_n)$ satisfies a formula $\Phi$ in respect to $\mathcal{T}$ we can also check if $(q_1, \dots, q_n)$ satisfies 
$\Phi$ in respect to $\mathcal{T}'$. 

From now all LTS are reduced LTS with respect to some states of their state sets. In more detail, let $\mathcal{T} = (Q, \Sigma, P, \Delta, v)$ in this chapter a reduced LTS with respect to $q_1, \dots, q_r$, where $q_1, \dots, q_r \in \Sigma$. Because each state of those LTS can be reached by at least one $q_i$ it is possible to define a total order on their states. 

\begin{remark} \label{remark:transitive_relation}
    In~\cite{otto1999bisimulation} was shown that it is possible to define a $2$-adic formula $\Phi_<$ that defines a
    transitive relation $<$ such that $< \cap > = \emptyset$ and $< \cup > = \not\sim$. In this thesis $<$ defines a total
    order on states of a reduced LTS.
\end{remark}