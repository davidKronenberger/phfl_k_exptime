%%
%% Author: DKron
%% 17.08.2018
%%

\section{Lower Bound of PHFL$^{k + 1}_{tail}$}\label{sec:lowerBoundOfPhflTail}

The lower bound of PHFL$^{k+1}_{tail}$ can be proven similar to the lower bound of PHFL$^k$. The main idea is not to
show directly, that the lower bound of PHFL$^{k+1}_{tail}$ is \expspace{$k$} but rather by detour over the
bisimulation-invariant fragment of HO(PFP)$^{k+1}$. In the first subsection we show that a Turing Machine that is in
$k$-EXPSPACE can be encoded by a HO(PFP)$^{k+1}$ formula. The next subsection uses this statement to show that the
lower bound of PHFL$^{k+1}_{tail}$ is \expspace{$k$} by encoding formulas of HO(PFP)$^{k+1}$ in PHFL$^{k+1}_{tail}$.

\subsection{$k$-EXPSPACE in HO(PFP)$^{k+1}$}\label{subsec:kExpspaceInHopfp}

In this subsection we want to show that a run of an $exp(k, f(n))$ space bounded DTM can be encoded by some
HO$^{k+1}$ formula. The main idea of this statement is an extension of the result of Abiteboul and
Vianu~\cite{abiteboul1995computing} into higher-order. They have shown that HO(PFP)$^1$ coincides with $0$-EXPSPACE.

%Because a finite $\sigma$-structure $\mathcal{A}$ for some relational signature $\sigma$ cannot be fed to a Turing
%%machine directly, we have to encode $\mathcal{A}$ to an input word $\langle \mathcal{A} \rangle$. We refer the
%interested reader to~\ref{}

\begin{lemma}
    \label{lemma:expspace_in_ho_pfp}
    Given a DTM $M = (Q, \Sigma, \Gamma, \delta, q_0, \Box, F, R)$ in $k$-EXPSPACE and some input word $w \in
    \Sigma^*$ then there exists a formula $\Psi$ in HO(PFP)$^{k+1}$ such that a run of $M$ on $w$ is accepting iff
    $\Psi$ is true.
\end{lemma}

\begin{proof}
    Let $M = (Q, \Sigma, \Gamma, \delta, q_0, \Box, F, R)$ be a $exp(k, f(n))$ space bounded DTM and $w
    \in \Sigma$ some input word. Furthermore, let $\sigma$ a relational signature, $\sigma_< = \sigma \cup
    <$ an extension of $\sigma$ and $\mathcal{A}$ a finite $\sigma_<$-structure including $<^\mathcal{A}$ that
    represents a linear ordering on the universe $\mathcal{U}$ of $\mathcal{A}$. Finally, let $\tau$ be an HO type of
    order $k + 1$.
    To prove this lemma we want to define a relational representation of the final configuration of $M$ on $w$
    as a partial fixpoint of some HO$^{k+1}$ formula. For this we set up the underlying HO$^{k+1}$ formula
    $\varphi$ such that the iterates $(F_\varphi^\mathcal{A})^{i+1}(\emptyset)$ over $D_\tau
    (\mathcal{U})$ represent the $i$-th configuration of $M$ on input $w$, for every $i$ until termination.

    Before we can define the configurations of $M$ in HO we have to make some preparations. Remember that a
    configuration of $M$ is a relation between the current state, the current reading head position and
    the current tape content represented by a function. These configurations will be combined in one relation $X$.
    Because the size of the formula we build have to be polynomial and the reading head of $M$ can be on
    one of $exp(k, f(n))$ cells for example, we have to encode the number in sets of order $k + 1$. Furthermore, to do
    not exceed the bound of order $k + 1$, we have to split the tape content function in this way that in one tuple
    $x$ of $X$ is just the current state, the current head position and one position of the tape with its content. By
    syntax of HO types each component of $x \in X$ have to be of the same type, so to access the different states and
    tape symbols they have to be numerated. $\{0, \dots, |Q| - 1\}$ for states and $\{0, \dots, |\Gamma| - 1\}$ for tape
    symbols.

    The next step is to define some abbreviations that we want to use in the definitions of the configurations. The
    first and most important abbreviation is the definition of orders of any HO type. These orders are defined similar
    to the defined formulas of Definition~\ref{definition:lower_bound_less}.
    \begin{align*}
        <^\odot(x, y) \coloneqq &\,<^\mathcal{A}(x, y) \\
        <^{\tau' \times \dots \times \tau'}(x_1, y_1, \dots, x_n, y_n) \coloneqq &\,\underset{i =
        1}{\overset{n}{\bigvee}}<^{\tau'}(x_i, y_i) \wedge \underset{j = 1}{\overset{i - 1}{\bigwedge}}
        \neg <^{\tau'}(y_j, x_j)\\
        <^{(\tau', \dots, \tau')}(X, Y) \coloneqq &\,\exists (x_1 \colon {\tau'}). \,\dots \exists(x_n \colon
        {\tau'}).\, Y(x_1, \dots, x_n)
        \\&\,\wedge \neg X(x_1, \dots, x_n)\,\wedge \forall (y_1 \colon {\tau'}). \,\dots
        \forall(y_1 \colon {\tau'}).\,\\&\,<^{\tau'\times \dots \times \tau'}
        (y_1, x_1, \dots, y_n, x_n) \\&\,\Rightarrow (X(y_1, \dots, y_n) \Rightarrow Y(y_1, \dots, y_n))
    \end{align*}
    With these formulas it is possible to define another two important abbreviations. On the one hand equality
    of two variables of arbitrary type and on the other hand the successor of a given element. If $X$ and
    $Y$ are two variables of type $\tau$ then the equality of $X$ and $Y$ is given by the formula
    \[X = Y \coloneqq \neg<^\tau(X, Y) \wedge \neg <^\tau(Y, X).\]
    Finally, if $X$ and $Y$ are two variables of type $\tau$ then the prove that $Y$ is the successor of $X$ is given by
    the formula
    \[next^{\tau}(X, Y) \coloneqq\; <^\tau(X, Y) \wedge \forall (Z \colon \tau).\, <^\tau(Z, Y) \Rightarrow\;<^\tau
    (Z, X).\]

    Now we are able to define the configurations in HO(PFP)$^k$. The first configuration is the initial configuration
    $C_0^M$. For input word $w$ this is given by the formula
    \begin{align*}
        \varphi_0(q, h, i, b) \coloneqq &\,q = q_0 \wedge h = 0 \wedge (\neg <^\tau(|w|, i) \Rightarrow b = w_{i})\,
        \vee\\&\,(\neg <^\tau (i, |w|) \wedge \neg i = |w| \Rightarrow b = \Box))
    \end{align*}
    where $q$ is the current state, $h$ the current head position, $i$ a tape index and $b$ the symbol on $i$. $q_0$,
    $0$, $|w|$, $w_{i}$ and $\Box$ are the numerical representations as sets of the same elements in $Q$ and $\Gamma$.

    To iterate through all configurations of $M$ on input $w$, we need a variable $X$ of type $\tau =
    (\tau', \tau', \tau', \tau')$ where $\tau'$ has order $k + 1$, so $X$ has order $k + 2$. On an iteration
    $(F_\varphi^\mathcal{A})^{i+1}(\emptyset)$ for the following formula $\varphi$ the variable $X$ includes
    all configurations that will be reached within $i$ transitions.
    \begin{align*}
        \varphi(X, q, h, i, b) \coloneqq &\,(\neg \exists (q_{old} \colon \tau').\, \neg\exists
        (h_{old}
        \colon \tau').\, \neg\exists (i_{old} \colon \tau').\, \neg\exists (b_{old} \colon \tau').\,\\
        &\, X(q_{old}, h_{old}, i_{old}, b_{old}) \wedge \varphi_0(q, h, i, b)) \,\vee \\
        &\,(\exists (q_{old} \colon \tau').\, \exists (h_{old} \colon \tau').\, \exists (i_{old} \colon
        \tau').\, \exists (b_{old} \colon \tau').\,\\
        &\, X(q_{old}, h_{old}, i_{old}, b_{old}) \wedge \xi(X, q, h, i, b))
    \end{align*}
    Note that $\varphi_0$ is invoked only in the first iteration and thus provides the correct initialisation. The
    formula $\xi$ collects the transitions of those tuples in $X$ according to the transition function $\delta$ of
    $M$. In each iteration exactly one configuration will be added to $X$ because $M$ is deterministic.
    The formula $\xi$ is given by
    \begin{align*}
        \xi(X, q, h, i, b) \coloneqq &\,\exists (q_{old} \colon \tau').\, \exists (h_{old} \colon
        \tau').\, \exists (i_{old} \colon \tau').\, \exists (b_{old} \colon \tau').\, \\
        &\,\exists (q_{new} \colon \tau').\, \exists (b_{new} \colon \tau').\,X(q_{old}, h_{old}, i_{old},
        b_{old}) \,\wedge \\
        &\, (\underset{\delta(q_{old}, b_{old}) = (q_{new}, b_{new}, d)}{\bigvee} q = q_{new} \wedge h =
        h_{old} + d \wedge i = i_{old}\,\wedge\\&\, ((\neg i = h_{old} \wedge b =
        b_{old}) \vee (i = h_{old} \wedge b = b_{new})))
    \end{align*}
    where $h = h_{old} + d$ depends on $d$ and is given by
    \[h = h_{old} + d \coloneqq
    \begin{cases}
        next^\tau(h_{old}, h),  & \text{if } d = L\\
        next^\tau(h, h_{old}),  & \text{if } d = R\\
        h = h{old},  & \text{if } d = N
    \end{cases}\]

    Because $M$ terminates the formula
    \[\psi_0(q', h', i', b') \coloneqq [\mathit{PFP}\;\varphi(X, q, h, i, b)](q', h', i', b') \]
    is guaranteed to define the relational description of the final configuration of $M$ on input word $w$.
    Finally, the formula
    \[\psi \coloneqq \exists (h' \colon \tau').\, \exists (i' \colon \tau').\, \exists (b' \colon \tau').\,
    \underset{q' \in F}{\bigvee} \psi_0(q', h', i', b')\]
    defines the acceptance of $M$ on input $w$.
\end{proof}

\subsection{Bisimulation Invariant HO(PFP)$^{k+1}$ to PHFL$^{k+1}_{tail}$}\label{subsec:bisimulationInvariantHopfptoPhfl}

As mentioned in the introduction of this section the main idea is not to show directly, that the lower bound of
PHFL$^{k+1}_{tail}$ is \expspace{$k$} but rather by detour over the bisimulation-invariant fragment of HO(PFP)
$^{k+1}$. In the previous subsection we have seen that a DTM in \expspace{$k$} can performed by a HO(PFP)$^{k+1}$
formula. This statement we want to use and want to encode a given bisimulation-invariant HO(PFP)$^{k+1}$ into
PHFL$^{k+1}_{tail}$. Because the encoding of the HO$^{k+1}$ part is already proven in
Section~\ref{sec:existential_quantifiers_in_phfl} and Section~\ref{sec:lowerBoundOfPhfl} to be in PHFL$^k$ and these
encodings are all tail-recursive the PFP operator is the only kind of HO(PFP)$^{k+1}$ formula that we have to encode
in this subsection.

\begin{lemma}
    \label{lemma:ho_pfp_equals_phfl_tail}
    Let $r \geq 1$ and $k \geq 0$. For every bisimulation invariant formula $\Phi$ of HO(PFP)$^{k + 1}$ there is a
    PHFL$^{k+1}_{tail}$ formula $\Psi$ such that $\mathcal{Q}_\Phi^r = \mathcal{Q}_\Psi^r$.
\end{lemma}

\begin{proof}
    We have to encode the PFP operator...
\end{proof}

The combination of Lemma~\ref{lemma:ho_pfp_equals_phfl_tail}, Lemma~\ref{lemma:expspace_in_ho_pfp} and
Theorem~\ref{theorem:phfl_k_plus_1_tail_in_k_expspace} proves the following theorem.

\begin{theorem}
    Let $k \geq 0$. PHFL$^{k+1}$ captures \expspace{$k$} over labeled transition systems.
\end{theorem}
