%%
%% Author: DKron
%% 17.08.2018
%%

\section{Lower Bound of PHFL$^{k + 1}_{tail}$}\label{sec:lowerBoundOfPhflTail}

The lower bound of PHFL$^{k+1}_{tail}$ can be proven similar to the lower bound of PHFL$^k$. The main idea is not to
show directly, that the lower bound of PHFL$^{k+1}_{tail}$ is \expspace{$k$} but rather by detour over the
bisimulation-invariant fragment of HO(PFP)$^{k+1}$. In the first subsection we show that a Turing Machine that is in
$k$-EXPSPACE can be encoded by a HO(PFP)$^{k+1}$ formula. The next subsection uses this statement to show that the
lower bound of PHFL$^{k+1}_{tail}$ is \expspace{$k$} by encoding formulas of HO(PFP)$^{k+1}$ in PHFL$^{k+1}_{tail}$.

\subsection{$k$-EXPSPACE in HO(PFP)$^{k+1}$}\label{subsec:kExpspaceInHopfp}

In this subsection we want to show that a run of an $exp(k, f(n))$ space bounded DTM can be encoded by some
HO$^{k+1}$ formula. The main idea of this statement is an extension of the result of Abiteboul and
Vianu~\cite{abiteboul1995computing} into higher-order. They have shown that HO(PFP)$^1$ coincides with $0$-EXPSPACE.

%Because a finite $\sigma$-structure $\mathcal{A}$ for some relational signature $\sigma$ cannot be fed to a Turing
%%machine directly, we have to encode $\mathcal{A}$ to an input word $\langle \mathcal{A} \rangle$. We refer the
%interested reader to~\ref{}

\begin{lemma}
    \label{lemma:expspace_in_ho_pfp}
    Given a DTM $M = (Q, \Sigma, \Gamma, \delta, q_0, \Box, F, R)$ in $k$-EXPSPACE there exists a formula $\Psi$ in HO(PFP)$^{k+1}$ over signature $\sigma$ such that for all variable mappings $\eta$ and all $\sigma$-structures $\mathcal{A}$ it holds $\mathcal{A}, \eta \models \Psi$ exactly then if the run of $M$ on the standard coding of $(\mathcal{A}, \eta)$ is accepting.
\end{lemma}

\begin{proof}
    Let $M = (Q, \Sigma, \Gamma, \delta, q_0, \Box, F, R)$ be a $exp(k, f(n))$ space bounded DTM. Furthermore, let $\sigma$ be a relational signature, $\sigma_< = \sigma\;\cup
    <$ an extension of $\sigma$ and $\mathcal{A}$ a finite $\sigma_<$-structure including $<^\mathcal{A}$ that
    represents a linear ordering on the universe $\mathcal{U}$ of $\mathcal{A}$. Finally, let $\tau$ be an HO type of
    order $k + 1$ and $\alpha$ a variable mapping.
    To prove this lemma we want to define a relational representation, of the final configuration of $M$ that has the standardized encoding of $(\mathcal{A}, \alpha)$\footnote{The standardized encoding of structures is a non-trivial problem. Because the description of the encoding goes beyond the scope of this thesis, we only refer to~\cite{abiteboul1995computing} for further informations about.}, abbreviated with $w$, as input word,
    as a partial fixpoint of some HO(PFP)$^{k+1}$ formula. For this we set up the underlying HO$^{k+1}$ formula
    $\varphi$ such that the iterates over $(F_\varphi^\mathcal{A})^{i+1}(\emptyset)$ over $D_\tau
    (\mathcal{U})$ represent the $i$-th configuration of $M$ on input standard coding of $(\mathcal{A}, \alpha)$, for every $i$ until termination.

    Before we can define the configurations of $M$ in HO we have to make some preparations. Remember that a
    configuration of $M$ is a relation between the current state, the current reading head position and
    the current tape content represented by a function. These configurations will be combined in one relation $X$.
    Because the size of the formula we build have to be polynomial and the reading head of $M$ can be on
    one of $exp(k, f(n))$ cells for example, we have to encode the number in sets of order $k + 1$. Furthermore, to do
    not exceed the bound of order $k + 1$, we have to split the tape content function in this way that in one tuple
    $x$ of $X$ is just the current state, the current head position and one position of the tape with its content. By
    syntax of HO types each component of $x \in X$ have to be of the same type, so to access the different states and
    tape symbols they have to be numerated. $\{0, \dots, |Q| - 1\}$ for states and $\{0, \dots, |\Gamma| - 1\}$ for tape
    symbols.

    The next step is to define some abbreviations that we want to use in the definitions of the configurations. The
    first and most important abbreviation is the definition of orders of any HO type. These orders are defined similarly
    to the defined formulas of Definition~\ref{definition:lower_bound_less_higher}. A tuple $x$ is smaller then a tuple $y$ if there is a position $i$ where $x_i < y_i$ and there is no position $j<i$ where $x_j > y_j$. A set $X$ is smaller then a tuple $Y$ if there is a $x \in Y$ such that $x \not\in X$ and there is no $y<x$ such that $y\in X$ but $y\not\in Y$.
    \begin{align*}
        <^\odot(x, y) \coloneqq &\,<^\mathcal{A}(x, y) \\
        <^{\tau' \times \dots \times \tau'}(x_1, y_1, \dots, x_n, y_n) \coloneqq &\,\underset{i =
        1}{\overset{n}{\bigvee}}<^{\tau'}(x_i, y_i) \wedge \underset{j = 1}{\overset{i - 1}{\bigwedge}}
        \neg <^{\tau'}(y_j, x_j)\\
        <^{(\tau', \dots, \tau')}(X, Y) \coloneqq &\,\exists (x_1 \colon {\tau'}). \,\dots \exists(x_n \colon
        {\tau'}).\, Y(x_1, \dots, x_n)
        \\&\,\wedge \neg X(x_1, \dots, x_n)\,\wedge \forall (y_1 \colon {\tau'}). \,\dots
        \forall(y_1 \colon {\tau'}).\,\\&\,<^{\tau'\times \dots \times \tau'}
        (y_1, x_1, \dots, y_n, x_n) \\&\,\Rightarrow (X(y_1, \dots, y_n) \Rightarrow Y(y_1, \dots, y_n))
    \end{align*}
    With these formulas it is possible to define another two important abbreviations. On the one hand equality
    of two variables of arbitrary type and on the other hand the successor of a given element. If $X$ and
    $Y$ are two variables of type $\tau$ then the equality of $X$ and $Y$ is given by the formula
    \[X = Y \coloneqq \neg<^\tau(X, Y) \wedge \neg <^\tau(Y, X).\]
    Finally, if $X$ and $Y$ are two variables of type $\tau$ then the prove that $Y$ is the successor of $X$ is given by
    the formula
    \[next^{\tau}(X, Y) \coloneqq\; <^\tau(X, Y) \wedge \forall (Z \colon \tau).\, <^\tau(Z, Y) \Rightarrow\;<^\tau
    (Z, X).\]

    Now we are able to define the configurations in HO(PFP)$^k$. The first configuration is the initial configuration
    $C_0^M$. For input word $w$ this is given by the formula
    \begin{align*}
        \varphi_0(q, h, i, b) \coloneqq &\,q = q_0 \wedge h = 0 \wedge (\neg <^\tau(|w|, i) \Rightarrow b = w_{i})\,
        \vee\\&\,(\neg <^\tau (i, |w|) \wedge \neg i = |w| \Rightarrow b = \Box))
    \end{align*}
    where $q$ is the current state, $h$ the current head position, $i$ a tape index and $b$ the symbol on $i$. $q_0$,
    $0$, $|w|$, $w_{i}$ and $\Box$ are the numerical representations as sets of the same elements in $Q$ and $\Gamma$.

    To iterate through all configurations of $M$ on input $w$, we need a variable $X$ of type $\tau =
    (\tau', \tau', \tau', \tau')$ where $\tau'$ has order $k + 1$, so $X$ has order $k + 2$. On an iteration
    $(F_\varphi^\mathcal{A})^{i+1}(\emptyset)$ for the following formula $\varphi$ the variable $X$ includes
    all configurations that will be reached within $i$ transitions.
    \begin{align*}
        \varphi(X, q, h, i, b) \coloneqq &\,(\neg \exists (q_{old} \colon \tau').\, \neg\exists
        (h_{old}
        \colon \tau').\, \neg\exists (i_{old} \colon \tau').\, \neg\exists (b_{old} \colon \tau').\,\\
        &\, X(q_{old}, h_{old}, i_{old}, b_{old}) \wedge \varphi_0(q, h, i, b)) \,\vee \\
        &\,(\exists (q_{old} \colon \tau').\, \exists (h_{old} \colon \tau').\, \exists (i_{old} \colon
        \tau').\, \exists (b_{old} \colon \tau').\,\\
        &\, X(q_{old}, h_{old}, i_{old}, b_{old}) \wedge \xi(X, q, h, i, b))
    \end{align*}
    Note that $\varphi_0$ is invoked only in the first iteration and thus provides the correct initialisation. The
    formula $\xi$ collects the transitions of those tuples in $X$ according to the transition function $\delta$ of
    $M$. In each iteration exactly one configuration will be added to $X$ because $M$ is deterministic.
    The formula $\xi$ is given by
    \begin{align*}
        \xi(X, q, h, i, b) \coloneqq &\,\exists (q_{old} \colon \tau').\, \exists (h_{old} \colon
        \tau').\, \exists (i_{old} \colon \tau').\, \exists (b_{old} \colon \tau').\, \\
        &\,\exists (q_{new} \colon \tau').\, \exists (b_{new} \colon \tau').\,X(q_{old}, h_{old}, i_{old},
        b_{old}) \,\wedge \\
        &\, (\underset{\delta(q_{old}, b_{old}) = (q_{new}, b_{new}, d)}{\bigvee} q = q_{new} \wedge h =
        h_{old} + d \wedge i = i_{old}\,\wedge\\&\, ((\neg i = h_{old} \wedge b =
        b_{old}) \vee (i = h_{old} \wedge b = b_{new})))
    \end{align*}
    where $h = h_{old} + d$ depends on $d$ and is given by
    \[h = h_{old} + d \coloneqq
    \begin{cases}
        next^\tau(h_{old}, h),  & \text{if } d = L\\
        next^\tau(h, h_{old}),  & \text{if } d = R\\
        h = h{old},  & \text{if } d = N
    \end{cases}\]

    Because $M$ terminates the formula
    \[\psi_0(q', h', i', b') \coloneqq [\mathit{PFP}\;\varphi(X, q, h, i, b)](q', h', i', b') \]
    is guaranteed to define the relational description of the final configuration of $M$ on input word $w$.
    Finally, the formula
    \[\psi \coloneqq \exists (h' \colon \tau').\, \exists (i' \colon \tau').\, \exists (b' \colon \tau').\,
    \underset{q' \in F}{\bigvee} \psi_0(q', h', i', b')\]
    defines the acceptance of $M$ on input $w$.
\end{proof}

\subsection{Bisimulation Invariant HO(PFP)$^{k+1}$ to PHFL$^{k+1}_{tail}$}\label{subsec:bisimulationInvariantHopfptoPhfl}

As mentioned in the introduction of this section the main idea is not to show directly, that the 
lower bound of PHFL$^{k+1}_{tail}$ is \expspace{$k$} but rather by detour over the 
bisimulation-invariant fragment of HO(PFP)$^{k+1}$. In the previous subsection we have seen 
 that a DTM in $k$-EXPSPACE can be performed by an HO(PFP)$^{k+1}$
formula. By encoding the bisimulation-invariant fragment of HO(PFP)$^{k+1}$ into
PHFL$^{k+1}_{tail}$ combined with the knowledge that $k$-EXPSPACE is captured by 
HO(PFP)$^{k+1}$ leads to the lower bound of PHFL$^{k+1}_{tail}$. In
Section~\ref{sec:existential_quantifiers_in_phfl} and Section~\ref{sec:lowerBoundOfPhfl} we 
have shown that the HO$^{k+1}$ part can be encoded in PHFL$^k$. It is easy to prove that the 
encoded formulas are all tail-recursive\footnote{The used PHFL$^0$ formula $\Phi_\sim$ (Example~\ref{example:phfl_order_0}) is, indeed, not tail-recursive, but over finite LTS it can be  encoded to a tail-recursive PHFL$^1$ formula $\Psi$ such that the query defined by $\Phi_\sim$ is equivalent to the query defined by $\Psi$~\cite{lange2014capturing_long}.}. It follows that the HO$^{k+1}$ part can also be 
encoded in  PHFL$^{k+1}_{tail}$. The PFP operator is the only kind of HO(PFP)$^{k+1}$ formula 
that we have to encode in this subsection to get the lower bound of PHFL$^{k+1}_{tail}$.

Before we give the definition of the transforming function, we define a PHFL formula for HO formulas that uses the PFP operator.

\begin{definition}
Let $d$ be the constant as described in Remark~\ref{remark:dimension} and $X$ a HO variable of HO type $\tau = (\tau', \dots, \tau')$ where $\tau' \neq \odot$. Furthermore, let
    $\Phi$ be a PHFL$^k$
    formula, then $PFP^\tau X.\,\Phi$
    is a PHFL$^k$ formula with dimension $d$ defined as:
    \begin{align*}
     PFP^\tau X. \, \Phi \coloneqq &\Big(\mu (F \colon T(\tau) \rightarrow \bullet).\,\lambda (X \colon T(\tau)).\, \big(X\,\wedge \forall^{\tau'}X_1.\, \dotsb \forall^{\tau'}X_n.\, \\&( (\dotsb (X X_1) \dotsb) X_n \Leftrightarrow \Phi(X, X_1, \dots, X_n) ) \vee F(\Phi(X)\big)\Big)\bot_{T(\tau)}
\end{align*}    
    In case of $\tau = (\odot, \dots, \odot)$ let $PFP^{(\odot, \dots, \odot)} X.\,\Phi$ defined as:
    \begin{align*}
    PFP^{(\odot, \dots, \odot)} X.\,\Phi \coloneqq & \Big(\mu (F \colon \bullet \rightarrow \bullet).\,\lambda (X \colon \bullet).\, \big(X \wedge \forall_1 \dotsb \forall_n \\&(X \Leftrightarrow \Phi(X)) \vee F(\Phi(X)\big)\Big)\,\bot 
    \end{align*}
\end{definition}

Now we are able to define the function that translates a bisimulation invariant HO(PFP)$^{k+1}$
formula to a PHFL$^{k+1}_{tail}$. 

\begin{definition}
    \label{definition:lower_bounds_phfl_formula_function_pfp}
   Define $F$ as the function that maps a bisimulation invariant HO(PFP)$^{k+1}$ formula $\varphi$ to a PHFL$^{k+1}_{tail}$ formula with dimension $d$, where $d$ and $s$ are the constants as described in Remark~\ref{remark:dimension} and $\Phi_\sim$ is the formula of Example~\ref{example:phfl_order_0}, then $F$ is defined
    inductive on $\varphi$ as follows:
    \begin{align*}
        F(p(x_i)) \coloneqq &\, p_{2s+i} \\
        F(a(x_i, x_j)) \coloneqq &\, \langle a \rangle_{2s+i} \{(2s+i, 2s+j, \\
        &\,3, \dots, d)\} \Phi_\sim \\
        F(\Phi \vee \Psi) \coloneqq &\, F(\Phi) \vee F(\Psi) \\
        F(\neg \Phi) \coloneqq &\, \neg F(\Phi) \\
        F(\exists (x_i \colon \odot).\,\Phi) \coloneqq &\, \exists_{2s+i} F(\Phi) \\
        F(\exists (X \colon \tau).\,\Phi) \coloneqq &\, \exists^\tau X.\,F(\Phi(X)) \\
        F([PFP\;\Phi(X, x_{i_1}, \dots, x_{i_n})](v_{j_1}, \dots, v_{j_n})) \coloneqq &\,\{(j_1, \dots, j_n, n + 1, \dots, d)\} \\
        &\,PFP^{(\odot, \dots, \odot)} X.\, F(\Phi) \\
        F([PFP\;\Phi(X, X_1, \dots, X_n)](V_1, \dots, V_n)) \coloneqq &\,(\dotsb \big(PFP^\tau X.\, F(\Phi)\big)\,V_1)\dotsb)\,V_n \\
        F(X(x_{i_1}, \dots, x_{i_n})) \coloneqq &\, \{(2s+i_1, \dots, 2s+i_n, \\
        &\,n + 1, \dots, d)\}X\\
        F(X(X_1, \dots, X_n)) \coloneqq &\, (\dotsb (X\,X_1)\dotsb)\,X_n
    \end{align*}
\end{definition}

The last step is to show that the semantics of a given HO(PFP)$^{k+1}$ formula coincides with the semantics of the by function $F$ encoded PHFL$^{k+1}_{tail}$ formula. As mentioned in Section~\ref{sec:lower_bounds_preparation} without loss of generality the statement can be proven by consider only  reduced LTS. 

\begin{lemma}
    \label{lemma:ho_pfp_equals_phfl_tail}
    Let $f \geq 1$ and $k \geq 0$. For every bisimulation-invariant formula $\Phi$ of HO(PFP)$^{k + 1}$ there is a
    PHFL$^{k+1}_{tail}$ formula $\Psi$ such that the $f$-adic query $\mathcal{Q}_\Phi^f$ that belongs to $\Phi$ is equal to the $f$-adic query  $\mathcal{Q}_\Psi^f$ that belongs to $\Psi$.
\end{lemma}

\begin{proof}
    This lemma can be proven by showing for all HO(PFP)$^{k+1}$ formulas $\Phi$ with first-order variables $x_1,
    \dots, x_q$, all reduced LTS $\mathcal{T} = (Q, \Sigma, P,
    \Delta, v)$ with respect to $\emph{q}_r = q_1, \dots, q_r$ and all variable mappings $\eta$ that it holds that $\mathcal{T}, \eta \models \Phi$ iff $\emph{q} =
    (\emph{q}_s, \emph{q}_s, \emph{q}_q, \emph{q}_r)$ and $\emph{q} \in \llbracket
   F(\Phi)\rrbracket^{\eta_V}_{\mathcal{T}}$, where $\emph{q}_s = q_1',\dots,q_s'$ is a sequence of $s$ placeholders used for the interaction of second-order variables, $\emph{q}_q = \eta(x_1), \dots, \eta(x_q)$ is a sequence of first-order variables that are mapped by $\eta$ where $\eta(x_i) = q_0$ if $x_i$ is bound to a quantifier, $q_0, q_1', \dots, q_s' \in Q$ are arbitrary states, $F$ is the formula function of
    Definition~\ref{definition:lower_bounds_phfl_formula_function} and $\eta_V$ the variable mapping of
    Definition~\ref{definition:lower_bound_variable_function}. This statement can be proven by induction over formula
    $\Phi$.
    Because the correctness proof of the non-fixpoint formulas is very similar to the correctness proof of Theorem~\ref{theorem:ho_lfp_equals_phfl} we concentrate us on showing correctness of the PFP operators.
    \begin{compactitem}
    \item In case of $\Phi = [PFP\,\Psi(X, x_{i_1}, \dots, x_{i_n})](v_{j_1}, \dots, v_{j_n})$, where $X$ is a
        free variable in $\Psi$ of HO type $(\odot, \dots, \odot)$ and $x_{i_1}, \dots, x_{i_n}$ are free first-order
        variables of $\Psi$ and $v_{j_1}, \dots, v_{j_n}$ are first-order variables of $\Phi$, then it follows that
        $\mathcal{T}, \eta \models \Phi$ exactly then if $(\eta(v_{j_1}), \dots, \eta(v_{j_n})) \in PFP
        (F_\Psi^\mathcal{T})$. 
        By Definition~\ref{definition:pfp} this is the 
        case when there is an $i$ such that ${F_\Psi^\mathcal{T}}^i(\emptyset) = {F_\Psi^		
        \mathcal{T}}^{i+1}(\emptyset)$ and $(\eta(v_{j_1}), \dots, \eta(v_{j_n})) \in {F_\Psi^\mathcal{T}}^i(\emptyset)$. By Definition~\ref{definition:induced_operator} $(\eta(v_{j_1}), \dots, 		
        \eta(v_{j_n})) \in {F_\Psi^\mathcal{T}}^i(\emptyset)$ if $\mathcal{T}, \eta \models \Psi({F_\Psi^
        \mathcal{T}}^{i}(\emptyset), \eta(v_{j_1}), \dots, $ $\eta(v_{j_n}))$. By induction hypothesis 
        this is exactly the case when 
\begin{align*}
        (\emph{q}_s, &\emph{q}_s, \eta(x_1), \dots, \eta(v_{g(1)-1}), \eta(v_{g(1)}), \eta(v_{g(1)+1}), \dots \eta(v_{g(2)-1}),\\& \eta
            (v_{g(2)}), \eta(v_{g(2)+1}), \dots, \eta(v_{g(n)-1}), \eta(v_{g(n)}), \eta(v_{g(n)+1}), \dots, \eta
            (x_f), \\&\emph{q}_q, \emph{q}_r) \in \llbracket
        F(\Psi) \rrbracket^{\eta_V}_\mathcal{T},
        \end{align*}
         where $g: \{1, \dots, n\} $ $\rightarrow \{j_1, \dots, j_n\}$ is a function.        
        Note that $X$ is set to ${F_\Psi^\mathcal{T}}^{i}(\emptyset)$.
		         
         If we use $\lambda$-approximation and $\beta$-reduction on PFP$^{(\odot, \dots, \odot)} X.\,\Psi$ we can 
         see that it can be summarized to
         \begin{align*}
         \varphi \coloneqq \underset{i}{\bigvee} \Big(&\Psi^i(\bot) \wedge \forall_1 \dotsb \forall_n
         \big(\Psi^i(\bot) \Leftrightarrow \Psi^{i+1}(\bot)\big)\Big)
         \end{align*}
		Note that 
		\[{F_\Psi^\mathcal{T}}^i(\emptyset) = \Psi^i(\bot).\] 
		This holds because by induction over $i$ obviously it holds
		$\emptyset = \bot$
		and 
		$F_\Psi^\mathcal{T}(\emptyset) = \Psi(\bot).$
		That means 
		${F_\Psi^\mathcal{T}}^{i+1}(\emptyset) = \Psi^{i+1}(\bot)$
		holds because by induction hypothesis it holds 
		${F_\Psi^\mathcal{T}}^{i}
		(\emptyset) = \Psi^{i}(\bot)$
		and 
		$F_\Psi^\mathcal{T}({F_\Psi^\mathcal{T}}^{i}(\emptyset))= \Psi\big(\Psi^{i}(\bot)\big).$
		
		Because it exists an $i$ such that ${F_\Psi^\mathcal{T}}^i(\emptyset) = {F_\Psi^		
        \mathcal{T}}^{i+1}(\emptyset)$ it follows that the right conjunct of $\varphi$
		\begin{align*}
		\forall_1 \dotsb \forall_n \big(\Psi^i(\bot) \Leftrightarrow \Psi^{i+1}(\bot)\big)
         \end{align*}
		holds also. The left conjunct of $\varphi$ returns then the function that we get through  
		the $i$-th application of $\Psi$ on the empty set.
		
		Because of the construction of variable mapping $\eta_V$ and $(\eta(v_{j_1}), \dots, 
        \eta(v_{j_n})) \in \eta(X)$ the
        tuple $(\eta(v_{j_1}), \dots, \eta(v_{j_n}), q_{n+1}', \dots, q_s', \emph{q}_s, \emph{q}_f,  
        \emph{q}_q,  \emph{q}_r)$ is also in $\eta_V(X)$.
         Because $(\eta(v_{j_1}), \dots, \eta(v_{j_n})) \in {F_\Psi^\mathcal{T}}^i(\emptyset)$ and $(\eta(v_{j_1}), \dots, \eta(v_{j_n}), $ $q_{n+1}', \dots, q_s', \emph{q}_s, \emph{q}_f,  
        \emph{q}_q,  \emph{q}_r) \in \eta_V(X)$ it follows with 
        \[F(\Phi) = \{(i_1, \dots, i_n)\} (PFP^{(\odot, \dots, \odot)} X.\, F(\Psi)) \]
        that 
        \begin{align*}
        (\emph{q}_s, &\emph{q}_s, \eta(x_1), \dots, \eta(v_{g(1)-1}), \eta(v_{g(1)}), \eta(v_{g(1)+1}), \dots \eta(v_{g(2)-1}),\\& \eta
            (v_{g(2)}), \eta(v_{g(2)+1}), \dots, \eta(v_{g(n)-1}), \eta(v_{g(n)}), \eta(v_{g(n)+1}), \dots, \eta
            (x_f), \\&\emph{q}_q, \emph{q}_r) \in \llbracket F(\Phi) \rrbracket_\mathcal{T}^{\eta_V}.
        \end{align*}

        \item In case of $\Phi = [PFP\,\Psi(X, X_1, \dots, X_n)](V_1, \dots, V_n)$, where $X$ is a
        free variable in $\Psi$ of HO type $(\tau, \dots, \tau)$ and $X_1, \dots, X_n$ are free 
        variables of $\Psi$ of type $\tau$ and $V_1, \dots, V_n$ are free variables of $\Phi$ also of 
        type $\tau$, then it follows that $\mathcal{T}, \eta \models \Phi$ exactly then if $(\eta(V_1), 
        \dots, \eta(V_n)) \in PFP(F_\Psi^\mathcal{T})$. By Definition~\ref{definition:pfp} this is the 
        case when there is an $i$ such that ${F_\Psi^\mathcal{T}}^i(\emptyset) = {F_\Psi^		
        \mathcal{T}}^{i+1}(\emptyset)$ and $(\eta(V_1), \dots, \eta(V_n)) \in {F_\Psi^\mathcal{T}}
        ^i(\emptyset)$. By Definition~\ref{definition:induced_operator} $(\eta(V_1), \dots, 		
        \eta(V_n)) \in {F_\Psi^\mathcal{T}}^i(\emptyset)$ if $\mathcal{T}, \eta \models \Psi({F_\Psi^
        \mathcal{T}}^{i}(\emptyset), \eta(V_1), \dots, $ $\eta(V_n))$. By induction hypothesis 
        this is exactly the case when $\emph{q} \in \llbracket F(\Psi) \rrbracket_\mathcal{T}
        ^{\eta_V}$. Note that $X$ is set to ${F_\Psi^\mathcal{T}}^{i}(\emptyset)$.
		         
         If we use $\lambda$-approximation and $\beta$-reduction on PFP$^\tau X.\,\Psi$ we can 
         see that it can be summarized to
         \begin{align*}
         \varphi \coloneqq \underset{i}{\bigvee} \Big(&\Psi^i(\bot_{T((\tau, \dots, \tau))}) \wedge \forall^{\tau'} X_1.\, \dotsb \forall^{\tau'} X_n.\,\\& 
         \big((\dotsb(\Psi^i(\bot_{T((\tau, \dots, \tau))}) X_1)\dotsb)X_n \Leftrightarrow \\&\;(\dotsb(\Psi^{i+1}(\bot_{T((\tau, \dots, \tau))}) X_1)\dotsb)X_n\big)\Big)
         \end{align*}
		Note that 
		\[{F_\Psi^\mathcal{T}}^i(\emptyset) = \Psi^i(\bot_{T((\tau, \dots, \tau))}).\] 
		This holds because by induction over $i$ obviously it holds
		\[\emptyset = \bot_{T((\tau, \dots, \tau))}\] 
		and 
		\[F_\Psi^\mathcal{T}(\emptyset) = \Psi(\bot_{T((\tau, \dots, \tau))}).\] 
		That means 
		\[{F_\Psi^\mathcal{T}}^{i+1}(\emptyset) = \Psi^{i+1}(\bot_{T((\tau, \dots, \tau))})\] 
		holds because by induction hypothesis it holds 
		\[{F_\Psi^\mathcal{T}}^{i}
		(\emptyset) = \Psi^{i}(\bot_{T((\tau, \dots, \tau))})\] 
		and 
		\[F_\Psi^\mathcal{T}({F_\Psi^\mathcal{T}}^{i}(\emptyset))= \Psi\big(\Psi^{i}(\bot_{T((\tau, \dots, \tau))})\big).\]
		Because it exists an $i$ such that ${F_\Psi^\mathcal{T}}^i(\emptyset) = {F_\Psi^		
        \mathcal{T}}^{i+1}(\emptyset)$ it follows that the right conjunct of $\varphi$
		\begin{align*}
		&\forall^{\tau'} X_1.\, \dotsb \forall^{\tau'} X_n.\,\\& 
         \big((\dotsb(\Psi^i(\bot_{T((\tau, \dots, \tau))}) X_1)\dotsb)X_n \Leftrightarrow \\&\;(\dotsb(\Psi^{i+1}(\bot_{T((\tau, \dots, \tau))}) X_1)\dotsb)X_n\big)
         \end{align*}
		holds also. The left conjunct of $\varphi$ returns then the function that we get through  
		the $i$-th application of $\Psi$ on the empty set.
		
		Because of the construction of variable mapping $\eta_V$ and $(\eta(V_1), \dots, \eta(V_n)) 
		\in \eta(X)$ it holds $(\dotsb\big(\eta_V(X)\,\eta_V(V_1)\big) \dotsb )\,\eta_V(V_n) = Q^d$
        and so 
         \[\emph{q} \in (\dotsb\big(\eta_V(X)\,\eta_V(V_1)\big) \dotsb )\,\eta_V(V_n).\]
         Because $(\eta(V_1), \dots, \eta(V_n)) \in {F_\Psi^\mathcal{T}}^i(\emptyset)$ and $
         \emph{q} \in (\dotsb\big(\eta_V(X)\,\eta_V(V_1)\big) \dotsb )\,$ $\eta_V(V_n)$ it follows with 
        \[F(\Phi) = (\dotsb((PFP^\tau X.\, F(\Phi))\, V1)\dotsb )\, V_n \]
        that $\emph{q} \in \llbracket F(\Phi) \rrbracket_\mathcal{T}^{\eta_V}$.
    \end{compactitem}
\end{proof}

The combination of Lemma~\ref{lemma:ho_pfp_equals_phfl_tail}, Lemma~\ref{lemma:expspace_in_ho_pfp} and
Theorem~\ref{theorem:phfl_k_plus_1_tail_in_k_expspace} proves the following theorem.

\begin{theorem}
    Let $k \geq 0$. PHFL$^{k+1}_{tail}$ captures \expspace{$k$} over labeled transition systems.
\end{theorem}