\section{Preliminaries}
This section introduces the most important definitions concerning picture languages. The notions are
mainly from~\cite{giammarresi1997twodimensional} and~\cite{cherubini2009picture}. We assume that
the reader is already familiar with basic notions of one-dimensional formal language theory.
\begin{definition}
	Let $\Sigma$ be a finite, non-empty set of symbols. Then $p$ is called a \emph{picture over
	$\Sigma$}, if $p$ is a finite rectangular array containing only symbols of $\Sigma$. The \emph{set
	of all pictures over $\Sigma$} is denoted by $\Sigma^{*,*}$. Any $L\subseteq\Sigma^{*,*}$ is called a
	\emph{picture language}.
\end{definition}
Let $p$ be a picture. The \emph{height} and \emph{width} of $p$ are the number of rows and the
number of columns and are denoted as $l_1(p)$ and $l_2(p)$, respectively. The \emph{size} of $p$ is
the pair $(l_1(p), l_2(p))$. The \emph{empty picture} is the only picture of size $(0, 0)$ and is
denoted by $\lambda$. The set $\Sigma^{h, k} \subset \Sigma^{*, *}$ denotes the \emph{set of
pictures over $\Sigma$ of size $(h, k)$}.

If we need evaluate the value of one specific pixel within the picture, we use $p(i, j)$ to return
the symbol in the picture $p$ at the vertical position $i$ and horizontal position $j$, where $1
\leq i \leq l_1(p)$ and $1 \leq j \leq l_2(p)$.

Sometimes it is necessary that a picture is surrounded by a special symbol to detect whether the
border of a picture has been reached. For this purpose, we use the symbol $\#$ as border symbol.
With $\hat{p}$ we denote the picture $p$ bordered with $\#$'s:
\begin{center}
	$\hat{p} =$ \scalebox{0.79}{\begin{tabular}{|C{2.75cm}|c|c|c|c|}
		\hline
		\hspace{0.99cm}\#\hspace{0.99cm} & \hspace{0.99cm}\#\hspace{0.99cm} &
		\hspace{1.45cm}\dots\hspace{1.45cm} & \# & \hspace{0.99cm}\#\hspace{0.99cm} \tabularnewline
		\hline
		\#       & $p(1, 1)$      & \dots    & $p(1, l_2(p))$      & \#	      \tabularnewline
		\hline
		\multirow{1}[12]{1cm}{\centering $\vdots$} & \rule[-2.4cm]{0pt}{0.9cm}
		\multirow{1}[12]{1cm}{\centering $\vdots$} & \multirow{1}[12]{1cm}{\centering $\ddots$} &
		\multirow{1}[12]{1cm}{\centering $\vdots$} &
		\multirow{1}[12]{1cm}{\centering $\vdots$}
		\tabularnewline
		\hline
		\#       & $p(l_1(p), 1)$ & \dots    & $p(l_1(p), l_2(p))$ & \#       \tabularnewline
		\hline
		\#       & \#             & \dots    & \#                  & \#       \tabularnewline
		\hline
	\end{tabular}
}.
\end{center} 
\begin{definition}
	Let $p \in \Sigma^{m,n}$. $q = p((i, j), (i', j'))$ is called a \emph{subpicture of $p$} if
	$1 \leq i < i' \leq m$ and $1 \leq j < j' \leq n$
	satisfying the following condition:
	\begin{compactitem}
	   \item for each $k, r (1 \leq k \leq i' - i + 1, 1 \leq r \leq j'- j + 1)$, $q(k, r) = p(k
                + i - 1, r + j - 1)$ holds.
	\end{compactitem}
\end{definition}

Sometimes it is necessary to look at all subpictures with a specific size of a picture.
Therefore, we denote the \emph{set of subpictures of size $(h, k)$ of a picture $p\in \Sigma^{*,*}$}
as 
\[B_{h, k}(p) = \{q \in \Sigma^{h,k} \mid q \text{ is a subpicture of } p\}.\] 

We are now able to look at some operations on pictures and languages. The concatenation of pictures
can be performed in two directions: horizontally and vertically.
Two pictures $p, q \in \Sigma^{*,*}$ can be horizontally concatenated, if $l_1(p) = l_1(q)$. We denote
the \emph{horizontal concatenation of $p$ and $q$} by $p \hcat{} q$. Similarly, the two pictures $p$
and $q$ can be concatenated vertically, if $l_2(p) = l_2(q)$. We then write $p \vcat{} q$. Moreover,
the empty picture $\lambda$ can always be concatenated and is the neutral element for both of the
concatenation operations.
\begin{definition}
	Let $L_1, L_2 \in \Sigma^{*,*}$ be two picture languages. The \emph{horizontal concatenation of
	$L_1$ and $L_2$} is defined by 
	\[L_1 \hcat{} L_2 = \{p \hcat{} q \mid p \in L_1, q \in L_2\}\]
\end{definition}
Similarly, the vertical concatenation of picture languages is defined. 

Another operation is the projection by mapping.
\begin{definition}
	Let $\Gamma$ and $\Sigma$ be two finite alphabets and $\pi: \Gamma \rightarrow \Sigma$ be a
	mapping. Let $p \in \Gamma^{m,n}$. \emph{$q \in \Sigma^{m,n}$ is a projection of $p$}, if $q(i, j)
	= \pi(p(i, j))$ for all $1 \leq i \leq m$ and $1 \leq j \leq n$. We write $q = \pi(p)$.
\end{definition}
Since the projection is not required to be injective, a picture can have more than one pre-image. We
define the \emph{projection of a language $L$} as 
\[L' = \pi(L) = \{q \mid q = \pi(p), p \in L\}.\] 
\begin{definition}
	Let $p \in \Sigma^{*, *}$. The \emph{clockwise rotated picture $p^R$ of $p$} is 
	\begin{center}
	
		$p^R = $ \scalebox{0.79}{\begin{tabular}{|C{2.75cm}|C{1cm}|C{1cm}|}
			\hline
			$p(l_1(p), 1)$ & \hspace{1.45cm}\dots\hspace{1.45cm}    & $p(1, 1)$ \tabularnewline
			\hline
			\rule[-2.4cm]{0pt}{0.9cm}
        \multirow{1}[12]{1cm}{\centering $\vdots$} & \multirow{1}[12]{1cm}{\centering $\ddots$} &
        \multirow{1}[12]{1cm}{\centering $\vdots$}  \tabularnewline
			\hline
			$p(l_1(p), l_2(p))$ & \dots    & \hspace{0.3cm}$p(1, l_2(p))$\hspace{0.3cm} \tabularnewline
			\hline
		\end{tabular}
	}.
	\end{center}
\end{definition}
Counterclockwise rotation can be defined similarly. With this operator, it is possible to define the
\emph{rotation of a language $L$} by 
\[L^R = \{p^R \mid p \in L\}.\] 

Based on this idea, it is possible to define horizontal and vertical mirroring and transposing of
pictures and languages.
% \begin{definition}
%     Let $p \in \Sigma^{*, *}$ be a picture. The \emph{vertical mirrored picture $p^{VM}$ of $p$} is 
%     \begin{center}
%         $p^{VM} = $ \begin{tabular}{|C{2.9cm}|C{1cm}|C{1cm}|}
%             \hline
%             \hspace{0.37cm}$p(l_1(p), 1)$\hspace{0.37cm} & \hspace{0.9cm}\dots\hspace{0.9cm}    &
%             $p(l_1(p), l_2(p))$ \tabularnewline
%             \hline
%             $\vdots$  & $\ddots$ & $\vdots$  \tabularnewline
%             \hline
%             $p(1, 1)$ & \dots    & $p(1, l_2(p))$ \tabularnewline
%             \hline
%         \end{tabular}
%     \end{center}
% \end{definition}
% Horizontal mirroring can be defined similarly. The \emph{vertical mirroring of a language $L$} is
% defined by \[L^{VM} = \{p^{VM} \mid p \in L\}.\]
% 
% Also possible is the transposition of pictures and languages and is to understand like the normal
% transposition over matrices.
% \begin{definition}
%     Let $p \in \Sigma^{*, *}$ be a picture. The \emph{transposed picture $p^{T}$ of $p$} is 
%     \begin{center}
%         $p^{T} = $ \begin{tabular}{|C{2.9cm}|C{1cm}|C{1cm}|}
%             \hline
%             $p(1, 1)$ & \hspace{0.9cm}\dots\hspace{0.9cm}    & $p(l_1(p), 1)$ \tabularnewline
%             \hline
%             $\vdots$  & $\ddots$ & $\vdots$  \tabularnewline
%             \hline
%             \hspace{0.37cm}$p(1, l_2(p))$\hspace{0.37cm} & \dots    & $p(l_1(p), l_2(p))$
%             \tabularnewline
%             \hline
%         \end{tabular}
%     \end{center}
% \end{definition}
% The \emph{transposed picture language of a pictue language $L$} is defined by
% \[L^{T} = \{p^{T} \mid p \in L\}.\]

Further operations on picture languages are union, intersection and complementation and will be
understood as ordinary set operations.