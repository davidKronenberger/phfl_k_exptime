At first, we will restrict the size of tiles to the sizes $(2, 1)$ and $(1, 2)$. These tiles are
called \emph{dominos}. We are now able to describe the languages recognized by dominos and the
projection of those languages.
\begin{definition}
	Let $\Delta$ be a finite set of dominos over $\Sigma \cup \{\#\}$. A language $L \subseteq
	\Sigma^{*, *}$ is \emph{hv-local} if $L$ is generated by $\Delta$ as follows: 
	\[L = \{p \in \Sigma^{*, *} \mid B_{2, 1}(\hat{p}) \cup B_{1, 2}(\hat{p}) \subseteq \Delta\}.\]
	We write $L = hv$-$LOC(\Delta)$. 
\end{definition}

% \begin{theorem}
% \label{theorem:local_hv_local}
% 	The family of hv-local languages is properly included in the family of local languages. 
% \end{theorem}
% 
% \begin{proof}
% 	It is easy to see, that the hv-local languages are included in the local languages, because we can
% 	create a tile for any combination of dominos. The language from
% 	Example~\ref{example:local_language} is a local language. But this language cannot be generated by
% 	dominos. It is impossible to ensure that there are only $1$'s on the diagonal.
% 	
% 	The detailed proof can be found in~\cite{giammarresi1997twodimensional}. 
% \end{proof}
Like tiling systems, we can now similarly define domino systems:
\newpage
\begin{definition}
	A quadruple $\tau = (\Sigma, \Gamma, \Delta, \pi)$ is called a \emph{domino system} (\emph{DS}),
	where
	\begin{compactitem}
		\item $\Sigma$ and $\Gamma$ are two finite alphabets,
		\item $\pi: \Gamma \rightarrow \Sigma$ is a mapping and
		\item $\Delta$ is a finite set of dominos over $\Gamma\cup\{\#\}$.
	\end{compactitem}
	The language generated by $\tau$ is $L(\tau) = \pi(hv$-$LOC(\Delta))$. 
\end{definition}

The family of languages generated by domino systems is denoted by $\familyOf{DS}$. 

It can be shown that $\familyOf{DS} = \familyOf{TS}$ (see~\cite{giammarresi1997twodimensional}).
Furthermore, it can be shown that the non-deterministic version of on-line tessellation automata
(see Section~\ref{ota}) is equal to $\familyOf{TS}$. All these approaches lead to the same class of
picture languages, which is called REC.
\label{domino_systems}