Tiling systems were introduced by D. Giammaresi and A. Restivo
in~\cite{giammarresi1992recognizable}. They searched for a possibility to define recognizable
picture languages by extending the projection of local languages from the one-dimensional case to
two dimensions.

Let $p \in \Sigma^{*, *}$ be a picture. We call $B_{2, 2}(\hat{p})$ the \emph{set of tiles of the
picture $p$} where a \emph{tile} is a picture of size $(2, 2)$. Regard that $B_{2, 2}(\hat{p})$
contains any subpicture of the bordered picture $\hat{p}$ of size $(2, 2)$. We can now restate the
definition of two-dimensional local languages from~\cite{cherubini2009picture}:
\begin{definition}
	Let $\Theta$ be a finite set of tiles over $\Sigma \cup \{\#\}$. A language $L \subseteq \Sigma^{*,
	*}$ is \emph{local}, if $L$ is generated by $\Theta$ as follows: 
	\[L = \{p \in \Sigma^{*, *} \mid B_{2, 2}(\hat{p}) \subseteq \Theta\}.\]
	We write $L = LOC(\Theta)$.
\end{definition}
To underline the power of local languages, we review a suitable example
from~\cite{giammarresi1997twodimensional}.
\begin{example}
\label{example:local_language}
	Let $\Theta$ be the following finite set of tiles over $\Sigma \cup \lbrace\#\rbrace$, where
	$\Sigma = \lbrace 0, 1\rbrace$.
	\newline
	$\Theta = \left\lbrace
	\begin{tabular}{|C{0.7cm}|C{0.7cm}|}
	\hline
	$\#$ & $\#$ \tabularnewline
	\hline
	$\#$ & $1$  \tabularnewline
	\hline
	\end{tabular}\right.$,
	\begin{tabular}{|C{0.7cm}|C{0.7cm}|}
	\hline
	$1$  & $\#$ \tabularnewline
	\hline
	$\#$ & $\#$ \tabularnewline
	\hline
	\end{tabular}, 
	\begin{tabular}{|C{0.7cm}|C{0.7cm}|}
	\hline
	$\#$ & $\#$ \tabularnewline
	\hline
	$0$  & $\#$ \tabularnewline
	\hline
	\end{tabular},
	\begin{tabular}{|C{0.7cm}|C{0.7cm}|}
	\hline
	$\#$ & $0$  \tabularnewline
	\hline
	$\#$ & $\#$ \tabularnewline
	\hline
	\end{tabular}, 
	\begin{tabular}{|C{0.7cm}|C{0.7cm}|}
	\hline
	$\#$ & \hspace{0.06cm}$1$\hspace{0.06cm} \tabularnewline
	\hline
	$\#$ & $0$ \tabularnewline
	\hline
	\end{tabular}, 
	\begin{tabular}{|C{0.7cm}|C{0.7cm}|}
	\hline
	$\#$ & \hspace{0.06cm}$0$\hspace{0.06cm} \tabularnewline
	\hline
	$\#$ & $0$ \tabularnewline
	\hline
	\end{tabular}, 
	\begin{tabular}{|C{0.7cm}|C{0.7cm}|}
	\hline
	\hspace{0.06cm}$0$\hspace{0.06cm} & $\#$ \tabularnewline
	\hline
	$0$ & $\#$ \tabularnewline
	\hline
	\end{tabular}, 
	\begin{tabular}{|C{0.7cm}|C{0.7cm}|}
	\hline
	\hspace{0.06cm}$0$\hspace{0.06cm} & $\#$ \tabularnewline
	\hline
	$1$ & $\#$ \tabularnewline
	\hline
	\end{tabular},
	\begin{tabular}{|C{0.7cm}|C{0.7cm}|}
	\hline
	$0$  & $0$  \tabularnewline
	\hline
	$\#$ & $\#$ \tabularnewline
	\hline
	\end{tabular}, 
	\begin{tabular}{|C{0.7cm}|C{0.7cm}|}
	\hline
	$0$  & $1$  \tabularnewline
	\hline
	$\#$ & $\#$ \tabularnewline
	\hline
	\end{tabular}, 
	\begin{tabular}{|C{0.7cm}|C{0.7cm}|}
	\hline
	$\#$ & $\#$ \tabularnewline
	\hline
	$1$  & $0$  \tabularnewline
	\hline
	\end{tabular}, 
	\begin{tabular}{|C{0.7cm}|C{0.7cm}|}
	\hline
	$\#$ & $\#$ \tabularnewline
	\hline
	$0$  & $0$  \tabularnewline
	\hline
	\end{tabular}, 
	\begin{tabular}{|C{0.7cm}|C{0.7cm}|}
	\hline
	\hspace{0.06cm}$1$\hspace{0.06cm} & \hspace{0.06cm}$0$\hspace{0.06cm} \tabularnewline
	\hline
	$0$ & $1$ \tabularnewline
	\hline
	\end{tabular}, 
	\begin{tabular}{|C{0.7cm}|C{0.7cm}|}
	\hline
	\hspace{0.06cm}$0$\hspace{0.06cm} & \hspace{0.06cm}$1$\hspace{0.06cm} \tabularnewline
	\hline
	$0$ & $0$ \tabularnewline
	\hline
	\end{tabular}, 
	\begin{tabular}{|C{0.7cm}|C{0.7cm}|}
	\hline
	\hspace{0.06cm}$0$\hspace{0.06cm} & \hspace{0.06cm}$0$\hspace{0.06cm} \tabularnewline
	\hline
	$1$ & $0$ \tabularnewline
	\hline
	\end{tabular}, 
	$\left.\begin{tabular}{|C{0.7cm}|C{0.7cm}|}
	\hline
	\hspace{0.06cm}$0$\hspace{0.06cm} & \hspace{0.06cm}$0$\hspace{0.06cm} \tabularnewline
	\hline
	$0$ & $0$ \tabularnewline
	\hline
	\end{tabular}
	\right\rbrace$ 
\end{example}
The first four tiles are the only tiles that can be used in the four corners. Thus, top-left and
bottom-right corners always contain $1$s, whereas the other corners always contain $0$s. The
following eight tiles are for vertical and horizontal borders. It can be seen that these tiles do
not allow any other $1$s except those in the appropriate corners. The last four tiles are forcing
the picture to have $1$s on the diagonal and $0$s otherwise.
Therefore \[LOC(\Theta) = \lbrace p \in \Sigma^{n,n} \mid n \geq 1, p(i, i) = 1, p(i, j) = 0\text{
where } 1 \leq i, j \leq l_1(p) \text{ and } i \neq j\rbrace\] is the language of square-shaped
pictures with $1$s on the main diagonal and $0$s everywhere else.

As this example underlines, the generative power of local languages is restricted. It is e.g.
impossible to generate square-shaped pictures with a one-letter alphabet.

Due to the fact that $LOC$ is the natural extension of string local
languages~\cite{cherubini2009picture}, we can extend the definition of one-dimensional regular
languages to the two-dimensional case:
\begin{definition}
	The quadruple $\tau = (\Sigma, \Gamma, \Theta, \pi)$ is called a \emph{tiling system} (\emph{TS}),
	where
	\begin{compactitem}
		\item $\Sigma$ and $\Gamma$ are two finite alphabets, 
		\item $\pi: \Gamma \rightarrow \Sigma$ is a mapping and
		\item $\Theta$ is a finite set of tiles over $\Gamma \cup \{\#\}$.
	\end{compactitem}
	The language recognized by $\tau$ is $L(\tau) = \pi(LOC(\Theta))$.
\end{definition}
The \emph{family of languages generated by tiling systems} is called $\familyOf{TS}$. This
definition solves the restriction of shaping pictures with one letter.
\begin{example}
	Let $\tau = (\lbrace a \rbrace, \lbrace 0, 1 \rbrace, \Theta, \pi)$ be a TS with
	$\Theta$ from Example~\ref{example:local_language} and $\pi(0) = \pi(1) = a$.
\end{example}
It is obvious that this TS recognizes the language $L(\tau) = \{p \in \lbrace a
\rbrace^{n, n} \mid n \geq 0\}$, which contains square-shaped pictures of $a$'s.
\label{tiling_systems}