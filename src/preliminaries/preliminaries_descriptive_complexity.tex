%%
%% Author: Davidov
%% 27.04.2018
%%

\subsection{Descriptive Complexity}\label{subsec:descriptiveComplexity}

One way to describe complexity classes is with the help of \textit{turing machines}~\cite{hopcroft1994einfuehrung}.

\begin{definition}
    The seven-tuple $M = (Q, \Sigma, \Gamma, \delta, q_0, \Box, F)$ is called a \emph{turing machine} (\emph{TM}),
    where
    \begin{compactitem}
        \item $Q$ is the finite set of states,
        \item $\Sigma$ is the input alphabet,
        \item $\Gamma$ is the working alphabet with $\Sigma \subset \Gamma$,
        \item $\delta \subseteq (Q \setminus F) \times \Gamma \times Q \times \Gamma \times \{L, R, N\}$ is the
        transition relation,
        \item $q_0 \in Q$ is the initial state,
        \item $\Box \in \Gamma \setminus \Sigma$ is the blank symbol and
        \item $F \in Q$ is the set of final states.
    \end{compactitem}
\end{definition}

Known from computational complexity theory~\cite{papadimitriou1994complexity}, the time and space classes
can be defined by functions. These functions have as input a natural number that represents the length of the input of a
turing machine. In case of time classes the output of the function is the maximal number of configuration steps for
all inputs of the given length. In case of space classes the output is based on the biggest configuration while
computing an input of the given length.

\begin{definition}
    Let $f: \mathbb{N} \rightarrow \mathbb{N}$ be a polynomial function, then $\emph{exp}: \mathbb{N} \times \mathbb{N}
    \rightarrow \mathbb{N}$ is a function defined inductive as follows:
    \begin{compactitem}
        \item $exp(0, f(n)) = f(n)$,
        \item $exp(i, f(n)) = 2^{exp(i - 1, f(n))}$ for $i \geq 1$.
    \end{compactitem}
\end{definition}

With the help of the function $exp$, we are now able to define the complexity classes $k$-EXPTIME and $k$-EXPSPACE for
all $k \geq 1$.

\begin{definition}
    Let $k \in \mathbb{N} \setminus \{0\}$ and $f: \mathbb{N} \rightarrow \mathbb{N}$ be a polynomial function then
    $k$-EXPTIME $=$ TIME($exp(k, f(n))$) and $k$-EXPSPACE $=$ SPACE($exp(k, f(n))$).
\end{definition}

Remark that TIME is the maximal number of configuration steps and SPACE the biggest configuration of a turing machine
while computing an input.

From definition~\ref{definition:bisimulationInvariant}, definition~\ref{definition:queryBelongsToComplexityClass}
and the complexity classes $k$-EXPTIME and $k$-EXPSPACE follow the queries that we want to investigate.

\begin{definition}
    \label{definition:kExptimekExpspace}
    \exptime{$k$} are the bisimulation invariant queries that belong to complexity class $k$-EXPTIME and
    \expspace{$k$} are the bisimulation invariant queries that belongs to complexity class $k$-EXSPACE, where $k \geq
    1$.
\end{definition}

The main aim of \emph{descriptive complexity} is to describe the complexity classes known from the
computational complexity theory with logics. While the computational complexity theory distinguishes time and space
classes, the descriptive complexity theory characterizes classes without a reference to automaton models or
space and time bounds.

The first known result in the area of descriptive complexity comes from Fagin. In 1974 he showed that the complexity
class NP coincides with $\exists SO$~\cite{fagin1974generalized}, the existential fragment of second-order logic.

In the next chapter we introduce the polyadic higher order fixed point logic and a restriction called tail-recursive
devised by Lange and Lozes in~\cite{lange2014capturing}. We want to compare the in
definition~\ref{definition:kExptimekExpspace} presented queries with those logics to make a further contribution in
the theory of descriptive complexity.