
%%
%% Author: Davidov
%% 27.04.2018
%%

\section{Descriptive Complexity}\label{sec:descriptiveComplexity}

The main aim of \emph{descriptive complexity} is to describe the complexity classes known from
computational complexity theory with logics. While computational complexity theory distinguishes time and space
classes, descriptive complexity theory characterizes classes with logical resources instead of a reference to
automaton models or space and time bounds.

The first known result in the area of descriptive complexity comes from Fagin. In 1974 he showed that the complexity
class NP coincides with $\exists SO$~\cite{fagin1974generalized}, the existential fragment of second-order logic.

To describe complexity classes with logics we have to explain what complexity classes are from the viewpoint of
computational complexity theory. One way to describe complexity classes is with the help of \textit{Turing
Machines}~\cite{hopcroft1994einfuehrung}.
A Turing Machine is a theoretical model moving over a tape with symbols. The reading head of
those machines is positioned always over one cell of this tape and scans the symbol of the current cell. After scanning
the symbol, the machine is able to override the symbol and move the reading head one cell left, right or stand still.
Formally, a Turing Machine is the following.

\begin{definition}
\label{definition:dtm}
    A seven-tuple $M = (Q, \Sigma, \Gamma, \delta, q_0, \Box, F, R)$ is called a \emph{Deterministic Turing Machine}
    ($\mathit{DTM}$), if
    \begin{compactitem}
        \item $Q$ is the finite set of states,
        \item $\Sigma$ is the input alphabet,
        \item $\Gamma$ is the working alphabet with $\Sigma \subset \Gamma$,
        \item $\delta : (Q \setminus (F \cup R)) \times \Gamma \rightarrow Q \times \Gamma \times \{L, R, N\}$ is the
        transition function,
        \item $q_0 \in Q$ is the initial state,
        \item $\Box \in \Gamma \setminus \Sigma$ is the blank symbol,
        \item $F \subseteq Q$ is the set of accepting states and
        \item $R \subseteq Q, F \cap R = \emptyset$ is the set of rejecting states.
    \end{compactitem}
\end{definition}

\begin{example}
    \label{example:dtm}
    As an example for a $\mathit{DTM}$ let
    \[M = (\{q_0, q_f, q_r\}, \{a, b\}, \{a, b, \Box\}, \delta, q_0, \Box, \{q_f\}, \{q_r\})\]
    where $\delta(q_0, a)= (q_0, \Box, R)$, $\delta(q_0, b) = (q_r, b, L)$, $\delta(q_0, \Box) = (q_f, \Box, N)$.
    $M$ is a $\mathit{DTM}$ that accepts all input words that contain no symbol $b$, i.e. $L(M) = \{a\}^*$.
\end{example}

Configurations are snapshots of $\mathit{DTM}$s working on an input word. This includes the working tape, the current
state and the current position of the reading head. Formally, $C_i^M = (q_i, h_i, t_i)$ is called the $i$-th
configuration of a $\mathit{DTM}$ $M = (Q, \Sigma, \Gamma, \delta, q_0, \Box, F)$ of the computation on some input
word, where $q_i \in Q$ is the current state, $h_i \in \mathbb{N}$ is the reading head position and $t: \mathbb{N}
\rightarrow \Gamma$ represents the full tape content. In addition, $t(i)$ represents the content of tape cell $i$.

\begin{definition}
    Let $C_i^M = (q_i, h_i, t_i)$, $C_j^M = (q_j, h_j, t_j)$ be two configurations of a
    $\mathit{DTM}$ $M = (Q, \Sigma, \Gamma, \delta, q_0, \Box, F, R)$ with $i \neq j$. $C_j^M$ is the next configuration
    of $C_i^M$ written as $C_i^M \rightarrow_M C_j^M$ iff
    \begin{compactitem}
        \item $j = i + 1$,

        \item $h_j = h_i + d$,

        \item $t_j(h_i) = a$, $t_j(k) = t_i(k)$, $k \neq h_i$,

        \item $\delta(q_i, t_i(h_i)) = (q_j, a, D)$
    \end{compactitem}
    where $D \in \{L, R, N\}$, $d = f(D)$ and $f(L) = -1$, $f(R) = 1$ and $f(N) = 0$. $C_i^M \rightarrow_M C_j^M$ is
    called a \emph{transition} of $M$.
\end{definition}

The start configuration for an input word $w = a_1\dots a_n$ of a $\mathit{DTM}$ $M$ is $C_0^M = (q_0, 0, h_{0})$,
where $q_0$ is the initial state of $M$ and $h_0(k) = a_k$ for $1 \leq k \leq n$ and $h_0(l) = \Box$ for $l \neq k$. A
run of $\mathit{DTM}$ $M$ on input word $w$ is a sequence of transitions, $C_0^M \rightarrow_M C_1^M \rightarrow_M
C_2^M \rightarrow_M \dots$. A run of $\mathit{DTM}$ $M$ on input word $w$ is terminating, if there is a
configuration $C_n^M$ for some $n \in \mathbb{N}$ such that the state of $C_n^M$ is either an accepting or rejecting
state of $M$. A run is accepting if the state of $C_n^M$ is an accepting state of $M$.

\begin{example}
    \label{example:run_of_dtm}
    Let $M$ be the $\mathit{DTM}$ from Example~\ref{example:dtm} and $w_1 = aaba$, $w_2 = aaaa$ two input words. For
    better readability we illustrate tape content functions as infinite words. The run of $M$ on $w_1$ is
    \begin{align*}
        C_0^M = &\,(q_0, 0, aaba\Box \dotsb) \rightarrow_M (q_0, 1, \Box aba\Box \dotsb) \rightarrow_M  (q_0, 2, \Box \Box
        ba\Box \dotsb)\\&\, \rightarrow_M (q_r, 1, \Box \Box ba\Box \dotsb) = C_3^M
    \end{align*}
    and the run of $M$ on $w_2$ is
    \begin{align*}
        C_0^M =\,&(q_0, 0, aaaa\Box \dotsb) \rightarrow_M (q_0, 1, \Box aaa\Box \dotsb) \rightarrow_M (q_0, 2, \Box \Box
        aa\Box \dotsb) \\&\,\rightarrow_M
        (q_0, 3, \Box \Box \Box a\Box \dotsb) \rightarrow_M (q_0, 4, \Box \Box \Box \Box \Box \dotsb) \\&\,\rightarrow_M
        (q_f, 4, \Box \Box \Box \Box \Box \dotsb) = C_5^M
    \end{align*}
    Note, that both runs are terminating. The run on input word $w_2$ is an accepting run whereas the run on input word $w_1$ is a rejecting one.
\end{example}

Note, that it is possible to define $\mathit{DTM}$s that do not accept or reject any input words. For example, let
$M = (\{q_0\}, \{a\}, \{a, \Box\}, \delta, \emptyset, \emptyset)$, where $\delta(q_0, x) = (q_0, x, N)$ with $x \in
\{a, \Box\}$. $M$ is a DTM where any calculation of an input word $w$ looks as follows $q_0w \rightarrow_M q_0w
\rightarrow_M \dots$. It never reaches an accepting or rejecting state. In this thesis, we are only interested in
$\mathit{DTM}$s that reach an accepting or rejecting state in finite time on any input word. Those $\mathit{DTM}$s
we are calling terminating $\mathit{DTM}$s.

Known from computational complexity theory~\cite{papadimitriou1994complexity}, the time and space classes
can be defined by functions. These functions take as input a natural number that represents the length of an input
word of a terminating $\mathit{DTM}$. In case of time classes the output of the functions depends on the number of
configuration steps. In case of space classes the output is based on the longest transition.

\begin{definition}
    Let $M$ be a terminating $\mathit{DTM}$. $\mathit{TIME}(n):= max(\mathit{STEPS}$ $(w)\mid |w| = n)$, where
    $\mathit{STEPS}(w)$ is the number of transitions of $M$ running on input word $w$. $\mathit{SPACE}(n) := max
    (\mathit{STORAGE}(w)\mid |w| = n)$, where $\mathit{STORAGE}(w) := max(h_i\mid i\in\{1, \dots m\} \text{ and }
    C_i^M = (q_i, h_i, t_i))$ is the most right head position of $M$ while $M$ runs on input word $w$ and $C_0^M
    \rightarrow_M
    C_1^M \rightarrow_M \dotsb \rightarrow_M C_m^M$ is this run of $M$ on $w$.
\end{definition}

\begin{example}
    \label{example:time_and_space}
    Let $w_1$ and $w_2$ be the two words from Example~\ref{example:run_of_dtm} and $M$ the terminating DTM of Example~\ref{example:dtm}. It is easy to see, that $\mathit{STEPS}
    (w_1) = 3$, $\mathit{STEPS}(w_2) = 5$, $\mathit{STORAGE}(w_1) = 2$ and $\mathit{STORAGE}(w_1) = 4$ for $M$ running on $w_1$ and $w_2$. Because the maximal number of steps for an input word of length $4$ occurs if the input word consists only of symbol
    $a$, it holds that $\mathit{TIME}(4) = 5$ or more general that $\mathit{TIME}(n) = n + 1$. $\mathit{SPACE}(n) = n$
    holds, because the length of the input word will only shrink and we do not move to the right side of the word.
\end{example}

It is possible to group the $\mathit{DTM}$s by functions. These groups are the computational complexity classes.
In this thesis we are interested in exponential time classes and exponential space classes.

\begin{definition}
    Let $f: \mathbb{N} \rightarrow \mathbb{N}$ be a polynomial function, then $\emph{exp}: \mathbb{N} \times \mathbb{N}
    \rightarrow \mathbb{N}$ is a function defined inductive as follows:
    \begin{compactitem}
        \item $exp(0, f(n)) = f(n)$,
        \item $exp(i, f(n)) = 2^{exp(i - 1, f(n))}$ for $i \geq 1$.
    \end{compactitem}
\end{definition}

With the help of the function $exp$, we are able to define the complexity classes $k$-EXPTIME and $k$-EXPSPACE for
all $k \geq 1$.

\begin{definition}
    \label{definition:k_exptime_and_k_expspace}
    $k$-EXPTIME is the set of all those problems $P$ where a $\mathit{DTM}$ $M$ and a polynomial function $f: \mathbb{N}
    \rightarrow \mathbb{N}$ exist such that $M$ can decide $P$ in $\mathit{TIME}$($exp(k, f(n))$).
    $k$-EXPSPACE is the set of all those problems $Q$ where a $\mathit{DTM}$ $M$ and a polynomial function $f: \mathbb{N}
    \rightarrow \mathbb{N}$ exist such that $M$ can decide $Q$ in $\mathit{SPACE}$($exp(k, f(n))$).
\end{definition}

Keep in mind, that $\mathit{TIME}$ is the maximal number of transitions and $\mathit{SPACE}$ the tallest configuration of a
$\mathit{DTM}$ when running on an input word.

An example for a problem that is in $1$-EXPSPACE is the problem to recognize
if two regular expressions represent different languages~\cite{meyer1972equivalence}. To check if a $\mathit{DTM}$
holds in at most $k$ steps is a problem that lies in $1$-EXPTIME. In~\cite{bruse2017space} it was proven that the
model checking problem for $\mathit{HFL}^{k + 1}_{tail}$ is in $k$-EXPSPACE. $\mathit{HFL}^{k + 1}_{tail}$ is the
class of all tail-recursive $1$-adic PHFL formulas with order at most $k + 1$. See
Section~\ref{sec:polyadichigherorderfixpointlogic} for further information about PHFL.

As mentioned in Section~\ref{sec:bisimulationInvariance} the queries defined in Definition~\ref{definition:query} can be categorized. The first category is defined in Definition~\ref{definition:bisimulationInvariant}. Here we define a second category that describes which complexity class a query belongs to. Because the queries are defined over LTS and the DTMs in this definition have to work on LTS, a standardized encoding of these LTS has to be used as input word. This standardized encoding transforms a given LTS with a tupel of states to a readable string.

\begin{definition}
    \label{definition:queryBelongsToComplexityClass}
    Let $\mathcal{T}$ be a LTS with $\mathcal{T} = (Q, \Sigma, P, \Delta, v)$ and $(q_1, \dots, q_{r}) \in Q^r$.
    A query $\mathcal{Q}$ belongs to complexity class $\mathcal{C}$ if there is a DTM (see
    Definition~\ref{definition:dtm}) in $\mathcal{C}$ for deciding on the standardized encoding of $(\mathcal{T}, (q_1, \dots,
    q_{r}))$ as input whether $(q_1, \dots, q_{r}) \in \mathcal{Q}^\mathcal{T}$.
\end{definition}

\begin{example}{\cite{lange2014capturing}}
    Let $\mathcal{Q}$ be the query and $\mathcal{T}$ the LTS of Example~
    \ref{example:query_bisimulation}. Because bisimilarity can be decided in P\footnote{To 
    decide if two states in an LTS are bisimilar can easy checked by a special kind of breadth-
    first search. On each iteration step we check if the states of a pair in the bisimulation have 
    same properties and if looking at the left state all possible actions leading to any state $q
    $ can also be made from the right state with same action leading to a state $p$. The pair $
    (q, p)$ will be add to the bisimulation ignoring duplicates. The same will be checked looking 
    at the right state first. Starting with the two states that have to be checked on bisimilarity this 
    iteration proceeds while a pair is added to the bisimulation and all conditions are fulfilled. Note that the state set of an LTS is finite. This means this process terminates after finite many steps.}, 
    there is an 
    algorithm in P for deciding on input $(\mathcal{T}, (q_1, q_2))$ whether $(q_1, q_2) \in 
    \mathcal{Q}^\mathcal{T}$, that means $\mathcal{Q}$ belongs to 
    complexity class P.
\end{example}

From Definition~\ref{definition:bisimulationInvariant}, Definition~\ref{definition:queryBelongsToComplexityClass}
and the complexity classes $k$-EXP-TIME and $k$-EXPSPACE of Defintion~\ref{definition:k_exptime_and_k_expspace}
the queries we want to investigate can be desired.

\begin{definition}
    \label{definition:kExptimekExpspace}
    \exptime{$k$} are the bisimulation invariant queries that belong to complexity class $k$-EXPTIME and
    \expspace{$k$} are the bisimulation invariant queries that belong to complexity class $k$-EXSPACE, where $k \geq 0$.
\end{definition}

In the introduction of this section, we mentioned that the main aim of descriptive complexity is to describe complexity classes with logics. The next definition defines how a complexity class is captured by a logic. 

\begin{definition}
A complexity class $\mathcal{C}$ is captured by a logic $L$ if for all queries $\mathcal{Q}$ that belongs to $\mathcal{C}$ there is a formula $\Phi$ of $L$ such that for all LTS $\mathcal{T} = (Q, \Sigma, P, \Delta, v)$ there is a variable mapping $\eta$ and it holds $(q_1, \dots, q_{r}) \in \mathcal{Q}^\mathcal{T}$ iff $(q_1, \dots, q_{r}) \in \llbracket \Phi \rrbracket^\eta_\mathcal{T}$, where $q_1, \dots, q_r \in Q$ and $\llbracket \Phi \rrbracket^\eta_\mathcal{T}$ is the semantics of $\Phi$ under $\mathcal{T}$ and $\eta$.
\end{definition}

