%%
%% Author: Davidov
%% 16.05.2018
%%

\section{Bisimulation Invariance}\label{sec:bisimulationInvariance}

First of all, we need the definition of \textit{labeled transition systems}. A labeled transition system is a graph
with labeled vertices and edges. Formally, it is the following.

\begin{definition}
    \label{definition:lts}
    A quintuple $\mathcal{T} = (Q, \Sigma, P, \Delta, v)$ is called a \emph{labeled transition system} (\emph{LTS}),
    where
    \begin{compactitem}
        \item $Q$ is a set of states,
        \item $\Sigma$ is a finite set of actions,
        \item $P$ is a finite set of propositions,
        \item $\Delta \subseteq Q \times \Sigma \times Q$ is the labeled transition relation and
        \item $v: Q \rightarrow 2^P$ is a function that maps each state to a set of propositions.
    \end{compactitem}
\end{definition}

For all $q_1, q_2 \in Q$ and all $a \in \Sigma$ we write $q_1 \overset{a}{\rightarrow} q_2$ for $(q_1, a, q_2) \in
\Delta$.

\begin{example}
    \label{example:lts}
    As mentioned above, LTS can be seen as a graph with labeled vertices and edges. For example, a LTS is
    $\mathcal{T} = (\{q_1, q_2, q_3, q_4, q_5\}, \{a, b\}, \{p_1,$ $ p_2\}, \Delta, v)$, where $\Delta = \{(q_1, a,
q_2)
    , (q_2, a, q_3), (q_2, b, q_3), (q_2, b, q_5), (q_5, a, q_4), (q_4, $ $a, q_3), (q_4, b, q_3), (q_4, b, q_1)\}$,
    $v(q_1) = v(q_2) = v(q_4) = v(q_5) = \{p_1\}$ and $v(q_3) = \emptyset$. $\mathcal{T}$ can be visualized as
    follows.
\begin{center}
    \begin{tikzpicture}[]
                  \node [place] (q1) [label=above:$p_1$] {$q_1$};
                  \node  (temp) [left=of q1] {};
                  \node  (label) [left=of temp] {$\mathcal{T}:$};
                  \node [place] (q3) [below=of q1] {$q_3$};
                  \node [place] (q5) [below=of q3,label=above:$p_1$] {$q_5$};
                  \node [place] (q2) [left=of q3,label=left:$p_2$] {$q_2$}
                  edge [post,bend left] node[above] {a} (q3)
                  edge [post,bend right] node[above] {b} (q3)
                  edge [post,bend right] node[left, below] {b} (q5)
                  edge [pre,bend left] node[left, above] {a} (q1);
                  \node [place] (q4) [right=of q3,label=right:$p_2$] {$q_4$}
                  edge [post,bend left] node[auto, swap] {b} (q3)
                  edge [post,bend right] node[auto, swap] {a} (q3)
                  edge [pre,bend left] node[right, below] {a} (q5)
                  edge [post,bend right] node[right, above] {b} (q1);
\end{tikzpicture}
\end{center}
\end{example}

On this systems or rather on the states of the systems it is possible to define relations. The
following relation describes those states that have the same behaviour.

\begin{definition}
    Let be $\mathcal{T}_1 = (Q_1, \Sigma, P, \Delta_1, v_1)$ and $\mathcal{T}_2 = (Q_2, \Sigma, P, \Delta_2, v_2)
    $ two LTS. A \emph{bisimulation} is a binary relation $R \subseteq Q_1 \times Q_2$ that fulfills for all $(q_1,
    q_2) \in R$
    \begin{compactitem}
        \item $v_1 (q_1) = v_2 (q_2)$,
        \item for all $a \in \Sigma$ and all $q_1' \in Q_1$, if $q_1 \overset{a}{\rightarrow} q_1'$, then there
        is a state $q_2' \in Q_2$, such that $q_2 \overset{a}{\rightarrow} q_2'$ and $(q_1', q_2') \in R$ and
        \item for all $a \in \Sigma$ and all $q_2' \in Q_2$, if $q_2 \overset{a}{\rightarrow} q_2'$, then there is a
        state $q_1' \in Q_1$, such that $q_1 \overset{a}{\rightarrow} q_1'$ and $(q_1', q_2') \in R$.
    \end{compactitem}
    We call two states $q_1 \in Q_1$, $q_2 \in Q_2$ \emph{bisimilar}, noted as $(\mathcal{T}_1, q_1) \sim
    (\mathcal{T}_2, q_2)$, if there
    is a bisimulation $R$ such that $(q_1, q_2) \in R$.
\end{definition}

\begin{example}
    \label{example:bisimilar}
    Let $\mathcal{T}$ be like in Example~\ref{example:lts}. In this example, we compare states of the same LTS. It is
    true, that $(\mathcal{T}, q_1) \sim (\mathcal{T}, q_5)$ and $(\mathcal{T}, q_2) \sim (\mathcal{T}, q_4)$ because
    $R = \{(q_1, q_5), (q_2, q_4), (q_3, q_3)\}$ is a bisimulation and includes $(q_1, q_5)$ and $(q_2, q_4)$. It is
    easy to prove, that $\sim$ is an equivalence relation. So it holds, that $(\mathcal{T}, q_5) \sim (\mathcal{T},
    q_1)$ and $(\mathcal{T}, q_4) \sim (\mathcal{T}, q_2)$. Moreover, all reflexive pairs are bisimilar. Note, that
    $(\mathcal{T}, q_1) \not\sim (\mathcal{T}, q_3)$ because $v(q_1) \neq v(q_3)$ and $(\mathcal{T}, q_1)
    \not\sim (\mathcal{T}, q_2)$ because $q_2$ can make $b$-transitions but $q_1$ has no $b$-transitions.
\end{example}

Because $\sim$ is an equivalence relation it can be used to build a factor structure over an LTS.

\begin{definition}
    \label{definition:reduced_lts}
    Let $\mathcal{T} = (Q, \Sigma, P, \Delta, v)$ be an LTS and $(q_1, \dots, q_n) \in Q^n$ a tuple then we call $
    (\mathcal{T}, q_1, \dots, q_n)$ \emph{reduced} iff all states of $\mathcal{T}$ are reachable by at
    least one $q_i$, where $i \in \{1, \dots, n\}$, and if $\sim$ coincides with equality. For all LTS $\mathcal{T} =
    (Q, \Sigma, P, \Delta, v)$ and all tuples of states $(q_1, \dots, q_n) \in Q^n$ we associate to $(\mathcal{T}, q_1,
    \dots, q_n)$ the reduced tuple $RED(\mathcal{T}, q_1, \dots, q_n)$ by factoring $\mathcal{T}$ with respect to
    $\sim$ and pruning all states of $\mathcal{T}_\sim$ that cannot be reached by at least one $q_i$. We call the resulting
    graph the \emph{reduced LTS} of $\mathcal{T}$ with respect to $q_1, \dots, q_n$.
\end{definition}

Furthermore, we can describe properties of LTS. \textit{Queries} are one way to describe these properties.

\begin{definition}{\cite{otto1999bisimulation}}
    \label{definition:query}
    A \emph{$r$-adic query} $\mathcal{Q}$ is a mapping that associates with each LTS $\mathcal{T} = (Q, \Sigma, P,
    \Delta, v)$ a subset $\mathcal{Q}^{\mathcal{T}}$ of $Q^r$ such that any isomorphism $f:
    \mathcal{T} \simeq \mathcal{T}'$ is also an isomorphism of the expansions $(\mathcal{T}, \mathcal{Q}^{\mathcal{T}})$
    and $({\mathcal{T}}', \mathcal{Q}^{{\mathcal{T}}'})$.
\end{definition}

The queries defined in Definition~\ref{definition:query} can be categorized. Here we are interested in two of these
categories. The first category is called \textit{bisimulation invariant}. This category describes those queries that
can not distinguish bisimilar states. In~\cite{otto1999bisimulation} this property is defined over so called
\textit{Kripke structures}. A Kripke structure is a transition system. Note, that transition systems have only one
type of actions.

\begin{definition}
    \label{definition:bisimulationInvariant}
    Let $\mathcal{T}$, $\mathcal{T}'$ be two LTS with $\mathcal{T} = (Q, \Sigma, P, \Delta, v)$
    and $\mathcal{T}' = (Q', \Sigma, P,$ $ \Delta', v')$. Moreover, let be $(q_1, \dots, q_r) \in Q^r$ and $({q_1}',
    \dots, {q_r}') \in {Q^r}'$.

    A query $\mathcal{Q}$ is called \emph{bisimulation invariant} if $(\mathcal{T}, q_i) \sim (\mathcal{T}', q_i')$
    for all $1 \leq i \leq r$ implies that $(q_1, \dots, q_r) \in \mathcal{Q}^\mathcal{T}$ iff $({q_1}',
    \dots, {q_r}') \in \mathcal{Q}^{{\mathcal{T}}'}$.
\end{definition}

The second category of a query tells us which complexity class a query belongs to.

\begin{definition}
    \label{definition:queryBelongsToComplexityClass}
    Let $\mathcal{T}$ be a LTS with $\mathcal{T} = (Q, \Sigma, P, \Delta, v)$ and $(q_1, \dots, q_{r}) \in Q^r$.
    A query $\mathcal{Q}$ belongs to complexity class $\mathcal{C}$ if there is a DTM (see
    Section~\ref{sec:descriptiveComplexity}) in $\mathcal{C}$ for deciding on input $(\mathcal{T}, (q_1, \dots,
    q_{r}))$ whether $(q_1, \dots, q_{r}) \in \mathcal{Q}^\mathcal{T}$.
\end{definition}

\begin{example}{\cite{lange2014capturing}}
    \label{example:query_bisimulation}
    Let $\mathcal{Q}$ be a $2$-adic query that maps a LTS to the set of pairs of states that are bisimilar. As
    example, let $\mathcal{T}$ the LTS from Example~\ref{example:lts}. The query $\mathcal{Q}$ maps $\mathcal{T}$ to
    the following set of pairs:
    \[\mathcal{Q}^\mathcal{T} = \{(q_1, q_1), \dots, (q_5, q_5), (q_1, q_5), (q_5, q_1), (q_2, q_4), (q_4,
    q_2)\}\]
    Because bisimilarity can be decided in P, there is an algorithm in P for deciding on input $(\mathcal{T}, (q_1,
    q_2))$ whether $(q_1, q_2) \in \mathcal{Q}^\mathcal{T}$, i.e. $\mathcal{Q}$ belongs to complexity class P.
    Furthermore, as mentioned in Example~\ref{example:bisimilar}, $\sim$ is an
    equivalence relation, it follows that $\mathcal{Q}$ is bisimulation invariant.
\end{example}

From Definition~\ref{definition:bisimulationInvariant}, Definition~\ref{definition:queryBelongsToComplexityClass}
and the complexity classes $k$-EXP-TIME and $k$-EXPSPACE of Defintion~\ref{definition:k_exptime_and_k_expspace} follow
the queries that we want to investigate.

\begin{definition}
    \label{definition:kExptimekExpspace}
    \exptime{$k$} are the bisimulation invariant queries that belong to complexity class $k$-EXPTIME and
    \expspace{$k$} are the bisimulation invariant queries that belong to complexity class $k$-EXSPACE, where $k \geq 0$.
\end{definition}

In the next chapter we introduce the polyadic higher order fixed point logic and a restriction called tail-recursive
devised by Lange and Lozes in~\cite{lange2014capturing}. We want to compare the in
Definition~\ref{definition:kExptimekExpspace} presented queries with these logics to make a further contribution in
the theory of descriptive complexity.