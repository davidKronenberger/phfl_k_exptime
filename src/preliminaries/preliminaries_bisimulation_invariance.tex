%%
%% Author: Davidov
%% 16.05.2018
%%

\subsection{Bisimulation Invariance}\label{subsec:bisimulationInvariance}

First of all, we need the definition of \textit{labeled transition systems}. A labeled transition system is a graph
with labeled vertices and edges. Formally, it is the following.

\begin{definition}
    \label{definition:lts}
    A quintuple $\mathcal{T} = (Q, \Sigma, P, \Delta, \nu)$ is called a \emph{labeled transition system} (\emph{LTS}),
    where
    \begin{compactitem}
        \item $Q$ is a set of states,
        \item $\Sigma$ is a finite set of actions,
        \item $P$ is a finite set of propositions,
        \item $\Delta \subseteq Q \times \Sigma \times Q$ is the labeled transition relation and
        \item $\nu: Q \rightarrow 2^P$ is a function that maps each state to a set of propositions.
    \end{compactitem}
\end{definition}

For all $q_1, q_2 \in Q$ and all $a \in \Sigma$ we write $q_1 \overset{a}{\rightarrow} q_2$ for $(q_1, a, q_2) \in
\Delta$. On this systems or rather on the states of the systems it is possible to define relations. The
following relation describes those states that have the same behaviour. For this, let be $\mathcal{T}_1 = (Q_1,
\Sigma_1, P_1, \Delta_1, \nu_1)$ and $\mathcal{T}_2 = (Q_2, \Sigma_2, P_2, \Delta_2, \nu_2)$ two LTSs.

\begin{definition}
    A \emph{bisimulation} is a binary relation $R \subseteq Q_1 \times Q_2$ that fulfills for all $(q_1, q_2) \in R$
    \begin{compactitem}
        \item $\nu_1 (q_1) = \nu_2 (q_2)$,
        \item for all $a_1 \in \Sigma_1$ and all $q_1' \in Q_1$, if $q_1 \overset{a_1}{\rightarrow} q_1' \in
        \Delta_1$, then there is a state $q_2' \in Q_2$ with $a_1 \in \Sigma_2$, $q_2
        \overset{a_1}{\rightarrow} q_2' \in \Delta_2$ and $(q_1', q_2') \in R$ and
        \item for all $a_2 \in \Sigma_2$ and all $q_2' \in Q_2$, if $q_2 \overset{a_2}{\rightarrow} q_2' \in
        \Delta_2$, then there is a state $q_1' \in Q_1$ with $a_2 \in \Sigma_1$, $q_1
        \overset{a_2}{\rightarrow} q_1' \in \Delta_1$ and $(q_1', q_2') \in R$.
    \end{compactitem}
    We call two states $q_1 \in Q_1$, $q_2 \in Q_2$ \emph{bisimilar}, noted as $(\mathcal{T}_1, q_1) \sim
    (\mathcal{T}_2, q_2)$, if there
    is a bisimulation $R$ such that $(q_1, q_2) \in R$.
\end{definition}

Furthermore, we can describe properties of LTS. \textit{Queries} are one way to describe these properties. A query
is a function that maps a LTS $\mathcal{T} = (Q, \Sigma, P, \Delta, \nu)$ to a subset of $Q$.

\begin{definition}
    \label{definition:query}
    A function $\mathcal{Q} : \mathscr{T} \rightarrow \mathscr{Q}, \mathcal{T} \mapsto Q^\mathcal{T}$ is called a
    \emph{query},
    where
    \begin{compactitem}
        \item $\mathscr{T}$ is the set of all LTS,
        \item $\mathscr{Q}$ is the set of all sets of states and
        \item $\mathcal{T} = (Q^\mathcal{T}, \Sigma^\mathcal{T}, P^\mathcal{T}, \Delta^\mathcal{T}, \nu^\mathcal{T})
        \in \mathscr{T}$ is a LTS.
    \end{compactitem}
\end{definition}

The queries defined in Definition~\ref{definition:query} can be categorized. Here we are interested in two of these
categories. The first category is called \textit{bisimulation invariant}. This category describes those queries that
can't distinguish bisimilar states. In~\cite{otto1999bisimulation} this property is defined over so called
\textit{Kripke structures}. A Kripke structure
is a transition system. Remark, that transition systems  have only one type of actions. This means the edges of the
graph haven't labels.

\begin{definition}
    \label{definition:bisimulationInvariant}
    Let $\mathcal{T}$, $\mathcal{T}' \in \mathscr{T}$ be two LTSs with $\mathcal{T} = (Q, \Sigma, P, \Delta, \nu)$
    and $\mathcal{T}' = (Q', \Sigma', P', \Delta', \nu')$. Furthermore, let be $q \in Q$ and $q' \in Q'$.

    A query $\mathcal{Q}$ is called \emph{bisimulation invariant} if $(\mathcal{T}, q) \sim (\mathcal{T}', q')$
    implies that
    $q \in \mathcal{Q}(\mathcal{T})$ iff $q' \in \mathcal{Q}(\mathcal{T'})$.
\end{definition}

The second category of a query tells us which complexity class a query belongs to.

\begin{definition}
    \label{definition:queryBelongsToComplexityClass}
    Let $\mathcal{T} \in \mathscr{T}$ be a LTS with $\mathcal{T} = (Q, \Sigma, P, \Delta, \nu)$ and $q \in Q$.

    A query $\mathcal{Q}$ belongs to a complexity class $\mathcal{C}$ if there is an algorithm in $\mathcal{C}$ for
    deciding on input $(\mathcal{T}, q)$ whether $q \in \mathcal{Q}(\mathcal{T})$.
\end{definition}

This definition leads us to the next chapter and the definitions for descriptive complexity.