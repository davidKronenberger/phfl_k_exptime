%%
%% Author: Davidov
%% 16.05.2018
%%

\section{Polyadic Higher Order Fixpoint Logic}\label{sec:polyadichigherorderfixpointlogic}

In this section, we present a logic with name Polyadic Higher Order Fixpoint Logic, abbreviated with PHFL, that was
introduced by M. Lange and E. Lozes in~\cite{lange2014capturing}. It is defined over LTS (see
Definition~\ref{definition:lts}) and extends the polyadic modal $\mu$-calculus~\cite{otto1999bisimulation} with
higher order fixpoints like M. Viswanathan and R. Viswanathan it did with monadic case of modal
$\mu$-calculus~\cite{kozen1983results} in~\cite{viswanathan2004higher}. The logic of M. Viswanathan and R
.Viswanathan with name higher order fixed point logic is a combination of propositional logic, modal operators and
a simply typed $\lambda$-calculus with fixed point operators. 

\subsection{PHFL Types}\label{subsec:phflTypes}

Before defining formulas of PHFL we need to introduce the PHFL types. These definitions are guided
by~\cite{viswanathan2004higher} and~\cite{lange2014capturing}.

\begin{definition}
    \emph{PHFL types} are given by the grammar
    \[\sigma, \tau \Coloneqq \bullet \mid \sigma^v \rightarrow \tau,\]
    where $v$ is called \textit{variance}. The \emph{variances} of PHFL are defined by the grammar
    \[v \Coloneqq + \mid - \mid 0.\]
\end{definition}

All types will be interpreted as a partially ordered sets. Partial orders are relations that are reflexive, transitive
and antisymmetric. Let $\mathcal{A} = (A, \leq_A)$ and $\mathcal{B} = (B, \leq_B)$ be two partial orders. Then
$\mathcal{A} \rightarrow \mathcal{B}$ is the partial order of monotone functions ordered pointwise, i.e.
\[\mathcal{A} \rightarrow \mathcal{B} = \{f\colon A\rightarrow B \mid \text{ for all } x,y \in A.\,x\leq_A y \text{
implies }
f(x)
\leq_B f(y)\}\]
and the ordering relation is given by
\[f \leq_{\mathcal{A}\rightarrow\mathcal{B}} g\text{ iff } \text{ for all } x\in \mathcal{A}.\,f(x) \leq_{\mathcal{B}} g
(x)\].

\begin{definition}
    Let $\mathcal{T} = (Q, \Sigma, P, \Delta, v)$ be a LTS and $d \in \mathbb{N}$ the dimension of PHFL,
    then $\llbracket\tau\rrbracket_\mathcal{T}$ the
    semantics
    of type $\tau$ is defined by $\tau$ as follows:
        \[\llbracket\tau\rrbracket_\mathcal{T}=
        \begin{cases}
            (\mathcal{P}(Q^d), \subseteq),  & \text{if }\tau = \bullet\\
            ((\llbracket\sigma_1\rrbracket_\mathcal{T})^v \rightarrow \llbracket\sigma_2\rrbracket_\mathcal{T}, \leq_{
            (\llbracket\sigma_1\rrbracket_\mathcal{T})^v \rightarrow \llbracket\sigma_2\rrbracket_\mathcal{T}}), &
            \text{if }\tau = \sigma_1^v\rightarrow \sigma_2,
        \end{cases}\]
    where for any partial order $\mathcal{A} = (A, \leq_A)$, $\mathcal{A}^v = (A, \leq_A^v)$ is a partial order
    with $\leq_A^+ = \leq_A$, $\leq_A^- = \{(a, b) \mid (b, a) \in \leq_A\}$ and $\leq_A^0 = \leq_A^+ \cap \leq_A^-$.
\end{definition}

The partial orders $\llbracket\tau\rrbracket_\mathcal{T}$ for any PHFL type $\tau$ are complete lattices. That means we
have meets and joins, denoted by $\sqcap_{\llbracket\tau\rrbracket_\mathcal{T}}$ and
$\sqcup_{\llbracket\tau\rrbracket_\mathcal{T}}$, and least and greatest elements, denoted by
$\bot_{\llbracket\tau\rrbracket_\mathcal{T}}$ and $\top_{\llbracket\tau\rrbracket_\mathcal{T}}$ for any subset of
$\llbracket\tau\rrbracket_\mathcal{T}$. This ensures that the least and greatest fixpoint over all monotone PHFL types
exist~\cite{tarski1955lattice}. See Section~\ref{sec:fixpoints} for further information about fixpoints.

\begin{definition}
    The \emph{maximal arity} $ma(\tau)$ and the \emph{order} $ord(\tau)$ of a PHFL type $\tau$ are defined
    inductively on
    $\tau$ as follows:
\[ma(\tau)=
\begin{cases}
    1, & \text{if }\tau = \bullet\\
    max(\{n\} \cup \{ma(\tau_i)\mid1,\dots,n\}), &
    \text{if }\tau = \tau_1\rightarrow\dots\rightarrow\tau_n\rightarrow\bullet
\end{cases}\]
\[ord(\tau)=
\begin{cases}
    0, & \text{if }\tau = \bullet\\
    max(\{1 + ord(\sigma_1), ord(\sigma_2)\}), & \text{if }\tau = \sigma_1 \rightarrow \sigma_2
\end{cases}\]
\end{definition}

\subsection{PHFL Syntax}\label{subsec:phflSyntax}

Next, we want to define the syntax of PHFL formulas.

\begin{figure}
    \caption{Derivation Rules for PHFL formulas.}
    \label{figure:phfl-typing-rules}
    \begin{mathpar}
        \Gamma \vdash \top \colon \bullet \and
        \Gamma \vdash p_i \colon \bullet \and
        \inferrule{\Gamma \vdash \Phi \colon \bullet}{\Gamma \vdash \langle a \rangle_i \Phi \colon \bullet} \and
        \inferrule{\Gamma \vdash \Phi \colon \bullet}{\Gamma \vdash \{\emph{i} \leftarrow \emph{j}\}\Phi \colon
        \bullet} \and
        \inferrule{\Gamma^-\vdash\Phi\colon \bullet}{\Gamma \vdash \neg \Phi \colon \bullet} \and
        \inferrule{\Gamma\vdash\Phi \colon \tau \\ \Gamma\vdash\Psi\colon \tau}{\Gamma \vdash \Phi \vee
        \Psi \colon  \tau} \and
        \inferrule{v \in \{+, 0\} }{\Gamma, X^v \colon\tau \vdash X\colon\tau} \and
        \inferrule{\Gamma,X^v\colon\sigma\vdash \Phi\colon\tau}{\Gamma\vdash \lambda (X^v \colon \tau)
        .\Phi\colon\sigma^v\rightarrow\tau} \and
        \inferrule{\Gamma,X^+ \colon \tau \vdash \Phi\colon\tau}{\Gamma \vdash \mu (X \colon\tau). \Phi\colon\tau} \and
        \inferrule{\Gamma\vdash \Phi\colon\sigma^+ \rightarrow \tau \\ \Gamma\vdash\Psi\colon\sigma}{\Gamma \vdash
        \Phi\,\Psi \colon \tau} \and
        \inferrule{\Gamma\vdash \Phi\colon\sigma^- \rightarrow \tau \\ \Gamma^-\vdash\Psi\colon\sigma}{\Gamma \vdash
        \Phi\,\Psi \colon \tau} \and
        \inferrule{\Gamma\vdash \Phi\colon\sigma^0 \rightarrow \tau \\ \Gamma \vdash \Psi\colon\sigma \\ \Gamma^-
        \vdash\Psi\colon\sigma}{\Gamma \vdash \Phi\,\Psi \colon \tau} \and
    \end{mathpar}
\end{figure}

\begin{definition}
    Let $P$ a set of propositions, $\Sigma$ a set of actions and $\mathcal{V} = \{X_1, X_2, \dots\}$ a countable
    infinite
    set of variables, then
    \emph{$d$-adic PHFL formulas} $\Phi, \Psi,\dots$ are defined by the grammar
    \begin{align*}
        \Phi,\Psi\Coloneqq&\top \mid p_i \mid \Phi \vee \Psi \mid \neg \Phi \mid \langle a \rangle_i \Phi \mid
        \{\emph{j}\,\} \Phi \mid X \mid \lambda (X^v\colon\tau).\Phi \mid \Phi\,\Psi\mid  \mu (X\colon\tau).\Phi
    \end{align*}
    where
    \begin{compactitem}
        \item $\emph{j} = (e(1), \dots, e(d))$ and $e: \{1, \dots, d\} \rightarrow \{1, \dots, d\}$,
        \item $i \in \{1, \dots, d\}$
        \item $v$ is a variance,
        \item $\tau$ is a type,
        \item $p \in P$,
        \item $a \in \Sigma$ and
        \item $X \in \mathcal{V}$.
    \end{compactitem}
\end{definition}

For convenience, we use some other further standard notations like $\Phi \wedge \Psi$, $[a]_i\Phi$, $\nu
X \colon \tau.\Phi$ or $\Phi \Leftrightarrow \Psi$. Note, that this logic is defined over LTS. The formulas are
often interpreted as a game played by two players moving pebbles along the transitions of an LTS. The two players
are called Prover and Refuter. So, $p_i$ can be interpreted as, the position of the $i$-th pebble fulfills
property $p$. $\langle a \rangle_i \Phi$ means, Prover has to move the $i$-th pebble along an $a$-transition and
check if there holds $\Phi$. With the formula $\{\emph{j}\,\} \Phi$ is mentioned, that all pebbles
are moved from Prover to the positions described by tuple $\emph{j}$ and after this $\Phi$ have to
be fulfilled. The player who has to move the pebbles changes on negations. $\lambda (X^v\colon\tau).\Phi$ is
interpreted as a function that expects arguments of $(\llbracket\tau\rrbracket_\mathcal{T})^v$. We see, that the
formulas also have types. For this, we have ensure that a formula is well-typed.

\begin{definition}
    Let $X_1, \dots, X_n$ variables, $\Phi$ a PHFL formula, $v_1, \dots, v_n$ variances and $\tau, \tau_1, \dots,
    \tau_n$ types, then $\Gamma = X_1^{v_1}\colon \tau_1, \dots X_n^{v_n} \colon \tau_n$ is
    called a \emph{type environment} and $\Gamma \vdash \Phi\colon\tau$
    is called a \emph{type judgement}. Let $\Gamma^- = X_1^{v_1^-}\colon \tau_1, \dots
    X_n^{v_n^-} \colon \tau_n$ be a type environment then $\Gamma^- = X_1^{v_1^-}\colon \tau_1, \dots
    X_n^{v_n^-} \colon \tau_n$, where $-^- = +$, $+^- = -$ and $0^- = 0$.
\end{definition}

A type judgment is called \textit{derivable} if it generates a derivation tree according to the rules of
Figure~\ref{figure:phfl-typing-rules}. This type system ensures that we do not create senseless formulas like
$\langle a \rangle_i p_j p_k$. Furthermore, it can order formulas with semantics and guarantees so monotonicity. A
formula $\Phi$ is called \textit{well-typed} if the type judgement $\emptyset \vdash \Phi:\tau$ is derivable for some
type $\tau$. Note, that we are here only interested in well-typed formulas. For those formulas where also the variable
types are obvious, we omit the type on the variables.

\subsection{PHFL Semantics}\label{subsec:phflSemantics}

\begin{figure}
    \caption{Semantics of PHFL formulas.}
    \label{figure:phfl-semantics}
    \begin{align*}
        \llbracket \Gamma \vdash \top \colon \bullet \rrbracket^\eta_\mathcal{T} =\,& Q^d\\
        \llbracket \Gamma \vdash p_i \colon \bullet \rrbracket^\eta_\mathcal{T} =\,& \{(q_1, \dots, q_d) \in Q^d \mid p \in v
        (q_i)\}\\
        \llbracket \Gamma \vdash \langle a \rangle_i \Phi \colon \bullet \rrbracket^\eta_\mathcal{T} =\,& \{(q_1,
        \dots, q_d) \in Q^d \mid \text{ it exists } \\& ({q'}_1, \dots, {q'}_d) \in \llbracket \Gamma \vdash \Phi \colon
\bullet \rrbracket^\eta_\mathcal{T} \text{ such that } \\&q_i \overset{a}{\rightarrow} {q'}_i \text{ and for all } j
        \neq
        i \text{ holds } q_j = {q'}_j\}\\
        \llbracket \Gamma \vdash \Phi \vee \Psi \colon \tau \rrbracket^\eta_\mathcal{T} =\,& \llbracket \Gamma \vdash \Phi
        \colon \tau \rrbracket ^\eta_\mathcal{T} \sqcup_\tau \llbracket \Gamma \vdash \Psi \colon \tau \rrbracket ^\eta_\mathcal{T}\\
        \llbracket \Gamma \vdash \neg \Phi \colon \bullet \rrbracket^\eta_\mathcal{T} =\,& Q^d \setminus \llbracket
        \Gamma^- \vdash \Phi
        \colon \bullet \rrbracket ^\eta_\mathcal{T}\\
        \llbracket \Gamma \vdash \{\emph{j}\} \Phi \colon \bullet \rrbracket^\eta_\mathcal{T} =\,&
        \{(q_1, \dots, q_d) \in Q^d \mid \\ &(q_{j_1}, \dots, q_{j_d}) \in \llbracket \Gamma \vdash \Phi
        \colon \bullet
        \rrbracket ^\eta_\mathcal{T}\}\\
        \llbracket \Gamma, X \colon \tau \vdash X \colon \tau \rrbracket^\eta_\mathcal{T} =\,& \eta(X)\\
        \llbracket \Gamma \vdash \mu (X \colon \tau).\,\Phi \colon \tau \rrbracket^\eta_\mathcal{T} =\,&
        \bigsqcap\,_{\llbracket\tau\rrbracket_\mathcal{T}} \{\mathcal{X} \in \llbracket \tau \rrbracket_\mathcal{T}
        \mid \\
        &\llbracket \Gamma, X^+ \colon \tau \vdash \Phi \colon \tau \rrbracket^{\eta[X \mapsto \mathcal{X}]}_\mathcal{T}
        \leq_{\llbracket \tau \rrbracket_\mathcal{T}} \mathcal{X}\}\\
        %\llbracket         \Gamma
 %       \vdash
  %      \lambda X^+ \colon \tau.\Phi \colon \tau \rrbracket ^\eta_\mathcal{T}\\
        \llbracket \Gamma \vdash \lambda (X^v \colon \sigma).\,\Phi \colon \sigma^v \rightarrow \tau \rrbracket
        ^\eta_\mathcal{T} =\,& F \in \llbracket \sigma^v \rightarrow \tau \rrbracket_\mathcal{T} \text{ such that for
        all }
        \mathcal{X} \in \llbracket \sigma \rrbracket_\mathcal{T}.\, \\
        &F(\mathcal{X}) = \llbracket \Gamma, X^v \colon \sigma \vdash \Phi \colon \tau \rrbracket^{\eta[X \mapsto
        \mathcal{X}]}_\mathcal{T}\\
        \llbracket \Gamma \vdash \Phi\,\Psi \colon \tau \rrbracket^\eta_\mathcal{T} =\,& \llbracket \Gamma \vdash \Phi
        \colon \sigma
        ^v \rightarrow \tau \rrbracket ^\eta_\mathcal{T}(\llbracket \Gamma \vdash \Psi \colon \sigma \rrbracket ^\eta_\mathcal{T})
    \end{align*}
\end{figure}

To define the semantics of PHFL formulas we need a mapping $\eta$ that associates to each variable an element of its
type semantics, i.e. $\eta(X) \in \llbracket\tau\rrbracket_\mathcal{T}$ for $X$ of type $\tau$. Let $\Phi$ be a
well-typed formula of type $\tau$, $\mathcal{T}$ a LTS and $\eta$ a variable mapping, then the semantics
$\llbracket\Gamma \vdash \Phi\colon \tau \rrbracket^\eta_\mathcal{T}$ are defined inductively on $\Phi$ which maps to
an element of $\llbracket\tau\rrbracket_\mathcal{T}$ as explained in Figure~\ref{figure:phfl-semantics}.
Note, that $\eta[X \mapsto \mathcal{X}]$ is a mapping $\eta'$ that is equal to $\eta$ but $\eta'(X) = \mathcal{X}$.

In this thesis we are interested in PHFL formulas that have a specific order. For this, a formula $\Phi$ has order $k$
if $k = max(\{ord(\tau)\mid \Psi \colon \tau$ \textit{is a subformula of} $\Phi\})$. The set of formulas that have
order at most $k$ is denoted by PHFL$^k$.

\begin{example}{\cite{lange2014capturing}}
    \label{example:phfl_order_0}
    The following $2$-adic PHFL$^0$ formula $\Phi_\sim$ describes bisimilarity i.e. it denotes
    those pairs $(q_1, q_2)$ such that $q_1 \sim q_2$ and vice versa.
    \begin{align*}
        \Phi_\sim = \nu (X \colon \bullet).\,
        \underset{a \in \Sigma}{\bigwedge} [a]_1 \langle a \rangle_2 X \wedge [a]_2 \langle a \rangle_1 X \wedge
        \underset{p \in P}{\bigwedge} p_1 \Leftrightarrow p_2
    \end{align*}
\end{example}

%\begin{example}{\cite{lange2014capturing}}
%    \label{example:phfl_order_2}
%    The following $2$-adic PHFL$^1$ formula $\Phi$ describes trace equivalence of two states in a LTS i.e. it denotes
%    those pairs $(q_1, q_2)$ for which $q_1$ has the same traces as $q_2$ and vice versa.
 %   \begin{align*}
 %       \Phi = &(\mu (F \colon \bullet^0 \rightarrow (\bullet^0 \rightarrow \bullet)).\,
 %       \lambda (X \colon \bullet).\, \lambda (Y \colon \bullet).\, \\&X \Leftrightarrow Y \wedge
 %       \underset{a \in \Sigma}{\bigwedge} F \langle a \rangle_1 X \langle a \rangle_2 Y)\top \top
 %   \end{align*}
%\end{example}

\begin{remark}
To simplify PHFL formulas with fixpoint and $\lambda$ operators we will use \emph{fixpoint unfolding principle} and \emph{$\beta$-reduction}. Because we work on finite structures and the semantics of PHFL is invariant under $\beta$-reduction we can use those procedures without loss of generality.
\end{remark}

Before looking at the tail-recursive fragment see the following definition of $r$-adic queries that are associated to a
closed $d$-adic formula $\Phi$. A PHFL formula $\Phi$ is closed if the type judgement $\Gamma \vdash \Phi \colon \bullet$ is derivable for at least one type environment $\Gamma$.

\begin{definition}
\label{definition:query_associated_to_formula}
    Given a dimension $d$ of PHFL, a type environment $\Gamma$, a variable assignment $\eta$ and a closed $d$-adic PHFL
    formula $\Phi$ we call a $r$-adic query $\mathcal{Q}^r_\Phi$ associated to $\Phi$ if there is for all LTS
    $\mathcal{T}$, all variable mappings $\eta$ and all $(q_1, \dots, q_r) \in {\mathcal{Q}^r_\Phi}^\mathcal{T}$ a $(s_1, \dots, s_d) \in
    \llbracket \Gamma \vdash \Phi \colon \bullet \rrbracket^\eta_\mathcal{T}$ such that $q_i = s_i$ for all $i \in
    \{1, \dots, min(\{r, d\})\}$.
\end{definition}

For example $\mathcal{Q}^2_{\Phi_\sim}$ is the same query as in Example~\ref{example:query_bisimulation}, where
$\Phi_\sim$ is the formula from Example~\ref{example:phfl_order_0}.

\subsection{Tail-Recursive PHFL}\label{subsec:tail-recursivePhfl}


\begin{figure}
    \caption{Derivation Rules for PHFL formulas that shall be tail-recursive.}
    \label{figure:phfl-tail-recursive}
    \begin{mathpar}
        \bar{Y} \vdash tail(p_i, \bar{X}) \and
        \inferrule{X \in \bar{X} \cup \bar{Y}}{\bar{Y} \vdash tail(X, \bar{X})} \and
        \inferrule{\bar{Y} \vdash tail(\Phi, \emptyset)}{\bar{Y} \vdash tail(\neg \Phi, \bar{X})} \and
        \inferrule{\bar{Y} \vdash tail(\Phi, \bar{X})}{\bar{Y} \vdash tail(\{\emph{j}\} \Phi,
        \bar{X})} \and
        \inferrule{\bar{Y} \vdash tail(\Phi, \bar{X}) \\ \bar{Y} \vdash tail(\Psi, \bar{X})}{\bar{Y} \vdash tail
        (\Phi \vee \Psi, \bar{X})} \and
        \inferrule{\bar{Y} \vdash tail(\Phi, \bar{X})}{\bar{Y} \vdash tail(\langle a \rangle_i \Phi, \bar{X})} \and
        \inferrule{\bar{Y} \vdash tail(\Phi, \emptyset)}{\bar{Y} \vdash tail([a]_i \Phi, \bar{X})} \and
        \inferrule{\bar{Y} \cup \{Z\} \vdash tail(\Phi, \bar{X})}{\bar{Y} \vdash tail(\lambda Z^v \colon \tau . \Phi,
        \bar{X})} \and
        \inferrule{\bar{Y} \vdash tail(\Phi, \emptyset) \\ \bar{Y} \vdash tail(\Psi, \bar{X})}{\bar{Y} \vdash tail
        (\Phi \wedge \Psi, \bar{X})} \and
        \inferrule{\bar{Y} \vdash tail(\Phi, \bar{X}) \\ \bar{Y} \vdash tail(\Psi, \emptyset)}{\bar{Y} \vdash tail
        (\Phi\,\Psi, \bar{X})} \and
        \inferrule{\bar{Y} \vdash tail(\Phi, \bar{X} \cup \{Z\})}{\bar{Y} \vdash tail(\mu Z \colon \tau . \Phi,
        \bar{X})} \and
        \inferrule{\bar{Y} \vdash tail(\Phi, \bar{X} \cup \{Z\})}{\bar{Y} \vdash tail(\nu Z \colon \tau . \Phi,
        \bar{X})}
    \end{mathpar}
\end{figure}

Next, we want to define a fragment of PHFL formulas. This fragment is called tail-recursive and ensures that
some combinations of subformulas do not appear in an PHFL formula. For this, let the logical connective
$\wedge$, the modality operators $[a]_i$ and the greatest fixpoint operator $\nu$ be further primitives of PHFL formula
syntax. Intuitively, tail-recursive PHFL formulas are PHFL formulas where all fixpoint variables do not occur freely
under the operators $\neg$ and $[a]_i$ nor in $\Psi$ of formulas of the type $\Phi\,\Psi$ or $\Psi \wedge \Phi$.

\begin{definition}
    A closed PHFL formula $\Phi$ is called \emph{tail-recursive} if $\emptyset \vdash tail(\Phi, \emptyset)$ is
    derivable via the rules in Figure~\ref{figure:phfl-tail-recursive}.
\end{definition}

The set of all tail-recursive PHFL formulas that have order at most $k$ is denoted by PHFL$^k_{tail}$.

\begin{example}{\cite{lange2014capturing}}
    Looking at Figure~\ref{figure:phfl-tail-recursive}, we can see, that $\Phi_1 = \mu X.[a]_1 X$ is not tail
    recursive, because $X$ occurs under $[a]_1$. Also $\Phi_2 = \mu F .\lambda X. (F X) \wedge (F(F X))$
    is not tail-recursive because $F$ occurs on the left side of the logical operator $\wedge$. The second reason why
    $\Phi_2$ is not tail-recursive is because $F$ also occurs in $F X$ of subformula $F (F X)$. 
    Note that all those formulas that do not use fixpoints are obviously tail-recursive. An 
    example of a formula that uses a fixpoint and is also tail-recursive is given by 
    $\Phi_3 = (\mu F. \lambda X. p_1 \underset{a \in \Sigma}{\bigvee} F \langle a \rangle_1  X) \bot$ and denotes those states that can reach a state where property $p$ holds.
    
\end{example}