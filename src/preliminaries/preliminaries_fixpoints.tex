%%
%% Author: Davidov
%% 16.05.2018
%%

\section{Fixpoints}\label{sec:fixpoints}

To define the polyadic higher-order fixpoint logic and the higher-order logic with least and partial fixpoints, we want to consider in this section fixpoints in general. The first fixpoint we consider is the least fixpoint.

\begin{definition}
   Let $F\colon A \rightarrow A$ be an operator on a finite set $A$, then $x \in A$
   is called a \emph{fixpoint} of $F$ if $F(x) = x$. Let $x$ be a fixpoint of $F$ and $\sqsubseteq$ an partial order on $A$, then $x$ is called the \emph{least
   fixpoint} of $F$, abbreviated as $\mathit{LFP}$($F$), if all other fixpoints $y$ of $F$ are bigger with respect to $sqsubset$ as $x$ i.e. $x
   \sqsubseteq y$ for all fixpoints $y$ of $F$. A fixpoint $x$ is called the \emph{greatest fixpoint} if $y \sqsubseteq x$ for all fixpoints $y$ of $F$.
\end{definition}

From Knaster-Tarski~\cite{tarski1955lattice} we know that if an operator $F\colon A \rightarrow 
A$ is monotone the least and greatest fixpoints of $F$ exists. $F$ is monotone iff for all $x, y
 \in A$ if $x \sqsubseteq y$ then $F(x) \sqsubseteq F(y)$ holds.

\begin{example}
    \label{example:lfp}
    Peano's induction axiom defined as HO$^2$ formula in Example~\ref{example:ho} can be also defined with a least
    fixpoint operator. This example is mainly from~\cite{hetzl2017higher}. Let $A = \mathbb{Z}$ and $F(X) = X \cup \{0\} \cup
    \{x + 1 \mid x \in X\}$ the operator on $A$. Obviously, $F$ is monotone and fixpoints of $F$ are of the form
    $[-k, \infty[~\subseteq\mathbb{Z}$ for $k \in \mathbb{N}$. The least fixpoint of operator $F$ describes Peano's
    induction axiom, i.e. $\mathit{LFP}(F) = \mathbb{N}$.
\end{example}

Next, we define the partial fixpoint. Since the
$\mathit{LFP}$ restricts the operator to be monotone to exist, the partial fixpoint need not any
restriction on the operator.

\begin{definition}
    Let $F\colon A \rightarrow A$ be an operator on a finite set $A$, then the \emph{partial
    fixpoint} of $F$, abbreviated as $\mathit{PFP}$($F$), is defined as follows:
    \[\mathit{PFP}(F)\coloneqq\begin{cases}
               F^{i+1}(\emptyset)=F^i(\emptyset),  & \text{if such } i \in \{0,\dots,|A|\} \text{ exists}\\
               \emptyset, & \text{otherwise,}
    \end{cases}\]
    where $F^0(\emptyset) = \emptyset$, $F^1(\emptyset) = F(\emptyset)$, $F^2(\emptyset) = F(F(\emptyset))$, and so on.
\end{definition}

Note, that for monotone $F$ holds $\mathit{PFP}(F)$ equals $\mathit{LFP}(F)$.	