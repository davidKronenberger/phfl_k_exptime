\chapter{Preliminaries}
\label{ch:preliminaries}
This chapter introduces all necessary definitions to prove that PHFL$^k =$~\exptime{$k$} and
PHFL$^{k+1}_{tail} =$~\expspace{$k$}. The notions are mainly from~\cite{immerman1999descriptive},
~\cite{papadimitriou1994complexity},~\cite{otto1999bisimulation},~\cite{freireMartins2011descriptive}
and~\cite{lange2014capturing}.

We assume that the reader is already familiar with basic notions of first order logic and 
computational complexity. In the first section we define a special kind of graph called labelled transition systems and 
define some operations on it. Additionally, we define queries and special forms of queries. 
In the next section we give some information on fixpoints that are used in the section that follows. 
There the logic PHFL is defined. In Section~\ref{sec:descriptiveComplexity} we present the 
descriptive complexity and the complexity classes \exptime{$k$} and \expspace{$k$}. In the last section we define the higher-order logic and combinations with LFP and PFP.

%%
%% Author: Davidov
%% 16.05.2018
%%

\subsection{Bisimulation Invariance}\label{subsec:bisimulationInvariance}

First of all, we need the definition of \textit{labeled transition systems}. A labeled transition system is a graph
with labeled vertices and edges. Formally, it is the following.

\begin{definition}
    A quintuple $\mathcal{T} = (Q, \Sigma, P, \Delta, \nu)$ is called a \emph{labeled transition system} (\emph{LTS}),
    where
    \begin{compactitem}
        \item $Q$ is a set of states,
        \item $\Sigma$ is a finite set of actions,
        \item $P$ is a finite set of propositions,
        \item $\Delta \subseteq Q \times \Sigma \times Q$ is the labeled transition relation and
        \item $\nu: Q \rightarrow 2^P$ is a function that maps each state to a set of propositions.
    \end{compactitem}
\end{definition}

For all $q_1, q_2 \in Q$ and all $a \in \Sigma$ we write $q_1 \overset{a}{\rightarrow} q_2$ for $(q_1, a, q_2) \in
\Delta$. On this systems or rather on the states of the systems it is possible to define relations. The
following relation describes those states that have the same behaviour. For this, let be $\mathcal{T}_1 = (Q_1,
\Sigma_1, P_1, \Delta_1, \nu_1)$ and $\mathcal{T}_2 = (Q_2, \Sigma_2, P_2, \Delta_2, \nu_2)$ two LTSs.

\begin{definition}
    A \emph{bisimulation} is a binary relation $R \subseteq Q_1 \times Q_2$ that fulfills for all $(q_1, q_2) \in R$
    \begin{compactitem}
        \item $\nu_1 (q_1) = \nu_2 (q_2)$,
        \item for all $a_1 \in \Sigma_1$ and all $q_1' \in Q_1$, if $q_1 \overset{a_1}{\rightarrow} q_1' \in
        \Delta_1$, then there is a state $q_2' \in Q_2$ with $a_1 \in \Sigma_2$, $q_2
        \overset{a_1}{\rightarrow} q_2' \in \Delta_2$ and $(q_1', q_2') \in R$ and
        \item for all $a_2 \in \Sigma_2$ and all $q_2' \in Q_2$, if $q_2 \overset{a_2}{\rightarrow} q_2' \in
        \Delta_2$, then there is a state $q_1' \in Q_1$ with $a_2 \in \Sigma_1$, $q_1
        \overset{a_2}{\rightarrow} q_1' \in \Delta_1$ and $(q_1', q_2') \in R$.
    \end{compactitem}
    We call two states $q_1 \in Q_1$, $q_2 \in Q_2$ \emph{bisimilar}, noted as $(\mathcal{T}_1, q_1) \sim
    (\mathcal{T}_2, q_2)$, if there
    is a bisimulation $R$ such that $(q_1, q_2) \in R$.
\end{definition}

Furthermore, we can describe properties of LTS. \textit{Queries} are one way to describe these properties. A query
is a function that maps a LTS $\mathcal{T} = (Q, \Sigma, P, \Delta, \nu)$ to a subset of $Q$.

\begin{definition}
    \label{definition:query}
    A function $\mathcal{Q} : \mathscr{T} \rightarrow \mathscr{Q}, \mathcal{T} \mapsto Q^\mathcal{T}$ is called a
    \emph{query},
    where
    \begin{compactitem}
        \item $\mathscr{T}$ is the set of all LTS,
        \item $\mathscr{Q}$ is the set of all sets of states and
        \item $\mathcal{T} = (Q^\mathcal{T}, \Sigma^\mathcal{T}, P^\mathcal{T}, \Delta^\mathcal{T}, \nu^\mathcal{T})
        \in \mathscr{T}$ is a LTS.
    \end{compactitem}
\end{definition}

The queries defined in Definition~\ref{definition:query} can be categorized. Here we are interested in two of these
categories. The first category is called \textit{bisimulation invariant}. This category describes those queries that
can't distinguish bisimilar states. In~\cite{otto1999bisimulation} this property is defined over so called
\textit{Kripke structures}. A Kripke structure
is a transition system. Remark, that transition systems  have only one type of actions. This means the edges of the
graph haven't labels.

\begin{definition}
    \label{definition:bisimulationInvariant}
    Let $\mathcal{T}$, $\mathcal{T}' \in \mathscr{T}$ be two LTSs with $\mathcal{T} = (Q, \Sigma, P, \Delta, \nu)$
    and $\mathcal{T}' = (Q', \Sigma', P', \Delta', \nu')$. Furthermore, let be $q \in Q$ and $q' \in Q'$.

    A query $\mathcal{Q}$ is called \emph{bisimulation invariant} if $(\mathcal{T}, q) \sim (\mathcal{T}', q')$
    implies that
    $q \in \mathcal{Q}(\mathcal{T})$ iff $q' \in \mathcal{Q}(\mathcal{T'})$.
\end{definition}

The second category of a query tells us which complexity class a query belongs to.

\begin{definition}
    \label{definition:queryBelongsToComplexityClass}
    Let $\mathcal{T} \in \mathscr{T}$ be a LTS with $\mathcal{T} = (Q, \Sigma, P, \Delta, \nu)$ and $q \in Q$.

    A query $\mathcal{Q}$ belongs to a complexity class $\mathcal{C}$ if there is an algorithm in $\mathcal{C}$ for
    deciding on input $(\mathcal{T}, q)$ whether $q \in \mathcal{Q}(\mathcal{T})$.
\end{definition}

This definition leads us to the next chapter and the definitions for descriptive complexity.

%%
%% Author: Davidov
%% 16.05.2018
%%

\section{Fixpoints}\label{sec:fixpoints}

To define the polyadic higher-order fixpoint logic and the higher-order logic with least and partial fixpoints, we want to consider in this section fixpoints in general. The first fixpoint we consider is the least fixpoint.

\begin{definition}
   Let $F\colon A \rightarrow A$ be an operator on a finite set $A$, then $x \in A$
   is called a \emph{fixpoint} of $F$ if $F(x) = x$. Let $x$ be a fixpoint of $F$ and $\sqsubseteq$ an partial order on $A$, then $x$ is called the \emph{least
   fixpoint} of $F$, abbreviated as $\mathit{LFP}$($F$), if all other fixpoints $y$ of $F$ are bigger with respect to $sqsubset$ as $x$ i.e. $x
   \sqsubseteq y$ for all fixpoints $y$ of $F$. A fixpoint $x$ is called the \emph{greatest fixpoint} if $y \sqsubseteq x$ for all fixpoints $y$ of $F$.
\end{definition}

From Knaster-Tarski~\cite{tarski1955lattice} we know that if an operator $F\colon A \rightarrow 
A$ is monotone the least and greatest fixpoints of $F$ exists. $F$ is monotone iff for all $x, y
 \in A$ if $x \sqsubseteq y$ then $F(x) \sqsubseteq F(y)$ holds.

\begin{example}
    \label{example:lfp} Let $\mathcal{T} = (Q, \Sigma, P, \Delta, v)$ be an LTS and $F: Q^2 \rightarrow Q^2$ an operator on $Q^2$ defined as 
\begin{align*}
    F(X) =\, &\{(q, p) \in Q^2 \mid v(q) \neq v(p)\}\, \cup \\&
    \{(q,p) \in Q^2 \mid \text{it exists } a\in\Sigma \text{ with } q\overset{a}{\rightarrow} q' \text{ such that for all } p' \in Q \\&\text{ it holds } (p, a, p')\not\in\Delta\}\,\cup \\&\{(q,p) \in Q^2 \mid  \text{ it exists } a\in\Sigma \text{ with } p\overset{a}{\rightarrow} p' \text{ such that for all } q' \in Q \\&\text{ it holds } (q, a, q')\not\in\Delta\}\,\cup \\& \{ (q, p) \in Q^2 \mid \text{ there exists } (q', p') \in X \text{ such that there exists } a\in\Sigma \\&\text{ with } q\overset{a}{\rightarrow} q' \text{ and } p\overset{a}{\rightarrow} p' \}
\end{align*}   
Then $LFP(F)$ represents all those pairs of states $(q, p)$ such that $q\not\sim p$.
\end{example}

Next, we define the partial fixpoint. Since the
$\mathit{LFP}$ restricts the operator to be monotone to exist, the partial fixpoint need not any
restriction on the operator.

\begin{definition}
    Let $F\colon A \rightarrow A$ be an operator on a finite set $A$, then the \emph{partial
    fixpoint} of $F$, abbreviated as $\mathit{PFP}$($F$), is defined as follows:
    \[\mathit{PFP}(F)\coloneqq\begin{cases}
               F^{i+1}(\emptyset)=F^i(\emptyset),  & \text{if such } i \in \{0,\dots,|A|\} \text{ exists}\\
               \emptyset, & \text{otherwise,}
    \end{cases}\]
    where $F^0(\emptyset) = \emptyset$, $F^1(\emptyset) = F(\emptyset)$, $F^2(\emptyset) = F(F(\emptyset))$, and so on.
\end{definition}

Note, that for monotone $F$ holds $\mathit{PFP}(F)$ equals $\mathit{LFP}(F)$. 

\begin{example}
\label{example:pfp}
Let $F$ be the operator of Example~\ref{example:lfp}. Because $F$ is obviously monotone it holds that $PFP(F)$ also represents all those pairs of states $(q, p)$ such that $q\not\sim p$.
\end{example}

%%
%% Author: Davidov
%% 16.05.2018
%%

\subsection{Polyadic Higher Order Fixpoint Logic}\label{subsec:polyadicHigherOrderFixpointLogic}

In this section, we present a logic with name Polyadic Higher Order Fixpoint Logic, abbreviated with PHFL, that was
introduced by M. Lange and E. Lozes in~\cite{lange2014capturing}. It is defined over LTS (see
Definition~\ref{definition:lts}) and extends the polyadic modal $\mu$-calculus~\cite{otto1999bisimulation} with
higher order fixpoints like M. Viswanathan and R. Viswanathan it did with monadic case of modal
$\mu$-calculus~\cite{kozen1983results} in~\cite{viswanathan2004higher}. The logic of M. Viswanathan and R
.Viswanathan with name higher order fixed point logic is a combination of propositional logic, modality operators and
a simply typed $\lambda$-calculus with fixed point operators. The higher order cases of PHFL and a restriction called
tail-recursive for this higher order cases we are interested to compare with the in
Chapter~\ref{subsec:descriptiveComplexity} introduced complexity classes \exptime{$k$} and \expspace{$k$}.

\subsubsection{PHFL Types}

Before defining formulas of PHFL we need to introduce the PHFL types. These definitions are guided
by~\cite{viswanathan2004higher} and~\cite{lange2014capturing}.

\begin{definition}
    The syntax of \emph{PHFL types} are given by the grammar
    \[\sigma, \tau ::= \bullet \mid \sigma^\nu \rightarrow \tau,\]
    where $\nu$ is called \textit{variance}. The \emph{variances} of PHFL are defined by the grammar
    \[\nu ::= + \mid - \mid 0.\]
\end{definition}

All types will be interpreted as a partially ordered sets. Partial orders are relations that are reflexiv, transitiv
and antisymmetric. Let $\mathcal{A} = (A, \leq_A)$ and $\mathcal{B} = (B, \leq_B)$ be two partial orders then
$\mathcal{A} \rightarrow \mathcal{B}$ is the partial order of monotone functions ordered pointwise.

\[\mathcal{A} \rightarrow \mathcal{B} = \{f:A\rightarrow B \mid \forall x,y \in A. x\leq_A y\Rightarrow f(x)\leq_B f(y)\}\]

\begin{definition}
    Let $\mathcal{T} = (Q, \Sigma, P, \Delta, \nu)$ be a LTS, then $\mathcal{T}\llbracket\tau\rrbracket$ the semantics
    of type $\tau$ is defined by $\tau$ as follows:
        \[\mathcal{T}\llbracket\tau\rrbracket=
        \begin{cases}
            (\mathcal{P}(Q^d), \subseteq),  & \text{if }\tau = \bullet\\
            (\mathcal{T}\llbracket\sigma_1\rrbracket)^\nu \rightarrow \mathcal{T}\llbracket\sigma_2\rrbracket, &
            \text{if }\tau = \sigma_1^\nu\rightarrow \sigma_2
        \end{cases},\]
    where for any partial order $\mathcal{A} = (A, \leq_A)$, $\mathcal{A}^\nu = (A, \leq_A^\nu)$ is a partial order
    with $\leq_A^+ = \leq_A$, $\leq_A^- = \{(a, b) \mid (b, a) \in \leq_A\}$ and $\leq_A^0 = \leq_A^+ \cap \leq_A^-$.
\end{definition}

The partial orders $\mathcal{T}\llbracket\tau\rrbracket$ for any PHFL type $\tau$ are complete lattices. That means we
have meets and joins, denoted by $\sqcap_{\mathcal{T}\llbracket\tau\rrbracket}$ and
$\sqcup_{\mathcal{T}\llbracket\tau\rrbracket}$, and least and greatest elements, denoted by
$\bot_{\mathcal{T}\llbracket\tau\rrbracket}$ and $\top_{\mathcal{T}\llbracket\tau\rrbracket}$ for any subset of
$\mathcal{T}\llbracket\tau\rrbracket$. This ensures that the least fixpoint over all monotone PHFL types
exist~\cite{tarski1955lattice}. See Chapter~\ref{chapter:ho_plus_lfp} for further information about fixpoints.

\begin{definition}
    The \emph{arity} $ma(\tau)$ and the \emph{order} $ord(\tau)$ of a PHFL type $\tau$ are defined inductively on
    $\tau$ as follows:
\[ma(\tau)=
\begin{cases}
    1, & \text{if }\tau = \bullet\\
    max(\{n\} \cup \{ma(\tau_i)\mid1,\dots,n\}), &
    \text{if }\tau = \tau_1\rightarrow\dots\rightarrow\tau_n\rightarrow\bullet
\end{cases}\]
\[ord(\tau)=
\begin{cases}
    0, & \text{if }\tau = \bullet\\
    max(\{1 + ord(\sigma_1), ord(\sigma_2)\}), & \text{if }\tau = \sigma_1 \rightarrow \sigma_2
\end{cases}\]
\end{definition}

Next, we want to define the syntax of PHFL formulas.

\subsubsection{PHFL Syntax}

\begin{figure}
    \caption{Derivation Rules for PHFL formulas.}
    \label{figure:phfl-typing-rules}
    \begin{mathpar}
        \Gamma \vdash \top \colon \bullet \and
        \Gamma \vdash p_i \colon \bullet \and
        \inferrule{\Gamma \vdash \Phi \colon \bullet}{\Gamma \vdash \langle a \rangle_i \Phi \colon \bullet} \and
        \inferrule{\Gamma \vdash \Phi \colon \bullet}{\Gamma \vdash \{\emph{i} \leftarrow \emph{j}\}\Phi \colon
        \bullet} \and
        \inferrule{\Gamma^-\vdash\Phi\colon \tau}{\Gamma \vdash \neg \Phi \colon \tau} \and
        \inferrule{\Gamma\vdash\Phi \colon \tau \\ \Gamma\vdash\Psi\colon \tau}{\Gamma \vdash \Phi \vee
        \Psi \colon  \tau} \and
        \inferrule{\nu \in \{+, 0\} }{\Gamma, X^\nu \colon\tau \vdash X\colon\tau} \and
        \inferrule{\Gamma,X^\nu\colon\sigma\vdash \Phi\colon\tau}{\Gamma\vdash \lambda X^\nu \colon \tau
        .\Phi\colon\sigma^\nu\rightarrow\tau} \and
        \inferrule{\Gamma,X^+ \colon \tau \vdash \Phi\colon\tau}{\Gamma \vdash \mu X \colon\tau. \Phi\colon\tau} \and
        \inferrule{\Gamma\vdash \Phi\colon\sigma^+ \rightarrow \tau \\ \Gamma\vdash\Psi\colon\sigma}{\Gamma \vdash
        \Phi\Psi \colon \tau} \and
        \inferrule{\Gamma\vdash \Phi\colon\sigma^- \rightarrow \tau \\ \Gamma^-\vdash\Psi\colon\sigma}{\Gamma \vdash
        \Phi\Psi \colon \tau} \and
        \inferrule{\Gamma\vdash \Phi\colon\sigma^0 \rightarrow \tau \\ \Gamma \vdash \Psi\colon\sigma \\ \Gamma^-
        \vdash\Psi\colon\sigma}{\Gamma \vdash \Phi\Psi \colon \tau} \and
    \end{mathpar}
\end{figure}

\begin{definition}
    Let $P$ a set of propositions, $\Sigma$ a set of actions and $X, Y, \dots$ a finite set
    of variables, then
    \emph{PHFL formulas} $\Phi, \Psi,\dots$ are defined by the grammar
    \[\Phi,\Psi::=\top \mid p_i \mid \Phi \vee \Psi \mid \neg \Phi \mid \langle a \rangle_i \Phi \mid \{\emph{i}
    \leftarrow \emph{j}\} \Phi \mid X \mid \lambda X^\nu\colon\tau.\Phi \mid \Phi \Psi \mid \mu X\colon\tau.\Phi,
    where\]
    \begin{compactitem}
        \item $\emph{i} = (i_1,\dots, i_n)$, $\emph{j} = (j_1,\dots,j_n)$, $i, j, i_1,\dots,i_n, j_1, \dots j_n, n\in
        \mathbb{N}$,
        \item $\nu$ is an arbitrary variance,
        \item $\tau$ is an arbitrary type,
        \item $p \in P$ is an arbitrary property and
        \item $a \in \Sigma$ is an arbitrary action.
    \end{compactitem}
\end{definition}

Conveniently, we use some other further standard notations like $\Phi \wedge \Psi$, $[a]_i\Phi$, $\nu
X \colon \tau.\Phi$ or $\Phi \Leftrightarrow \Psi$. Remark, that this logic is defined over LTS. The formulas are
often interpreted as a game played by two players  moving pebbles along the transitions of an LTS. So, $p_i$ can be
interpreted as, the position of the $i$-th pebble fulfills property $p$. $\langle a \rangle_i \Phi$ means, move the
$i$-th pebble along an $a$-transition and check if there holds $\Phi$. With the formula $\{\emph{i}\leftarrow
\emph{j}\} \Phi$ is mentioned, that all pebbles from tuple $\emph{j}$ are moved to the tuple $\emph{i}$ and after
this $\Phi$ have to be fulfilled. It is important that $\emph{i}$ and $\emph{j}$ have the same size. $\lambda
X^\nu\colon\tau.\Phi$ is interpreted as a function that expects arguments of
$\mathcal{T}\llbracket\tau^\nu\rrbracket$. We can see, that the formulas can also have types. For this, we have
ensure that a formula is well-typed.

\begin{definition}
    Let $X_1, \dots, X_n$ variables, $\Phi$ a PHFL formula, $\nu_1, \dots, \nu_n$ variances and $\tau, \tau_1, \dots,
    \tau_n$ types, then $\Gamma = X_1^{\nu_1}\colon \tau_1, \dots X_n^{\nu_n} \colon \tau_n$ is
    called a \emph{type environment} and $\Gamma \vdash \Phi:\tau$
    is called a \emph{type judgement}. Let $\Gamma^- = X_1^{\nu_1^-}\colon \tau_1, \dots
    X_n^{\nu_n^-} \colon \tau_n$ be a type environment then $\Gamma^- = X_1^{\nu_1^-}\colon \tau_1, \dots
    X_n^{\nu_n^-} \colon \tau_n$, where $-^- = +$, $+^- = -$ and $0^- = 0$.
\end{definition}

A type judgment is called \textit{derivable} if it generates a derivation tree respectively to the rules of
Figure~\ref{figure:phfl-typing-rules}. A formula $\Phi$ is called \textit{well-typed} if the type
judgement $\vdash \Phi:\tau$ is derivable for some type $\tau$.



\subsubsection{PHFL Semantics}

To define the semantics of PHFL formulas we need mapping $\eta$ that associates each variable to an element of its
type semantics, i.e. $\eta(X) \in \mathcal{T}\llbracket\tau\rrbracket$ for $X$ of type $\tau$. Let $\Phi$ be a
well-typed formula of type $\tau$ and $eta$ a variable mapping, then the semantics $\llbracket\Gamma \vdash \Phi
\colon \tau \rrbracket(\eta)$ is defined inductively on $\Phi$ which maps to an element of
$\mathcal{T}\llbracket\tau\rrbracket$ as explained in Figure~\ref{figure:phfl-semantics}.
Remark, that $\eta[X \mapsto \mathcal{X}]$ is a mapping $\eta'$ that is equal to $\eta$ but $\eta'(X) = \mathcal{X}$ and
$\emph{q} \overset{a, i}{\rightarrow} \emph{q'}$ stands for $q_i \overset{a}{\rightarrow} {q_i}'$ and for all $j \neq i$
holds
$q_j = {q_j}'$.

\begin{figure}
    \caption{Semantics of PHFL formulas.}
    \label{figure:phfl-semantics}
    \begin{align*}
        \llbracket \Gamma \vdash \top \colon \bullet \rrbracket(\eta) =\text{ }& Q^d\\
        \llbracket \Gamma \vdash \langle a \rangle_i \Phi \colon \bullet \rrbracket(\eta) =\text{ }& \{\emph{q} \in Q^d \mid
        \exists \emph{q'} \in \llbracket \Gamma \vdash \Phi \colon \bullet \rrbracket . \emph{q}
        \overset{a, i}{\rightarrow} \emph{q'}\}\\
        \llbracket \Gamma \vdash \Phi \vee \Psi \colon \tau \rrbracket(\eta) =\text{ }& \llbracket \Gamma \vdash \Phi
        \colon \tau \rrbracket (\eta) \sqcup_\tau \llbracket \Gamma \vdash \Psi \colon \tau \rrbracket (\eta)\\
        \llbracket \Gamma \vdash \neg \Phi \colon \tau \rrbracket(\eta) =\text{ }& Q^d \setminus \llbracket \Gamma^- \vdash \Phi
        \colon \bullet \rrbracket (\eta)\\
        \llbracket \Gamma \vdash \{\emph{i} \leftarrow \emph{j}\} \Phi \colon \bullet \rrbracket(\eta) =\text{ }&
        \{\{\emph{i} \leftarrow \emph{j}\}(\emph{q}) \mid \emph{q} \in \llbracket \Gamma \vdash \Phi \colon \bullet
        \rrbracket (\eta)\}\\
        \llbracket \Gamma, X \colon \tau \vdash X \colon \tau \rrbracket(\eta) =\text{ }& \eta(X)\\
        \llbracket \Gamma \vdash \mu X \colon \tau .\Phi \colon \tau \rrbracket(\eta) =\text{ }&
        \sqcap_{\mathcal{T}\llbracket\tau\rrbracket} \{\mathcal{X} \in \mathcal{T}\llbracket \tau \rrbracket \mid \\
        &\llbracket \Gamma, X^+ : \tau \vdash \Phi \colon \tau \rrbracket(\eta[X \mapsto \mathcal{X}])
        \leq_{\mathcal{T}\llbracket \tau \rrbracket} \mathcal{X}\}\\
        %\llbracket         \Gamma
 %       \vdash
  %      \lambda X^+ \colon \tau.\Phi \colon \tau \rrbracket (\eta)\\
        \llbracket \Gamma \vdash \lambda X^\nu \colon \sigma. \Phi \colon \sigma^\nu \rightarrow \tau \rrbracket
        (\eta) =\text{ }& F \in \mathcal{T}\llbracket \sigma^\nu \rightarrow \tau \rrbracket \text{ s.t. }\\
        &\forall \mathcal{X} \in
        \mathcal{T}\llbracket \sigma \rrbracket. F(\mathcal{X}) = \llbracket \Gamma, X^\nu \colon \sigma \vdash \Phi \colon \tau
        \rrbracket (\eta[X \mapsto \mathcal{X}])\\
        \llbracket \Gamma \vdash \Phi \Psi \colon \tau \rrbracket(\eta) =\text{ }& \llbracket \Gamma \vdash \Phi \colon \sigma
        ^\nu \rightarrow \tau \rrbracket (\eta)(\llbracket \Gamma \vdash \Psi \colon \sigma \rrbracket (\eta)))
    \end{align*}
\end{figure}

In this thesis we are interested in PHFL formulas that have a specific order. For this, a formula $\Phi$ have order $k$
if $k = max(\{ord(\tau)\mid \mu X \colon \tau. \Psi \text{ is a subformula of } \Phi\})$. The set of formulas that have
order at most $k$ is denoted by PHFL$^k$.

\subsubsection{tail-recursive PHFL}

Next, we want to define a restriction on PHFL formulas. This restriction is called tail-recursive and ensures that
some combinations of subformulas don't appear in an PHFL formula. For this, let the logical connective
$\wedge$ and the modality operators $[a]_i$ be further primitives of PHFL formula syntax. The restriction of tail
recursive formulas is in more detail, that least fixpoint variables do not occur freely under the new operators
$\wedge$ and $[a]_i$, nor in $\Psi$ of formula $\Phi\Psi$.

\begin{definition}
    A closed PHFL formula $\Phi$ is called \emph{tail-recursive} if $\emptyset \vdash tail(\Phi, \emptyset)$ is
    derivable via the rules in Figure~\ref{figure:phfl-tail-recursive}.
\end{definition}

\begin{figure}
    \caption{Derivation Rules for PHFL formulas that shall be tail-recursive.}
    \label{figure:phfl-tail-recursive}
    \begin{mathpar}
        \bar{Y} \vdash tail(p_i, \bar{X}) \and
        \inferrule{X \in \bar{X} \cup \bar{Y}}{\bar{Y} \vdash tail(X, \bar{X})} \and
        \inferrule{\bar{Y} \vdash tail(\Phi, \emptyset)}{\bar{Y} \vdash tail(\neg \Phi, \bar{X})} \and
        \inferrule{\bar{Y} \vdash tail(\Phi, \bar{X})}{\bar{Y} \vdash tail(\{\emph{i} \leftarrow \emph{j}\} \Phi,
        \bar{X})} \and
        \inferrule{\bar{Y} \vdash tail(\Phi, \bar{X}) \\ \bar{Y} \vdash tail(\Psi, \bar{X})}{\bar{Y} \vdash tail
        (\Phi \vee \Psi, \bar{X})} \and
        \inferrule{\bar{Y} \vdash tail(\Phi, \bar{X})}{\bar{Y} \vdash tail(\langle a \rangle_i \Phi, \bar{X})} \and
        \inferrule{\bar{Y} \vdash tail(\Phi, \emptyset)}{\bar{Y} \vdash tail([a]_i \Phi, \bar{X})} \and
        \inferrule{\bar{Y} \vdash tail(\Phi, \emptyset) \\ \bar{Y} \vdash tail(\Psi, \bar{X})}{\bar{Y} \vdash tail
        (\Phi \wedge \Psi, \bar{X})} \and
        \inferrule{\bar{Y} \vdash tail(\Phi, \bar{X}) \\ \bar{Y} \vdash tail(\Psi, \emptyset)}{\bar{Y} \vdash tail
        (\Phi \Psi, \bar{X})} \and
        \inferrule{\bar{Y} \cup \{Z\} \vdash tail(\Phi, \bar{X})}{\bar{Y} \vdash tail(\lambda Z^\nu \colon \tau . \Phi,
        \bar{X})} \and
        \inferrule{\bar{Y} \vdash tail(\Phi, \bar{X} \cup \{Z\})}{\bar{Y} \vdash tail(\mu Z \colon \tau . \Phi,
        \bar{X})}
    \end{mathpar}
\end{figure}

The set of all tail-recursive PHFL formulas that have order at most $k$ is denoted by PHFL$^k_{tail}$.

%%
%% Author: Davidov
%% 27.04.2018
%%

\subsection{Descriptive Complexity}\label{subsec:descriptiveComplexity}

One way to describe complexity classes is with the help of \textit{turing machines}~\cite{hopcroft1994einfuehrung}.

\begin{definition}
    The seven-tuple $M = (Q, \Sigma, \Gamma, \delta, q_0, \Box, F)$ is called a \emph{turing machine} (\emph{TM}),
    where
    \begin{compactitem}
        \item $Q$ is the finite set of states,
        \item $\Sigma$ is the input alphabet,
        \item $\Gamma$ is the working alphabet with $\Sigma \subset \Gamma$,
        \item $\delta \subseteq (Q \setminus F) \times \Gamma \times Q \times \Gamma \times \{L, R, N\}$ is the
        transition relation,
        \item $q_0 \in Q$ is the initial state,
        \item $\Box \in \Gamma \setminus \Sigma$ is the blank symbol and
        \item $F \in Q$ is the set of final states.
    \end{compactitem}
\end{definition}

Known from computational complexity theory~\cite{papadimitriou1994complexity}, the time and space classes
can be defined by functions. These functions have as input a natural number that represents the length of the input of a
turing machine. In case of time classes the output of the function is the maximal number of configuration steps for
all inputs of the given length. In case of space classes the output is based on the biggest configuration while
computing an input of the given length.

\begin{definition}
    Let $f: \mathbb{N} \rightarrow \mathbb{N}$ be a polynomial function, then $\emph{exp}: \mathbb{N} \times \mathbb{N}
    \rightarrow \mathbb{N}$ is a function defined inductive as follows:
    \begin{compactitem}
        \item $exp(0, f(n)) = f(n)$,
        \item $exp(i, f(n)) = 2^{exp(i - 1, f(n))}$ for $i \geq 1$.
    \end{compactitem}
\end{definition}

With the help of the function $exp$, we are now able to define the complexity classes $k$-EXPTIME and $k$-EXPSPACE for
all $k \geq 1$.

\begin{definition}
    Let $k \in \mathbb{N} \setminus \{0\}$ and $f: \mathbb{N} \rightarrow \mathbb{N}$ be a polynomial function then
    $k$-EXPTIME $=$ TIME($exp(k, f(n))$) and $k$-EXPSPACE $=$ SPACE($exp(k, f(n))$).
\end{definition}

Remark that TIME is the maximal number of configuration steps and SPACE the biggest configuration of a turing machine
while computing an input.

From definition~\ref{definition:bisimulationInvariant}, definition~\ref{definition:queryBelongsToComplexityClass}
and the complexity classes $k$-EXPTIME and $k$-EXPSPACE follow the queries that we want to investigate.

\begin{definition}
    \label{definition:kExptimekExpspace}
    \exptime{$k$} are the bisimulation invariant queries that belong to complexity class $k$-EXPTIME and
    \expspace{$k$} are the bisimulation invariant queries that belongs to complexity class $k$-EXSPACE, where $k \geq
    1$.
\end{definition}

The main aim of \emph{descriptive complexity} is to describe the complexity classes known from the
computational complexity theory with logics. While the computational complexity theory distinguishes time and space
classes, the descriptive complexity theory characterizes classes without a reference to automaton models or
space and time bounds.

The first known result in the area of descriptive complexity comes from Fagin. In 1974 he showed that the complexity
class NP coincides with $\exists SO$~\cite{fagin1974generalized}, the existential fragment of second-order logic.

In the next chapter we introduce the polyadic higher order fixed point logic and a restriction called tail-recursive
devised by Lange and Lozes in~\cite{lange2014capturing}. We want to compare the in
definition~\ref{definition:kExptimekExpspace} presented queries with those logics to make a further contribution in
the theory of descriptive complexity.

%%
%% Author: Davidov
%% 16.05.2018
%%


\section{Higher Order Logic}\label{sec:higherOrderLogic}

For comparing the complexity classes with PHFL, we have to detour over combinations of extensions of FO. The first well
known extension is called Higher Order Logic~\cite{vanBenthem2001higher}, abbreviated with HO. In HO we
increase the expressive power of FO by allowing relation variables of any order. For this, we have to define the
types of higher order variables.

\subsection{HO Syntax}\label{subsec:hoSyntax}

\begin{definition}
    \emph{HO types} are given by the grammar
    \[ \tau \Coloneqq \odot \mid (\tau, \dots, \tau) \]
\end{definition}

The HO type of individuals is $\tau = \odot$. These objects have the \textit{order} $1$. The HO type $\tau = (\tau',
\dots, \tau')$ is that of relations between objects of HO type $\tau'$ and has the order $1 + order(\tau')$. For
each HO type we have a countably infinite set of variables. Furthermore, let $\sigma$ be a signature over a
relational vocabulary i.e. $\sigma$ just contains relation symbols.

\begin{definition}
    Let $\mathcal{V} = \{X_1, X_2, \dots \}$ a countable infinite set of variables $\sigma$ a signature, then \emph{HO
    formulas} $\Phi, \Psi, \dots$ over $\sigma$ are defined by the grammar
    \[\Phi, \Psi \Coloneqq R(x_1, \dots, x_n) \mid Y(y_1, \dots, y_n) \mid \neg \Phi \mid \Phi \vee \Psi \mid \exists
    (X \colon \tau).\,\Phi\]
    where
    \begin{compactitem}
        \item $R \in \sigma$ is a relation of individuals with arity $n$ and $x_1, \dots, x_n \in \mathcal{V}$ of HO
        type $\odot$,
        \item $Y \in \mathcal{V}$ of HO type $(\tau', \dots, \tau')$ and $y_1, \dots, y_n \in \mathcal{V}$ of HO type
        $\tau'$ and
        \item $X \in \mathcal{V}$ of HO type $\tau$.
    \end{compactitem}
\end{definition}

\subsection{HO Semantics}\label{subsec:hoSemantics}

The first step in direction of the semantics of HO formulas is to interpret the universes of the different HO types.

\begin{definition}
    Let $\mathcal{A}$ be a $\sigma$-structure over universe $\mathcal{U}$ then the universes of the
    HO types are defined inductively as follows:
    \begin{compactitem}
        \item $D_\odot(\mathcal{U}) = \mathcal{U}$,
        \item $D_{(\tau, \dots, \tau)}(\mathcal{U}) = \mathcal{P}(D_{\tau}(\mathcal{U})^n)$
    \end{compactitem}
\end{definition}

Moreover, $\alpha$ is a function that assigns all variables to an element of the appropriate universe, i.e. if
variable $X$ is of HO type $\tau$, then $\alpha(X) \in D_{\tau}(\mathcal{U})$. With $\alpha[X \rightarrow \mathcal{X}]$,
where $\mathcal{X} \in D_\tau(\mathcal{U})$ and $X$ of HO type $\tau$, we mean the variable assignment $\alpha'$,
where $\alpha'(X) = \mathcal{X}$ and $\alpha'(Y) = \alpha(Y)$ for all $Y \neq X$.

\begin{definition}
    Let $\mathcal{A}$ be a $\sigma$-structure and $\alpha$ a variable assignment over universe $\mathcal{U}$. The
    semantics of a HO formula is defined inductively as follows:
    \begin{compactitem}
        \item $\mathcal{A}, \alpha \models R(x_1, \dots x_n)$ iff $(\alpha(x_1), \dots
        \alpha(x_n)) \in R^{\mathcal{A}}$,
        \item $\mathcal{A}, \alpha \models Y(y_1, \dots y_n)$ iff $(\alpha(y_1), \dots
        \alpha(y_n)) \in \alpha(Y)$,
        \item $\mathcal{A}, \alpha \models \neg\varphi$ iff $\mathcal{A}, \alpha\not\models\varphi$,
        \item $\mathcal{A}, \alpha \models \varphi \vee \psi$ iff $\mathcal{A}, \alpha\models\varphi$ or $\mathcal{A},
        \alpha\models\psi$,
        \item $\mathcal{A}, \alpha \models \exists (X\colon\tau).\,\varphi$ iff there exists $\mathcal{X} \in D_{\tau}
        (\mathcal{U})$ with $\mathcal{A}, \alpha[X \rightarrow \mathcal{X}] \models \varphi$
        \end{compactitem}
\end{definition}

We can categorize the formulas by the order of all occurring variables. With HO$^k$ we mean the set of all those
formulas whose variables have order less or equal $k$.

\begin{example}{\cite{vanBenthem2001higher}}
    \label{example:ho}
    The following formula $\varphi$ describes Peano's induction axiom. Peano's induction axiom reveals that every set
    of natural numbers, which contains $0$ and is also closed under immediate successors, contains all natural numbers.
    \[\varphi = \forall (P\colon(\odot)).\,(P(0) \wedge \forall (I\colon\odot).\,(P(I) \Rightarrow P(I + 1)) \Rightarrow
    \forall (N \colon\odot).\,(P(N)))\]
    Note, that $\varphi$ lies in $\mathit{HO}^2$, i.e. it is a second order logic formula.
\end{example}

Similarly to Definition~\ref{definition:query_associated_to_formula} we define $r$-adic queries that are associated to HO formulas with $f$ free first-order variables. 

\begin{definition}
\label{definition:query_associated_to_formula_ho}
    Given a signature $\sigma$ and a closed HO
    formula $\Phi$ with free first-order variables $x_1, \dots, x_f$ we call a $r$-adic query $\mathcal{Q}^r_\Phi$ associated to $\Phi$ if there is for all $\sigma$-structures
    $\mathcal{A}$, all variable mappings $\eta$ and all $(q_1, \dots, q_r) \in {\mathcal{Q}^r_\Phi}^\mathcal{A}$ a $\mathcal{A}, (x_1, \dots, x_f) \models
     \Phi$ such that $q_i = \eta(x_i)$ for all $i \in
    \{1, \dots, min(\{r, f\})\}$.
\end{definition}

\subsection{HO + LFP}
\label{subsec:hoPlusLfp}

Another possibility to extend FO is to add operators that are not expressible in FO. Here, we are interested in two
of them, the least fixpoint and the partial fixpoint operator. Instead of defining the operators for FO we are
here interested to define this operators for HO. First, we regard the least fixpoint operator.
Like in~\cite{freireMartins2011descriptive} we want to define special operators that are working on HO type universes.

\begin{definition}
    Let $\sigma$ an arbitrary signature, $X$ a relation variable of HO type $\tau = (\tau', \dots, \tau')$,
    $\tau'$ an arbitrary HO type, $x_1, \dots x_n$ variables of HO type $\tau'$ and $\varphi(X, x_1, \dots, x_k)$ a
    formula over $\sigma$ with free variables $X, x_1, \dots, x_k$. For each $\sigma$-structure $\mathcal{A}$ with
    universe $\mathcal{U}$, $\varphi(A, a_1, \dots, a_k)$ induces the operator
    \begin{align*}
        F_\varphi^\mathcal{A}\colon\mathscr{P}(D_\tau(\mathcal{U})) &\longrightarrow \mathscr{P}(D_\tau(\mathcal{U}))\\
        A &\longmapsto F_\varphi^\mathcal{A}(A) \coloneqq \{(a_1, \dots, a_n) \mid \mathcal{A} \models \varphi(A, a_1,
        \dots, a_n)\}
    \end{align*}
    where $a_1 \dots, a_n \in D_{\tau'}(\mathcal{U})$.
\end{definition}

To make $F_\varphi^\mathcal{A}$ monotone we have to restrict $\varphi(X, x_1, \dots, x_k)$ in this way that variable
$X$ occurs in an even number of negations within of $\varphi$~\cite{freireMartins2011descriptive}. Those formulas
are called \textit{positive in} $X$. With this information we are able to define the least fixpoint operator for HO
formulas, denoted by HO($\mathit{LFP}$).

\begin{definition}
    Let $\sigma$ be a signature. The set of \emph{HO($\mathit{LFP}$) formulas} enhances the set of HO formulas with the
    following formation rule:
    \begin{compactitem}
        \item $[\mathit{LFP}\;\varphi(X, x_1, \dots, x_n)](v_1, \dots, v_n)$ is a HO
        ($\mathit{LFP}$) formula over $\sigma$ with free variables $v_1, \dots, v_n$ iff $\varphi(X, x_1, \dots, x_n)
        $ is a HO($\mathit{LFP}$) formula with free variables $X, x_1, \dots, x_n$, if $\varphi$ is positive in
        $X$, $X$ has HO type $\tau = (\tau', \dots, \tau')$, $x_1, v_1, \dots, x_n, v_n$ have HO type $\tau'$.
    \end{compactitem}
\end{definition}

As in HO, with HO($\mathit{LFP}$)$^k$ we mean the set of all those HO($\mathit{LFP}$) formulas whose variables have
order less or equal $k$.

\begin{definition}
    Let $\mathcal{A}$ be a $\sigma$-structure and $\alpha$ a variable assignment over universe $\mathcal{U}$. The
    semantics of a HO($\mathit{LFP}$) formula extends that of HO formulas with the following definition:
    \begin{compactitem}
        \item $\mathcal{A}, \alpha \models [\mathit{LFP}\;\varphi(X, x_1, \dots, x_n)](v_1, \dots,
        v_n)$ iff $(\alpha(v_1), \dots, \alpha(v_n)) \in \mathit{LFP}$ $(F_\varphi^\mathcal{A})$.
    \end{compactitem}
\end{definition}

\begin{example}{\cite{freireMartins2011descriptive}}
    \label{example:ho_lfp} This example describes the reachability of two
    vertices in a graph. For this, let $\sigma = \{E\colon(\odot, \odot)\}$ a signature where $E$ represents the
    edges in a graph. Then
    \[[LFP\;(E(x_1, x_2) \vee \exists z\colon\odot.\,(E(x_1, z) \wedge X(z, x_2)))](s, t)\]
    describes the reachability of vertex $t$ from vertex $s$. Note, that $[LFP\;($ $E(x_1, x_2) \vee
    \exists z\colon\odot.\,(E(x_1, z) \wedge X(z, x_2)))]$ describes the transitive closure of $E$. This formula is
    in $HO(LFP)^1$.
    %\[[LFP_{X,x}(\forall X:(\odot).\,(\forall Y:\odot.\,(X(Y) \vee Y = 0 \vee \exists Z:\odot.\,(X(Z) \wedge Y = (Z + 1)))
%))]\]
\end{example}

\subsection{HO + PFP}\label{subsec:ho+Pfp}

Next, we define the partial fixpoint operator for HO formulas~\cite{schewe2006fixpoint}. Since the
$\mathit{LFP}$ operator restricts formulas to be positive in a variable, the partial fixpoint operator has not any
restriction. With this knowledge we can define and
add the partial fixpoint operator to HO formulas, denoted as HO($\mathit{PFP}$).

\begin{definition}
    Let $\sigma$ be a signature. The set of \emph{HO($\mathit{PFP}$) formulas} enhances the set of HO formulas with the
    following formation rule:
    \begin{compactitem}
        \item $[\mathit{PFP}\;\varphi(X, x_1, \dots, x_n)](v_1, \dots, v_n)$ is a HO
        ($\mathit{PFP}$) formula over $\sigma$ with free variables $v_1, \dots, v_n$ iff $\varphi(X, x_1, \dots, x_n)
        $ is a HO($\mathit{PFP}$) formula with free variables $X, x_1, \dots, x_n$, where $X$ has HO type $\tau =
        (\tau', \dots, \tau')$, $x_1, v_1, \dots, x_n, v_n$ have HO type $\tau'$.
    \end{compactitem}
\end{definition}

HO($\mathit{PFP}$)$^k$ is the set of all those HO($\mathit{PFP}$) formulas whose variables have order less or equal $k$.

\begin{definition}
    Let $\mathcal{A}$ be a $\sigma$-structure and $\alpha$ a variable assignment over universe $\mathcal{U}$. The
    semantics of a HO($\mathit{PFP}$) formula extends that of HO formulas with the following definition:
    \begin{compactitem}
        \item $\mathcal{A}, \alpha \models [\mathit{PFP}\;\varphi(X, x_1, \dots, x_n)](v_1, \dots,
        v_n)$ iff $(\alpha(v_1), \dots, \alpha(v_n)) \in \mathit{PFP}$ $(F_\varphi^\mathcal{A})$.
    \end{compactitem}
\end{definition}

\begin{example}{\cite{abiteboul1995computing}}
    \begin{align*}
    [PFP\;(&G(x, y) \wedge \neg \exists z\colon\odot.\,(G(x, z) \wedge G(z, y)) \vee \\
    &\exists z\colon\odot.\,(\neg G(x, z)\wedge\neg G(z, x) \wedge \neg G(y, z) \wedge G(z, y)))](s, t)
    \end{align*}
    This formula describes a graph $G$, where all edges between $(s, t)$ will be removed, if there is a path of
    length $2$ between $s$ and $t$. If there is a vertex that is not directly connected to $s$ or $t$, an edge $(s,
    t)$ will be inserted. This process will be iterated while some changes occurs.
\end{example}
