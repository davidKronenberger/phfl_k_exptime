\section{Preliminaries}
\label{sec:preliminaries}
This chapter introduces all necessary definitions to prove that PHFL$^k = $~\exptime{$k$} and
PHFL$^{k-1}_{tail} =$~\expspace{$k$}. The notions are mainly from~\cite{immerman1999descriptive},
~\cite{papadimitriou1994complexity},~\cite{otto1999bisimulation} and~\cite{lange2014capturing}.

We assume that the reader is already familiar with basic notions of first order logic and computational complexity.

%%
%% Author: Davidov
%% 27.04.2018
%%

\subsection{Descriptive Complexity}\label{subsec:descriptiveComplexity}

One way to describe complexity classes is with the help of \textit{Turing Machines}~\cite{hopcroft1994einfuehrung}.

\begin{definition}
    The seven-tuple $M = (Q, \Sigma, \Gamma, \delta, q_0, \Box, F, R)$ is called a \emph{Deterministic Turing Machine}
    ($\mathit{DTM}$),
    where
    \begin{compactitem}
        \item $Q$ is the finite set of states,
        \item $\Sigma$ is the input alphabet,
        \item $\Gamma$ is the working alphabet with $\Sigma \subset \Gamma$,
        \item $\delta : (Q \setminus (F \cup R)) \times \Gamma \rightarrow Q \times \Gamma \times \{L, R, N\}$ is the
        transition function,
        \item $q_0 \in Q$ is the initial state,
        \item $\Box \in \Gamma \setminus \Sigma$ is the blank symbol,
        \item $F \subseteq Q$ is the set of accepting states and
        \item $R \subseteq Q$ is the set of rejecting states.
    \end{compactitem}
\end{definition}

\begin{example}
    \label{example:dtm}
    As an example for a $\mathit{DTM}$ let
    \[M = (\{q_0, q_f, q_r\}, \{a, b\}, \{a, b, \Box\}, \delta, q_0, \Box, \{q_f\}, \{q_r\})\]
    where $\delta(q_0, a)= (q_0, \Box, R)$, $\delta(q_0, b) = (q_r, b, L)$, $\delta(q_0, \Box) = (q_f, \Box, N)$.
    $M$ is a $\mathit{DTM}$ that accepts all input words that contain no symbol $b$, i.e. $L(M) = \{a\}^*$.
\end{example}

Configurations are snapshots of $\mathit{DTM}$s working on an input word. This includes the working tape, the current
state and the current position of the reading head. Formally, $C_i^M(w) = \Gamma^m \cdot Q \cdot (\Gamma^n | \Box)$
is called the $i$-th configuration of a $\mathit{DTM}$ $M = (Q, \Sigma, \Gamma, \delta, q_0, \Box, F)$ for input word
$w \in \Sigma^*$, where $m \geq 0$ and $n \geq 1$. In addition, $\Gamma^m$ represents the word of all symbols left
from the reading head position to the last symbol that is unequal to the blank symbol. $\Gamma^n$ represents the word
of all symbols right of the reading head position to the last symbol that is unequal to the blank symbol. If there
are no symbols unequal to the blank symbol right of the reading head position, the blank symbol $\Box$ of $M$ is used
instead of $\Gamma^n$.

\begin{definition}
    Let $C_i^M(w) = \gamma_1^i\dots\gamma_{m_i}^i{q_i}{\gamma_1^i}'\dots{\gamma_{n_i}^i}'$, $C_j^M(w) =
    \gamma_1^j\dots\gamma_{m_j}^j{q_j}{\gamma_1^j}'\dots{\gamma_{n_j}^j}'$ two configurations of a $\mathit{DTM}$ $M = (Q, \Sigma,
    \Gamma,
    \delta, q_0, \Box, F)$ for input word $w \in \Sigma^*$ with $i \neq j$. $C_j^M(w)$ is the next configuration
    of $C_i^M(w)$ written as $C_i^M(w) \rightarrow_M C_j^M(w)$ iff $j = i + 1$ and
    \begin{compactitem}
        \item $m_j = m_i - 1$, $\gamma_1^j = \gamma_1^i, \dots \gamma_{m_j}^j = \gamma_{m_j}^i$, $n_j = n_i + 1$,
        ${\gamma_1^j}' = \gamma_{m_i}^i, {\gamma_2^j}' = a, {\gamma_3^j}' = {\gamma_2^i}' \dots {\gamma_{n_j}^j}' =
        {\gamma_{{n_j}- 1}^i}'$ and $\delta(q_i, {\gamma_1^i}') = (q_j, a, L)$ or
        \item $m_j = m_i + 1$, $\gamma_1^j = \gamma_1^i, \dots \gamma_{m_j-1}^j = \gamma_{m_j-1}^i, \gamma_{m_j}^j
        = a$, $n_j = n_i - 1$, ${\gamma_1^j}' = {\gamma_2^i}', \dots {\gamma_{n_j}^j}' = {\gamma_{{n_j}+1}^i}'$ and
$\delta (q_i, {\gamma_1^i}') = (q_j, a, R)$ or
        \item $m_j = m_i$, $\gamma_1^j = \gamma_1^i, \dots \gamma_{m_j}^j = \gamma_{m_j}^i$, $n_j = n_i$, ${\gamma_1^j}'
= a, {\gamma_2^j}' = {\gamma_2^i}' \dots {\gamma_{n_j}^j}' = {\gamma_{n_j}^i}'$ and $\delta
        (q_i, {\gamma_1^i}') = (q_j, a, N)$.
    \end{compactitem}
    $C_i^M(w) \rightarrow_M C_j^M(w)$ is called a \emph{transition} of $M$ on $w$.
\end{definition}

The start configuration for an input word $w$ of a $\mathit{DTM}$ $M$ is $C_0^M(w) = q_0w$, where $q_0$ is the
initial state of $M$. A run of $\mathit{DTM}$ $M$ on input word $w$ is a sequence of transitions, $C_0^M(w)
\rightarrow_M C_1^M(w) \rightarrow_M C_2^M(w) \rightarrow_M \dots \rightarrow_M C_n^M(w)$, where $C_n^M(w)$ includes
either an accepting or rejecting state. A run is accepting if there is a $n \in \mathcal{N}$ such that $C_0^M(w)
\rightarrow_M C_1^M(w) \rightarrow_M \dots \rightarrow_M C_n^M(w)$ and $C_n^M(w)$ contains an accepting state of $M$.

\begin{example}
    \label{example:run_of_dtm}
    Let $M$ be the $\mathit{DTM}$ from Example~\ref{example:dtm} and $w_1 = aaba$, $w_2 = aaaa$ two input words. The
    run of $M$ on $w_1$ is
    \[C_0^M(w_1) = q_0aaba \rightarrow_M q_0aba \rightarrow_M q_0ba \rightarrow_M q_r\Box ba = C_3^M(w_1)\]
    and the run of $M$ on $w_2$ is
    \[C_0^M(w_2) = q_0aaaa \rightarrow_M q_0aaa \rightarrow_M q_0aa \rightarrow_M q_0a \rightarrow_M q_0\Box
    \rightarrow_M q_f\Box = C_5^M(w_2)\]
\end{example}

Remark, that it is possible to define $\mathit{DTM}$s that does not accept or reject any input word. For example, let
$M = (\{q_0\}, \{a\}, \{a, \Box\}, \delta, \emptyset, \emptyset)$, where $\delta(q_0, x) = (q_0, x, N)$ with $x \in
\{a, \Box\}$. $M$ is a DTM where any calculation of an input word $w$ looks as follows $q_0w \rightarrow_M q_0w
\rightarrow_M \dots$. It never reaches an accepting or rejecting state. In this thesis, we are only interested in
$\mathit{DTM}$s that reaches an accepting or rejecting state in finite time on any input word.

Known from computational complexity theory~\cite{papadimitriou1994complexity}, the time and space classes
can be defined by functions. These functions have as input a natural number that represents the length of an input
word of a $\mathit{DTM}$. In case of time classes the output of the functions depends on the number of configuration
steps. In case of space classes the output is based on the longest transition.

\begin{definition}
    Let $M$ be a $\mathit{DTM}$. $\mathit{TIME}(n):= max(\mathit{STEPS}(w)\mid |w| = n)$, where $\mathit{STEPS}(w)$
    is the number of transitions while $M$ computes $w$. $\mathit{SPACE}(n) := max(\mathit{STORAGE}(w)\mid |w| = n)$,
    where $\mathit{STORAGE}(w) := max(length(C_i^M)\mid i\in\{1, \dots m\})$ is the longest configuration while $M$
    computes $w$ and $C_0^M(w) \rightarrow_M C_1^M(w) \rightarrow_M \dots \rightarrow_M C_m^M(w)$ is a run of $M$ on
    $w$.
\end{definition}

\begin{example}
    
\end{example}

Now, it is possible to group the $\mathit{DTM}$s by functions. Those groups are our computational complexity classes.
In this thesis we are interested in exponential time classes and exponential space classes.

\begin{definition}
    Let $f: \mathbb{N} \rightarrow \mathbb{N}$ be a polynomial function, then $\emph{exp}: \mathbb{N} \times \mathbb{N}
    \rightarrow \mathbb{N}$ is a function defined inductive as follows:
    \begin{compactitem}
        \item $exp(0, f(n)) = f(n)$,
        \item $exp(i, f(n)) = 2^{exp(i - 1, f(n))}$ for $i \geq 1$.
    \end{compactitem}
\end{definition}

With the help of function $exp$, we are now able to define the complexity classes $k$-EXPTIME and $k$-EXPSPACE for
all $k \geq 1$.

\begin{definition}
    Let $k \in \mathbb{N} \setminus \{0\}$ and $f: \mathbb{N} \rightarrow \mathbb{N}$ be a polynomial function then
    $k$-EXPTIME $=$ $\mathit{TIME}$($exp(k, f(n))$) and $k$-EXPSPACE $= \mathit{SPACE}(exp(k, f(n)))$.
\end{definition}

Remark, that $\mathit{TIME}$ is the maximal number of transitions and $\mathit{SPACE}$ the biggest configuration of a
$\mathit{DTM}$ while computing a word.

From Definition~\ref{definition:bisimulationInvariant}, Definition~\ref{definition:queryBelongsToComplexityClass}
and the complexity classes $k$-EXPTIME and $k$-EXPSPACE follow the queries that we want to investigate.

\begin{definition}
    \label{definition:kExptimekExpspace}
    \exptime{$k$} are the bisimulation invariant queries that belong to complexity class $k$-EXPTIME and
    \expspace{$k$} are the bisimulation invariant queries that belong to complexity class $k$-EXSPACE, where $k \geq 1$.
\end{definition}

The main aim of \emph{descriptive complexity} is to describe the complexity classes known from
computational complexity theory with logics. While computational complexity theory distinguishes time and space
classes, descriptive complexity theory characterizes classes with logical resources instead of a reference to
automaton models or space and time bounds.

The first known result in the area of descriptive complexity comes from Fagin. In 1974 he showed that the complexity
class NP coincides with $\exists SO$~\cite{fagin1974generalized}, the existential fragment of second-order logic.

In the next chapter we introduce the polyadic higher order fixed point logic and a restriction called tail-recursive
devised by Lange and Lozes in~\cite{lange2014capturing}. We want to compare the in
Definition~\ref{definition:kExptimekExpspace} presented queries with these logics to make a further contribution in
the theory of descriptive complexity.

\begin{definition}
	Let $\Sigma$ be a finite, non-empty set of symbols. Then $p$ is called a \emph{picture over
	$\Sigma$}, if $p$ is a finite rectangular array containing only symbols of $\Sigma$. The \emph{set
	of all pictures over $\Sigma$} is denoted by $\Sigma^{*,*}$. Any $L\subseteq\Sigma^{*,*}$ is called a
	\emph{picture language}.
\end{definition}
Let $p$ be a picture. The \emph{height} and \emph{width} of $p$ are the number of rows and the
number of columns and are denoted as $l_1(p)$ and $l_2(p)$, respectively. The \emph{size} of $p$ is
the pair $(l_1(p), l_2(p))$. The \emph{empty picture} is the only picture of size $(0, 0)$ and is
denoted by $\lambda$. The set $\Sigma^{h, k} \subset \Sigma^{*, *}$ denotes the \emph{set of
pictures over $\Sigma$ of size $(h, k)$}.

If we need evaluate the value of one specific pixel within the picture, we use $p(i, j)$ to return
the symbol in the picture $p$ at the vertical position $i$ and horizontal position $j$, where $1
\leq i \leq l_1(p)$ and $1 \leq j \leq l_2(p)$.

Sometimes it is necessary that a picture is surrounded by a special symbol to detect whether the
border of a picture has been reached. For this purpose, we use the symbol $\#$ as border symbol.
With $\hat{p}$ we denote the picture $p$ bordered with $\#$'s:
\begin{center}
	$\hat{p} =$ \scalebox{0.79}{\begin{tabular}{|C{2.75cm}|c|c|c|c|}
		\hline
		\hspace{0.99cm}\#\hspace{0.99cm} & \hspace{0.99cm}\#\hspace{0.99cm} &
		\hspace{1.45cm}\dots\hspace{1.45cm} & \# & \hspace{0.99cm}\#\hspace{0.99cm} \tabularnewline
		\hline
		\#       & $p(1, 1)$      & \dots    & $p(1, l_2(p))$      & \#	      \tabularnewline
		\hline
		\multirow{1}[12]{1cm}{\centering $\vdots$} & \rule[-2.4cm]{0pt}{0.9cm}
		\multirow{1}[12]{1cm}{\centering $\vdots$} & \multirow{1}[12]{1cm}{\centering $\ddots$} &
		\multirow{1}[12]{1cm}{\centering $\vdots$} &
		\multirow{1}[12]{1cm}{\centering $\vdots$}
		\tabularnewline
		\hline
		\#       & $p(l_1(p), 1)$ & \dots    & $p(l_1(p), l_2(p))$ & \#       \tabularnewline
		\hline
		\#       & \#             & \dots    & \#                  & \#       \tabularnewline
		\hline
	\end{tabular}
}.
\end{center} 
\begin{definition}
	Let $p \in \Sigma^{m,n}$. $q = p((i, j), (i', j'))$ is called a \emph{subpicture of $p$} if
	$1 \leq i < i' \leq m$ and $1 \leq j < j' \leq n$
	satisfying the following condition:
	\begin{compactitem}
	   \item for each $k, r (1 \leq k \leq i' - i + 1, 1 \leq r \leq j'- j + 1)$, $q(k, r) = p(k
                + i - 1, r + j - 1)$ holds.
	\end{compactitem}
\end{definition}

Sometimes it is necessary to look at all subpictures with a specific size of a picture.
Therefore, we denote the \emph{set of subpictures of size $(h, k)$ of a picture $p\in \Sigma^{*,*}$}
as 
\[B_{h, k}(p) = \{q \in \Sigma^{h,k} \mid q \text{ is a subpicture of } p\}.\] 

We are now able to look at some operations on pictures and languages. The concatenation of pictures
can be performed in two directions: horizontally and vertically.
Two pictures $p, q \in \Sigma^{*,*}$ can be horizontally concatenated, if $l_1(p) = l_1(q)$. We denote
the \emph{horizontal concatenation of $p$ and $q$} by $p \hcat{} q$. Similarly, the two pictures $p$
and $q$ can be concatenated vertically, if $l_2(p) = l_2(q)$. We then write $p \vcat{} q$. Moreover,
the empty picture $\lambda$ can always be concatenated and is the neutral element for both of the
concatenation operations.
\begin{definition}
	Let $L_1, L_2 \in \Sigma^{*,*}$ be two picture languages. The \emph{horizontal concatenation of
	$L_1$ and $L_2$} is defined by 
	\[L_1 \hcat{} L_2 = \{p \hcat{} q \mid p \in L_1, q \in L_2\}\]
\end{definition}
Similarly, the vertical concatenation of picture languages is defined. 

Another operation is the projection by mapping.
\begin{definition}
	Let $\Gamma$ and $\Sigma$ be two finite alphabets and $\pi: \Gamma \rightarrow \Sigma$ be a
	mapping. Let $p \in \Gamma^{m,n}$. \emph{$q \in \Sigma^{m,n}$ is a projection of $p$}, if $q(i, j)
	= \pi(p(i, j))$ for all $1 \leq i \leq m$ and $1 \leq j \leq n$. We write $q = \pi(p)$.
\end{definition}
Since the projection is not required to be injective, a picture can have more than one pre-image. We
define the \emph{projection of a language $L$} as 
\[L' = \pi(L) = \{q \mid q = \pi(p), p \in L\}.\] 
\begin{definition}
	Let $p \in \Sigma^{*, *}$. The \emph{clockwise rotated picture $p^R$ of $p$} is 
	\begin{center}
	
		$p^R = $ \scalebox{0.79}{\begin{tabular}{|C{2.75cm}|C{1cm}|C{1cm}|}
			\hline
			$p(l_1(p), 1)$ & \hspace{1.45cm}\dots\hspace{1.45cm}    & $p(1, 1)$ \tabularnewline
			\hline
			\rule[-2.4cm]{0pt}{0.9cm}
        \multirow{1}[12]{1cm}{\centering $\vdots$} & \multirow{1}[12]{1cm}{\centering $\ddots$} &
        \multirow{1}[12]{1cm}{\centering $\vdots$}  \tabularnewline
			\hline
			$p(l_1(p), l_2(p))$ & \dots    & \hspace{0.3cm}$p(1, l_2(p))$\hspace{0.3cm} \tabularnewline
			\hline
		\end{tabular}
	}.
	\end{center}
\end{definition}
Counterclockwise rotation can be defined similarly. With this operator, it is possible to define the
\emph{rotation of a language $L$} by 
\[L^R = \{p^R \mid p \in L\}.\] 

Based on this idea, it is possible to define horizontal and vertical mirroring and transposing of
pictures and languages.
% \begin{definition}
%     Let $p \in \Sigma^{*, *}$ be a picture. The \emph{vertical mirrored picture $p^{VM}$ of $p$} is 
%     \begin{center}
%         $p^{VM} = $ \begin{tabular}{|C{2.9cm}|C{1cm}|C{1cm}|}
%             \hline
%             \hspace{0.37cm}$p(l_1(p), 1)$\hspace{0.37cm} & \hspace{0.9cm}\dots\hspace{0.9cm}    &
%             $p(l_1(p), l_2(p))$ \tabularnewline
%             \hline
%             $\vdots$  & $\ddots$ & $\vdots$  \tabularnewline
%             \hline
%             $p(1, 1)$ & \dots    & $p(1, l_2(p))$ \tabularnewline
%             \hline
%         \end{tabular}
%     \end{center}
% \end{definition}
% Horizontal mirroring can be defined similarly. The \emph{vertical mirroring of a language $L$} is
% defined by \[L^{VM} = \{p^{VM} \mid p \in L\}.\]
% 
% Also possible is the transposition of pictures and languages and is to understand like the normal
% transposition over matrices.
% \begin{definition}
%     Let $p \in \Sigma^{*, *}$ be a picture. The \emph{transposed picture $p^{T}$ of $p$} is 
%     \begin{center}
%         $p^{T} = $ \begin{tabular}{|C{2.9cm}|C{1cm}|C{1cm}|}
%             \hline
%             $p(1, 1)$ & \hspace{0.9cm}\dots\hspace{0.9cm}    & $p(l_1(p), 1)$ \tabularnewline
%             \hline
%             $\vdots$  & $\ddots$ & $\vdots$  \tabularnewline
%             \hline
%             \hspace{0.37cm}$p(1, l_2(p))$\hspace{0.37cm} & \dots    & $p(l_1(p), l_2(p))$
%             \tabularnewline
%             \hline
%         \end{tabular}
%     \end{center}
% \end{definition}
% The \emph{transposed picture language of a pictue language $L$} is defined by
% \[L^{T} = \{p^{T} \mid p \in L\}.\]

Further operations on picture languages are union, intersection and complementation and will be
understood as ordinary set operations.