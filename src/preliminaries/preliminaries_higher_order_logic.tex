%%
%% Author: Davidov
%% 16.05.2018
%%


\section{Higher Order Logic}\label{sec:higherOrderLogic}

For comparing the complexity classes with PHFL, we use combinations of extensions of FO as intermediate logics. The first well
known extension is called Higher Order Logic~\cite{vanBenthem2001higher}, abbreviated with HO. In HO we
increase the expressive power of FO by allowing quantification over relations of any order. Therefore, we have to define the
types of higher order variables.

\subsection{Syntax of HO}\label{subsec:hoSyntax}

\begin{definition}
    \emph{HO types} are given by the grammar
    \[ \tau \Coloneqq \odot \mid (\tau, \dots, \tau) \]
\end{definition}

The HO type of individuals is $\tau = \odot$. These objects have \textit{order} $1$. The HO type $\tau = (\tau',
\dots, \tau')$ is that of relations between objects of HO type $\tau'$ and has order $1 + order(\tau')$. For
each HO type we have a countable infinite set of variables. Furthermore, let $\sigma$ be a signature over a
relational vocabulary i.e. $\sigma$ only contains relation symbols.

\begin{definition}
    Let $\mathcal{V} = \{X_1, X_2, \dots \}$ a countable infinite set of variables, $P$ a set of propositions and $\Sigma$ a set of actions, then \emph{HO
    formulas} over $P$ and $\Sigma$ are defined by the grammar
    \[\Phi, \Psi \Coloneqq p(X_1) \mid a(Y_1, Y_2) \mid Y(Z_1, \dots, Z_n) \mid \neg \Phi \mid \Phi \vee \Psi \mid \exists
    (X \colon \tau).\,\Phi\]
    where
    \begin{compactitem}
        \item $p \in P$ and $X_1 \in \mathcal{V}$ of HO type $\odot$,
        \item $a \in \Sigma$ and $Y_1, Y_2 \in \mathcal{V}$ of HO type $\odot$,
        \item $Y \in \mathcal{V}$ of HO type $(\tau', \dots, \tau')$ and $Z_1, \dots, Z_n \in \mathcal{V}$ of HO type $\tau'$ and
        \item $X \in \mathcal{V}$ of HO type $\tau$.
    \end{compactitem}
\end{definition}

\subsection{Semantics of HO}\label{subsec:hoSemantics}

At first, we define the universes of the different HO types.

\begin{definition}
    Let $\mathcal{T} = (Q, \Sigma, P, \Delta, v)$ be an LTS then the universes of the HO types are defined inductively as follows:
    \begin{compactitem}
        \item $D_\odot(Q) = Q$,
        \item $D_{(\tau, \dots, \tau)}(Q) = \mathcal{P}(D_{\tau}(Q)^n)$
    \end{compactitem}
\end{definition}

Moreover, $\eta$ is a variable mapping that assigns every variable to an element of the appropriate universe, i.e. if
variable $X$ is of HO type $\tau$, then $\eta(X) \in D_{\tau}(Q)$. With $\eta[X \rightarrow \mathcal{X}]$,
where $\mathcal{X} \in D_\tau(Q)$ and $X$ of HO type $\tau$, we mean the variable assignment $\eta'$,
where $\eta'(X) = \mathcal{X}$ and $\eta'(Y) = \eta(Y)$ for all $Y \neq X$.

\begin{definition}
    Let $\mathcal{T} = (Q, \Sigma, P, \Delta, v)$ be an LTS and $\eta$ a variable mapping over universe $Q$. The
    semantics of a HO formula is defined inductively as follows:
    \begin{compactitem}
        \item $\mathcal{T}, \eta \models p(X_1)$ iff $p \in v(\eta(X_1))$,
        \item $\mathcal{T}, \eta \models a(Y_1, Y_2)$ iff $(\eta(Y_1), a, \eta(Y_2)) \in \Delta$,
        \item $\mathcal{T}, \eta \models Y(Z_1, \dots Z_n)$ iff $(\eta(Z_1), \dots
        \eta(Z_n)) \in \eta(Y)$,
        \item $\mathcal{T}, \eta \models \neg\Phi$ iff $\mathcal{T}, \eta\not\models\Phi$,
        \item $\mathcal{T}, \eta \models \Phi \vee \Psi$ iff $\mathcal{T}, \eta\models\Phi$ or $\mathcal{T},
        \eta\models\Psi$,
        \item $\mathcal{T}, \eta \models \exists (X\colon\tau).\,\Phi$ iff there exists $\mathcal{X} \in D_{\tau}
        (Q)$ with $\mathcal{T}, \eta[X \rightarrow \mathcal{X}] \models \Phi$
        \end{compactitem}
\end{definition}

We can categorize the formulas by the order of all occurring variables. With HO$^k$ we mean 
the set of all those formulas whose variables have order less or equal $k$.
%\begin{example}{\cite{vanBenthem2001higher}}
%    \label{example:ho}
 %   The following formula $\varphi$ describes Peano's induction axiom. Peano's induction axiom reveals that every set
%    of natural numbers, which contains $0$ and is also closed under immediate successors, contains all natural numbers.
   % \[\varphi = \forall (P\colon(\odot)).\,(P(0) \wedge \forall (I\colon\odot).\,(P(I) \Rightarrow P(I + 1)) \Rightarrow
    %\forall (N \colon\odot).\,(P(N)))\]
   % Note, that $\varphi$ lies in $\mathit{HO}^2$, i.e. it is a second order logic formula.
%\end{example}
Similar to Definition~\ref{definition:query_associated_to_formula} we define $r$-adic queries that are defined by HO formulas with $f$ free first-order variables. 

\begin{definition}
\label{definition:query_associated_to_formula_ho}
    Given a set of propositions $P$, a set of actions $a$ and a closed HO
    formula $\Phi$ with free first-order variables $X_1, \dots, X_f$ we call $\mathcal{Q}^r_\Phi$ a $r$-adic query defined by $\Phi$ if for all LTS $\mathcal{T}$, all variable mappings $\eta$ and all $(q_1, \dots, q_r) \in {\mathcal{Q}^r_\Phi}^\mathcal{T}$ there is a $\mathcal{T}, \eta \models
     \Phi$ such that $q_i = \eta(x_i)$ for all $i \in
    \{1, \dots, min(\{r, f\})\}$.
\end{definition}

Finally, we define bisimulation-invariant HO formulas.

\begin{definition} 
\label{definition:bisimulation_invariant_ho_formula}
	Given a HO formula $\Phi$ with $f$ free first-order variables and the $f$-adic query $\mathcal{Q}^f_\Phi$ defined by $\Phi$ then we call $\Phi$ bisimulation-invariant if $\mathcal{Q}^r_\Phi$ is bisimulation-invariant.
\end{definition}

\subsection{HO + LFP}
\label{subsec:hoPlusLfp}

Another possibility to extend FO is to add operators that are not expressible in FO. Here, we are interested in two
of them, the least fixpoint and the partial fixpoint operator. Instead of defining the operators for FO we want
to define these operators for HO. At first, we regard the least fixpoint operator.
Like in~\cite{freireMartins2011descriptive} we want to define special operators that are working on HO type universes.

\begin{definition}
\label{definition:induced_operator}
    Let $P$ be an set of propositions, $\Sigma$ a set of actions, $X$ a relation variable of HO type $\tau = (\tau', \dots, \tau')$,
    $\tau'$ an arbitrary HO type, $X_1, \dots X_n$ variables of HO type $\tau'$ and $\varphi(X, X_1, \dots, X_k)$ a
    formula over $P$ and $\Sigma$ with free variables $X, X_1, \dots, X_k$. For each LTS $\mathcal{T}$ with
    state set $Q$ and each variable mapping $\eta$, the formula $\varphi(X, X_1, \dots, X_k)$ induces the operator
    \begin{align*}
        F_\Phi^{\mathcal{T},\eta}\colon\mathscr{P}(D_\tau(Q)) &\longrightarrow \mathscr{P}(D_\tau(Q))\\
        A &\longmapsto F_\Phi^\mathcal{T}(A) \coloneqq \{(A_1, \dots, A_n) \mid \mathcal{T}, \eta \models \varphi(A, A_1,
        \dots, A_n)\}
    \end{align*}
    where $A_1 \dots, A_n \in D_{\tau'}(Q)$.
\end{definition}

To make $F_\Phi^{\mathcal{T}, \eta}$ monotone we have to restrict $\varphi(X, X_1, \dots, X_k)$ in a way that variable
$X$ occurs under an even number of negations within of $\Phi$~\cite{freireMartins2011descriptive}. Those formulas
are called \textit{positive in} $X$. Therefore, we are able to define the least fixpoint operator for HO
formulas, denoted by HO($\mathit{LFP}$).

\begin{definition}
    Let $P$ be a set of propositions and $\Sigma$ a set of actions. The set of \emph{HO($\mathit{LFP}$) formulas} enhances the set of HO formulas with the
    following formation rule:
    \begin{compactitem}
        \item $[\mathit{LFP}\;\Phi(X, X_1, \dots, X_n)](V_1, \dots, V_n)$ is a HO
        ($\mathit{LFP}$) formula over $P$ and $\Sigma$ with free variables $V_1, \dots, V_n$ iff $\Phi(X, X_1, \dots, X_n)
        $ is a HO($\mathit{LFP}$) formula with free variables $X, X_1, \dots, X_n$, if $\Phi$ is positive in
        $X$, variable $X$ has HO type $\tau = (\tau', \dots, \tau')$ and $X_1, V_1, \dots, X_n, V_n$ have HO type $\tau'$.
    \end{compactitem}
\end{definition}

Similar to HO$^k$, the set HO($\mathit{LFP}$)$^k$ denotes all those HO($\mathit{LFP}$) formulas whose variables have types of
order less or equal $k$ with one exception: The variable $X$ occurring in formulas of kind $[\mathit{LFP}\;\varphi(X, x_1, \dots, x_n)](v_1, \dots, v_n)$ is the only one that can have a type of an order $k+1$.

\begin{definition}
    Let $\mathcal{T}$ be an LTS and $\eta$ a variable mapping over universe $Q$. The
    semantics of an HO formula are extended to the semantics of HO($\mathit{LFP}$) via the following definition:
    \begin{compactitem}
        \item $\mathcal{T}, \eta \models [\mathit{LFP}\;\varphi(X, X_1, \dots, X_n)](V_1, \dots,
        V_n)$ iff $(\eta(V_1), \dots, \eta(V_n)) \in \mathit{LFP}$ $(F_\varphi^{\mathcal{T}, \eta})$.
    \end{compactitem}
\end{definition}

\begin{example}
    \label{example:ho_lfp} 
    Let $\mathcal{T} = (Q, \Sigma, P, \Delta, v)$ be an LTS. Furthermore, let 
\begin{align*}
\Phi(X, X_1, X_2) &= \underset{p\in P}{\bigvee} \big(p(X_1) \neq p(X_2)\big)\\&
 \vee \underset{a\in\Sigma}{\bigvee} \big(\exists (Y_1 \colon \odot).\, a(X_1, Y_1) \wedge \forall (Y_2 
 \colon \odot).\, a(X_2,Y_2) \Rightarrow X(Y_1, Y_2)\big) \\&
\vee \underset{a\in\Sigma}{\bigvee} \big(\exists (Y_2 \colon \odot).\, a(X_2, Y_2) \wedge \forall (Y_1 
\colon \odot).\, a(X_1,Y_1) \Rightarrow X(Y_1, Y_2)\big)
\end{align*}
be a HO$^2$ formula with free first-order variables $x_1$ and $x_2$ and free second-order variable $X$.
Then $LFP(F_\Phi^{\mathcal{T}, \eta})$ describes the same set as in Example~\ref{example:lfp}, the set of all those pairs of states $(q, p)$ in $\mathcal{T}$ such that $q\not\sim p$ holds. With $[LFP\; \Phi(X, X_1, X_2)](q, p)$ we can check if $q \not\sim p$, where $q, p \in Q$. 
\end{example}

\subsection{HO + PFP}\label{subsec:ho+Pfp}

Next, we define the partial fixpoint operator for HO formulas~\cite{schewe2006fixpoint}. While the
$\mathit{LFP}$ operator restricts formulas to be positive in a variable, the partial fixpoint operator does not have any
restriction. Therefore, we can define and
add the partial fixpoint operator to HO formulas, denoted as HO($\mathit{PFP}$).

\begin{definition}
    Let $P$ be a set of propositions and $\Sigma$ a set of actions. The set of \emph{HO($\mathit{PFP}$) formulas} enhances the set of HO formulas with the
    following formation rule:
    \begin{compactitem}
        \item $[\mathit{PFP}\;\Phi(X, X_1, \dots, X_n)](V_1, \dots, V_n)$ is a HO
        ($\mathit{PFP}$) formula over $P$ and $\Sigma$ with free variables $V_1, \dots, V_n$ iff $\Phi(X, X_1, \dots, X_n)
        $ is a HO($\mathit{PFP}$) formula with free variables $X, X_1, \dots, X_n$, where $X$ has HO type $\tau =
        (\tau', \dots, \tau')$ and $X_1, V_1, \dots, X_n, V_n$ have HO type $\tau'$.
    \end{compactitem}
\end{definition}

HO($\mathit{PFP}$)$^k$ denotes the set of all those HO($\mathit{PFP}$) formulas whose variables have types of
order less or equal $k$ with one exception: The variable $X$ occurring in formulas of kind $[\mathit{PFP}\;\varphi(X, X_1, \dots, X_n)](V_1, \dots, V_n)$ is the only one that can have a type of an order $k+1$.

\begin{definition}
    Let $\mathcal{T}$ be an LTS and $\eta$ a variable mapping over universe $Q$. The
    semantics of a HO formula are extended to the semantics of HO($\mathit{PFP}$) via the following definition:
    \begin{compactitem}
        \item $\mathcal{T}, \eta \models [\mathit{PFP}\;\varphi(X, X_1, \dots, X_n)](V_1, \dots,
        V_n)$ iff $(\eta(V_1), \dots, \eta(V_n)) \in \mathit{PFP}$ $(F_\Phi^{\mathcal{A},\eta})$.
    \end{compactitem}
\end{definition}

\begin{example}
\label{example:ho_pfp}
Let $\mathcal{T} = (Q, \{a\}, P, \Delta, v)$ be an LTS with $Q = \{1, \dots, n\}$ and $\Delta = \{(x, a, y) \mid x, y \in Q \text{ and } x < y\}$. Furthermore, let
\begin{align*}
	\Phi(X,X_1) = X(X_1) \wedge \exists (X_2 \colon \odot).\, a(X_2, X_1) \wedge X(X_2)\,\vee \\ 
	\neg X(X_1) \wedge \forall (X_2 \colon \odot).\, a(X_2, X_1) \Rightarrow X(X_2)
\end{align*}
be a HO$^1$ formula. Then $F_\Phi^{\mathcal{T}, \eta}$ describes the same operator as in Example~\ref{example:pfp} and therefore $PFP(F_\Phi^{\mathcal{A},\eta}) = \varnothing$ holds. Note that $\mathcal{T}, \eta \not\models [PFP\; \Phi(X, $ $X_1)](V_1)$ for any variable mapping $\eta$.
\end{example}