%%
%% Author: Davidov
%% 16.05.2018
%%


\subsection{Higher Order Logic}\label{subsec:higherOrderLogic}

For comparing the complexity classes with PHFL, we have to detour over three extensions of FO. The first well known
extension is called Higher Order Logic~\cite{vanBenthem2001higher}, abbreviated with HO. In HO we
increase the expressive power of FO by allowing relation variables over any order. For this, we have to define the
types of higher order variables.

\begin{definition}
    \emph{HO types} are defined inductive as follows:
    \begin{compactitem}
        \item $\tau = \odot$ is a HO type,
        \item $\tau = (\tau_1, \dots, \tau_n)$ is a HO type, if $\tau_1, \dots, \tau_n$ are
        HO types.
    \end{compactitem}
\end{definition}

The HO type of individuals is $\tau = \odot$. These objects have the \textit{HO order} $0$. The HO type $\tau = (\tau_1,
\dots, \tau_n)$ is that of relations between objects of HO types $\tau_1, \dots, \tau_n$ and have the HO order $1 + max
(order(\tau_1), \dots, order(\tau_n))$.

For each HO type we have a countable infinite set of variables. The \textit{terms} of HO type $\odot$ are
generated as in FO. Terms of a higher HO type are just a variable of that HO type. Furthermore, let $\sigma$ be
the signature holding the constants, functions and relations of arbitrary HO types.

\begin{definition}
    The set of \emph{HO formulas} over $\sigma$ is defined inductively as follows:
    \begin{compactitem}
        \item $X = Y$ is a HO formula over $\sigma$ if $X$ and $Y$ are variables of the same type,
        \item $R(t_1, \dots, t_n)$ is a HO formula over $\sigma$ if $R \in \sigma$ is a relation with arity $n$ and
        $t_1, \dots, t_n$ are terms of type $\odot$,
        \item $X(t_1, \dots, t_n)$ is a HO formula over $\sigma$ if $X$ is a variable of type $(\tau_1, \dots, \tau_n)$
        and $t_i$ is a term over $\tau_i$, with $i \in \{1, \dots, n\}$,
        \item if $\phi$ and $\psi$ are two HO formulas over $\sigma$, then $\neg\phi$, $\phi\wedge\psi$ and $\phi
        \vee \psi$ are also HO formulas over $\sigma$,
        \item if $\phi$ is a HO formula over $\sigma$ and $X$ a variable of arbitrary type, then $\exists X\phi$ and
        $\forall X\phi$ are also HO formulas over $\sigma$.
    \end{compactitem}
\end{definition}

The first step in direction of the semantics of HO formulas is to interpret the universes of the different HO types.

\begin{definition}
    Let $\mathcal{U}$ be the global universe. The universes of the HO types are defined inductively as follows:
    \begin{compactitem}
        \item $D_\odot(\mathcal{U}) = \mathcal{U}$,
        \item $D_{(\tau_1, \dots, \tau_n)}(\mathcal{U}) = \mathcal{P}(D_{\tau_1} \times \dots \times D_{\tau_n})$
    \end{compactitem}
\end{definition}

Moreover, $\alpha$ is a function that assigns all variables to an element of the appropriated universe, i.e. if
variable $X$ is of type $\tau$, then $\alpha(X) \in D_{\tau}(\mathcal{U})$.

\begin{definition}
    Let $\mathcal{A}$ be a structure over $\sigma$ and $\alpha$ a variable allocation over $\mathcal{U}$. The
    satisfaction of a HO formula is defined inductively as follows:
    \begin{compactitem}
        \item $\mathcal{A}, \alpha \models (X = Y)$ iff $\alpha(X) = \alpha(Y)$
        \item $\mathcal{A}, \alpha \models R(t_1, \dots t_n)$ iff $(t_1^{\mathcal{A}}[\alpha], \dots
        t_n^{\mathcal{A}}[\alpha]) \in R^{\mathcal{A}}$, where $t_i^{\mathcal{A}}[\alpha]$ is the value of term
        $t_i$ under $\alpha$ in $\mathcal{A}$ defined as in FO,
        \item $\mathcal{A}, \alpha \models X(t_1, \dots t_n)$ iff $(t_1^{\mathcal{A}}[\alpha], \dots
        t_n^{\mathcal{A}}[\alpha]) \in \alpha(X)$, where $t_i^{\mathcal{A}}[\alpha]$ is the value of term
        $t_i$ under $\alpha$ in $\mathcal{A}$ defined as in FO,
        \item $\mathcal{A}, \alpha \models \neg\phi$ iff $\mathcal{A}, \alpha\not\models\phi$,
        \item $\mathcal{A}, \alpha \models \phi \wedge \psi$ iff $\mathcal{A}, \alpha\models\phi$ and $\mathcal{A},
        \alpha\models\psi$,
        \item $\mathcal{A}, \alpha \models \phi \vee \psi$ iff $\mathcal{A}, \alpha\models\phi$ or $\mathcal{A},
        \alpha\models\psi$,
        \item $\mathcal{A}, \alpha \models \exists X\phi$ iff there is an allocation of $X$, $a = \alpha(X) \in D_{\tau}
        (\mathcal{U})$, that $\mathcal{A}, \alpha[X/a] \models \phi$, where $\tau$ is the type of $X$ and
        \item $\mathcal{A}, \alpha \models \forall X\phi$ iff all possible allocations of $X$, $a = \alpha(X) \in
        D_{\tau}(\mathcal{U})$ holds that $\mathcal{A}, \alpha[X/a] \models \phi$, where $\tau$ is the type of $X$.
        \end{compactitem}
\end{definition}

We can categorize the formulas by the order of all variables. With HO$^k$ we mean the set of all those formulas whose
variables have order less or equal $k$.

Another extension of FO
HOL , HOL + PFP, HOL + LFP