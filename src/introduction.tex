\section{Introduction}

In this thesis we compare two different definitions of deterministic recognizable picture
languages. Furthermore, we examine some closure properties of the corresponding classes of
languages.

The first definition of deterministic recognizable picture languages is from K. Reinhardt and B.
Borchert and was introduced in~\cite{borchert2006deterministically} in 2006. This definition uses a
domino tiling system to recognize a local picture by a deterministic process. They abbreviated their
family of all deterministically recognizable picture languages as \emph{DREC}.

In 2007 another and completely different definition was introduced by M. Anselmo, D. Giammarresi
and M. Madonia in~\cite{anselmo2007determinism}. This definition uses a tiling system to recognize
a picture by proceeding from one corner to the diagonally opposite corner. They also abbreviated
their family of all deterministic recognizable picture languages by DREC. In 2009 V. Lonati and M.
Pradella called these languages ``diagonal deterministic recognizable picture languages'' because of
its procedure and abbreviated the class of all these languages as \emph{$Diag$-DREC}
(see~\cite{lonati2009snake}). To avoid ambiguity, we will also adapt the abbreviation
$Diag$-DREC in this paper.

At first, we review some basic definitions of picture languages. A picture is a two-dimensional
array of symbols and a picture language is a set of pictures. Furthermore, we describe some
operations on pictures and languages.

The following section describes the usage of tiling systems (presented
in~\cite{giammarresi1992recognizable}) to generate pictures. This approach is based on the idea of
the description of a picture language by a set of tiles. The set of picture languages which are
generated by tiling systems is denoted as the class of recognizable picture languages abbreviated
with \emph{REC}. This class can be described in many other ways e.g. with a domino system, which is
presented in Section~\ref{domino_systems}.

Every approach that generates REC recognizes a picture non-deterministically. This is the reason
for the two definitions of DREC and $Diag$-DREC which will be presented in the next two chapters. We
regard closure properties that are known for $Diag$-DREC and prove some missing closure properties
of $Diag$-DREC and DREC. In~\cite{anselmo2007determinism} it is shown that $Diag$-DREC is closed
under complement and rotation. We show that $Diag$-DREC is not closed under intersection and union.
In the second part of these two chapters we prove that DREC is closed under rotation and
intersection.

The next part presents a special kind of two-dimensional cellular automaton which accepts
picture languages. These automata are called two-dimensional deterministic on-line tesselation
automata, abbreviated with \emph{2DOTA}, and were introduced in 1977 in~\cite{inoue1977properties}
by K. Inoue and A. Nakamura. It should be noted that the non-deterministic version is another
approach to describe REC. 

In the last section we compare $Diag$-DREC and DREC. In so doing we show that $Diag$-DREC is a
strict subset of DREC.