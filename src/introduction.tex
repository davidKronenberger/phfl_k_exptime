\chapter{Introduction}\label{ch:introduction}

The descriptive complexity theory describes the complexity classes known from computational complexity theory with logics. The key advantage is that complexity classes are characterized by logical resources instead of referring to automaton models or space and time bounds. The first known result in the area of descriptive complexity comes from Fagin. In 1974 he showed that the well-known complexity class NP coincides with $\exists SO$, the existential fragment of second-order logic. 

The complexity classes that are of interest in this thesis are those which their problems can be solved in $k$-fold exponential time or uses $k$-fold exponential space. These prominent clases are abbreviated by $k$-EXPTIME and $k$-EXPSPACE, respectively. The logic that captures the bisimulation-invariant $k$-EXPTIME, abbreviated by \exptime{$k$}, is called Polyadic Higher-Order Fixpoint Logic and was introduced by M. Lange and E. Lozes in~\cite{lange2014capturing} abbreviated by PHFL. 

PHFL is a modal fixpoint logic that extends the Higher-Order Fixpoint Logic, abbreviated by HFL, from M. Viswanathan and R. Viswanathan~\cite{viswanathan2004higher} with the Polyadic $\mu$-Calculus from M. Otto~\cite{otto1999bisimulation}. HFL extends the modal $\mu$-calculus by a simply typed $\lambda$-calculus which allows to define higher-order functions on predicates.

This thesis is about to proof that the logic PHFL that uses formulas with order at most $k$, abbreviated with PHFL$^k$, captures \exptime{$k$} where $k > 1$. Due to the fact, that the above statement is also true for $k = 0$~\cite{otto1999bisimulation} and $k = 1$~\cite{lange2014capturing} we were able to verify, that PHFL$^k$ captures \exptime{$k$} for any $k \geq 0$ on finite labelled transition systems. Furthermore, we will show that a restriction of PHFL called tail-recursive that uses formulas with order at most $k+1$, abbreviated with PHFL$^{k+1}_{tail}$, captures the bisimulation-invariant $k$-EXPSPACE, abbreviated with \expspace{$k$} where $k > 1$. In analogy to the exponential time classes, it was also proven that PHFL$^{k+1}_{tail}$ captures \expspace{$k$} for any $k \geq 0$. The results presented in this paper are divided in two parts. 

In the first part it is shown that the upper bounds of the expressive power of PHFL$^k$ and PHFL$^{k+1}_{tail}$ are $k$-EXPTIME and $k$-EXSPACE, repectively. This is shown by a reduction from the semantics of PHFL$^k$ and PHFL$^{k+1}_{tail}$ to the semantics of HFL$^k$ and HFL${k+1}_{tail}$, respectively. Because it is known that the model checking problems of HFL$^k$ and HFL${k+1}_{tail}$ are in $k$-EXPTIME and $k$-EXPSPACE the same holds for PHFL$^k$ and PHFL$^{k+1}_{tail}$, respectively.

The logics PHFL$^k$ and PHFL$^{k+1}_{tail}$ being as least as expressible as \exptime{$k$} and \expspace{$k$} is shown in the second part. This can be proven by encoding the run of a Turing Machine as queries. As another possibility the higher-order logic extended with least fixpoints, abbreviated with HO(LFP)$^{k+1}$, and higher-order logic extended with partial fixpoints, abbreviated with HO(PFP)$^{k+1}$ can be used. Higher-Order logic compared to first-order logic allows quantification over sets, sets of sets and so on. Therefore, the proofs of the lower bounds are divided in two sub parts. ,

In the first sub part it is shown that the lower bound of the expressive power of HO(LFP)$^{k+1}$ and HO(PFP)$^{k+1}$ are \exptime{$k$} and \expspace{$k$} respectively. C. Freire and A. Martins showed in~\cite{freireMartins2011descriptive} that HO(LFP)$^{k+1}$ is at least as expressive as \exptime{$k$}. The logic HO(PFP)$^{k+1}$ being as least as expressible as \expspace{$k$} will be shown in this thesis by encoding the run of a Turing Machine as queries.

In the second sub part we show that the bisimulation-invariant logic of HO(LFP)$^{k+1}$ and HO(PFP)$^{k+1}$ can be encoded by PHFL$^k$ and PHFL$^{k+1}_{tail}$, respectively. This required a lot of effort since there are no quantifiers in PHFL and PHFL and HO have different sets of types. Quantifiers can be simulated by orders for each type of the bound variable. These orders make it possible to define successors and these are helpful to iterate over the scope. Note that the base type of PHFL denotes sets of elements whereas the base type of HO denotes elements. This problem is non-trivial but can be solved by using the polyadicity of PHFL. 

\section*{Structure of the Thesis}
The struture of the thesis is as follows. In Section~\ref{sec:bisimulationInvariance} we review queries and all necessary definitions for the term bisimulation-invariance. Section~\ref{sec:fixpoints} explaines the term fixpoints and some variants of them. In Section~\ref{sec:polyadichigherorderfixpointlogic} we use fixpoints to define PHFL and the tail-recursive restriction of it. The following section explaines descriptive complexity in more detail and defines \exptime{$k$} and \expspace{$k$}. In the last section of Chapter~\ref{ch:preliminaries} we define the intermediate logics HO(LFP)$^{k+1}$ and HO(PFP)$^{k+1}$. Chapter~\ref{ch:upperBounds} shows that the upper bounds of the expressive power of PHFL$^k$ and PHFL$^{k+1}_{tail}$ are $k$-EXPTIME and $k$-EXSPACE, repectively. In the last chapter we show that the lower bounds of the expressive power of PHFL$^k$ and PHFL$^{k+1}_{tail}$ are \exptime{$k$} and \expspace{$k$}, respectively. In Section~\ref{sec:lower_bounds_preparation} we identify an order on the base type of HO and explain how the proof can be simplified by defining constrains of the structures. In Section~\ref{sec:existential_quantifiers_in_phfl} we show how the quantification of any type can be encoded by PHFL formulas. The next section defines the encoding for any bisimulation-invariant HO(LFP)$^k$ formula in PHFL$^k$ and verifies its correctness. As a consequence the lower bound of the expressive power of PHFL$^k$ is \exptime{$k$}. The last section defines the encoding of the partial fixpoint operator of HO(PFP)$^{k+1}$ in PHFL$^{k+1}_{tail}$ in a similar manner as the previous Section. Finally, the correctness of this encoding is shown and analogue to PHFL$^k$ this means that the lower bound of the expressive power of PHFL$^{k+1}_{tail}$ is \expspace{$k$}.