\chapter{Introduction}\label{ch:introduction}

This thesis make a contribute to the descriptive complexity. Descriptive complexity describes the complexity classes known from computational complexity theory with logics. The big advantage is that complexity classes are characterized with logical resources instead of a reference to automaton models or space and time bounds. The first known result in the area of descriptive complexity comes from Fagin. In 1974 he showed that the complexity class NP coincides with $\exists SO$, the existential fragment of second-order logic. 

In this thesis the complexity classes that we are interested in are those where their problems can be solved in $k$-fold exponential time, abbreviated with $k$-EXPTIME, or uses $k$-fold exponential space, abbreviated with $k$-EXPSPACE. The logic that captures the bisimulation-invariant $k$-EXPTIME, abbreviated with \exptime{$k$}, is called Polyadic Higher-Order Fixpoint Logic, abbreviated with PHFL and was introduced by M. Lange and E. Lozes in~\cite{lange2014capturing}. 

PHFL is a modal fixpoint logic that extends the Higher-Order Fixpoint Logic, abbreviated with HFL, from M. Viswanathan and R. Viswanathan~\cite{viswanathan2004higher} with the Polyadic $\mu$-Calculus from M. Otto~\cite{otto1999bisimulation}. HFL extends the modal $\mu$-calculus by a simply typed $\lambda$-calculus to allow to define higher-order functions on predicates.

The contribution to descriptive complexity is to show that the logic PHFL that uses formulas with order at most $k$, abbreviated with PHFL$^k$, captures \exptime{$k$} where $k > 1$.  Furthermore, we will show that a restriction of PHFL called tail-recursive that uses formulas with order at most $k+1$, abbreviated with PHFL$^{k+1}_{tail}$, captures the bisimulation-invariant $k$-EXPSPACE, abbreviated with \expspace{$k$} where $k > 1$. For $0 \leq k \leq$ both statements were proven by M. Lange and E. Lozes in~\cite{lange2014capturing}. Both of these results are divided in three parts. 

The first part is to show that the upper bounds of the expressive power of PHFL$^k$ and PHFL$^{k+1}_{tail}$ are $k$-EXPTIME and $k$-EXSPACE, repectively. This will be shown by a reduction from the semantics of PHFL$^k$ and PHFL$^{k+1}_{tail}$ to the semantics of HFL$^k$ and HFL${k+1}_{tail}$, respectively. Because it is known that the model checking problem of HFL$^k$ and HFL${k+1}_{tail}$ is in $k$-EXPTIME and $k$-EXPSPACE the same holds also for PHFL$^k$ and PHFL$^{k+1}_{tail}$, respectively.

That the lower bounds of the expressive power of PHFL$^k$ and PHFL$^{k+1}_{tail}$ are \exptime{$k$} and \expspace{$k$} can be proven by encoding the run of a Turing Machine as queries. Another possibility is to use the higher-order logic one extended with least fixpoints, abbreviated with HO(LFP)$^{k+1}$, and one with partial fixpoints, abbreviated with HO(PFP)$^{k+1}$. Higher-Order logic compared to first-order logic allows quantification over sets, sets of sets and so on. That the lower bound of of the expressive power of HO(LFP)$^{k+1}$ is \exptime{$k$} is shown by C. Freire and A. Martins in~\cite{freireMartins2011descriptive}. That the lower bound of expressive power of HO(PFP)$^{k+1}$ is \expspace{$k$} will be shown in this thesis by encoding the run of a Turing Machine as queries.

For the last part we will show that the bisimulation-invariant logic of HO(LFP)$^{k+1}$ and HO(PFP)$^{k+1}$ can be encoded in PHFL$^k$ and PHFL$^{k+1}_{tail}$, respectively. This part is the most complex one of this thesis. The reason is that we have no quantifiers in PHFL and the set of types of PHFL and HO are totally different. The quantifiers have to be simulated by defining orders for any type. With the orders it is possible to define successors and with those we can iterate over the scope. Note that the base type of PHFL denotes sets of elements while the base type of HO denotes elements. Also this is a non-trivial problem but can be solved by using the polyadicity of PHFL. 

In Section~\ref{sec:bisimulationInvariance} we present queries and give all the definitions that are necessary for the term bisimulation-invariance. Section~\ref{sec:fixpoints} explaines the term fixpoints and special variants of them. In Section~\ref{sec:polyadichigherorderfixpointlogic} we use the fixpoints to define PHFL and the tail-recursive restriction of it. The following section explaines descriptive complexity in more detail and defines \exptime{$k$} and \expspace{$k$}. In the last section of Chapter~\ref{ch:preliminaries} we define the intermediate logics HO(LFP)$^{k+1}$ and HO(PFP)$^{k+1}$. Chapter~\ref{ch:upperBounds} shows that the upper bounds of the expressive power of PHFL$^k$ and PHFL$^{k+1}_{tail}$ are $k$-EXPTIME and $k$-EXSPACE, repectively. In the last chapter we show that the lower bounds of the expressive power of PHFL$^k$ and PHFL$^{k+1}_{tail}$ are \exptime{$k$} and \expspace{$k$}, respectively. In more detail in Section~\ref{sec:lower_bounds_preparation} we remark an order of the base type of HO and explain how to make the proof easier by defining constrains of the structures. In Section~\ref{sec:existential_quantifiers_in_phfl} we show how the quantification of any type can be encoded by PHFL formulas. The next section defines the encoding for any bisimulation-invariant HO(LFP)$^k$ formula in PHFL$^k$ and shows its correctness what has as consequence that the lower bound of the expressive power of PHFL$^k$ is. The last section uses parts of the previous section and defines the encoding of the partial fixpoint operator of HO(PFP)$^{k+1}$ in PHFL$^{k+1}_{tail}$. After showing also the correctness of this encoding the lower bound of the expressive power of PHFL$^{k+1}_{tail}$ is \expspace{$k$}.