\section{Deterministic On-line Tessellation Automata}
In this chapter we present the definition of the deterministic on-line tesselation automata.
This automata model is needed to show the inclusion of $Diag$-DREC in DREC. In the following $L(M)$
describes the language of all pictures $p$ that are accepted by an automaton $M$.

The following automata model is a restricted type of a two-dimensional cellular automaton with the
name two-dimensional online tessellation automaton abbreviated as \emph{2OTA}. A cellular
automaton is an array of ``cells", where each cell contains a pixel of the input picture and a state. The automaton
moves diagonally over the picture to compute the state of each cell. After the computation, the
automaton accepts a picture if the cell at the bottom right corner contains an accepting state. The
2OTA was introduced by K. Inoue and A. Nakamura in
1977(see~\cite{inoue1977properties}).

In this automata model, each cell only receives its state at one step at a time. Before this time
step the state of these cells are undefined. Exactly at that moment when the two neighbours on the
top and the left of a cell $c$ have a defined state, $c$ can define its own state depending on its symbol and the states of these neighbours.
\begin{definition} A \emph{deterministic two-dimensional online tessellation automaton}, abbreviated
as 2DOTA, is a 5-tuple $M = (Q, \Sigma, \delta, q_0, F)$, where
\begin{compactitem}
    \item $Q$ is a finite set of states,
    \item $\Sigma$ is a finite set of input symbols,
    \item $q_0 \in Q$ is the initial state,
    \item $F \subseteq Q$ is the set of accepting states and
    \item $\delta: Q \times Q \times \Sigma \rightarrow Q$ is the control function.
\end{compactitem}
\end{definition} 
The automaton $M$ starts at time step $t = 0$ on a picture $p \in \Sigma^{*,*}$. At this time step,
the initial state $q_0$ is associated with every position of the first row and the first column of
$\hat{p}$.
One time step later, the top left corner of $p$ associates the state $\delta(q_0, q_0, p(1, 1))$. At
time step $t = 2$, the position $p(1, 2)$ and $p(2, 1)$ define their states and so on. That means
that $M$ moves along the first main diagonal over the picture to define the state of each cell which
is parallel to the second main diagonal dependent of the current position. The automaton $M$ accepts
a picture $p$ if an accepting state is associated to position $(l_1(p), l_2(p))$ after the
computation.

Now we show an example of a 2DOTA. First let $S = \{p \mid p\in\Sigma^{*,*}, l_1(p) = l_2(p)\}$ be
the language of all quadratic pictures over $\Sigma = \lbrace 0 \rbrace$. The following definition
of a 2DOTA $M$ is one possible construction such that $L(M) = S$ holds. It is obtained
from~\cite{giammarresi1997twodimensional}.
\begin{example}
$M = (\{q_0, q_1, q_2, q_a\}, \Sigma, \delta, q_0, \{q_a\})$ \\
Remark that the following table of transitions does not contain symbols, because $M$ has a unary
alphabet:
\begin{center}
\begin{tabular}{C{0.74cm}|C{0.74cm}|C{0.8cm}|C{0.8cm}|C{0.8cm}}
$\delta$ & $q_0$ & $q_1$ & $q_2$ & $q_a$ \tabularnewline
\hline
$q_0$    & $q_a$ & -     & $q_2$ & $q_2$ \tabularnewline
\hline
$q_1$    & $q_1$ & $q_1$ & -     & -     \tabularnewline
\hline
$q_2$    & -     & $q_a$ & $q_2$ & $q_2$ \tabularnewline
\hline
$q_a$    & $q_1$ & $q_1$ & -     & -     \tabularnewline
\end{tabular}
\end{center}
\label{example_2dota}
\end{example}
$M$ associates the final state $q_a$ to every position in the first main diagonal of an input
picture $p$. For all positions in $p$ above the first main diagonal $M$ associates the state $q_1$
and for all positions below the state $q_2$. If, after the execution of $M$, the final state $q_a$
is at the right bottom corner, the input picture is a square. 

We now have a look at how $M$ will operate with the following picture $p \in S$:
\begin{center}
$p = $\begin{tabular}{|C{0.6cm}|C{0.6cm}|C{0.6cm}|}
\hline
0  & 0  & 0 \tabularnewline
\hline
0  & 0  & 0 \tabularnewline
\hline
0  & 0  & 0 \tabularnewline
\hline
\end{tabular}, 
$ \hat{p} = $\begin{tabular}{|C{0.77cm}|C{0.7cm}|C{0.7cm}|C{0.7cm}|C{0.7cm}|}
\hline
\# & \# & \# & \# & \# \tabularnewline
\hline
\# & 0  & 0  & 0  & \# \tabularnewline
\hline
\# & 0  & 0  & 0  & \# \tabularnewline
\hline
\# & 0  & 0  & 0  & \# \tabularnewline
\hline
\# & \# & \# & \# & \# \tabularnewline
\hline
\end{tabular}.
\end{center}
First, $M$ appends the initial state to all cells of the first row and first column of $\hat{p}$.
The following time steps should be self-explanatory:
\begin{center}
$\overset{t = 0}{\rightarrow}$
\begin{tabular}{|C{1.1cm}|C{1.1cm}|C{1.1cm}|C{1.1cm}|C{1.1cm}|}
\hline
\#$q_0$ & \#$q_0$ & \#$q_0$ & \#$q_0$ & \#$q_0$ \tabularnewline
\hline
\#$q_0$ & 0       & 0       & 0       & \#      \tabularnewline
\hline
\#$q_0$ & 0       & 0       & 0       & \#      \tabularnewline
\hline
\#$q_0$ & 0       & 0       & 0       & \#      \tabularnewline
\hline
\#$q_0$ & \#      & \#      & \#      & \#      \tabularnewline
\hline
\end{tabular}
$\overset{t = 1}{\rightarrow}$
\begin{tabular}{|C{1.1cm}|C{1.1cm}|C{1.1cm}|C{1.1cm}|C{1.1cm}|}
\hline
\#$q_0$ & \#$q_0$ & \#$q_0$ & \#$q_0$ & \#$q_0$ \tabularnewline
\hline
\#$q_0$ & 0$q_a$  & 0       & 0       & \#      \tabularnewline
\hline
\#$q_0$ & 0       & 0       & 0       & \#      \tabularnewline
\hline
\#$q_0$ & 0       & 0       & 0       & \#      \tabularnewline
\hline
\#$q_0$ & \#      & \#      & \#      & \#      \tabularnewline
\hline
\end{tabular}
\end{center}
\begin{center}
$\overset{t = 2}{\rightarrow}$
\begin{tabular}{|C{1.1cm}|C{1.1cm}|C{1.1cm}|C{1.1cm}|C{1.1cm}|}
\hline
\#$q_0$ & \#$q_0$ & \#$q_0$ & \#$q_0$ & \#$q_0$ \tabularnewline
\hline
\#$q_0$ & 0$q_a$  & 0$q_1$  & 0       & \#      \tabularnewline
\hline
\#$q_0$ & 0$q_2$  & 0       & 0       & \#      \tabularnewline
\hline
\#$q_0$ & 0       & 0       & 0       & \#      \tabularnewline
\hline
\#$q_0$ & \#      & \#      & \#      & \#      \tabularnewline
\hline
\end{tabular}
$\overset{t = 3}{\rightarrow}$
\begin{tabular}{|C{1.1cm}|C{1.1cm}|C{1.1cm}|C{1.1cm}|C{1.1cm}|}
\hline
\#$q_0$ & \#$q_0$ & \#$q_0$ & \#$q_0$ & \#$q_0$ \tabularnewline
\hline
\#$q_0$ & 0$q_a$  & 0$q_1$  & 0$q_1$  & \#      \tabularnewline
\hline
\#$q_0$ & 0$q_2$  & 0$q_a$  & 0       & \#      \tabularnewline
\hline
\#$q_0$ & 0$q_2$  & 0       & 0       & \#      \tabularnewline
\hline
\#$q_0$ & \#      & \#      & \#      & \#      \tabularnewline
\hline
\end{tabular}
\end{center}
\begin{center}
$\overset{t = 4}{\rightarrow}$
\begin{tabular}{|C{1.1cm}|C{1.1cm}|C{1.1cm}|C{1.1cm}|C{1.1cm}|}
\hline
\#$q_0$ & \#$q_0$ & \#$q_0$ & \#$q_0$ & \#$q_0$ \tabularnewline
\hline
\#$q_0$ & 0$q_a$  & 0$q_1$  & 0$q_1$  & \#      \tabularnewline
\hline
\#$q_0$ & 0$q_2$  & 0$q_a$  & 0$q_1$  & \#      \tabularnewline
\hline
\#$q_0$ & 0$q_2$  & 0$q_2$  & 0       & \#      \tabularnewline
\hline
\#$q_0$ & \#      & \#      & \#      & \#      \tabularnewline
\hline
\end{tabular}
$\overset{t = 5}{\rightarrow}$
\begin{tabular}{|C{1.1cm}|C{1.1cm}|C{1.1cm}|C{1.1cm}|C{1.1cm}|}
\hline
\#$q_0$ & \#$q_0$ & \#$q_0$ & \#$q_0$ & \#$q_0$ \tabularnewline
\hline
\#$q_0$ & 0$q_a$  & 0$q_1$  & 0$q_1$  & \#      \tabularnewline
\hline
\#$q_0$ & 0$q_2$  & 0$q_a$  & 0$q_1$  & \#      \tabularnewline
\hline
\#$q_0$ & 0$q_2$  & 0$q_2$  & 0$q_a$  & \#      \tabularnewline
\hline
\#$q_0$ & \#      & \#      & \#      & \#      \tabularnewline
\hline
\end{tabular}.
\end{center}
As we can see, $M$ has finished its run over $p$ after five time steps. Note that every
computation of a 2DOTA consists of $l_1(p) + l_2(p) - 1$ steps.

One important theorem for the proof of Theorem~\ref{diag_drec_subsetequal_drec}, which states that
$Diag$-DREC is a subset of DREC, is the following theorem. It was proven by M. Anselmo, D.
Giammarresi and M. Madonia in~\cite{anselmo2007determinism}.
\begin{theorem}
The class $Diag$-DREC is equal to the closure by rotation of $\familyOf{2DOTA}$. 
\label{diag_drec_equal_closure_rotation_2dota}
\end{theorem}
More information about this type of model can be found in~\cite{giammarresi1997twodimensional}.
\label{ota}