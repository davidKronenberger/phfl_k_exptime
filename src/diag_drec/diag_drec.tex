To define this kind of picture languages we use tiling systems. First, we will present the
$c2c$-deterministic tiling systems, where $c2c \in \{tl2br, tr2bl, bl2tr, br2tl\}$ describes a
scanning direction from one corner to its diagonal opposite corner. These tiling systems were
introduced in 2007 by M. Anselmo, D. Giammarresi and M. Madonia in~\cite{anselmo2007determinism}.
The main idea is based on the 2DOTA presented in Section~\ref{ota}.
This kind of deterministic tiling system scans a picture $p \in \Sigma^{*,*}$ from one corner to the
diagonally opposite one. While scanning $p$ the system creates the local picture $p' \in LOC(\Theta)
\subseteq \Gamma^{*,*}$. It is important that every position $(i, j)$ of $p$ can only be read for the
$tl2br$-deterministic tiling system if all the positions above and left from $(i, j)$ were
already read i.e. all positions of the subpicture $q = p((1, 1), (i, j))$ of $p$ without position
$(i, j)$. Formally, it is the following:
\begin{definition}
A tiling system $\tau = (\Sigma, \Gamma, \Theta, \pi)$ is \emph{tl2br-deterministic} if for any
$\gamma_1, \gamma_2, \gamma_3 \in \Gamma \cup \{\#\}$ and $a \in \Sigma$
there exists at most one tile $\begin{tabular}{|C{0.8cm}|C{0.8cm}|} 
\hline
$\gamma_1$ & $\gamma_2$ \tabularnewline
\hline
$\gamma_3$ & $\gamma_4$ \tabularnewline
\hline
\end{tabular} \in \Theta$, with $\pi(\gamma_4) = a$ or $\gamma_4 = \#$.

Similarly the $c2c$-deterministic tiling system is defined for any corner-to-corner
direction $c2c$.
\end{definition}
$L \in REC$ is a \emph{diagonal deterministic recognizable picture language} iff it can be
recognized by a $c2c$-deterministic tiling system for some $c2c$. M. Anselmo, D. Giammarresi
and M. Madonia have denoted the class of all diagonal deterministic recognizable picture languages
by DREC in~\cite{anselmo2007determinism}. Later V. Lonati and M. Pradella called this family $Diag$-DREC
in~\cite{lonati2009snake}. In the following we denote the \emph{family of all diagonal deterministic
recognizable picture languages} as $Diag$-DREC, too.

As an example for a tiling system that is $tl2br$-deterministic we can cite
from~\cite{anselmo2007determinism} the language $L_{fr = fc}$. Let $L_{fr = fc} = \{p \in
\Sigma^{*,*} \mid l_1(p) = l_2(p) \text{ and } p(1, i) = p(i, 1) \text{ for } 1 \leq i \leq
l_1(p)\}$ be the language of squares over a two-letter alphabet $\Sigma = \{a, b\}$, where the
first row is equal to the first column. Please note that $L_{fr = fc} \in REC$. Let us consider an
example of an $tl2br$-deterministic tiling system which recognizes $L_{fr = fc}$
(see~\cite{anselmo2007determinism}).
\begin{example}
$\tau = (\Sigma, \Gamma, \Theta, \pi)$ is a tiling system, where 
\begin{compactitem}
  \item $\Gamma = \{{x}_{y}^{z}$ with $x, y \in \Sigma, z \in \{0, 1, 2\}\}$,
  \item $\Theta = \left\lbrace 
  \begin{tabular}{|C{0.95cm}|C{0.95cm}|}
\hline
 $w_{r}^{0}$ & \hspace{0.07cm}$x_{r}^{0}$\hspace{0.07cm} \tabularnewline
\hline
 \hspace{0.07cm}$y_{s}^{0}$\hspace{0.07cm} & $z_{s}^{0}$\tabularnewline
\hline
\end{tabular}
  \right.$, 
\begin{tabular}{|C{0.95cm}|C{0.95cm}|}
\hline
 $w_{r}^{0}$ & \hspace{0.07cm}$x_{r}^{1}$\hspace{0.07cm} \tabularnewline
\hline
 \hspace{0.07cm}$y_{s}^{0}$\hspace{0.07cm} & $z_{s}^{0}$\tabularnewline
\hline
\end{tabular}, 
\begin{tabular}{|C{0.95cm}|C{0.95cm}|}
\hline
 $w_{r}^{1}$ & \hspace{0.07cm}$x_{s}^{2}$\hspace{0.07cm} \tabularnewline
\hline
 \hspace{0.07cm}$y_{s}^{0}$\hspace{0.07cm} & $z_{s}^{1}$\tabularnewline
\hline
\end{tabular}, 
\begin{tabular}{|C{0.95cm}|C{0.95cm}|}
\hline
 $w_{r}^{2}$ & \hspace{0.07cm}$x_{s}^{2}$\hspace{0.07cm} \tabularnewline
\hline
 \hspace{0.08cm}$y_{r}^{1}$\hspace{0.08cm} & $z_{s}^{2}$\tabularnewline
\hline
\end{tabular}, 
\begin{tabular}{|C{0.95cm}|C{0.95cm}|}
\hline
 $w_{r}^{2}$ & \hspace{0.07cm}$x_{s}^{2}$\hspace{0.07cm} \tabularnewline
\hline
 \hspace{0.08cm}$y_{r}^{2}$\hspace{0.08cm} & $z_{s}^{2}$\tabularnewline
\hline
\end{tabular},

\begin{tabular}{|C{0.95cm}|C{0.95cm}|}
\hline
 \hspace{0.1cm}$\#$\hspace{0.1cm} & \hspace{0.1cm}$\#$\hspace{0.1cm} \tabularnewline
\hline
 $\#$ & $\#$ \tabularnewline
\hline
\end{tabular},
\begin{tabular}{|C{0.95cm}|C{0.95cm}|}
\hline
 \hspace{0.1cm}$\#$\hspace{0.1cm} & \hspace{0.1cm}$\#$\hspace{0.1cm} \tabularnewline
\hline
 $\#$ & $z_{z}^{1}$\tabularnewline
\hline
\end{tabular},
\begin{tabular}{|C{0.95cm}|C{0.95cm}|}
\hline
 $\#$ & $x_{x}^{0}$\tabularnewline
\hline
 \hspace{0.1cm}$\#$\hspace{0.1cm} & \hspace{0.1cm}$\#$\hspace{0.1cm} \tabularnewline
\hline
\end{tabular},
\begin{tabular}{|C{0.95cm}|C{0.95cm}|}
\hline
 \hspace{0.1cm}$\#$\hspace{0.1cm} & \hspace{0.1cm}$\#$\hspace{0.1cm} \tabularnewline
\hline
 $y_{y}^{2}$ & $\#$ \tabularnewline
\hline
\end{tabular},
\begin{tabular}{|C{0.95cm}|C{0.95cm}|}
\hline
 $w_w^1$ & $\#$ \tabularnewline
\hline
 \hspace{0.1cm}$\#$\hspace{0.1cm} & \hspace{0.1cm}$\#$\hspace{0.1cm} \tabularnewline
\hline
\end{tabular},
\begin{tabular}{|C{0.95cm}|C{0.95cm}|}
\hline
 \hspace{0.1cm}$\#$\hspace{0.1cm} & \hspace{0.1cm}$\#$\hspace{0.1cm} \tabularnewline
\hline
 $y_{y}^{1}$ & $z_{z}^{2}$\tabularnewline
\hline
\end{tabular},

\begin{tabular}{|C{0.95cm}|C{0.95cm}|}
\hline
 \hspace{0.1cm}$\#$\hspace{0.1cm} & \hspace{0.1cm}$\#$\hspace{0.1cm} \tabularnewline
\hline
 $y_{y}^{2}$ & $z_{z}^{2}$ \tabularnewline
\hline
\end{tabular},
\begin{tabular}{|C{0.95cm}|C{0.95cm}|}
\hline
 $w_{w}^{0}$ & $x_{x}^{0}$ \tabularnewline
\hline
 \hspace{0.1cm}$\#$\hspace{0.1cm} & \hspace{0.1cm}$\#$\hspace{0.1cm} \tabularnewline
\hline
\end{tabular},
\begin{tabular}{|C{0.95cm}|C{0.95cm}|}
\hline
 $w_{w}^{0}$ & $x_{x}^{1}$ \tabularnewline
\hline
 \hspace{0.1cm}$\#$\hspace{0.1cm} & \hspace{0.1cm}$\#$\hspace{0.1cm} \tabularnewline
\hline
\end{tabular},
\begin{tabular}{|C{0.95cm}|C{0.95cm}|}
\hline
 $w_{w}^{2}$ & \hspace{0.1cm}$\#$\hspace{0.1cm} \tabularnewline
\hline
 \hspace{0.08cm}$y_{y}^{2}$\hspace{0.08cm} & $\#$ \tabularnewline
\hline
\end{tabular},
\begin{tabular}{|C{0.95cm}|C{0.95cm}|}
\hline
 $w_{w}^{2}$ & \hspace{0.1cm}$\#$\hspace{0.1cm} \tabularnewline
\hline
 \hspace{0.08cm}$y_{y}^{1}$\hspace{0.08cm} & $\#$\tabularnewline
\hline
\end{tabular},
\begin{tabular}{|C{0.95cm}|C{0.95cm}|}
\hline
 \hspace{0.1cm}$\#$\hspace{0.1cm} & \hspace{0.07cm}$x_{x}^{1}$\hspace{0.07cm} \tabularnewline
\hline
 $\#$ & $z_{z}^{0}$ \tabularnewline
\hline
\end{tabular},

$\left.
\begin{tabular}{|C{0.95cm}|C{0.95cm}|}
\hline
 \hspace{0.1cm}$\#$\hspace{0.1cm} & \hspace{0.07cm}$x_{x}^{0}$\hspace{0.07cm} \tabularnewline
\hline
 $\#$ & $z_{z}^{0}$ \tabularnewline
\hline
\end{tabular} \mid r, s, w, x, y, z \in \Sigma
\right\rbrace$ 
\item $\pi({x}_{y}^{z}) = x$,
\item the upper index $0$ describes a position below the diagonal, $1$ a position on the
diagonal and $2$ a position above the diagonal,
\item and the lower index of the upper right, lower left and lower right of each tile have
at least one neighbour, which has the same lower index. Remark that the upper and the left neighbour
of each lower right symbol of a tile, which has a $1$ as the upper index, both have the same lower
index as this element.
\end{compactitem}
\label{diag_drec_example}
\end{example}
For a better visualization of the language $L_{fr=fc}$ and the tiling system $\tau$, an example of a
picture $p \in L_{fr=fc}$ together with the corresponding local picture $p'$ can be found
below:
\vspace{1.2em}
\begin{center}
\begin{tabular}{lr}
$p = $ \begin{tabular}{|C{0.65cm}|C{0.65cm}|C{0.65cm}|C{0.65cm}|C{0.65cm}|}
\hline
 $a$ & $a$ & $b$ & $b$ & $a$ \tabularnewline
\hline
 $a$ & $b$ & $b$ & $a$ & $a$ \tabularnewline
\hline
 $b$ & $b$ & $a$ & $a$ & $b$ \tabularnewline
\hline
 $b$ & $b$ & $a$ & $a$ & $a$ \tabularnewline
\hline
 $a$ & $a$ & $a$ & $a$ & $b$ \tabularnewline
\hline
\end{tabular},
&
$p' = $ \begin{tabular}{|C{0.8cm}|C{0.8cm}|C{0.8cm}|C{0.8cm}|C{0.8cm}|}
\hline
 ${a}_{a}^{1}$ & ${a}_{a}^{2}$ & ${b}_{b}^{2}$ & ${b}_{b}^{2}$ & ${a}_{a}^{2}$\tabularnewline
\hline
 ${a}_{a}^{0}$ & ${b}_{a}^{1}$ & ${b}_{b}^{2}$ & ${a}_{b}^{2}$ & ${a}_{a}^{2}$\tabularnewline
\hline
 ${b}_{b}^{0}$ & ${b}_{b}^{0}$ & ${a}_{b}^{1}$ & ${a}_{b}^{2}$ & ${b}_{a}^{2}$\tabularnewline
\hline
 ${b}_{b}^{0}$ & ${b}_{b}^{0}$ & ${a}_{b}^{0}$ & ${a}_{b}^{1}$ & ${a}_{a}^{2}$\tabularnewline
\hline
 ${a}_{a}^{0}$ & ${a}_{a}^{0}$ & ${a}_{a}^{0}$ & ${a}_{a}^{0}$ & ${b}_{a}^{1}$\tabularnewline
\hline
\end{tabular}
\end{tabular}.
\end{center}
\vspace{1.2em}
We can see that $\tau$ is $tl2rb$-deterministic and $L(\tau) = L_{fr=fc}$ and hence $L_{fr=fc} \in
Diag$-DREC.

Remark that this tiling system is not $br2tl$-deterministic, $bl2tr$-deterministic or
$tr2bl$-deterministic. The two tiles \begin{tabular}{|C{0.8cm}|C{0.8cm}|}
\hline
 $a_{a}^1$ & $a_{a}^2$ \tabularnewline
\hline
 $a_{a}^0$ & $b_{a}^1$ \tabularnewline
\hline
\end{tabular}, 
\begin{tabular}{|C{0.8cm}|C{0.8cm}|}
\hline
 $a_{b}^1$ & $a_{a}^2$ \tabularnewline
\hline
 $a_{a}^0$ & $b_{a}^1$ \tabularnewline
\hline
\end{tabular} $\in \Theta$ with $\pi(a_{a}^1) = \pi(a_{b}^1) = a$ are the reason why $\tau$ is not
$br2tl$-deterministic. The two tiles \begin{tabular}{|C{0.8cm}|C{0.8cm}|}
\hline
 $a_{a}^2$ & $a_{a}^2$ \tabularnewline
\hline
 $a_{a}^2$ & $a_{a}^2$ \tabularnewline
\hline
\end{tabular}, 
\begin{tabular}{|C{0.8cm}|C{0.8cm}|}
\hline
 $a_{a}^2$ & $a_{a}^2$ \tabularnewline
\hline
 $a_{a}^1$ & $a_{a}^2$ \tabularnewline
\hline
\end{tabular} $\in \Theta$ with $\pi(a_{a}^2) = \pi(a_{a}^1) = a$ are the reason why $\tau$ is not
$tr2bl$-deterministic and the two tiles 
\begin{tabular}{|C{0.8cm}|C{0.8cm}|}
\hline
 $a_{a}^0$ & $a_{a}^0$ \tabularnewline
\hline
 $a_{a}^0$ & $a_{a}^0$ \tabularnewline
\hline
\end{tabular}, 
\begin{tabular}{|C{0.8cm}|C{0.8cm}|}
\hline
 $a_{a}^0$ & $a_{a}^1$ \tabularnewline
\hline
 $a_{a}^0$ & $a_{a}^0$ \tabularnewline
\hline
\end{tabular} $\in \Theta$ with $\pi(a_{a}^0) = \pi(a_{a}^1) = a$ are the reason why $\tau$ is not
$bl2tr$-deterministic.
 
 \vspace{0.5cm}
After all this information about diagonal deterministic tiling systems we will discuss some closure
properties.
\label{diag_drec}