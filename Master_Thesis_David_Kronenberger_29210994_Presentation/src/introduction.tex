
\section{$k$-EXPTIME/$_{\raise.17ex\hbox{$\scriptstyle\sim$}}$}

\begin{frame}

    \begin{example}
        \begin{center}
            \begin{tikzpicture}[]
                \node [place] (q1) [label=above:$p_1$] {$q_1$};
                \node  (temp) [left=of q1] {};
                \node  (label) [left=of temp] {$\mathcal{T}:$};
                \node [place] (q3) [below=of q1] {$q_3$};
                \node [place] (q5) [below=of q3,label=above:$p_1$] {$q_5$};
                \node [place] (q2) [left=of q3,label=left:$p_2$] {$q_2$}
                edge [post,bend left] node[above] {a} (q3)
                edge [post,bend right] node[above] {b} (q3)
                edge [post,bend right] node[left, below] {b} (q5)
                edge [pre,bend left] node[left, above] {a} (q1);
                \node [place] (q4) [right=of q3,label=right:$p_2$] {$q_4$}
                edge [post,bend left] node[auto, swap] {b} (q3)
                edge [post,bend right] node[auto, swap] {a} (q3)
                edge [pre,bend left] node[right, below] {a} (q5)
                edge [post,bend right] node[right, above] {b} (q1);
            \end{tikzpicture}
        \end{center}
    \end{example}

\end{frame}

\begin{frame}
    \begin{definition}
        Let be $\mathcal{T}_1 = (Q_1, \Sigma, P, \Delta_1, v_1)$ and $\mathcal{T}_2 = (Q_2, \Sigma, P, \Delta_2, v_2)
        $ two LTS. A \emph{bisimulation} is a binary relation $R \subseteq Q_1 \times Q_2$ that fulfills for all $(q_1,
        q_2) \in R$
        \begin{compactitem}
            \item $v_1 (q_1) = v_2 (q_2)$,
            \item for all $a \in \Sigma$ and all $q_1' \in Q_1$, if $q_1 \overset{a}{\rightarrow} q_1'$, then there
            is a state $q_2' \in Q_2$, $q_2 \overset{a}{\rightarrow} q_2'$ and $(q_1', q_2') \in R$ and
            \item for all $a \in \Sigma$ and all $q_2' \in Q_2$, if $q_2 \overset{a}{\rightarrow} q_2'$, then there is a
            state $q_1' \in Q_1$, $q_1 \overset{a}{\rightarrow} q_1'$ and $(q_1', q_2') \in R$.
        \end{compactitem}
        We call two states $q_1 \in Q_1$, $q_2 \in Q_2$ \emph{bisimilar}, noted as $(\mathcal{T}_1, q_1) \sim
        (\mathcal{T}_2, q_2)$, if there
        is a bisimulation $R$ such that $(q_1, q_2) \in R$.
    \end{definition}
\end{frame}

\begin{frame}

    \begin{example}
        \begin{center}
            \begin{tikzpicture}[]
                \node [place] (q1) [label=above:$p_1$] {$q_1$};
                \node  (temp) [left=of q1] {};
                \node  (label) [left=of temp] {$\mathcal{T}:$};
                \node [place] (q3) [below=of q1] {$q_3$};
                \node [place] (q5) [below=of q3,label=above:$p_1$] {$q_5$};
                \node [place] (q2) [left=of q3,label=left:$p_2$] {$q_2$}
                edge [post,bend left] node[above] {a} (q3)
                edge [post,bend right] node[above] {b} (q3)
                edge [post,bend right] node[left, below] {b} (q5)
                edge [pre,bend left] node[left, above] {a} (q1);
                \node [place] (q4) [right=of q3,label=right:$p_2$] {$q_4$}
                edge [post,bend left] node[auto, swap] {b} (q3)
                edge [post,bend right] node[auto, swap] {a} (q3)
                edge [pre,bend left] node[right, below] {a} (q5)
                edge [post,bend right] node[right, above] {b} (q1);
            \end{tikzpicture}
        \end{center}
    \end{example}

\end{frame}

\section{PHFL}

\begin{frame}

    \begin{definition}
        \emph{PHFL types} are given by the grammar
        \[\sigma, \tau \Coloneqq \bullet \mid \sigma^v \rightarrow \tau,\]
        where $v$ is called \textit{variance}. The \emph{variances} of PHFL are defined by the grammar
        \[v \Coloneqq + \mid - \mid 0.\]
    \end{definition}

    \begin{definition}
        \[\llbracket\tau\rrbracket_\mathcal{T}=
        \begin{cases}
            (\mathcal{P}(Q^d), \subseteq),  & \text{if }\tau = \bullet\\
            ((\llbracket\sigma_1\rrbracket_\mathcal{T})^v \rightarrow \llbracket\sigma_2\rrbracket_\mathcal{T}, \leq_{
            (\llbracket\sigma_1\rrbracket_\mathcal{T})^v \rightarrow \llbracket\sigma_2\rrbracket_\mathcal{T}}), &
            \text{if }\tau = \sigma_1^v\rightarrow \sigma_2.
        \end{cases}\]
    \end{definition}

\end{frame}

\begin{frame}

    \begin{definition}
        \begin{align*}
            \Phi,\Psi\Coloneqq&\top \mid p_i \mid \Phi \vee \Psi \mid \neg \Phi \mid \langle a \rangle_i \Phi \mid
            \{\emph{i}
            \leftarrow \emph{j}\,\} \Phi \mid X \mid \lambda (X^v\colon\tau).\Phi \mid \\&\Phi \Psi\mid
            \mu (X\colon\tau).\Phi
        \end{align*}
    \end{definition}

\end{frame}

\begin{frame}
\begin{example}
\label{example:phfl_order_0}
The following $2$-adic PHFL$^0$ formula $\Phi$ describes bisimilarity i.e. it denotes
those pairs $(q_1, q_2)$ such that $q_1 \sim q_2$ and vice versa.
\begin{align*}
\Phi = \nu (X \colon \bullet).\,
\underset{a \in \Sigma}{\bigwedge} [a]_1 \langle a \rangle_2 X \wedge [a]_2 \langle a \rangle_1 X \wedge
\underset{p \in P}{\bigwedge} p_1 \Leftrightarrow p_2
\end{align*}
\end{example}
\end{frame}

\section{Upper Bounds}

\begin{frame}
\begin{example}
\begin{center}
\begin{tikzpicture}[]
\node [place] (q1) {$1$};
\node  (temp1) [left=of q1] {};
\node  (label) [left=of q1] {$\mathcal{T}:$};
\node  (temp2) [right=of q1] {};
\node [place] (q2) [below=of temp2,label=left:$p$] {$2$}
edge [pre] node[left] {a} (q1);
\node [place] (q3) [right=of temp2,label=right:$q$] {$3$}
edge [pre] node[auto, swap] {b} (q1)
edge [pre] node[right] {c} (q2);
\end{tikzpicture}
\end{center}
\end{example}
\end{frame}

\begin{frame}

\begin{example}
\begin{center}
\resizebox{.8\textwidth}{!}{
\begin{tikzpicture}[]
\node [place] (q11) {$(1, 1)$}
edge [loop] node[above] {$S_d$} ();
\node (label) [left=of q11] {$\mathcal{T}_d:$};
\node (temp1) [right=of q11] {};
\node [place] (q12) [right=of temp1,label=above:{$p_2$}] {$(1, 2)$}
edge [pre] node[above] {$a_2$} (q11);
\node  (temp2) [right=of q12] {};
\node [place] (q13) [right=of temp2, label=above:{$q_2$}] {$(1, 3)$}
edge [pre] node[above] {$c_2$} (q12)
edge [pre, bend right=40] node[above] {$b_2$} (q11);
\node [place] (q21) [below=of q11,label=left:{$p_1$}] {$(2, 1)$}
edge [post, bend right=10] node[auto, swap] {$s_{(2, 1)}$} (q12)
edge [pre, bend left=10] node[auto] {$s_{(2, 1)}$} (q12)
edge [pre] node[left] {$a_1$} (q11);
\node [place] (q22) [below=of q12,label=right:{$p_1, p_2$}] {$(2, 2)$}
edge [pre] node[below] {$a_2, s_{(1, 1)}$} (q21)
edge [pre, bend left=25] node[right] {$a_1, s_{(2, 2)}$} (q12);
\node [place] (q23) [below=of q13,label=right:{$p_1, q_2$}] {$(2, 3)$}
edge [pre, bend left=30] node[below] {$c_2$} (q22)
edge [post, bend right=30] node[above] {$s_{(1, 1)}$} (q22)
edge [pre] node[right] {$a_1$} (q13);
\node [place] (q31) [below=of q21,label=below:{$q_1$}] {$(3, 1)$}
edge [pre, bend left=80] node[left] {$b_1$} (q11)
edge [pre] node[left] {$c_1$} (q21);
\node [place] (q32) [below=of q22,label=below:{$q_1, p_2$}] {$(3, 2)$}
edge [pre] node[above] {$a_2$} (q31)
edge [pre, bend left=25] node[left] {$c_1$} (q22)
edge [post, bend right=25] node[right] {$s_{(2, 2)}$} (q22);
\node [place] (q33) [below=of q23,label=below:{$q_1, q_2$}] {$(3, 3)$}
edge [pre] node[below] {$c_2, s_{(1, 1)}$} (q32)
edge [pre] node[right] {$c_1, s_{(2, 2)}$} (q23)
edge [pre, bend right=100] node[right] {$s_{(2, 2)}$} (q13)
edge [pre, bend left=40] node[right, below] {$s_{(1, 1)}$} (q31);
\end{tikzpicture}}
\end{center}
\end{example}

\end{frame}

\begin{frame}
\begin{definition}
\begin{align*}
F(\top) &= \top\\
F(X) &= X\\
F(p_i) &= p_i\\
F(\langle a \rangle_i \psi) &= \langle a_i \rangle F(\psi) \\
F(\psi \vee \psi') &= F(\psi) \vee F(\psi') \\
F(\neg \psi) &= \neg F(\psi) \\
F(\{\emph{i} \leftarrow \emph{j}\,\} \psi) &= \langle s_{(e(1), \dots, e(d))} \rangle F(\psi)  \\
F(\mu (X \colon \tau).\,\psi) &= \mu (X \colon T(\tau)).\,F(\psi) \\
F(\lambda (X^v \colon \tau).\, \psi) &= \lambda (X^v \colon T(\tau)).\, F(\psi) \\
F(\psi{\psi'}) &= F(\psi)F({\psi'})
\end{align*}

\end{definition}
\end{frame}

\section{HO + LFP}

\begin{frame}

\begin{definition}
\emph{HO types} are defined inductive as follows:
\begin{compactitem}
\item $\tau = \odot$ is a HO type,
\item $\tau = (\tau_1, \dots, \tau_n)$ is a HO type, if $\tau_1, \dots, \tau_n$ are
HO types.
\end{compactitem}
\end{definition}

\begin{definition}
Let $\mathcal{A}$ be a $\sigma$-structure over universe $\mathcal{U}$ then the universes of the
HO types are defined inductively as follows:
\begin{compactitem}
\item $D_\odot(\mathcal{U}) = \mathcal{U}$,
\item $D_{(\tau_1, \dots, \tau_n)}(\mathcal{U}) = \mathcal{P}(D_{\tau_1}(\mathcal{U}) \times \dots \times
D_{\tau_n}(\mathcal{U}))$
\end{compactitem}
\end{definition}

\end{frame}

\begin{frame}

\begin{definition}
The set of \emph{HO formulas} over $\sigma$ is defined inductively as follows:
\begin{compactitem}
\item $R(x_1, \dots, x_n)$ is a HO formula over $\sigma$ if $R \in \sigma$ is a relation with arity $n$ and
$x_1, \dots, x_n$ are variables of type $\odot$,
\item $X(x_1, \dots, x_n)$ is a HO formula over $\sigma$ if $X$ is a variable of type $(\tau_1, \dots, \tau_n)$
and $x_i$ is a variable of type $\tau_i$, with $i \in \{1, \dots, n\}$,
\item if $\varphi$ and $\psi$ are two HO formulas over $\sigma$, then $\neg\varphi$, $\varphi\wedge\psi$ and $\varphi
\vee \psi$ are also HO formulas over $\sigma$,
\item if $\varphi$ is a HO formula over $\sigma$ and $X$ a variable of arbitrary type $\tau$, then $\exists
X\colon\tau.\,\varphi$ and
$\forall X\colon\tau.\,\varphi$ are also HO formulas over $\sigma$.
\end{compactitem}
\end{definition}

\end{frame}

\begin{frame}

\begin{definition}
Let $\sigma$ an arbitrary signature, $X$ a relation variable of HO type $\tau = (\tau_1, \dots, \tau_k)$,
$\tau_1, \dots \tau_k$ arbitrary HO types, $x_1, \dots, x_k$ variables of HO type $\tau_1$, \dots, $\tau_k$
respectively and $\varphi(X, x_1, \dots, x_k)$ a formula over $\sigma$ with free variables $X, x_1,
\dots, x_k$. For each $\sigma$-structure $\mathcal{A}$ with universe $\mathcal{U}$, $\varphi(X, x_1, \dots, x_k)$
induces the operator
\begin{align*}
F_\varphi^\mathcal{A}\colon\mathscr{P}(D_\tau(\mathcal{U})) &\longrightarrow \mathscr{P}(D_\tau(\mathcal{U}))\\
A &\longmapsto F_\varphi^\mathcal{A}(A) \coloneqq \{(a_1, \dots, a_k) \mid \mathcal{A} \models \varphi(A, a_1,
\dots, a_k)\}
\end{align*}
where $a_1 \in D_{\tau_1}(\mathcal{U})$, \dots, $a_k \in D_{\tau_k}(\mathcal{U})$.
\end{definition}
\end{frame}

\begin{frame}

\begin{definition}
Let $\mathcal{A}$ be a $\sigma$-structure and $\alpha$ a variable assignment over universe $\mathcal{U}$. The
semantics of a HO($\mathit{LFP}$) formula extends that of HO formulas with the following definition:
\begin{compactitem}
\item $\mathcal{A}, \alpha \models [\mathit{LFP}_{X, x_1, \dots, x_k}\varphi(X, x_1, \dots, x_k)](v_1, \dots,
v_k)$ iff $(\alpha(v_1), \dots, \alpha(v_k)) \in \mathit{LFP}(F_\varphi^\mathcal{A})$.
\end{compactitem}
\end{definition}

\end{frame}

\begin{frame}

\begin{definition}
Let $F\colon \mathscr{P}(A) \rightarrow \mathscr{P}(A)$ be an operator on a finite set $A$, then the \emph{partial
fixpoint} of $F$, abbreviated as $\mathit{PFP}$($F$), is defined as follows:
\[\mathit{PFP}(F)\coloneqq\begin{cases}
F^{i+1}(\emptyset)=F^i(\emptyset),  & \text{if such } i \in \{0,\dots,|A|\} \text{ exists}\\
\emptyset, & \text{otherwise,}
\end{cases}\]
where $F^0(\emptyset) = \emptyset$, $F^1(\emptyset) = F(\emptyset)$, $F^2(\emptyset) = F(F(\emptyset))$, and so on.
\end{definition}

\end{frame}